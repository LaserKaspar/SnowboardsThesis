\chapter{Projektdokumentation}

\section{Meilensteine}

\subsection{Prototype (23.09)}
Die Hauptmechaniken wurden Implementiert, und erste Systeme Getestet. Der Art-Style von unserem Spiel wurde ausgearbeitet, und dazu wurden erste Style-Guides, Charaktere, Farbpaletten und Target-Experience festgelegt.

\subsection{Iteration Interim 1 (20.10)}
Es wurden die Spell-Types und des Raum-Design definiert und implementiert. Der Gameloop des Spieles funktionierte erstmals komplett. Es wurden außerdem erste Playtests in einem Test-Level durchgeführt, um mögliche Probleme mit dem Design herauszufinden.

\subsection{Iteration Interim 2 (22.12)}
Ein Großteil der Grafiken und 3D-Modelle wurden importiert. Außerdem wurde die Mana-Menge und andere Variablen anhand von Playtests, die durchgeführt wurden, um früher auf Fehler im Design zu stoßen und sie zu beheben, angepasst. Die Technik des Spieles ist auch größtenteils fertiggestellt.

\subsection{Vertical Slice (30.03)}
Ein Vertical Slice des Projekts wurde finalisiert. Der Gameloop ist nun ohne Errors durchspielbar, und die gewünchte Experience, Art/Design und Performance wurden gut umgesetzt. Es gibt außerdem ein vollständiges Presskit inklusive Trailer.

\section{Playtest Protokolle \& Ergebnisse}
Es wurden bereits mehrere Playtests durchgeführt. Die Ersten mussten aufgrund unerwarteter Fehler frühzeitig abgebrochen werden. Die Playtests, die Problemlos abgelaufen sind, haben allerdings die gewünschte Experience erzeugt. Nach den Playtests, wurden die Teilnehmenden anschließend darum gebeten einige Fragen zu beantworten.

\subsection{Welcher Spell ist der Lustigste?}
Der Großteil der PC- als auch der VR-Spielenden gaben an, dass die Bombe der beste Spell sei. Auf Platz zwei war die Mikado-Barrikade, da damit gute Kombinationen erzielt werden konnten.

\subsection{Welcher Spell ist der Unlustigste?}
Die wolke wurde von vielen Spielenden als "`sinnlos"' oder "`Unfair"' beschrieben. Das ist wahrscheinlich darauf zurückzuführen, dass sie erst relativ spät Einfluss auf die Snowboarder hat und nicht sehr genau platziert werden kann, also nicht wirklich ein Zusammenhang von Wolken-kill zu VR-Skill besteht.

\subsection{In welcher Rolle hattest du mehr Spaß?}
Ca. 65\% der Spielenden gaben an, dass sie als Snowboarder mehr Spaß hatten. Manche hatten leider auch Probleme mit Motion-Sickness.

\subsection{Gab es Momente wo du stark mitgefiebert hast?}
Die Spannung wenn nur nur noch ein Snowboarder im Spiel ist, ist sehr intensiv. Vor allem gegen Ende einer Runde.

\subsection{Wie würdest du die Gesamterfahrung bewerten?}
Der Durchschnittliche Score der Gesamterfahrung beträgt ~7.5/10 Punkten. Die Erfahrung der Snowboarder war laut Umfrage etwas besser. "`Ist für ein paar runden ganz witzig."'

\subsection{Gab es Momente in denen dir langweilig war?}
Manche Snowboarder haben sich überfordert gefühlt, wenn sie bereits mit einer erfahrenen Gruppe gespielt haben. Viele VR-Spielende gaben an, dass die Wartezeiten zwischen den Runden verkürzt werden sollte.

%\includepdf[page=-]{formulare/PlaytestProtokolle}

\section{Begleitprotokolle}
%\includepdf[page=-]{formulare/Zeitprotokolle}
