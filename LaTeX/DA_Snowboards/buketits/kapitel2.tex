\chapter{Mögliche Performance-Optimierungen} \label{simon_performance}
\section{Level Of Detail}
Der Begriff "LOD" (Level of Detail) bezieht sich auf die Komplexität der 3D-Darstellung eines 3D-Models. Das LOD kann mit der Distanz vom Betrachter oder nach Wichtigkeit des Objektes verringert oder erhöht werden. LOD kann die Geschwindigkeit und Effizienz des Renderings erhöhen, indem die Arbeitslast in der Grafikpipeline verringert wird. Diese Performance-Verbesserung wird dadurch erhalten, dass beispielsweise weniger Vertices gerendert werden müssen. Die verringerte visuelle Qualität ist oft nicht bemerkbar, da Objekte sich dann in erhöhter Distanz befinden oder sich schnell bewegen. LOD wird meistens nur auf Geometriedetails angewendet, jedoch kann dieses Konzept auch anders angewendet werden. Theoretisch kann man auch Shader und Pixelkomplexität basierend auf diesen Parametern abändern. Das sogenannte "Mip-Mapping", das seit einigen Jahren existiert, kann ebenfalls eine höhere Rendering-Qualität bieten.
\cite{_lod_formeshes}

\begin{figure}[H]
	\centering
	\includegraphics[width=11cm]{images/buketits_lod}
	\caption{Ein Mesh mit zwei verschiedenen Detailgraden\cite{_lod_formeshes}}
\end{figure}


\section{Culling}
Beim 3D-Rendering beschreibt der Begriff "Culling" das frühzeitige Aussortieren von Objekten jeglicher Art, die nicht zum endgültigen Bild beitragen. Es gibt viele Techniken, die Objekte in verschiedenen Bereichen der Rendering-Pipeline auszusortieren. Einige dieser Techniken werden vollständig in Software auf der CPU ausgeführt, andere werden von der Hardware (GPU) unterstützt und wieder andere sind in die Grafikkarte integriert. Es ist hilfreich, alle diese Techniken zu verstehen, um eine gute Leistung zu erzielen. Um viel Rechenkraft zu sparen, ist es wichtig, frühzeitig und in größerem Umfang zu Cullen. Wichtig ist auch, dass das Culling an sich selbst nicht zu viel Leistung und Speicherplatz kostet.
\cite{_cryengine_culling}
Normalerweise erfolgt das Culling der Engine auf der Draw Call Ebene. Beim Culling werden nicht die einzelnen Triangles gecullt, sondern ganze Objekte. Das bedeutet, dass es sinnvoll sein kann, große Objekte in mehrere Teile aufzuteilen, sodass nicht alle Teile zusammen gerendert werden.\cite{_cryengine_culling}
Was vom Culling ausgeblendet wird, muss auch nicht von der GPU berechnet werden.
\cite{_cryengine_culling}

\subsection{Beispiel}
Ein Objekt, bei dem Culling angewandt werden könnte, ist beispielsweise ein Haus, das einen Innenraum besitzt. Dieses Haus besteht aus 5 Millionen Triangles. Sieht man das Gebäude, dann sieht man also 5 Millionen Triangles. Sieht man es nicht, rendert man 5 Millionen Triangles weniger. Nun könnte man optimieren.\cite{_cryengine_culling}
Der wichtigste Schritt bei so einem Haus wäre, den Innenraum von der äußeren Fassade zu trennen. So kann man den Innenraum Cullen, falls man sich außerhalb des Gebäudes befindet und spart beispielsweise die Hälfte der Triangles ein. Umgekehrt kann man die Fassade Cullen, wenn man sich innerhalb des Gebäudes aufhält.\cite{_cryengine_culling}
Zusätzlich kann man jetzt noch Level of Detail für die äußere Fassade verwenden, um Triangles zu sparen, während man nicht in unmittelbarer Nähe des Gebäudes befindet. Sollte man ein detailliertes Gebäude implementieren wollen, ist es oft klug, dieses Gebäude in mehrere Teile zu teilen und auf diesen zusätzlich LODs zu verwenden. Culling ist also wesentlich mehr als nur eine Funktion und es muss vorher überlegt werden, wo und wie es angewandt wird.
\cite{_cryengine_culling}

\subsection{Frustum-Culling}
"Frustrum-Culling" ist eine wichtige Technik, die in so gut wie jeder ernst zu nehmenden Engine vorhanden ist. Die Grundlage des Frustrum-Cullings ist, Objekte, die sich nicht in der Sehpyramide befinden (also nicht gesehen werden und keinen Einfluss auf das endgültige Bild haben), nicht zu rendern und somit Rechenleistung zu sparen.
\cite{_cryengine_culling}

"Occlusion-Culling" überprüft, ob Objekte oder auch Polygone von anderen verdeckt werden und rendert diese dann nicht. Diese Technik ist besonders bei Szenen mit hoher Tiefenkomplexität äußerst wichtig und kann zu erheblichen Performance-Einsparungen führen.
\cite{_cryengine_culling}

\begin{figure}[H]
	\centering
	\includegraphics[width=12cm]{images/buketits_frustrum}
	\caption{Sehpyramide ohne Culling, mit Frustrum Culling und mit Frustrum und Occlusion Culling\cite{_culling}}
\end{figure}

\subsection{Backface-Culling}
"Backface-Culling" ist mittlerweile ein Standardfeature, dass die Ausrichtung der Polygone eines Meshes überprüft und damit festlegt, ob diese gerendert werden sollen.
\cite{_cryengine_culling}

\section{Draw-Calls}
In einem Spiel werden unterschiedliche Aufgaben entweder von GPU oder CPU berechnet.
Aufgaben, die die CPU und die GPU weitergibt, nennt man "Draw-Calls". Man kann sich Draw-Calls bildhaft als Briefe vorstellen, die die CPU und die GPU weitersendet. Umso mehr Draw-Calls geschickt werden, umso mehr Draw-Calls muss die GPU verarbeiten. Das Empfangen eines Draw-Calls dauert in der Regel länger als diesen Call zu verarbeiten. Grundsätzlich (ohne andere Optimierungen) wird für jedes einzelne Mesh ein Draw-Call versendet. Somit ist es vorteilhaft, wenn möglich, viele kleine Objekte in ein großes Mesh zu kombinieren. So gibt es weniger Draw-Calls, jedoch werden die Draw-Calls größer. Man muss also eine gute Balance zwischen Menge und Größe finden. Grundsätzlich kann man jedoch grob behaupten, dass man möglichst wenige Draw-Calls senden sollte. Draw-Calls werden aber nicht nur von Meshes verursacht, sondern können beispielsweise auch aus Schattenberechnungen und Reflektionen entstehen.
\cite{_drawcallbatching}

\subsection{Dynamic-Draw-Call-Batching}
Die Unity-Engine stapelt Draw-Calls von mehreren Game-Objects, die das gleiche Material verwenden. Somit wird die Anzahl der Draw-Calls reduziert und die Last auf der Grafikkarte verringert. Die besten Ergebnisse können erzielt werden, indem dasselbe Material auf möglichst viele Objekte angewandt wird. Um diese Technik anzuwenden, muss man jedoch auf ein paar Regulierungen achten. Mesh-Renderer, Trail-Renderer, Line-Renderer, Particle-Systems und Sprite-Renderer unterstützen Draw-Call-Batching. Andere Rendering Komponenten unterstützen diese Technik nicht. Abgesehen davon muss die Rendering-Komponente dieselbe sein. Das heißt, dass beispielsweise.Mesh-Renderer und Trail-Renderer nicht gebatcht werden. Sollte man mehrere Materialien verwenden, ist es auch möglich, die gleiche Textur zu einzusetzen und damit Performance zu sparen. Dieser Vorgang heißt "Texture-Atlasing".
\cite{_drawcallbatching}

\subsection{Static-Draw-Call-Batching}
Die Unity-Engine kann Static-Batching während der Build und Runtime durchführen. Generell sollte jedes Objekt, das in einer Szene existiert, durch den Editor in der Build Time gebatcht werden. Sollte ein Objekt erst später erzeugt werden, kann dieses jedoch auch mithilfe der Runtime API gebatcht werden. Wenn die Runtime API verwendet wird, kann das Rootobjekt auch später noch skaliert, rotiert und transformiert werden. Die einzelnen Meshes, aus denen dieser Batch besteht, jedoch nicht. Um Static-Batching verwenden zu können, muss die Gruppe an Game-Objects einige Kriterien erfüllen:
\cite{_drawcallbatching}

"`The GameObject is active.
The GameObject has a Mesh Filter
component, and that component is enabled.
The Mesh Filter component has a reference to a Mesh.
The mesh is read/write enabled.
The mesh has a vertex count greater than 0.
The mesh has not already been combined with another Mesh.
The GameObject has a Mesh Renderer
component, and that component is enabled.
The Mesh Renderer component does not use any Material with a shader
that has the DisableBatching tag set to true.
Meshes you want to batch together use the same vertex attributes. For example Unity can batch meshes that use vertex position, vertex normal, and one UV with one another, but not with meshes that use vertex position, vertex normal, UV0, UV1, and vertex tangent."'
(siehe \cite{_drawcallbatching})


\section{Object-Pooling}
Wenn Game-Objects schnell erstellt und gelöscht werden müssen, ist Object-Pooling eine gute Technik, um Projekte zu optimieren und die CPU-Last zu verringern. Object-Pooling ist ein Design-Pattern, bei dem die benötigten Objekte bereits vor dem Start des Spiels erstellt und in ein „Zwischenlager“ geschoben werden. Bei Anwendung werden diese dann aus dem Lager geholt und anstatt diese Objekte nach Gebrauch zu zerstören, werden diese dann wieder ins Lager verschoben. Bei manchen Anwendungsfällen erhöht Object-Pooling die Performance erheblich. Ein gutes Beispiel wären unter anderem Projektile in einem Top-Down-Shooter.
\cite{_objectpooling}

\section{Coroutines vs Update}
In Unity werden sowohl Coroutines als auch die Update-Methode verwendet, um den Zustand eines Spiels oder einer Anwendung im Laufe der Zeit zu aktualisieren. Sie funktionieren jedoch auf leicht unterschiedliche Weise, was sich auf ihre Leistungsmerkmale auswirken kann. Die Update-Methode ist eine eingebaute Unity-Funktion, die einmal pro Frame aufgerufen wird und typischerweise zur Aktualisierung der Positionen, Geschwindigkeiten und anderer Eigenschaften von Objekten in der Szene verwendet wird. Diese Methode wird von Unity automatisch aufgerufen. Coroutines sind eine Möglichkeit, benutzerdefinierte Iterationen zu erstellen, die gleichzeitig mit der Update-Methode ausgeführt werden. Eine Coroutine ist im Wesentlichen eine Funktion, die Angehalten und zu einem späteren Zeitpunkt wieder aufgenommen werden kann. Coroutines können verwendet werden, um einen Prozess durchzuführen, der über einen bestimmten Zeitraum hinweg erfolgen soll, z. B. das Verschieben eines Objekts an eine neue Position oder das Abspielen einer Animation. Coroutines sind weniger Rechen-effizient als die Update-Methode. Wenn der Zustand eines Objekts sich einmal pro Frame aktualisieren soll, ist es in der Regel am besten, die Update-Methode zu verwenden. Dadurch wird sichergestellt, dass die Änderungen mit dem Rest des Spiels synchronisiert sind, und es benötigt weniger Rechenleistung. Wenn man eine Funktion immer wieder pausieren muss und diese vielleicht sogar in anderen Abständen als jeden Frame aufrufen will, dann ist eine Coroutine die bessere Wahl.
\cite{_dickinson2015unity}

\section{Statische -und Dynamische Collider}
Ein Objekt, auf dem sich ein Collider befindet, kann je nach den Einstellungen, die in dem zugehörigen Rigidbody gesetzt sind, unterschiedlich mit anderen Collidern interagieren. Die drei wichtigsten Einstellungsmöglichkeiten sind: Static-Collider(Kein Rigidbody), Rigidbody-Collider, Kinematic-Rigidbody-Collider.
\cite{_staticCollider}

\subsection{Static-Collider}
Ein statischer Collider ist ein Collider auf dem sich ein Collider aber kein Rigidbody befindet. Statische Collider werden normalerweise bei Levelgeometrie, die sich nicht bewegt, verwendet. Statische Collider, können von anderen Collider nicht bewegt werden. In bestimmten Fällen optimiert die Physics-Engine statische Collider, die sich nicht bewegen. Statische Collider können während der Programmlaufzeit an und aus geschalten und bewegt werden, ohne dass die Rechenzeit der Physics-Engine wesentlich erhöht wird.
\cite{_staticCollider}

\subsection{Rigidbody-Collider}
Ein Rigidbody-Collider ist ein Game-Object mit einem Collider und einem normalen nicht-kinematischem Rigidbody. Rigidbody-Collider sind vollständig von der Physics-Engine simuliert und können auf Kollisionen und Kräfte-Einwirkungen reagieren. Sie können mit anderen Objekten (auch statischen Collider) kollidieren und sind die am meisten verwendeten Collider in einem Spiel, die Physics verwenden.
\cite{_staticCollider}

\subsection{Kinematic-Rigidbody-Collider}
Ein Kinematic-Rigidbody-Collider ist ein Game-Object auf dem sich ein Collider und ein kinematischer Rigidbody befinden. Ein kinematischer Rigidbody ist ein Rigidbody, bei dem die "IsKinematic-Property" aktiviert ist. Ein kinematischer Rigidbody kann zwar anhand seines Transforms bewegt werden, aber er kann nicht wie der nicht kinematische Rigidbody, auf Kollisionen und Kräfte-Einwirkungen reagieren. Ein Kinematischer Rigidbody-Collider wird dann verwendet, wenn ein Objekt manchmal mit einem Collider bewegt wird, es aber sonst wie ein statischer Collider agiert.\cite{_staticCollider}
Ein Beispiel dafür wäre eine Schiebetür, die normalerweise ein unbewegliches Hindernis ist, sich aber wenn nötig öffnen kann. Im Gegensatz zu einem statischen Collider kann ein bewegender Kinematic-Rigidbody Reibung an andere Objekte weitergeben und "`weckt"' andere Rigidbodies bei Kontakt auf.
Auch wenn dieser sich nicht bewegt, hat dieser ein anderes Verhalten als ein statischer Collider. Beispielsweise kann ein Collider, der als Trigger gesetzt ist, nur Trigger Events empfangen, wenn sich ein Rigidbody auf dem Game-Object befindet. Sollte gewollt sein, dass dieser Trigger nicht von Physik beeinflusst werden kann, dann kann die "IsKinematic-Property" auf dem zugehörigen Rigidbody aktiviert werden. Ein Rigidbody kann immer zwischen normal und kinematisch mit Hilfe der "IsKinematic-Property", gewechselt werden. Ein Rigidbody kann immer zwischen normal und kinematisch mithilfe der "IsKinematic-Property" gewechselt werden. Ein gängiges Beispiel dafür wäre der "`Ragdoll"' Effekt, bei dem sich ein Charakter normalerweise mit Animationen bewegt, aber physikalisch geworfen werden kann. Beispielsweise könnte eine Explosion oder starke Kollision diesen Charakter schleudern. Es kann jedem Körperteil des Charakters ein Rigidbody zugewiesen werden, der auf "IsKinematic" gesetzt ist. Die Körperteile bewegen sich dann normal, mit der Animation, bis "IsKinematic" ausgeschalten wird und der Charakter sich wie ein Physikobjekt verhält. Jetzt könnte eine Explosion oder Kollision diesen Charakter werfen, um eine glaubwürdige Reaktion auf diese Explosion zu bieten.

\cite{_staticCollider}
