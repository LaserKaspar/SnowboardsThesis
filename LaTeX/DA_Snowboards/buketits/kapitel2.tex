\chapter{Mögliche Performance Optimierungen}
\section{Level Of Detail}
Dieses Dokument ist als vorwiegend technische Starthilfe für das
Erstellen einer Masterarbeit (oder Bachelorarbeit) mit \latex
gedacht und ist die Weiterentwicklung einer früheren
Vorlage\footnote{Nicht mehr verfügbar.} für das Arbeiten mit
Microsoft \emph{Word}. Während ursprünglich daran gedacht war, die
bestehende Vorlage einfach in \latex zu übernehmen, wurde rasch
klar, dass allein aufgrund der großen Unterschiede zum Arbeiten
mit \emph{Word} ein gänzlich anderer Ansatz notwendig wurde. Dazu
kamen zahlreiche Erfahrungen mit Diplomarbeiten in den
nachfolgenden Jahren, die zu einigen zusätzlichen Hinweisen Anlass gaben.


\section{Culling}
Dieses Dokument ist als vorwiegend technische Starthilfe für das
Erstellen einer Masterarbeit (oder Bachelorarbeit) mit \latex
gedacht und ist die Weiterentwicklung einer früheren
Vorlage\footnote{Nicht mehr verfügbar.} für das Arbeiten mit
Microsoft \emph{Word}. Während ursprünglich daran gedacht war, die
bestehende Vorlage einfach in \latex zu übernehmen, wurde rasch
klar, dass allein aufgrund der großen Unterschiede zum Arbeiten
mit \emph{Word} ein gänzlich anderer Ansatz notwendig wurde. Dazu
kamen zahlreiche Erfahrungen mit Diplomarbeiten in den
nachfolgenden Jahren, die zu einigen zusätzlichen Hinweisen Anlass gaben.


\section{Batching}
Dieses Dokument ist als vorwiegend technische Starthilfe für das
Erstellen einer Masterarbeit (oder Bachelorarbeit) mit \latex
gedacht und ist die Weiterentwicklung einer früheren
Vorlage\footnote{Nicht mehr verfügbar.} für das Arbeiten mit
Microsoft \emph{Word}. Während ursprünglich daran gedacht war, die
bestehende Vorlage einfach in \latex zu übernehmen, wurde rasch
klar, dass allein aufgrund der großen Unterschiede zum Arbeiten
mit \emph{Word} ein gänzlich anderer Ansatz notwendig wurde. Dazu
kamen zahlreiche Erfahrungen mit Diplomarbeiten in den
nachfolgenden Jahren, die zu einigen zusätzlichen Hinweisen Anlass gaben.

\section{Object Pooling}
Dieses Dokument ist als vorwiegend technische Starthilfe für das
Erstellen einer Masterarbeit (oder Bachelorarbeit) mit \latex
gedacht und ist die Weiterentwicklung einer früheren
Vorlage\footnote{Nicht mehr verfügbar.} für das Arbeiten mit
Microsoft \emph{Word}. Während ursprünglich daran gedacht war, die
bestehende Vorlage einfach in \latex zu übernehmen, wurde rasch
klar, dass allein aufgrund der großen Unterschiede zum Arbeiten
mit \emph{Word} ein gänzlich anderer Ansatz notwendig wurde. Dazu
kamen zahlreiche Erfahrungen mit Diplomarbeiten in den
nachfolgenden Jahren, die zu einigen zusätzlichen Hinweisen Anlass gaben.

\section{Multithreading}
Dieses Dokument ist als vorwiegend technische Starthilfe für das
Erstellen einer Masterarbeit (oder Bachelorarbeit) mit \latex
gedacht und ist die Weiterentwicklung einer früheren
Vorlage\footnote{Nicht mehr verfügbar.} für das Arbeiten mit
Microsoft \emph{Word}. Während ursprünglich daran gedacht war, die
bestehende Vorlage einfach in \latex zu übernehmen, wurde rasch
klar, dass allein aufgrund der großen Unterschiede zum Arbeiten
mit \emph{Word} ein gänzlich anderer Ansatz notwendig wurde. Dazu
kamen zahlreiche Erfahrungen mit Diplomarbeiten in den
nachfolgenden Jahren, die zu einigen zusätzlichen Hinweisen Anlass gaben.

\section{Coroutines vs Update}
Dieses Dokument ist als vorwiegend technische Starthilfe für das
Erstellen einer Masterarbeit (oder Bachelorarbeit) mit \latex
gedacht und ist die Weiterentwicklung einer früheren
Vorlage\footnote{Nicht mehr verfügbar.} für das Arbeiten mit
Microsoft \emph{Word}. Während ursprünglich daran gedacht war, die
bestehende Vorlage einfach in \latex zu übernehmen, wurde rasch
klar, dass allein aufgrund der großen Unterschiede zum Arbeiten
mit \emph{Word} ein gänzlich anderer Ansatz notwendig wurde. Dazu
kamen zahlreiche Erfahrungen mit Diplomarbeiten in den
nachfolgenden Jahren, die zu einigen zusätzlichen Hinweisen Anlass gaben.
