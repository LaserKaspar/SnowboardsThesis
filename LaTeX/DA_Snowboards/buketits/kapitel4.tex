\chapter{Fazit}
Die Entwicklung von asymmetrischen Virtual Reality Spielen birgt mehr Schwierigkeiten als ursprünglich erwartet. Wenn man von der Rechenintensivität absieht, zeigen sich einige Probleme erst mitten in der Entwicklungszeit. Mehrere Kameras gleichzeitig rendern zu müssen, senkt nicht nur die Performance, sondern erschwert auch die Möglichkeit, diese zu optimieren. Die verschiedenen Größenverhältnisse des VR-Charakters und der Snowboardenden verkomplizieren die Implementierung raumübergreifender Systeme, in der diese miteinander interagieren können. Die größte Schwierigkeit der Entwicklung bestand aus den vielen, sehr unterschiedlichen Systemen, die miteinander arbeiten müssen. Aufgrund der simplen 3D Modelle, die verwendet werden, konnte durch ein paar simplere Optimierungen schon sehr viel Rechenleistung gespart werden. Somit war es nicht notwendig, komplizierte Optimierungen zu Implementieren. Die Herausforderungen, die asymmetrisches Virtual Reality mit sich gebracht hat, konnten durch Zusammenarbeit gut bewältigt werden und das Arbeiten innerhalb des Teams hat bis auf ein paar Ausnahmen sehr gut funktioniert. Dadurch, dass die Artists nichts über den Code und das Game-Design des Spiels wissen mussten, wurden Informationen über Änderungen nur zwischen zwei Personen weitergegeben, was den Prozess erheblich erleichtert hat. Wenn \emph{Tricks ´n´ Treats} in Zukunft weiterentwickelt werden würde, müssten einige Änderungen durchgeführt werden. Um das Spiel grafisch zu skalieren, müsste Level of Detail implementiert werden und einige Systeme, welche das Local Multiplayer handhaben, müssten neu oder umgeschrieben werden, um zukünftig fehlerfrei zu bleiben. Sollten zukünftige Playtests zeigen, dass das Spiel unter den Testenden gut ankommt, ist es erstrebenswert, diese komplexen Probleme zu beheben und weitere Arbeitszeit in das Projekt zu investieren.
