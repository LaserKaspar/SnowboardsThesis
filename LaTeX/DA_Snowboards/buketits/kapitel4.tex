\chapter{Fazit}
Die Implementierung von asymmetrischen Virtual Reality Spielen birgt mehr Schwierigkeiten als ursprünglich erwartet. Wenn man von der Rechenintensivität absieht, zeigen sich einige Probleme erst mitten in der Entwicklungszeit. Mehrere Kameras gleichzeitig rendern zu müssen, senkt nicht nur die Performance, sondern erschwert auch die Möglichkeit diese zu Optimieren. Die verschiedenen Größenverhältnisse des VR-Charakters und der Snowboardenden erschweren die Implementierung raumübergreifender Systeme, in welcher diese miteinander interagieren können. Wenn Tricks ´n´ Treats in Zukunft weiterentwickelt werden würde, müssten einige Änderungen gemacht werden. Um das Spiel grafisch skalierbarer zu machen, müsste noch Level of Detail implementiert werden. Einige Systeme müssten neu oder umgeschrieben werden, um zukünftig fehlerfrei zu bleiben. Die Arbeit innerhalb des Teams hat, bis auf ein paar Ausnahmen, sehr gut funktioniert, weil innerhalb des Teams viel kommuniziert wurde. Dadurch, dass die Artists nichts über den Code und das Game Design des Spiels wissen mussten, wurden Informationen über Änderungen nur zwischen zwei Personen weitergegeben.
