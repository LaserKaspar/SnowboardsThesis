\chapter{Einleitung}
\label{cha:sa_Einleitung}

\section{Virtuelle Realität}
Die Virtuelle Realität (Virtual Reality, VR) ist eine computergenerierte virtuelle Umgebung, welche in Echtzeit berechnet wird. In dieser ist es möglich sich umzuschauen und je nach Anwendung kann man sich auch Bewegen und beliebig mit Objekten interagieren. Es gibt zwei verschiedene Möglichkeiten Virtual Reality zu erleben. Einerseits gibt es speziell angefertigte Räume, in welchen Großbildleinwände angebracht sind, andererseits gibt es auch Head-Mounted-Displays (VR Brillen, die man aufsetzt), welche hier im Fokus stehen werden. In der Virtuellen Realität gibt es eine breit gefächerte Reichweite an Spielen, in welchen man verschiedenste Situationen erleben kann. Anfangs waren Spiele für VR nur simple Testräume, in welchen man mit Objekten interagieren konnte. Mittlerweile ist es jedoch möglich riesige Spiele mit einer eigenen Geschichte zu erleben. Die Interaktionen, welche damals ein ganzes Spiel ausmachten, sind seit einiger Zeit nur noch die Basis, worauf neue Spiele aufbauen.

\subsection{Asymmetrische VR Spiele}
Die meisten VR-Spiele kapseln einen von der Außenwelt ab. Sobald man die VR-Brille aufsetzt, befindet man sich in einer neuen, digitalen Welt. Eine Zeit lang war die zwischenmenschliche Interaktion in der virtuellen Realität gar nicht möglich. Seit neuerem gibt es auch Spiele, welche unterstützen, dass VR-Spieler über das Internet miteinander spielen können (Online-Multiplayer). Somit braucht man jedoch zwei HMDs (VR-Brillen), zwei PCs und man kann in der Regel nicht nebeneinander spielen. Asymmetrische VR-Spiele versuchen, reguläre PC-Spiele mit VR-Spielen zu kombinieren. So können z.B. zwei Freunde mit einem PC und einer VR-Brille miteinander spielen. Das funktioniert z.B., indem die zwei Spieler sich in derselben Spielwelt befinden, jedoch unterschiedliche Charaktere steuern. Durch die verschiedenen Steuerungsmöglichkeiten (VR-Controller, Tastatur und Maus, Gamepad) ergibt sich oft automatisch eine bestimme Rollenverteilung. So könnte der PC-Spieler z.B. einen Menschen spielen, welcher durch ein Level manövrieren muss und zeitgleich steuert der VR-Spieler einen Riesen, welcher den Menschen behindern muss, indem er ihm z.B. Steine in den Weg legt.

\section{Warum Asymmetrisches VR}
Die größte Stärke und auch Nachteil der VR-Brille ist, dass sie einen von der Außenwelt abkapselt und dafür sorgt, dass man vollständig in die Spielwelt eintreten kann. Online-Multiplayer sorgt dafür, dass mehrere VR-Spieler miteinander spielen und interagieren können, jedoch gibt es für Interaktion und Kooperation zwischen VR-Spieler und PC-Spieler noch kein großes Angebot.

\section{Zielsetzung}
Das Ziel dieser Arbeit ist, die technische Implementierung und Optimierung von asymmetrischen VR-Spielen genauer zu Untersuchen und gefundene technische Probleme genauer zu erläutern.

