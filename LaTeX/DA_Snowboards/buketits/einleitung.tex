\chapter{Einleitung}
\label{cha:sa_Einleitung}

\section{Virtuelle Realität}
Die Virtuelle Realität (Virtual Reality, VR) ist eine computergenerierte virtuelle Umgebung, welche in Echtzeit berechnet wird. In dieser ist es möglich sich umzuschauen und je nach Anwendung kann man sich auch Bewegen und beliebig mit Objekten interagieren. Es gibt zwei verschiedene Möglichkeiten Virtual Reality zu erleben. Einerseits gibt es speziell angefertigte Räume, in welchen Großbildleinwände angebracht sind, andererseits gibt es auch Head-Mounted-Displays (VR Brillen, die man aufsetzt), welche hier im Fokus stehen werden. In der Virtuellen Realität gibt es eine breit gefächerte Reichweite an Spielen, in welchen man verschiedenste Situationen erleben kann. Anfangs waren Spiele für VR nur simple Testräume, in welchen man mit Objekten interagieren konnte. Mittlerweile ist es jedoch möglich riesige Spiele mit einer eigenen Geschichte zu erleben. Die Interaktionen, welche damals ein ganzes Spiel ausmachten, sind seit einiger Zeit nur noch die Basis, worauf neue Spiele aufbauen.

\subsection{Asymmetrische VR Spiele}
Die meisten VR-Spiele kapseln einen von der Außenwelt ab. Sobald man die VR-Brille aufsetzt, befindet man sich in einer neuen, digitalen Welt. Eine Zeit lang war die zwischenmenschliche Interaktion in der virtuellen Realität gar nicht möglich. Seit neuerem gibt es auch Spiele, welche unterstützen, dass VR-Spieler über das Internet miteinander spielen können (Online-Multiplayer). Somit braucht man jedoch zwei HMDs (VR-Brillen), zwei PCs und man kann in der Regel nicht nebeneinander spielen. Asymmetrische VR-Spiele versuchen, reguläre PC-Spiele mit VR-Spielen zu kombinieren. So können z.B. zwei Freunde mit einem PC und einer VR-Brille miteinander spielen. Das funktioniert z.B., indem die zwei Spieler sich in derselben Spielwelt befinden, jedoch unterschiedliche Charaktere steuern. Durch die verschiedenen Steuerungsmöglichkeiten (VR-Controller, Tastatur und Maus, Gamepad) ergibt sich oft automatisch eine bestimme Rollenverteilung. So könnte der PC-Spieler z.B. einen Menschen spielen, welcher durch ein Level manövrieren muss und zeitgleich steuert der VR-Spieler einen Riesen, welcher den Menschen behindern muss, indem er ihm z.B. Steine in den Weg legt.

\section{Zielsetzung}
Das Ziel dieser Arbeit ist, die technische Implementierung und Optimierung von asymmetrischen VR Spielen genauer zu Untersuchen und gefundene technische Probleme genauer zu erläutern.


\section{Warum Asymmetrisches VR}

Diplomarbeiten, Dissertationen und Bücher im
technisch-natur\-wissen\-schaft\-lichen Bereich werden
traditionell mithilfe des Textverarbeitungssystems \latex
\cite{Lamport94,Lamport95} gesetzt. Das hat gute Gründe, denn
\latex ist bzgl.\ der Qualität des Druckbilds, des Umgangs mit
mathematischen Elementen, Literaturverzeichnissen etc.\
unübertroffen und ist noch dazu frei verfügbar. Wer mit \latex
bereits vertraut ist, sollte es auch für die Diplomarbeit
unbedingt in Betracht ziehen, aber auch für den Anfänger sollte
sich die zusätzliche Mühe am Ende durchaus lohnen.

Für den professionellen elektronischen Buchsatz wurde früher
häufig \emph{Adobe Framemaker} verwendet, allerdings ist diese
Software teuer und komplex. Eine modernere Alternative dazu ist
\emph{Adobe InDesign}, wobei allerdings die Erstellung
mathematischer Elemente und die Verwaltung von Literaturverweisen
zur Zeit nur rudimentär unterstützt werden.%
\footnote{Angeblich werden aber für den (sehr sauberen) Schriftsatz 
in \emph{InDesign} ähnliche Algorithmen wie in \latex\ verwendet.}

Microsoft \emph{Word} gilt im Unterschied zu \latex, 
\emph{Framemaker} und \emph{InDesign} übrigens nicht als professionelle
Textverarbeitungssoftware, obwohl es immer häufiger auch von
großen Verlagen verwendet wird.%
\footnote{Siehe auch \url{http://latex.tugraz.at/mythen.php}.}
Das Schriftbild in \emph{Word}
lässt -- zumindest für das geschulte Auge -- einiges zu wünschen
übrig und das Erstellen von Büchern und ähnlich großen Dokumenten
wird nur unzureichend unterstützt. Allerdings ist \emph{Word} sehr
verbreitet, flexibel und vielen Benutzern zumindest oberflächlich
vertraut, sodass das Erlernen eines speziellen Werkzeugs wie
\latex\ ausschließlich für das Erstellen einer Diplomarbeit
manchen verständlicherweise zu mühevoll ist. Man sollte es daher
niemandem übel nehmen, wenn er/sie sich auch bei der Diplomarbeit
auf \emph{Word} verlässt. Im Endeffekt lässt sich mit etwas
Sorgfalt (und ein paar Tricks) auch damit ein durchaus akzeptables
Ergebnis erzielen. 
Für alle, die so denken, finden sich in
%Kap.~\ref{chap:Word} einige spezielle Hinweise zum Arbeiten mit
%\emph{Word}. 
Ansonsten sollten auch für \emph{Word}-Benutzer 
einige Teile dieses Dokuments von Interesse sein, insbesondere die
Abschnitte über Abbildungen und Tabellen und mathematische Elemente.

Übrigens, genau hier am Ende des Einleitungskapitels (und nicht
etwa in der Kurzfassung) ist der richtige Platz, um die
inhaltliche Gliederung der nachfolgenden Arbeit zu beschreiben.
Hier soll dargestellt werden, welche Teile (Kapitel) der Arbeit
welche Funktion haben und wie sie inhaltlich zusammenhängen. Auch
die Inhalte des \emph{Anhangs} -- sofern vorgesehen -- sollten hier
kurz beschrieben werden.

