\chapter{Überblick über die Materie}
\section{Virtuelle Realität}
Die Virtuelle Realität (Virtual Reality, VR) ist eine computergenerierte virtuelle Umgebung welche in Echtzeit berechnet wird. In dieser ist es möglich sich um zuschauen und je nach Anwendung kann man sich auch Bewegen und beliebig mit Objekten interagieren. Es gibt 2 verschiedene Möglichkeiten Virtual Reality zu erleben. Einerseits gibt es speziell angefertigte Räume, in welchen Großbildleinwände angebracht sind, andererseits gibt es auch Head-Mounted-Displays (VR Brillen,  die man aufsetzt), welche hier im Fokus stehen werden.


\subsection{Einführung in die Virtuelle Realität}
In der Virtuellen Realität gibt es eine breit gefächerte Reichweite an Spielen, in welchen man verschiedenste Situationen erleben kann. Anfangs waren Spiele für VR nur simple Testräume, in welchen man mit Objekten interagieren konnte. Mittlerweile ist es jedoch möglich riesige Spiele mit einer eigenen Geschichte zu erleben. Die Interaktionen, welche damals ein ganzes Spiel ausmachten, sind seit einiger Zeit nur noch die Basis, worauf neue Spiele aufbauen.

\subsection{Asymmetrische VR Spiele}
Die meisten VR Spiele kapseln einen von der Außenwelt ab. Sobald man die VR Brille aufsetzt, befindet man sich in einer neuen, digitalen Welt. Eine Zeit lang war die zwischenmenschliche Interaktion in der virtuellen Realität nicht möglich. Manche Spiele, welche etwas neuer sind, unterstützen, dass VR Spieler über das Internet miteinander spielen können(Online Multiplayer). Somit braucht man jedoch 2 HMDs(VR Brillen), 2 PCs und man kann in der Regel nicht nebeneinander spielen. Asymmetrische VR Spiele versuchen, reguläre PC Spiele mit VR Spielen zu kombinieren. So können z.B. zwei Freunde mit einem PC und einer VR Brille miteinander spielen. Das funktioniert z.B., indem die zwei Spieler sich in der selben Spielwelt befinden, jedoch unterschiedliche Charaktere steuern. Durch die verschiedenen Steuerungsmöglichkeiten (VR Controller, Tastatur und Maus, Gamepad) ergibt sich oft automatisch eine bestimme Rollenverteilung. So könnte der PC Spieler z.B. einen Menschen spielen, welcher durch ein Level manövrieren muss und zeitgleich steuert der VR Spieler einen Riesen, welcher den Menschen behindern muss, indem er ihm z.B. Steine in den Weg legt.  

\subsection{Verfügbare Brillen}
Es gibt mittlerweile viele verschiedene Anbieter für VR Brillen, welche verschieden gute HMDs in verschiedenen Preisreichweiten anbieten.
Das Steam Hardware Survey zeigt, dass momentane Marktführer Facebooks Meta (ehemals Oculus) mit der Oculus Quest und der Oculus Quest 2 im Low-Budget Bereich und Valve mit der Valve Index im High-Budget Bereich sind.

\begin{figure}[h]
	\centering
	\includegraphics{images/Steam_HardwareSurvey}
	\caption{\cite{_steam_hardware}}
\end{figure}

\section{Existierende Spiele}
Es gibt viele verschiedene VR Spiele und eine Breite Reichweite an Genres. Besonders relevant sind jedoch die asymmetrischen Spiele. Einige wichtige Beispiele sind hier aufgelistet.

\subsection{Keep Talking and Nobody Explodes}
Du bist alleine in einem Raum mit einer tickenden Zeitbombe gefangen. Deine Freunde haben das Handbuch, um die Bombe zu entschärfen, können sie aber nicht sehen, also müssen sie dir erklären, wie du sie entschärfen kannst.

\subsection{Davigo}
DAVIGO ist ein VR vs. PC "realitätsübergreifendes" Kampfspiel. Der VR-Spieler verkörpert einen Riesen und tritt gegen einen oder mehrere PC-Krieger in rasanten, explosiven Kämpfen an!

\subsection{Carly and the Reaperman}
Schlüpfen Sie in die Rolle der kürzlich verstorbenen Carly und rennen und springen Sie sich Ihren Weg durch unglaubliche Jump'n'Run-Herausforderungen - oder schlüpfen Sie in die Rolle des Reaperman und verändern Sie die Umgebung um Sie herum in Echtzeit, um Carly neue Orte zum Erkunden zu geben, gefährliche Hindernisse zu umgehen und ihr zu helfen, das Unmögliche zu erreichen.

\section{Erwartete Probleme}
Bei asymmetrischen VR Spielen kann man schon voraus sagen, dass es bei der Entwicklung einige Schwierigkeiten und Probleme geben wird. Wichtig ist dabei, dass man diese Probleme in verschiedene Bereiche unterteilt, um diese ab zu arbeiten. Es ist wichtig zu wissen, dass von nun an, um Verwirrung zu vermeiden 2 verschiedene Begriffe verwendet werden.Wenn von einem PC Spieler die Rede ist, wird von dem Spieler berichtet, welcher das Spiel mit Controller oder Tastatur und Maus erlebt. Der VR Spieler ist jener Spieler, welcher die VR Brille auf dem Kopf trägt und damit das Spiel spielt. Wenn man versucht asymmetrische VR Spiele in kleine Teile zu brechen merkt man, dass das Genre aus 2 Sub-Spielen besteht, welche ineinander eingearbeitet sind. Ein asymmetrisches VR Spiel besteht somit aus einem PC Spiel und einem VR Spiel. Diese zwei eigenen Teile arbeiten nun miteinander um eine Mehrspieler Erfahrung zu bieten.

\subsection{Motion Sickness}
Grundsätzlich steckt hinter dem Begriff Motion Sickness die Reisekrankheit. Die Reisekrankheit hat vermutlich jeder schon mal gespürt. Auslöser ist, dass Information von zwei oder mehrere Sinnen nicht miteinander übereinstimmen. Sollten die Augen eines Menschen die Information übermitteln, dass man sich nicht bewegt und der Gleichgewichtssinn nimmt trotzdem eine Bewegung wahr, kann somit passieren, dass man sich unwohl oder schlecht fühlt.

\subsection{Wenige Bilder pro Sekunde}
Zitat

\subsection{Platz und Bewegung}
Zitat

\subsection{Interaktionen}
Zitat