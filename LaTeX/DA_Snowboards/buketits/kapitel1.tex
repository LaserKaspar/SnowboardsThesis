\chapter{Überblick über die Materie}
\section{Virtuelle Realität}
Die Virtuelle Realität (Virtual Reality, VR) ist eine computergenerierte virtuelle Umgebung welche in Echtzeit berechnet wird. In dieser ist es möglich sich um zuschauen und je nach Anwendung kann man sich auch Bewegen und beliebig mit Objekten interagieren. Es gibt 2 verschiedene Möglichkeiten Virtual Reality zu erleben. Einerseits gibt es speziell angefertigte Räume, in welchen Großbildleinwände angebracht sind, andererseits gibt es auch Head-Mounted-Displays (VR Brillen,  die man aufsetzt), welche hier im Fokus stehen werden.


\subsection{Einführung in die Virtuelle Realität}
In der Virtuellen Realität gibt es eine breit gefächerte Reichweite an Spielen, in welchen man verschiedenste Situationen erleben kann. Anfangs waren Spiele für VR nur simple Testräume, in welchen man mit Objekten interagieren konnte. Mittlerweile ist es jedoch möglich riesige Spiele mit einer eigenen Geschichte zu erleben. Die Interaktionen, welche damals ein ganzes Spiel ausmachten, sind seit einiger Zeit nur noch die Basis, worauf neue Spiele aufbauen.

\subsection{Asymmetrische VR Spiele}
Die meisten VR Spiele kapseln einen von der Außenwelt ab. Sobald man die VR Brille aufsetzt befindet man sich in einer neuen, digitalen Welt. Eine Zeit lang war die zwischenmenschliche Interaktion in der Virtuellen Realität nicht möglich. Manche Spiele, welche etwas neuer sind, unterstützen dass VR Spieler über das Internet miteinander spielen können(Online Multiplayer). Somit braucht man jedoch 2 HMDs(VR Brillen), 2 PCs und man kann in der Regel nicht nebeneinander spielen. Asymmetrische VR Spiele versuchen, reguläre PC Spiele mit VR Spielen zu kombinieren. So können z.B. zwei Freunde mit einem PC und einer VR Brille miteinander spielen. Das funktioniert z.B., indem die zwei Spieler sich in der selben Spielwelt befinden, jedoch unterschiedliche Charaktere steuern. Durch die verschiedenen Steuerungsmöglichkeiten (VR Controller, Tastatur und Maus, Gamepad) ergibt sich oft automatisch eine bestimme Rollenverteilung. So könnte der PC Spieler z.B. einen Menschen spielen, welcher durch ein Level manövrieren muss und zeitgleich steuert der VR Spieler einen Riesen, welcher den Menschen behindern muss, indem er ihm z.B. Steine in den Weg legt.  

\subsection{Verfügbare Brillen}
Es gibt mittlerweile viele verschiedene Anbieter für VR Brillen, welche verschieden gute HMDs in verschiedenen Preisreichweiten anbieten.
Das Steam Hardware Survey zeigt, dass momentane Marktführer Facebooks Meta (ehemals Oculus) mit der Oculus Quest und der Oculus Quest 2 im Low-Budget Bereich und Valve mit der Valve Index im High-Budget Bereich sind.

\vspace{0.75cm}
\includegraphics{images/Steam_HardwareSurvey}

\section{Existierende Spiele}
Es gibt viele verschiedene VR Spiele und eine Breite Reichweite an Genres. Besonders relevant sind jedoch die asymmetrischen Spiele. Einige wichtige Beispiele sind hier aufgelistet.

\subsection{Keep Talking and Nobody Explodes}
YOU’RE ALONE
IN A ROOM WITH A BOMB.
Your friends have the info you need to defuse it.

But there’s a catch. They can’t see the bomb. So everyone will need to talk it out–fast!

\subsection{Davigo}
DAVIGO is a VR vs. PC "cross-reality" battle game. The VR player embodies a giant and faces off against one or more PC warriors in fast-paced, explosive combat!

\subsection{Carly and the Reaperman}
Step into the recently-dead shoes of Carly as you run and jump your way through incredible platforming challenges — or take up the mantle of the Reaperman and change the environment around you in real-time to give Carly new places to explore, avoid dangerous obstacles, and help her to achieve the impossible.

\section{Erwartete Probleme}
Dieses Dokument ist als vorwiegend technische Starthilfe für das
Erstellen einer Masterarbeit (oder Bachelorarbeit) mit \latex
gedacht und ist die Weiterentwicklung einer früheren
Vorlage\footnote{Nicht mehr verfügbar.} für das Arbeiten mit
Microsoft \emph{Word}. Während ursprünglich daran gedacht war, die
bestehende Vorlage einfach in \latex zu übernehmen, wurde rasch
klar, dass allein aufgrund der großen Unterschiede zum Arbeiten
mit \emph{Word} ein gänzlich anderer Ansatz notwendig wurde. Dazu
kamen zahlreiche Erfahrungen mit Diplomarbeiten in den
nachfolgenden Jahren, die zu einigen zusätzlichen Hinweisen Anlass gaben.

\subsection{Wenige Bilder pro Sekunde}
Zitat \cite{bobsch 123 bobsch}

\subsection{Bewegungskrankheit}
Zitat \cite{bobsch 123 bobsch}

\subsection{Platz und Bewegung}
Zitat \cite{bobsch 123 bobsch}

\subsection{Interaktionen}
Zitat \cite{bobsch 123 bobsch}