\chapter{Überblick über die Virtuelle Realität}


\subsection{Verfügbare Brillen}
Es gibt mittlerweile viele verschiedene Anbieter für VR-Brillen, welche verschieden gute HMDs in verschiedenen Preisreichweiten anbieten.
Das Steam Hardware Survey zeigt, dass momentane Marktführer Facebooks Meta (ehemals Oculus) mit der Oculus Quest und der Oculus Quest 2 im Low-Budget Bereich und Valve mit der Valve Index im High-Budget Bereich sind.


\begin{figure}[H]
	\centering
	\includegraphics{images/Steam_HardwareSurvey}
	\caption{\cite{_steam_hardware}}
\end{figure}

\section{Existierende Spiele}
Es gibt viele verschiedene VR-Spiele und eine Breite Reichweite an Genres. Besonders relevant sind jedoch die asymmetrischen Spiele. Einige wichtige Beispiele sind hier aufgelistet.

\subsection{Keep Talking and Nobody Explodes}
Du bist alleine in einem Raum mit einer tickenden Zeitbombe gefangen. Deine Freunde haben das Handbuch, um die Bombe zu entschärfen, können sie aber nicht sehen, also müssen sie dir erklären, wie du sie entschärfen kannst.
\cite{_steam_keeptalking}

\subsection{Davigo}
DAVIGO ist ein VR vs. PC "realitätsübergreifendes" Kampfspiel. Der VR-Spieler verkörpert einen Riesen und tritt gegen einen oder mehrere PC-Krieger in rasanten, explosiven Kämpfen an!

\begin{figure}[H]
	\centering
	\includegraphics[width=10cm]{images/DavigoScreenshot}
	\caption{\cite{_steam_davigo}}
\end{figure}

\subsection{Carly and the Reaperman}
Schlüpfen Sie in die Rolle der kürzlich verstorbenen Carly und rennen und springen Sie sich Ihren Weg durch unglaubliche Jump'n'Run-Herausforderungen - oder schlüpfen Sie in die Rolle des Reaperman und verändern Sie die Umgebung um Sie herum in Echtzeit, um Carly neue Orte zum Erkunden zu geben, gefährliche Hindernisse zu umgehen und ihr zu helfen, das Unmögliche zu erreichen.
\cite{_steam_carly}

\section{Erwartete Probleme}
Bei asymmetrischen VR-Spielen kann man schon voraussagen, dass es bei der Entwicklung einige Schwierigkeiten und Probleme geben wird. Wichtig ist dabei, dass man diese Probleme in verschiedene Bereiche unterteilt, um diese abzuarbeiten. Es ist wichtig zu wissen, dass von nun an, um Verwirrung zu vermeiden zwei verschiedene Begriffe verwendet werden. Wenn von einem PC Spieler die Rede ist, wird von dem Spieler berichtet, welcher das Spiel mit Controller oder Tastatur und Maus erlebt. Der VR-Spieler ist jener Spieler, welcher die VR-Brille auf dem Kopf trägt und damit das Spiel spielt. 

\subsection{Motion Sickness und VR-Krankheit} \label{simon_motionsickness}
Grundsätzlich steckt hinter dem Begriff Motion Sickness die Reisekrankheit. Die Reisekrankheit hat vermutlich jeder schon mal gespürt. Auslöser ist, dass Information von zwei oder mehreren Sinnen nicht miteinander übereinstimmen. Sollten die Augen eines Menschen die Information übermitteln, dass man sich nicht bewegt und der Gleichgewichtssinn nimmt trotzdem eine Bewegung wahr, kann somit passieren, dass man sich unwohl oder schlecht fühlt. Die VR-Krankheit ist jedoch eine Unterform der Reisekrankheit (so wie die Seekrankheit).
Die VR-Krankheit (engl. Virtual Reality Sickness) ist eine spezielle Form von Übelkeit, welche auftreten kann, wenn man sich in einer virtuellen Umgebung befindet. Die Hautsymptome sind Kopfschmerzen, Apathie, Müdigkeit und Unwohlsein. Bewegungsinstabilität und Stolpern sind auch mögliche Symptome. In weniger häufigen Fällen wurde Übelkeit und Erbrechen beobachtet. Die Symptome der VR-Krankheit ähneln denen der Seekrankheit sehr, jedoch gibt es einen entscheidenden Unterschied. Bei der Seekrankheit ist der Mensch physisch in Bewegung und seine Sinne geben zusätzlich zu visuellen Informationen auch noch Informationen über seine Bewegung. Im Gegensatz dazu tritt die VR-Krankheit ohne von außen induzierter Bewegung auf. Das heißt, dass die Übelkeit nur durch visuelle und akustische Änderung auftritt. Die VR-Krankheit ensteht vermutlich dadurch, dass die körperliche Selbstwahrnehmung (Propriozeption) von der Wahrnehmung des visuellen Cortex abweicht.

\vspace{1cm}
Mögliche Auslöser in VR:
\begin{itemize}
	\item Abrupte Bewegungen,
	\item Schnelle Bewegungen im Spiel,
	\item Bewegung und Drehung des Spielers durch einen Analogstick.
	\item Schnelle Positionsänderung in der Spielwelt wie z.B. Teleportation.
	\item Kamerafahrten (z.B. Achterbahnen)
\end{itemize}

(Quelle für gesamten Themenbereich)
\cite{_motionsickness}
\cite{_vr_quovadis}


\subsection{Was ist die Frame Rate?}
Die meisten haben schon einmal von der Bildrate oder dem Begriff (engl.)"Frames per second" (FPS) gehört. Die wissenschaftliche Definition der Bildrate sagt, dass es sich um die Anzahl der Bilder handelt, die in einer einzigen Sekunde gezeigt werden. Dabei ist egal in welchem Medium. 120 FPS bedeuten also, dass man in jeder Sekunde 120 Bilder sieht. Dadurch, dass in einem kurzen Zeitraum so viele Bilder gezeigt werden, sorgt es für eine flüssige Bewegung.
(vergleiche \cite{_vr_linde})

\subsection{Wenige Bilder pro Sekunde}
Wie viele FPS reichen?
Die Regel lautet: Je höher, desto besser. Alles unter 60 FPS führt wahrscheinlich zu Übelkeit, Desorientierung und allgemeinem Unbehagen, dass zu Motion Sickness führen kann. 120 FPS pro Auge sind am besten für ein gutes VR-Erlebnis geeignet, und alles, was darüber liegt, ist umso besser.

(vergleiche \cite{_vr_linde})

\subsection{Platz und Bewegung}
Je nach VR-Brille kann der Platzverbrauch des HMDs variieren. Manche Brillen brauchen zusätzlich zu den Controllern und des Head Mounted Displays auch noch Lighthouses(Stationäre Sensoren/Kameras, welche die Bewegung des Spielers nachvollziehen). Dieses können zusätzlich etwas Platz verbrauchen und den Spielraum etwas einschränken. Schnelle und hektische Bewegungen, welche durch ein gutes Spielerlebnis entstehen, können so oftmals darin enden einen Schrank, Schreibtisch oder Bildschirm zu beschädigen. Somit ist es also auch wichtig ein Spiel möglich platzsparend zu bauen um Spielern mit wenig Platz auch eine Möglichkeit zu bieten, das Spiel zu spielen.

\subsection{Interaktionen}
Gut ausgearbeitete Interaktionen sind eine wichtige Grundlage für ein VR-Spiel. Flüssige und interessante Interaktion mit der Spielwelt und den Objekten darin, kann ein Spiel um einen wesentlichen Bestandteil verbessern und ansprechender machen. Somit erhöhen bessere und auch realistischere Interaktionen die Immersion des Spielers.