\chapter{PC und VR Welt - Umsetzung}
\section{Level Aufbau}
Jedes Level in Tricks ´n´ Treats ist ähnlich aufgebaut. 
In der Mitte befindet sich der Berg, auf welchem die Snowboardenden fahren und rund herum befindet sich der Raum des VR Spielers. Die PC Spieler starten oben auf dem Berggipfel und fahren den Hang bis zum Ende der Piste hinab. Der VR Spieler kann mit einem Joystick auf seinem Controller den Berg rotieren um sich besser auf der Strecke zu orientieren und die Spieler in Sicht zu behalten. Währenddessen kann er hinter sich bestimmte Zauberkugeln aufheben und auf den PC Spielern anwenden.

\begin{figure}[h]
	\centering
	\includegraphics[width=13cm]{images/buketits_room}
	\caption{Editor View eines unfertigen Levels, Die rote Kapsel stellt den VR-Spieler dar}
\end{figure}

\section{VR Spieler}
Der VR Spieler besteht aus mehreren System welche zusammen arbeiten. In Tricks ´n´ Treats sind die Kamera und Hände mit Hilfe der SteamVR Integration für Unity gemacht. Der VR Charakter verwendet die physikalischen Hände und die "`Player"' Komponente, welche beide von SteamVR bereitgestellt werden. Ein großes Problem unseres Spiels (welches im nächsten Kapitel erklärt und gelöst wird) ist, dass der VR Charakter, welcher physikalische Hände besitzt nicht richtig skaliert werden kann. Sobald dieser skaliert wird, funktionieren die Hand-Collider nicht mehr planmäßig.

\subsection{Zauber und Interaktionssystem}
Für die Basis des Interaktionssystems in Tricks ´n´ Treats, dient das System, das SteamVR für Unity bereitstellt. Auf jedem Objekt, mit welchem der VR Spielende interagieren kann, befindet sich das "`Interactable"' Script, welches von SteamVR bereitgestellt ist. Unser eigenes System für die Zauberkugeln hakt sich in dieses Script ein und kann somit bestimmte Eigenschaften leichter implementieren. Somit haben wir also eine Hervorhebung für Objekte geschaffen mit denen interagiert werden und können gleichzeitig auf die Eingabe des VR Controllers reagieren. Wenn eine Zauberkugel mit einem Objekt kollidiert (während man sie in der Hand hält), wird dann mit dem, von Steam vorgegebenen, "`Velocity Estimator"' die Aufprallgeschwindigkeit ausgerechnet und je nach Geschwindigkeit ein Zauberspruch gewirkt.

\subsection{VR Umgebung}
t

\section{PC Spieler}
t

\subsection{Spieler Größe}
Zitat

\subsection{Fähigkeiten}
Zitat

\section{Lösung der erwarteten Probleme}
t

\subsection{Wenige Bilder pro Sekunde}
Zitat

\subsection{Bewegungskrankheit}
Zitat

\subsection{Platz und Bewegung}
Zitat

\subsection{Interaktionen}
Zitat