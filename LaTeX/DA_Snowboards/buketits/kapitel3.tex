\chapter{PC und VR Welt - Umsetzung}
\section{Level Aufbau}
Jedes Level in Tricks ´n´ Treats ist ähnlich aufgebaut. 
In der Mitte befindet sich der Berg, auf welchem die Snowboardenden fahren und rund herum befindet sich der Raum des VR Spielers. Die PC Spieler starten oben auf dem Berggipfel und fahren den Hang bis zum Ende der Piste hinab. Der VR Spieler kann mit einem Joystick auf seinem Controller den Berg rotieren um sich besser auf der Strecke zu orientieren und die Spieler in Sicht zu behalten. Währenddessen kann er hinter sich bestimmte Zauberkugeln aufheben und auf den PC Spielern anwenden.

\begin{figure}[h]
	\centering
	\includegraphics[width=13cm]{images/buketits_room}
	\caption{Editor View eines unfertigen Levels, Die rote Kapsel stellt den VR-Spieler dar}
\end{figure}

\section{VR Spieler}
Der VR Spieler besteht aus mehreren System welche zusammen arbeiten. In Tricks ´n´ Treats sind die Kamera und Hände mit Hilfe der SteamVR Integration für Unity gemacht. Der VR Charakter verwendet die physikalischen Hände und die "`Player"' Komponente, welche beide von SteamVR bereitgestellt werden. Ein großes Problem unseres Spiels (welches im nächsten Kapitel erklärt und gelöst wird) ist, dass der VR Charakter, welcher physikalische Hände besitzt nicht richtig skaliert werden kann. Sobald dieser skaliert wird, funktionieren die Hand-Collider nicht mehr planmäßig.

\subsection{Spell System}
Zitat

\subsection{VR Umgebung}
Zitat


\section{PC Spieler}
Dieses Dokument ist als vorwiegend technische Starthilfe für das
Erstellen einer Masterarbeit (oder Bachelorarbeit) mit \latex
gedacht und ist die Weiterentwicklung einer früheren
Vorlage\footnote{Nicht mehr verfügbar.} für das Arbeiten mit
Microsoft \emph{Word}. Während ursprünglich daran gedacht war, die
bestehende Vorlage einfach in \latex zu übernehmen, wurde rasch
klar, dass allein aufgrund der großen Unterschiede zum Arbeiten
mit \emph{Word} ein gänzlich anderer Ansatz notwendig wurde. Dazu
kamen zahlreiche Erfahrungen mit Diplomarbeiten in den
nachfolgenden Jahren, die zu einigen zusätzlichen Hinweisen Anlass gaben.

\subsection{Spieler Größe}
Zitat

\subsection{Fähigkeiten}
Zitat

\section{Erwartete Probleme}
Dieses Dokument ist als vorwiegend technische Starthilfe für das
Erstellen einer Masterarbeit (oder Bachelorarbeit) mit \latex
gedacht und ist die Weiterentwicklung einer früheren
Vorlage\footnote{Nicht mehr verfügbar.} für das Arbeiten mit
Microsoft \emph{Word}. Während ursprünglich daran gedacht war, die
bestehende Vorlage einfach in \latex zu übernehmen, wurde rasch
klar, dass allein aufgrund der großen Unterschiede zum Arbeiten
mit \emph{Word} ein gänzlich anderer Ansatz notwendig wurde. Dazu
kamen zahlreiche Erfahrungen mit Diplomarbeiten in den
nachfolgenden Jahren, die zu einigen zusätzlichen Hinweisen Anlass gaben.

\subsection{Wenige Bilder pro Sekunde}
Zitat

\subsection{Bewegungskrankheit}
Zitat

\subsection{Platz und Bewegung}
Zitat

\subsection{Interaktionen}
Zitat