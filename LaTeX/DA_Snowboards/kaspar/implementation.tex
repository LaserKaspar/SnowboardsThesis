\chapter{Implementation}

\section{Konzept}

[Konzept vom Spiel, kommt an den Anfang der gesamten Arbeit]

\section{Was muss bei VR-Game-Design beachtet werden?}

VR-Spiele bringen einige Limitationen mit sich, die es zu überwinden gilt. Bei der ersten Limitation handelt es sich um die Größe des Play-Spaces, also jenem Physischem Raum in dem sich VR-Spielende aufhalten. Das zweite Problem ist die Reisekrankheit (Motion-Sickness, Eine Körperliche Reaktion auf ungewöhnliche/unzusammenhängende Bewegung\cite[S. 533]{_art_of_gamedesign}), vor allem bei Party-Spielen muss ihr Einfluss möglichst gering sein, da das Spiel von einer breiten Masse gespielt werden können soll. Die Reisekrankheit limitiert die Bewegungsmöglichkeiten der Spielenden und diese wiederum die Größe und Gestaltung der virtuellen Umgebung.

Außerdem muss die Größe des Spielenden (Um sicherzustellen, dass alles womit Spielende interagieren sollen erreichbar ist), als auch Ängste, wie zum Beispiel "`Klaustrophobie"', beachtet werden. Im Falle von einem Party Spiel spielt auch das technische Know-How eine Rolle, da man sehr schnell in das Spiel eintauchen können soll.

\section{Stimmung des Spieles}
Das im Zuge dieser Arbeit entwickelte Spiel soll eine bestimmte Stimmung erzeugen. Am Anfang des Design-Prozesses muss diese Stimmung festgelegt werden.

\subsection{Core Pillars}

\subsubsection{Fast but Strategic}
Das Ziel des Spiels soll es nicht sein, möglichst schnell Spells zu aktivieren. Es muss etwas Strategie dahinter sein, damit Spielende gefordert werden und eine wertvolle Entscheidung treffen können. Da es sich um ein Party-Spiel und kein Strategie-Spiel handelt, sollte es möglich sein, schnelle Entscheidungen zu treffen. Um das zu ermöglichen, muss das Spell-System vergebend sein, und muss sich auf die aktuelle Spielsituation anpassen können\ref{_rubberbanding}. 

\subsubsection{Teamwork \& Kommunikation}
Außerdem muss Strategie und Absprache zwischen den PC-Spielenden angeregt werden, um weiter das Teamwork der Gruppe zu fördern. Wie am Anfang der Arbeit bereits definiert, braucht Teamwork ein gemeinsames Ziel. Das Ende der Strecke kommt hier recht schnell in den Sinn, allerdings ist das alleine nicht genug, um Kommunikation zwischen den Spielenden zu erzeugen, da es zu weit in der Zukunft liegt, und sich nicht verändert. Eine Lösung, und mögliche Lösungsansätze, für dieses Problem werden in Kapitel \ref{_teamwork_erzeugen} besprochen.

\subsubsection{God-Like}
Vor allem der VR-Spieler soll sich mächtig fühlen. Erreicht werden soll dieses Gefühl durch den deutlichen Größenunterschied zwischen Magier und Snowboarder und Combos, also der Möglichkeit mehrere Snowboarder gleichzeitig von der Strecke zu schmeißen. Es muss allerdings beachtet werden, dass das alleinige "`auf die Strecke schmeißen"' keinen Spaß machen wird, wenn davor nicht etwas dafür getan werden musste. Hier kommt die "`Lens of Challange"'\cite{_art_of_gamedesign} ins Spiel, man sollte besonders darauf achten, dass die Schwierigkeit das Ziel zu erreichen mit dem Skill des Spielenden übereinstimmt, außerdem ist es wichtig die Interaktion nicht zu einseitig zu gestalten. Das Gefühl der Macht kann nur erzeugt werden, wenn Spielende denken, es war ihre alleinige Leistung.

\subsection{Target Experiences}

Da es sich bei dem Projekt um ein asymmetrisches Spiel handelt, wurden für die PC- und die VR-Spielenden jeweils andere Target Experiences festgelegt.

\subsubsection{Snowboarder (PC)}
Für die Snowboarder sind folgende zwei Experiences am wichtigsten: die (Snowboarding-) Action und das Teamwork der Gruppe. Mithilfe der "`Lens of Essential Experience"'\cite[S. 55]{_art_of_gamedesign} kann versucht werden, diese Experiences zu erzeugen.

Um das Gefühl "`Snowboarden"' akkurat wiederzugeben wurden 3 wichtige Komponenten gefunden: Tricks, Stürzen und Beharrlichkeit. Man könnte auch sagen, dass die Kälte ein Teil des Snowboardens darstellt, jedoch wurde dagegen entschieden, da es dem Core Pillar "`God-Like"' wiederspricht, wenn sich die Snowboarder über die Körpertemperatur ihrer Charaktere Gedanken machen müssen.

Eine weitere wichtige Experience der Snowboarder ist das Teamwork, sogar so wichtig, dass es, wie oben erwähnt, zu einem der Core Pillars gemacht wurde. Wie Teamwork erzeugt werden soll, wird in Kapitel \ref{_teamwork_erzeugen} behandelt.

\subsubsection{Magier (VR)}
Für den Magier sind die folgenden zwei Experiences am wichtigsen: Macht und Multitasking. 

Diese, im Konkreten Göttliche/Magische Kraft, soll vor allem durch das Zaubern, also den Spells (\ref{_spell_design}]), und die Aneinanderreihung und taktische ("`Fast but Strategic"') Platzierung dieser erzeugt werden. Sie wird auch über den visuellen Größenunterschied der Snowboarder und dem Magier dargestellt.

Multitasking ist vor allem im Bereich der Interaktionen wichtig, der Magier sollte immer schon bevor er einen Spell platziert hat, darüber nachdenken, was der nächste Zug sein wird. Um des zu erreichen, sollten Spells nicht immer sofort verfügbar sein. Ziel ist es dadurch mehr Strategie in das Spiel zu bringen.

\section{Zusammentreffen beider Welten (PC + VR)}

\subsection{Streckendesign}
Der erste Punkt, und auch Grundstein für das restliche Design, ist das Konzept der Stecke. Es hat sowohl Einfluss auf die PC-Spielenden als auch auf den VR-Spielenden, da es den Ort darstellt, in dem beide miteinander Interagieren.

\subsubsection{Limitationen}
Der VR-Spielende muss die Möglichkeit haben, zu jedem gegebenen Zeitpunkt, mit einem Großteil der Strecke zu interagieren (Fallen platzieren, etc.). Außerdem ist es wichtig, dass sich Spells immer in Reichweite befinden. Die Strecke muss jedoch auch (aufgrund anderer technischer Limitationen) zu jedem Zeitpunkt komplett geladen und sichtbar sein.

Das Streckendesign muss es zulassen, die Strecke so anzupassen, dass eine Fahrt von oben nach unten ca. 3 Minuten dauert. Des Weiteren muss die Breite der Strecke genug Platz, um Hindernissen auszuweichen, und den Spielenden eine gewissen Toleranz für Fehler beim Lenken der Snowboarder lassen.

Zur Auswahl stehen drei verschiedene Streckendesigns, deren vor und Nachteile behandelt werden und begründet wird, warum schlussendlich für Streckendesign Nr. 3 entschieden wurde. Alle drei Designs versuchen die Bewegung des VR-Spielenden zu vermeiden, um das Gefühl der Reisekrankheit zu mindern.

\subsubsection{Designoption 1 - Spirale}
Die Snowboarder fahren, in diesem Design, in einer Spirale um den VR-Spielenden herum. Es ermöglicht eine freie Strecke für die Snowboarder, jedoch ist wenig Platz für VR-Spielende. Dieses Design ist auch sehr unrealistisch, da es sich hier nicht um eine klassische Ski-Strecke handeln kann, da Teile "`Schweben". Dieser Effekt wird weiters verstärkt, da es sich für die Spielenden so anfühlt, als wären sie in einer Wasserrutsche. Das kämpft alles mit der Experience des Snowboardens.

\subsubsection{Designoption 2 - Kristallkugel}
Dieses Design bringt auf der Seite der Snowboarder und des VR-Spielenden deutliche Verbesserungen. Da VR-Spielende durch eine Kristallkugel auf die Snowboarder herabschaut, kann die Strecke unglaublich frei sein, es kann harte Kurven und Richtungswechsel geben. Der VR-Spielende, hat die Möglichkeit, einen eigenen Raum zu bekommen, in dem der Zugang zu den Spells einfach ist. Der größte Nachteil ist hier die Technik, die im Hintergrund gebraucht wird. Technisch ist es sehr schwer umzusetzen, da die Welten des VR-Spielenden und der Snowboarder voneinander getrennt und über die Kristallkugel wieder zusammengebracht werden müssen.

\subsubsection{Designoption 3 - Berg}
Die Idee des Berges wurde am Anfang etwas vernachlässigt, da es sehr eingeschränkt erschien. Jedoch bringt der Berg neben der technisch um einiges leichteren Umsetzung wieder etwas Realismus und eine schöne Atmosphäre ins Spiel, die die Experience des Snowboardens verstärken kann. Auf diesem, aus der Sicht des VR-Spielenden, kleinen Berg, in der Mitte des VR-Play-Spaces, können die Snowboarder spiralförmig hinunterfahren. Wie bei der Spirale hat man hier aber auch das Problem, dass die Snowboarder permanent in die Kurve lenken müssen, dieses Problem wird in Kapitel \ref{_playercontroller} weiter behandelt. Der Berg kann automatisch mit den Snowboardern mitgedreht werden, damit sie immer im Sichtfeld des Magiers sind. Der Magier soll aber auch eine Möglichkeit haben diese Drehung zu überschreiben, damit auch Teile der Strecke die außerhalb des interagierbaren Bereiches liegen beeinflussen kann.

\subsection{Interaktionsdesign}
Da der Wettbewerb zwischen PC und VR asymmetrisch ist, müssen beide Parteien die jeweils andere Seite beeinflussen können, um es interessant zu machen. Der Magier muss durch Spells und andere Tools die Snowboarder behindern können und die Snowboarder müssen Einfluss auf die Welt des VR-Spieleden haben.

\subsubsection{Spell Design\label{_spell_design}}
Um dem VR-Spielenden das Gefühl von Zaubern zu übermitteln, braucht es neben der Möglichkeit zu Zaubern auch gute Interaktionen. Das Spell-System muss dem Magier die Möglichkeit geben zu Multitasken und Kombos zu machen, damit das Gefühlt der Macht aufkommt. Da der VR-Spielende zwei Hände hat, ist es naheliegend, dass beide Hände an unterschiedlichen Spells arbeiten können. Um eine gewisse Vielfalt in die möglichen Spells zu bringen, wurden drei Spell-Kategorien eingeführt.

\paragraph{Instant-Spell}
Sollen einen sofortigen Effekt auf die PC-Spielenden haben. Instant-Spells sollen sich besonders gut für die Kombination mit anderen Spell-Typen eignen. Sie dürfen auch in die Masse der Snowboarder platziert werden, da sie niemals sofort töten dürfen. Ein Beispiel hierfür wäre eine Eisfläche, die die Snowboarder zum Rutschen bringt.

\paragraph{Short-Term-Spell}
Dieser Spell-Typ braucht eine gewisse Zeit, um sich zu aktivieren. Somit müssen Spielende sich vorher etwas genauer überlegen, wo sie am Besten zu platzieren sind, und die Snowboarder haben etwas Zeit um auszuweichen. Short-Term-Spells dürfen auch tötlich enden, ein Beispiel hierfür wäre eine Bombe, die die Snowboarder in die Luft schleudert. Alle Short-Term-Spells müssen nach einer gewissen Zeit wieder verschwinden.

\paragraph{Long-Term-Spell}
Das Ziel von Long-Term-Spells ist es den Spielenden die Möglichkeit zu geben, etwas strategischer zu Spielen. Sie können nur begrenzt, und an vordefinierten orten aktiviert werden. Ziel ist es Spielende dazu zu bewegen, zu Beginn einer Runde abzuwägen einen Long-Term-Spell einzusetzen, und dafür weniger Instant- und Short-Term-Spells, oder darauf zu verzichten um flexibler zu sein. Beispiele hierfür wären das Absperren von gewissen Streckenteilen, oder das zum Einsturz bringen von Bäumen oder Geröll auf der Strecke.

\paragraph{Balancing-Cooldown}
Die Anzahl der Spells, welche der Magier verwenden kann, muss limitiert sein. Der einfachste Weg, um Spells zu limitieren, ist ein Cooldown, welcher verhindert, dass zwei Spells direkt nacheinander platziert werden können. Es wird jedoch bei diesen System schnell dazu kommen, dass VR-Spielende immer den "`Stärksten"'' Spell nehmen, und es dadurch schwer zu balancen ist.

\paragraph{Balancing-Mana\label{_mana}}
Bevor ein Spell platziert werden kann, muss er in einen Mana-Kessel getaucht werden, um ihn zu aktivieren. Dieser Mana-Kessel kann nur eine begrenzte Menge an Mana halten und befüllt sich über den Verlauf des Spieles von selbst. Ein Mana-Kessel bietet gleich mehrere Vorteile: Balancing und Physische-Interaktion. Mana kann dabei helfen Spells untereinander zu balancen, indem ihre Kosten erhöht oder verringert werden. Außerdem fügt der Kessel einen Zwischenschritt hinzu, das macht die Interaktion komplexer und interessanter. Sobald ein Spell einmal aufgeladen ist, ist man dazu gezwungen ist ihn auch zu platzieren, da er sonst seine Ladung wieder verliert.

Die Anzahl der Spielenden, die noch im Rennen sind, könnte sich auf die Mana-Regeneration auswirken, so entsteht ein negativer Feedback Loop, es wird für den Magier schwieriger zu gewinnen, wenn er kurz davor ist den letzten Snowboarder zu Fall zu bringen. Das erzeugt einen weiteren Pacing Spike, wie bei einem Bosskampf, am Ende einer Runde und hat einen automatischen Balancing\cite[S. 296]{_game_design_workshop}-Effekt.

\paragraph{Balancing-Interaction}
Ein weiterer Weg, wie Spells gebalanced werden könnten, wäre durch die Dauer / Komplexität der Interaktion. Ein großer Vorteil hierbei wäre, dass es den Spielenden etwas natürlicher vorkommt, als ein künstlicher Cooldown, dadurch sollte es auch weniger frustrierend sein, wenn gerade kein Spell bereit steht, da der Magier selbst dafür verantwortlich war. [Ich denke das heißt "`internal attribution"', gibts dafür Quellen?]

\subsection{Playercontroller \& Obstacles\label{_playercontroller}}
\subsubsection{Tricks}
Tricks können zum einen dazu genutzt werden, ein Risiko einzugehen, um die Respawn-Zeit zu verkürzen oder als eine Movement-Mechanic, die dabei hilft Obstacles auszuweichen. In das Spiel wird vermutlich eine Mischung beider Mechanics implementiert.

\subsubsection{Respawns}
Das Wiedereintreten von ausgeschiedenen Spielenden ist eine wichtige Mechanik, um das Gefühl des Snowboardens einzufangen und um den Party-Aspekt des Spieles zu fördern. Es gibt verschiedene Möglichkeiten zu regeln, wann Spielende wieder in die Runde eintreten.

\paragraph{Timer}
Eine sehr simple Möglichkeit wäre, Ausgeschiedene nach einer gewissen Zeit automatisch wieder dem Rennen hinzuzufügen. Das hätte den Vorteil, dass es zum einen sehr leicht umsetzbar ist und zum anderen den Spielenden eine kleine Pause verschafft und der Timer sich durch andere Faktoren beeinflussen ließe, was wiederum beim Balancing helfen kann.

\paragraph{Sekundäres Ziel für ausgeschiedene Spielende}
Ein weiterer Weg, die die ausgeschiedenen Spielenden besser ins Spiel inkludiert, wäre ein Sekundäres Ziel. Sie müssten, um wieder ins Rennen zu kommen, eine kleine Aufgabe lösen, bestimmte Knöpfe schnell hintereinander drücken (verbunden mit viel Game-Juice, um die durch die Spielenden wahrgenommene Intensität des Gameplays zu steigern), oder sie könnten den VR-Spielenden behindern mehr dazu in \ref{_rubberbanding}.

\subsubsection{Automatische Lenkhilfe}
Da die Spielenden bei dem Berg- oder Spiralen-Streckendesign permanent im Kreis lenken müssen, wäre es wichtig, dem entgegenzuwirken, um das Gefühl des "`im Kreis"'-fahrens zumindest etwas zu mindern. Hier wäre eine Automatische Lenkhilfe, die den Snowboarder automatisch nach unten fahren lässt, die einfachste Methode. Verbunden mit einer Lenkung, welche Relativ zur Kamera funktioniert, haben Spielende eine angemessene Lenk-Freiheit in beide Richtungen.

\subsubsection{Obstacles}
Obstacles können sowohl von der Strecke vorgegeben sein als auch durch den Magier platziert werden. Sie zwingen die Snowboarder Entscheidungen zu treffen, insbesondere im Falle von Positive Obstacles. Normale Obstacles sollten von den Snowboardern vermieden werden.

\paragraph{Positive Obstacles}
Positive Obstacles sind ähnlich wie Pickups in z.B. Mario Cart, nur sind sie in die Welt integriert. Sie haben einen Positiven Effekt auf die Gruppe, jedoch werden sie von VR-Spielenden eher angegriffen/beschützt.

\subsubsection{Spieler führen}
Wenn man Spielende für riskantes Fahren belohnt, entsteht eine Meaningful Decision. Es entsteht eine Risk-Reward Situation, in der sich schlechte Teammitglieder von den Guten abspalten und andere Wege nehmen. Mithilfe von Rampen und Slalom-Bögen, die die Respawn-Zeit verkürzen, kann man riskantes Fahren fördern.

\subsection{Teamwork erzeugen - Krone \& andere Lösungsansätze\label{_teamwork_erzeugen}}
Um die Snowboard-Gruppe dazu zu bewegen zu kooperieren, müssen sie ein gemeinsames Ziel haben. Dieses Ziel ist schlussendlich das Ende der Strecke jedoch braucht es auch ein kurzzeitiges Ziel wie zum Beispiel einem ausgeschiedenen Spielenden dabei zu helfen, wieder ins Rennen zu kommen. Mögliche Aufgaben dafür wären, das Einsammeln von bestimmten Punkten auf der Strecke, das Halten der Krone, oder eine gemeinsame Aktion (z.B. einen Trick machen).

\subsubsection{Coop-Points}
Bei Coop-Points müssen verschiede Spielende zur selben Zeit an zwei verschiedenen Orten sein. Dadurch muss es zwingend zu einer Absprache zwischen den Spielenden kommen, um auszumachen wer wohin fährt.

\subsubsection{Krone\label{_krone}}
Der erste Spielende hält immer die Krone. Wenn die Krone länger als eine gewisse Zeit vom selben Spielenden gehalten wird, kann ein ausgeschiedener Spielender wieder auf die Piste. Sollte die Krone von einem anderen Spielenden übernommen worden sein, bevor ein Snowboarder respawnt ist, wird der Timer wieder länger. Nach einem Respawn, muss die Krone an einen anderen Spielenden übergeben werden, um Teamwork und Kommunikation innerhalb des Teams zu fördern.

\subsubsection{Gemeinsame Aktion}
Eine gemeinsame Aktion wäre beispielsweise, dass alle Snowboarder zur selben Zeit Springen oder einen Trick ausführen. Das würde wieder ein großes Risiko mit sich bringen, da alle gleichzeitig sterben könnten.

\subsubsection{Rubberbanding [Ich finde nichts literarisches zu diesem Wort soll ich es durch dynamic balancing austauschen?]\label{_rubberbanding}}
Snowboarder, die weiter hinten sind, sollten schneller werden, damit sie mit der Gruppe mithalten können. Wie in \ref{_mana} angesprochen, ist es ach wichtig PC und VR untereinander zu balancen. Neben dem bereits angesprochenen, gibt es noch folgende weitere Möglichkeiten. Ausgeschiedene Spielende könnten den VR-Spieler daran hindern weitere Spells zu aktivieren oder gute Sicht auf die Strecke durch z.B. Nebel verhindern. Spells könnten auch dynamisch gestärkt oder geschwächt werden, indem man deren Range oder Dauer verändert.