\chapter{Implementation}

\section{Konzept}

[Konzept vom Spiel]

\section{Was muss bei VR-Game-Design beachtet werden?}

VR-Spiele bringen einige Limitationen mit sich, die es zu überwinden gilt. Die erste davon ist die Größe des Play-Spaces. Das zweite Problem ist Reisekrankheit (Motion-Sickness, [Definition + quelle]), vor allem bei Party-Spielen muss ihr Einfluss möglichst gering sein, da das Spiel von einer breiteren Masse von Menschen gespielt werden können soll. Die Reisekrankheit limitiert die Bewegungsmöglichkeiten der Spielenden und diese wiederum die Größe der virtuellen Umgebung. Beide Probleme hängen eng miteinander zusammen und es ist immer ein Balanceakt sie miteinander in Einklang zu bringen, wenn sich der Spieler kontinuierlich bewegen soll. 

Außerdem muss die Größe des Spielenden, um sicherzustellen, das alles, womit der Spielende interagieren soll, erreichbar ist, als auch Ängste, wie zum Beispiel "`Klaustrophobie"', beachtet werden. Im Falle von einem Party Spiel spielt auch das technische Know-How eine Rolle, da man sehr schnell in das Spiel eintauchen können soll.

\section{Stimmung \& Erfahrung}

Das im Zuge dieser Arbeit entwickelte Spiel soll eine bestimmte Stimmung erzeugen. Am Anfang des Design-Prozesses muss also festgelegt werden, wie eben diese Stimmung sein soll.

\subsection{Core Pillars}

\subsubsection{Fast but Strategic}
Das Ziel des Spiels ist es nicht nur so schnell es geht Spells auf der Strecke zu aktivieren. Es muss etwas Strategie dahinter sein, damit Spielende gefordert werden und eine wertvolle Entscheidung treffen. Da es sich aber immer noch um ein Party-Spiel und kein Strategie-Spiel handelt, sollte es trotzdem möglich sein schnelle Entscheidungen zu treffen. Um das zu ermöglichen, muss das Spell-System vergebend sein, und muss sich auf die aktuelle Spielsituation anpassen können. 

\subsubsection{Teamwork \& Kommunikation}
Außerdem muss ein bisschen Strategie/Absprache zwischen den PC-Spielern gewährleistet werden, was weiter zu dem Teamwork der Gruppe beiträgt. Wie am Anfang der Arbeit bereits definiert, braucht Teamwork ein gemeinsames Ziel. Das Ende der Strecke kommt hier recht schnell in den Sinn, allerdings ist das alleine nicht genug um Kommunikation zwischen den Spielenden zu erzeugen, da es zu weit in der Zukunft liegt, und sich nicht verändert. Eine Lösung, und mögliche Lösungsansätze, für dieses Problem wird in Kapitel \ref{_teamwork_erzeugen} besprochen.

\subsubsection{God-Like}
Vor allem der VR-Spieler soll sich mächtig fühlen. Erreicht werden soll dieses Gefühl durch den deutlichen Größenunterschied und Combos, also der möglichkeit mehrere Spieler gleichzeitig von der Strecke zu schmeißen, zu erzeugen.
[Lens of Power? = Lens of Challange]

\subsection{Target Experiences}

Da es sich bei dem Projekt um ein asymmetrisches Spiel handelt, wurden für die PC- und die VR-Spielenden jeweils andere Target Experiences festgelegt.

\subsubsection{Snowboarder (PC)}
Für die Snowboarder sind folgende zwei Experiences am wichtigsten: die (Snowboarding-) Action und das Teamwork der Gruppe. Mithilfe der "`Lens of Essential Experience"'\cite[S. 55]{_art_of_gamedesign} kann versucht werden, diese Experiences zu erzeugen.

Um das Gefühl Snowboarden akkurat wiederzugeben wurden 3 wichtige Komponenten gefunden: Tricks, Stürzen und Beharrlichkeit. Man könnte auch sagen, dass die Kälte ein Teil des Snowboardens darstellt, jedoch habe wurde dagegen entschieden, da es nicht mit dem Core Pillar "`God-Like"' zusammenpasst, wenn sich die Snowboarder über die Temperatur ihrer Charaktere Gedanken machen müssen.

Eine weitere wichtige Experience der Snowboarder ist das Teamwork, sogar so wichtig, dass ich sie, wie oben erwähnt, zu einem unserer Core Pillars gemacht habe. Wie Teamwork in unserem Spiel erzeugt werden soll, behandle ich, wie oben bereits erwähnt, in Kapitel \ref{_teamwork_erzeugen}.

\subsubsection{Magier (VR)}

Für den Magier sind die folgenden zwei Experiences am wichtigsen: Macht und Multitasking. Diese im Konkreten Göttliche/Magische Kraft, soll vor allem durch das Zaubern, also den Spells (\ref{_spell_design}]), und die Aneinanderreihung und taktische ("`Fast but Strategic"') Platzierung erzeugt werden. Sie wird auch über den visuellen Größenunterschied der Snowboarder und dem Magier dargestellt.

[
* Multitasken (Spells \& Combos)
* Ein Hauch Strategie
]

\section{Das Zusammentreffen der beiden Welten (PC + VR)}

\subsection{Streckendesign}

Der erste Punkt und auch Grundstein für das restliche Design ist das Design der Stecke. Es hat sowohl Einfluss auf die PC-Spielenden, als auch auf den VR-Spielenden, da es den Ort darstellt, in dem beide miteinander Interagieren.

\subsubsection{Limitationen}

Der VR-Spielende muss die Möglichkeit haben, zu jedem gegebenen Zeitpunkt, mit einem Großteil der Strecke zu interagieren (Fallen platzieren, etc.). Außerdem ist es wichtig, dass sich Spells immer in seiner Reichweite befinden, und er sich nicht zu viel bewegen muss. Die Strecke muss jedoch auch (aufgrund anderer technischer Limitationen) zu jedem Zeitpunkt komplett geladen und sichtbar sein.

Das Streckendesign muss es zulassen, die Strecke so anzupassen, dass eine Fahrt von oben nach unten ca. 3 Minuten dauert. Des weiteren muss die Breite der Strecke genug Platz lassen, um Hindernissen auszuweichen und die Spielenden eine gewissen Toleranz für Fehler bei dem Lenken der Snowboarder haben.

Zur Auswahl stehen drei verschiedene Streckendesigns, deren vor und Nachteile behandelt werden und begründet wird warum schlussendlich für Streckendesign Nr. 3 entschieden wurde. Alle drei Designs versuchen die Bewegung des VR-Spielers zu vermeiden, um das Gefühl der Reisekrankheit zu mindern.

\subsubsection{Designoption 1 - Spirale}

Die Snowboarder fahren, in diesem Design, in einer Spirale um den VR-Spielenden herum. Es ermöglicht eine relativ freie Strecke für die Snowboarder, jedoch ist wenig Platz für VR-Spielende. Dieses Design ist auch sehr unrealistisch, da es sich hier nicht um eine klassische Ski-Strecke handeln kann, da Teile "`Schweben", es fühlt sich an als wäre man in einer Wasserrutsche und das kämpft offensichtlich mit der Experience des Snowboardens.

\subsubsection{Designoption 2 - Kristallkugel}

Dieses Design bringt auf der Seite der Snowboarder und des VR-Spielenden deutliche Verbesserungen. Da VR-Spielende durch eine Kristallkugel auf die Snowboarder herab schaut, kann die Strecke unglaublich frei sein, es kann harte Kurven und Richtungswechsel geben. Der VR-Spieler, hat die Möglichkeit, einen eigenen schönen Raum zu bekommen, in dem er einen guten Zugang zu seinen Spells hat. Der größte Nachteil ist hier die Technik welche im Hintergrund gebraucht wird. Technisch ist es sehr schwer umzusetzen, da die Welten des VR-Spielenden und der Snowboarder voneinander getrennt und über die Kristallkugel wieder zusammengebracht werden müssen.

\subsubsection{Designoption 3 - Berg}

Die Idee des Berges wurde am Anfang etwas vernachlässigt, da es sehr eingeschränkt erschien aber der Berg bringt neben der technisch um einiges leichteren Umsetzung wieder etwas Realismus und eine schöne Atmosphäre ins Spiel, die die Experience des Snowboardens verstärken kann. Auf diesem, aus der Sicht des VR-Spielenden, kleinen Berg, in der Mitte des VR-Play-Spaces, können die Snowboarder Spiralenförmig hinunterfahren. Wie bei der Spirale hat man hier aber auch das Problem, dass die Snoaboarder permanent in die Kurve lenken müssen, dieses Problem wird in Kapitel \ref{_playercontroller} weiter behandelt. Der Berg wird außerdem automatisch mit den Snowboardern mitgedreht, damit sie immer im Sichtfeld des Magiers sind. Der Magier soll aber auch eine Möglichkeit haben diese Drehung zu überschreiben, damit er auch Teile der Strecke die außerhalb seines interagierbaren Bereiches liegen beeinflussen kann.

\subsection{Interaktionsdesign}

Da der Wettbewerb zwischen PC und VR asymmetrisch ist, müssen beide Parteien die jeweils andere Seite beeinflussen können, um es interessant zu machen. Der Magier muss durch Spells und andere Tools die Snowboarder behindern können und die Snowboarder müssen Einfluss auf die Welt des VR-Spieleden haben.

\subsubsection{Spell Design\label{_spell_design}}

Um dem VR-Spielenden das Gefühl von Zaubern zu übermitteln, braucht es neben der Möglichkeit zu Zaubern auch gute Interaktionen. Außerdem muss das Spell-System dem Magier die Möglichkeit geben zu Multitasken und Kombos zu machen, damit er sich Mächtig fühlt. Da der VR-Spieler zwei Hände hat, ist es naheliegend, dass beide Hände an einem unterschiedlichen Spells arbeiten können. Um eine gewisse Vielfalt in die möglichen Spells zu bringen, habe ich drei Spell-Kategorien eingeführt.

\paragraph{Instant-Spell}
\paragraph{Short-Term-Spell}
\paragraph{Long-Term-Spell}

\paragraph{Balancing-Cooldown}
Die Anzahl der Spells die der Magier verwenden kann, muss limitiert werden. Der einfachste Weg um Spells zu limitieren, ist ein Cooldown, der verhindert, dass man zwei Spells direkt nacheinander platziert.

\paragraph{Balancing-Mana}
Bevor ein Spell platziert werden kann, muss er in einen Mana-Kessel getaucht werden, um ihn zu aktivieren. Ein Mana-Kessel bietet gleich mehrere Vorteile: Balancing, Interaktion. Er kann dabei helfen das Spiel zu balancen, da jeder Spell unterschiedlich teuer sein kann und Mana limitiert ist. Außerdem fügt er einen zwischenschritt hinzu bevor man den Spell platzieren kann, das macht das spiel ein bisschen strategischer, da man sobald man den Spell einmal aufgeladen hat, dazu gezwungen ist ihn auch zu platzieren, da er sonst sein Mana wieder verliert. 

[Balancing-Interaktion: Dauer der Interaktionen nutzen um einen künstlichen Cooldown einzuführen]

\subsubsection{Einfluss der Snowboarder auf den Magier}

Es sollte auch für die PC-Spielenden eine Möglichkeit geben, den VR-Spielenden zu schwächen.

\paragraph{Nebel}
Ausgeschiedene Spielende könnten Nebel auf der Strecke platzieren, um die Sicht des Magiers zu hindern.

\paragraph{Mana}
Die Anzahl der Spielenden, die noch im Rennen sind könnte sich auf die Mana Menge des Magiers auswirken, so entsteht ein negativer Feedback Loop, es wird für den Magier schwieriger zu gewinnen, wenn er kurz davor ist zu gewinnen. Das erzeugt einen weiteren Pacing Spike, wie bei einem Bosskampf, am Ende einer Runde und hat einen Rubberbanding[Quelle?]-Effekt.

\subsection{VR-Room-Design}

Das Design des VR-Raumes geht Hand in Hand mit dem Design der Strecke. Je nach dem welches Stecken-Design gewählt wird, hat das unterschiedliche Auswirkungen auf das Design des VR-Raumes.

\subsection{Playercontroller \& Obstacles\label{_playercontroller}}
\subsubsection{Tricks}
Tricks können zum einen dazu genutzt werden, ein Risiko einzugehen um die Respawn-Zeit zu verkürzen oder als eine Movement-Mechanic, die dabei hilft Obstacles auszuweichen.

\subsubsection{Respawns}
Das Wiedereintreten von ausgeschiedenen Spielenden ist eine wichtige Mechanik um das Gefühl des Snowboardens einzufangen und um den Party-Aspekt des Spieles zu fördern. Es gibt verschiedene Möglichkeiten zu regeln, wann ein Spielende wieder in die Runde eintriten.

\paragraph{Timer}
Eine sehr simple Möglichkeit wäre, Ausgeschiedene nach einer gewissen Zeit automatisch wieder dem Rennen hinzuzufügen. Das hätte den Vorteil, dass es zum einen sehr leicht umsetzbar ist, es den Spielenden eine kleine Pause verschafft und der Timer sich durch andere Faktoren beeinflussen ließe.

\paragraph{Sekundäres Ziel für ausgeschiedene Spielende}
Ein weiterer Weg, der den augeschiedenen Spielenden etwas zu tun gibt, wäre ein Sekundäres Ziel. Sie müssten, um wieder ins Rennen zu kommen, eine kleine Aufgabe lösen, bestimmte Knöpfe schnell hintereinander drücken (verbunden mit viel Game-Juice, um die durch die Spielenden wahrgenommene Intensität des Gameplays zu steigern), oder den VR-Spieler behindern.

\subsubsection{Automatische Lenkhilfe}
Da die Spielenden bei dem Berg- oder Spiralen-Streckendesign permanent im Kreis lenken müssen, wäre es wichtig, dem entgegenzuwirken, um das Gefühl des "`im Kreis"'-fahrens zumindest etwas zu mindern.

\subsubsection{Obstacles}
Obstacles können sowohl von der Strecke vorgegeben sein, als auch von dem Magier platziert werden.

\paragraph{non-static-obstacles}
[ist das eine gute Idee?]

\paragraph{Positive Obstacles}
[Das Twinnie (verringert crown-timer wenn man reinfährt) - wie bringt man spieler dazu sie als etwas gutes zu erkennen]

\subsubsection{Spieler führen}
Wenn man Spielende für riskantes Fahren belohnt entsteht eine Meaningful Decision. Es entsteht eine Risk-Reward Situation, in der schlechte Team-Mitglieder sich von den Guten abspalten und andere Wege nehmen. Mithilfe von Rampen Slalom-Bögen, die die Respawn-Zeit verkürzen, kann man riskantes Fahren fördern.

\subsection{Teamwork erzeugen - Krone \& andere Lösungsansätze\label{_teamwork_erzeugen}}
Um die Snowbaord-Gruppe dazu zu bewegen zu Kooperieren, müssen sie ein gemeinsames Ziel haben. Dieses Ziel ist schlussendlich das Ende der Strecke jedoch braucht es auch ein kurzzeitiges Ziel wie zum Beispiel einem ausgeschiedenen Spielenden dabei zu helfen, wieder ins Rennen zu kommen. Mögliche Aufgaben dafür wären, das Einsammeln von bestimmten Punkten auf der Strecke, das Halten der Krone um einen ausgeschiedenen Spielenden wieder ins Rennen zu holen, oder eine gemeinsame Aktion.

\subsubsection{Coop-Points}
Bei Coop-Points müssen verschiede Spielende zur selben Zeit bei zwei verschiedenen Orten sein. Dadurch muss es zwingend zu einer Absprache zwischen den Spielenden geben, um auszumachen, wer wohin fährt.

\subsubsection{Krone}
Der erste Spielende hält immer die Krone. Wenn die Krone länger als eine gewisse Zeit vom selben Spielenden gehalten wird, kann ein ausgeschiedener Spielender wieder auf die Piste. Sollte die Krone von einem anderen Spielenden übernommen worden sein, bevor ein Snowboarder respawnt ist, wird der Timer wieder länger. Nach einem Respawn, muss die Krone an einen anderen Spielenden übergeben werden, um Teamwork und Kommunikation innerhalb des Teams zu fördern.

\subsubsection{Gemeinsame Aktion}
Eine gemeinsame Aktion wäre beispielsweise, dass alle Snowboarder zur selben Zeit Springen oder einen Trick ausführen. Das würde wieder ein großes Risiko mit sich bringen, da alle gleichzeitig sterben könnten.

\subsubsection{Rubberbanding - Balanceakt}
Snowboarder, die weiter hinten sind, sollten schneller werden, damit sie mit der Gruppe mithalten können. 

[Wie sehr verschwimmt Kapitel 2 und 3?]
[Playtests und balancing dann in Kapitel 3?]

\section{Playtests}

[Spieler Spawnen unerwartet und sterben sofort wieder - Interaktion bevor ein Spieler respawnt]

\section{Balancing}

[Spell \& Mana Balancing]
