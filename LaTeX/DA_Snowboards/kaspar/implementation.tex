\chapter{Implementation}

\section{Konzept}

[Nochmal?]

\section{Was muss bei VR-Game-Design beachtet werden?}

VR-Spiele bringen einige Limitationen mit sich, die es zu überwinden gilt. Die erste davon ist die Größe des Play-Spaces. Das zweite Problem ist Reisekrankheit (Motion-Sickness, [Definition + quelle]), vor allem bei Party-Spielen muss ihr Einfluss möglichst gering sein, da das Spiel von einer breiteren Masse von Menschen gespielt werden können soll. Die Reisekrankheit limitiert die Bewegungsmöglichkeiten der Spielenden und diese wiederum die Größe der virtuellen Umgebung. Beide Probleme hängen eng miteinander zusammen und es ist immer ein Balanceakt sie miteinander in Einklang zu bringen, wenn sich der Spieler kontinuierlich bewegen soll. 

Außerdem muss die Größe des Spielenden, um sicherzustellen, das alles, womit der Spielende interagieren soll, erreichbar ist, als auch Ängste, wie zum Beispiel "`Klaustrophobie"', beachtet werden. Im Falle von einem Party Spiel spielt auch das technische Know-How eine Rolle, da man sehr schnell in das Spiel eintauchen können soll.

\section{Stimmung \& Erfahrung}

Ich versuche in dem Spiel möglichst gut eine gewisse Stimmung zu erziehen. Am Anfang des Design-Prozesses muss also festgelegt werden, wie eben diese Stimmung sein soll.

\subsection{Core Pillars}

\subsubsection{Fast but Strategic}
Das ziel des Spiels ist es nicht nur so schnell es geht Spells auf der Strecke zu aktivieren. Es muss etwas Strategie dahinter sein, damit Spielende gefordert werden und eine wertvolle Entscheidung treffen. Da es sich aber immer noch um ein Party-Spiel und kein Strategie-Spiel handelt, sollte es trotzdem möglich sein schnelle Entscheidungen zu treffen. Um das zu ermöglichen, muss das Spell-System vergebend sein, und muss sich auf die aktuelle Spielsituation anpassen können. 

\subsubsection{Teamwork \& Kommunikation}
Außerdem muss ist ein bisschen Strategie/Absprache zwischen den PC-Spielern gewährleistet werden, was weiter zu dem Teamwork der Gruppe beiträgt. Wie am Anfang der Arbeit bereits definiert, braucht Teamwork ein gemeinsames Ziel. Das Ende der Strecke kommt hier recht schnell in den Sinn, allerdings, ist das alleine nicht genug um Kommunikation zwischen den Spielenden zu erzeugen, da es zu weit in der Zukunft liegt, und sich nicht verändert. Eine Lösung, und mögliche Lösungsansätze, für dieses Problem behandle ich in Kapitel \ref{_teamwork_erzeugen}.

\subsubsection{God-Like}
Vorallem der VR-Spieler soll sich mächtig fühlen. Geplant ist es dieses Gefühl vor allem durch den deutlichen Größenunterschied und Combos, also der möglichkeit mehrere Spieler gleichzeitig von der Strecke zu schmeißen, zu erzeugen.
[Lens of Power?]

\subsection{Target Experiences}

Da es sich bei unserem Spiel um ein asymmetrisches Spiel handelt, habe ich für jeweils die PC- und die VR-Spielenden andere Target Experiences festgelegt.

\subsubsection{Snowboarder (PC)}
Für die Snowboarder sind folgende zwei Experiences am wichtigsten: die (Snowboarding-) Action und das Teamwork der Gruppe. Ich fange damit an die Experience des Snowboardens etwas genauer zu untersuchen. Betrachten wir Snowboarden mal mit der "`Lens of Essential Experience"' [Brauche ich hier nochmal eine Quelle?].

Um das Gefühl Snowboarden akkurat wiederzugeben habe ich 3 wichtige Komponenten gefunden: Tricks, Stürzen und Beharrlichkeit. Man könnte auch sagen, dass die Kälte ein Teil des Snowboardens darstellt, jedoch habe ich mich dagegen entschieden, da es nicht mit unserem Core Pillar "`God-Like"' zusammenpasst, wenn sich die Snowboarder über die Temperatur ihrer Charaktere gedanken machen müssen.

Eine weitere wichtige Experience der Snowboarder ist das Teamwork, sogar so wichtig, dass ich sie, wie oben erwähnt, zu einem unserer Core Pillars gemacht habe. Wie Teamwork in unserem Spiel erzeugt werden soll, behandle ich, wie oben bereits erwähnt, in Kapitel \ref{_teamwork_erzeugen}.

\subsubsection{Magier (VR)}

Für den Magier sind die folgenden zwei Experiences am wichtigsen: Macht und Multitasking. Mit Macht, meine ich im Konkreten Göttliche/Magische Kraft, diese soll vorallem durch das Zaubern, also den Spells (\ref{_spell_design}]), und die Aneinanderreihung und Taktische ("`Fast but Strategic"') Platzierung erzeugt werden. Sie wird auch über den Visuellen Größenunterschied der Snowboarder und dem Magier dargestellt. 

[
* Multitasken (Spells \& Combos)
* Ein Hauch Strategie
]

\section{Das Zusammentreffen der beiden Welten (PC + VR)}

\subsection{Streckendesign}

Der erste Punkt und auch Grundstein für das restliche Design ist das Design der Stecke. Es hat sowohl Einfluss auf die PC-Spielenden, als auch auf den VR-Spielenden, da es den Ort darstellt, in dem beide miteinander Interagieren.

\subsubsection{Limitationen}

Der VR-Spieler muss die Möglichkeit haben, mit einen Großteil der Strecke zu jedem gegebenen Zeitpunkt interagieren (Fallen platzieren, etc.) zu können. Außerdem ist es wichtig, dass er selbst genug Platz hat um an seine Spells zu kommen, ohne sich viel bewegen zu müssen. Die Strecke darf ihn also nicht einschließen, da er sonst keine Möglichkeit hat an Spells zu kommen. Die Strecke muss jedoch auch (aufgrund anderer technischer Limitationen) zu jedem Zeitpunkt komplett geladen und sichtbar sein.

Die Länge der Strecke ist ein weiterer wichtiger Faktor, das Streckendesign muss es zulassen, die Strecke so anzupassen, dass eine Fahrt von oben nach unten ca. 3 Minuten dauert. Sie darf nicht zu eng sein, da die PC-Spieler sonst permanent in eine Richtung lenken müssten, und eine Seite der Strecke präferiert werden würde.

Zur Auswahl stehen drei verschiedene Streckendesigns, deren vor und Nachteile ich behandeln und begründen werde, warum ich mich schlussendlich für Streckendesign Nr. 3 entschieden habe. Alle drei Designs versuchen die Bewegung des VR-Spielers zu vermeiden, um das Gefühl der Reisekrankheit zu mindern.

\subsubsection{Designoption 1 - Spirale}

Die Snowboarder fahren, in diesem Design, in einer Spirale um den VR-Spielenden herum. Es ermöglicht eine relativ freie Strecke für die Snowboarder, jedoch ist wenig Platz für VR-Spielende. Dieses Design ist auch sehr unrealistisch, da es sich hier nicht um eine klassische Ski-Strecke handeln kann, da teile "`Schweben", es fühlt sich an als wäre man in einer Wasserrutsche und das kämpft offensichtlich mit der Experience des Snowboardens.

\subsubsection{Designoption 2 - Kristallkugel}

Dieses Design bringt auf der Seite der Snowboarder und des VR-Spielenden deutliche Verbesserungen. Da VR-Spielende durch eine Kristallkugel auf die Snowboarder herab schaut, kann die Strecke unglaublich frei sein, es kann harte Kurven und Richtungswechsel geben. Der VR-Spieler, hat die Möglichkeit, einen eigenen schönen Raum zu bekommen, in dem er einen guten Zugang zu seinen Spells hat. Der größte Nachteil ist hier die Technik welche im Hintergrund gebraucht wird. Technisch ist es sehr schwer umzusetzen, da die Welten des VR-Spielenden und der Snowboarder voneinander getrennt und über die Kristallkugel wieder zusammengebracht werden müssen.

\subsubsection{Designoption 3 - Berg}

Die Idee des Berges wurde am Anfang etwas vernachlässigt, da es sehr eingeschränkt erschin aber der Berg bringt neben der technisch um einiges leichteren Umsetzung wieder etwas Realismus und eine schöne Atmosphäre ins Spiel, die die Experience des Snowboardens verstärken kann. Auf diesem, aus der Sicht des VR-Spielenden, kleinen Berg, in der Mitte des VR-Play-Spaces, können die Snowboarder Spiralenförmig hinunterfahren. Wie bei der Spirale hat man hier aber auch das Problem, dass die Snoaboarder permanent in die Kurve lenken müssen, dieses Problem behandle ich weiter in Kapitel \ref{_playercontroller}. Der Berg wird außerdem automatisch mit den Snowboardern mitgedreht, damit sie immer im Sichtfeld des Magiers sind. Der magier soll aber auch eine möglichkeit haben diese Drehung zu überschreiben, damit er auch teile der Strecke die außerhalb seines interagierbaren Bereiches liegen beeinflussen kann.

\subsection{Interaktionsdesign}

Da der Wettbewerb zwischen PC und VR asymmetrisch ist, müssen beide Parteien die jeweils andere Seite beeinflussen können, um es interessant zu machen. Der Magier muss durch Spells und andere Tools die Snowboarder behindern können und die Snowboarder müssen Einfluss auf die Welt des VR-Spieleden haben.

\subsubsection{Spell Design\label{_spell_design}}

Um dem VR-Spielenden das Gefühl von Zaubern zu übermitteln, braucht es neben der Möglichkeit zu Zaubern auch gute Interaktionen. Außerdem muss das Spell-System dem Magier die Möglichkeit geben zu Multitasken und Kombos zu machen, damit er sich Mächtig fühlt. Da der VR-Spieler zwei Hände hat, ist es naheliegend, dass beide Hände an einem unterschiedlichen Spells arbeiten können.

\paragraph{Spell-Types}

\subsubsection{PC -> VR Interactions}

Es sollte auch für die PC-Spielenden eine Möglichkeit geben, den VR-Spielenden zu schwächen.

\subsection{VR-Room-Design}

Das Design des VR-Raumes geht Hand in Hand mit dem Design der Strecke. Je nach dem welches Stecken-Design gewählt wird, hat das unterschiedliche Auswirkungen auf das Design des VR-Raumes.

\subsection{Playercontroller \& Obstacles\label{_playercontroller}}
\subsubsection{Tricks}
\subsubsection{Respawns}
\subsubsection{Automatische Lenkhilfe}


\subsection{Teamwork erzeugen - Krone \& andere Lösungsansätze\label{_teamwork_erzeugen}}
Um die Snowbaord-Gruppe dazu zu bewegen zu Kooperieren, müssen sie ein gemeinsames Ziel haben. Dieses Ziel ist schlussendlich das Ende der Strecke


\subsubsection{Krone}

\subsubsection{Rubberbanding - Balanceakt}

\section{Target Experience vs Playtests}

\section{Balancing}
