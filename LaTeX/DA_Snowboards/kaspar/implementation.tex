\chapter{Implementation}

\section{Konzept}

\section{Was muss bei VR-Game-Design beachtet werden?}

VR-Spiele bringen einige Limitationen mit sich, die es zu überwinden geht. Die erste davon ist die Größe des Play-Spaces / der limitierte Platz. Das zweite Problem ist Reisekrankheit (Motion-Sickness, [Definition + quelle]), vor allem bei Party-Spielen muss ihr Einfluss möglichst gering sein, da das Spiel von einer breiteren Masse an Menschen gespielt werden können soll. Die Reisekrankheit limitiert die Bewegungsmöglichkeiten der Spielenden und die limitiert wiederum die Größe der virtuellen Umgebung. Beide Probleme hängen eng miteinander zusammen und es ist immer ein Balanceakt sie miteinander in Einklang zu bringen. 

Außerdem muss die Größe der Spieler und deren technisches Know-How, als auch Ängste beachtet werden. 

\subsection{Streckendesign}

Der erste Punkt und auch Grundstein für das restliche Design ist das Design der Stecke. Das Design der Strecke hat sowohl Einfluss auf die PC-Spielenden als auch auf den VR-Spielenden, da es den Ort darstellt, in dem beide miteinander Interagieren.

\subsubsection{VR}

Der VR-Spieler muss die Möglichkeit haben, mit einen Großteil der Strecke zu jedem gegebenen Zeitpunkt interagieren (Fallen platzieren, etc.) zu können. Außerdem ist es wichtig, dass er selbst genug Platz hat um an seine Spells zu kommen, ohne sich viel bewegen zu müssen. Die Strecke darf ihn also nicht einschließen, da er sonst keine Möglichkeit hat an Spells zu kommen. Die Strecke muss jedoch auch (aufgrund anderer Technischer Limitationen) zu jedem Zeitpunkt komplett geladen und sichtbar sein.

\subsubsection{PC}

Die Länge der Strecke ist ein weiterer Wichtiger Faktor, das Streckendesign muss es zulassen, die Strecke so anzupassen, dass eine Fahrt von oben nach unten ca. 3 Minuten dauert. Sie darf nicht zu eng sein, da die PC-Spieler sonst permanent in eine Richtung lenken müssten, und eine Seite der Strecke präferiert werden würde.

\subsubsection{Design}

Zur Auswahl stehen drei verschiedene Streckendesigns, dessen vor und Nachteile ich behandeln werde und begründen werde, warum ich mich schlussendlich für Streckendesign Nr. 3 entschieden habe. Alle drei Designs versuchen die Bewegung des VR-Spielers zu vermeiden, um Reisekrankheit zu vermeiden.

\paragraph{Designoption 1 - Spirale}

Die Snowboarder fahren, in diesem Design, in einer Spirale um den VR-Spielenden herum. Es ermöglicht eine relativ freie Strecke für die Snowboarder, jedoch ist wenig Platz für den, VR-Spielenden, und es man fühlt sich sehr eingeschlossen. Dieses Design ist auch sehr unrealistisch, da es sich hier nicht um eine klassische Ski-Strecke handeln kann, da teile "`Schweben", es fühlt sich an als wäre man in einer Wasserrutsche.

\paragraph{Designoption 2 - Kristallkugel}

Dieses Design bringt auf der Seite der Snowboarder und des VR-Spielenden deutliche Verbesserungen. Da der VR-Spieler durch eine Kristallkugel auf die Snowboarder herab schaut, kann die Strecke unglaublich frei sein, es kann harte Kurven und Richtungswechsel geben. Der VR-Spieler, hat die möglichkeit, einen eigenen schönen Raum zu bekommen, in dem er einen guten Zugang zu seinen Spells hat. Der größte Nachteil ist hier die Technik welche im Hintergrund gebraucht wird. Technisch ist es sehr schwer umzusetzen, da die Welten des VR-Spielenden und der Snowboarder voneinander getrennt und über die Kristallkugel wieder zusammengebracht werden müssen.

\paragraph{Designoption 3 - Berg}

Der Berg bring neben der Technisch um einiges leichteren Umsetzung wieder etwas Realismus ins Spiel. Es handelt sicht schlicht und einfach um einen kleinen berg, in der Mitte des VR-Play-Spaces, welcher automatisch, oder auch durch den VR-Spielenden, gedreht werden kann.

\subsection{Raum-Design}

\subsection{Playercontroller \& Obstacles}

\section{Was ist unsere Target-Experience?}

\section{Target Experience vs Playtests}
