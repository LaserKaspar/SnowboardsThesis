\chapter{Begriffsdefinitionen}

Bei "`Tricks 'n' Treats"' handelt es sich um ein asymmetrisches Couch-Party VR-Spiel. [Hier fehlt, denke ich, eine kurze Beschreibung von unserem Spiel.]

In diesem Kapitel werden die in "`Tricks 'n' Treats"' verwendeten Technologien erklärt und die grundlegenden Konzepte des Game-Designs die verwendet werden um die  Erlebnisse der und die Interaktionen zwischen den Spielenden zu beeinflussen.

\section{VR trifft Couchparty-coop}

\subsection{Was ist Virtual Reality?}

Virtual Reality bezeichnet Bilder und Töne, die von einem Computer erzeugt werden und dem Benutzer, der mit Hilfe von Sensoren mit ihnen interagieren kann, fast real erscheinen.\cite{_oxford_dict} Sie ist trotz anderer Anwendungsbereiche, dank der Immersion und Interaktionsmöglichkeiten die sie bietet, besonders in der Gaming-Industrie relevant geworden. [bitkom research]. Spielenden können neue Erlebnisse geboten werden, welche wiederum eigene  Game-Design-Fragen aufbringen die es zu beantworten gilt.

VR bringt neben der Interaktivität noch einen weiteren gewaltigen Vorteil: Intuition. Greifen, Umschauen, Bewegen, alles funktioniert wie man es erwartet. Gaming-Neulinge haben häufige Probleme die richtigen Tasten auf der Tastatur zu drücken und auch klassische Gamepads sind nicht optimal, in VR gibt es zum einen weniger Knöpfe die gedrückt werden können und zum anderen wird viel unterbewusster gearbeitet. Spielende verstehen in VR schneller wie er das Spiel gesteuert wird und können sich mehr Gedanken darüber machen was das Ziel des Spiels ist. Insbesondere für Party-Spiele (Zu denen wir gleich kommen werden.) ist das sehr wichtig. [VR ist ein gutes Medium für Party-Spiele -> Es ist "`einfach"' man versteht schneller was man tun muss]

[
Bild: 
	Gegenüberstellung von Interaktionen VR - Nicht VR (Greifen, Knöpfe)
	Bewegen? Drehen, Umschauen, Firstperson
]
[Greifen Tastatur-E vs VR-Grab (hl vs alyx)]
[field: interaction design]

\subsection{Was ist Couchparty?}

Couchparty-Spiele, oft auch nur Party-Spiele oder local-multiplayer-games, sind Computerspiele, die eine Gruppe (die sich meist untereinander kennt) gemeinsam spielt. Anders als bei klassischen Online-Multiplayer-Spielen werden diese zusammen in einem Raum gespielt. Sie sind jedoch nicht zu verwechseln mit LAN klassikern wie Counter Strike\footnote{Counter Strike ist eine Reihe von taktischen Multiplayer-Ego-Shootern [verbesserungswürdig]} oder Unreal Tournament\footnote{[Sehr ähnich wie cs:go? Anderes Beispiel?]} die meist nur bei leidenschaftlichen Gamern gespielt werden, da sie Vorbereitung, Know-How und teure Ausrüstung benötigen. Das Alleinstellungsmerkmal dieser Art von Spielen ist, dass diese auf nur einem Bildschirm vor einer Couch, daher der Name, aus gespielt werden und nur minimale Ausrüstung, wie zum beispielsweise mehrere Controller, wobei oft auch ein einfaches Handy reicht, benötigt wird.

In Couchparty-Spielen müssen Spielende schnell verstehen worum es geht, daher stützen sich viele Spiele auf "`Zufall"'. Dadurch können Anfangende schnell mit den anderen mitkommen, auch wenn sie das Spiel noch nie zuvor gespielt haben.

[Splitscreen vs Single Screen]

[Quellen?]

\subsection{Was sind coop-games?}

Coop-games (= cooperative-games/Kooperative-Spiele) sind eine besondere Art von Team-basierten-Spielen die besonders auf Gemeinschaft der Spielenden setzt und diese auch durch Game-Design voraussetzt. 

Ein Koalitions- oder Strategiespiel ist kooperativ, wenn die Spieler verbindliche Vereinbarungen über die Verteilung der Auszahlungen oder die Wahl der Strategien treffen können, auch wenn diese Vereinbarungen nicht durch die Spielregeln spezifiziert oder impliziert sind.\cite{_introduction_to_the_theory_of_cooperative_games}

Bei klassischen Team-basierten-Spielen ist Kommunikation zwar wichtig jedoch nicht essentiell. Aufgaben der Spielenden werden in coop-games so verteilt, dass ein Ziel nicht ohne die Zusammenarbeit der anderen erreicht werden kann. Es entsteht hierbei eine interessante Dynamik in der die Spielenden komplett voneinander abhängig sind. 

[symmetrische coop-games: Lego-games, Battle Block Theater / ist Overcooked asymmetrisch?] [Darf man andere Diplomarbeiten zitieren?]

\subsection{Was ist bedeutet Asymmetrie?}
Asymmetrie beschreibt im allgemeinen zwei Seiten oder Teile, die nicht die gleiche Größe oder Form haben.\cite{_oxford_dict}

In Videospielen beschreibt Asymmetrie das (Board-) Design. Die Spielenden haben in einem solchen Spiel ein hoch unterschiedliches action-set, das bedeutet sie haben einen anderen Einfluss auf die Spielwelt/andere Mitspieler. In Spielen mit asymmetrischen Wahlmöglichkeiten beginnt jeder Spielende jedoch typischerweise mit einer Reihe von Aktionen, die sich stark von denen der anderen unterscheiden, was das "`Balancing"' dieser Spiele besonders schwierig macht. Die Herausforderung besteht vor allem darin, den relativen Einfluss einer Aktion auf die Gewinnwahrscheinlichkeit zu bestimmen.\cite{_balancing_asymmetric_video_games}


