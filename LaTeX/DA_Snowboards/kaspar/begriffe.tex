\chapter{Begriffsdefinitionen}

Bei "`Tricks 'n' Treats"' handelt es sich um ein asymmetrisches Couch-Party VR-Spiel.

In diesem Kapitel werden die in "`Tricks 'n' Treats"' verwendeten Technologien erklärt und die grundlegenden Konzepte des Game-Designs, die verwendet werden, um die Erlebnisse der Spielenden und die Interaktionen zwischen ihnen zu beeinflussen.

\section{VR trifft Couchparty-coop}

\subsection{Was ist Virtual Reality?}

Virtual Reality bezeichnet Bilder und Töne, die von einem Computer erzeugt werden und dem Benutzer, der mit Hilfe von Sensoren mit ihnen interagieren kann, fast real erscheinen\cite{_oxford_dict}. Sie ist trotz anderer Anwendungsbereiche, dank der Immersion und Interaktionsmöglichkeiten, die sie bietet, besonders in der Gaming-Industrie relevant geworden\cite{_bitkom_vr}. Spielenden können neue Erlebnisse geboten werden, welche wiederum eigene Game-Design-Fragen aufwerfen, die es zu beantworten gilt.

VR bringt neben der Interaktivität noch einen weiteren gewaltigen Vorteil: Intuition. Greifen, Umschauen, Bewegen, alles funktioniert, wie man es erwartet. Gaming-Neulinge haben häufig Probleme die richtigen Tasten auf der Tastatur zu drücken und auch klassische Gamepads sind nicht optimal. In VR gibt es zum einen weniger Knöpfe an den Controllern, die gedrückt werden können und zum anderen können alltägliche Interaktionen, wie Umschauen / Orientieren und das in die Hand nehmen von Objekten, in VR unterbewusster durchgeführt werden und sind daher für Anfangende weniger frustrierend. Spielende verstehen in VR schneller wie das Spiel gesteuert wird und können sich mehr Gedanken darüber machen was das Ziel des Spiels ist\cite{_natural_interaction_in_augmented_reality_context}. Insbesondere für Party-Spiele (siehe \ref{_party_games}) ist das schelle Verstehen von der Steuerung und der intuitiven Ausführung von grundlegenden Aktionen in Stresssituationen sehr wichtig.

\begin{figure}[H]
	\centering
	\includegraphics[width=10cm]{images/kaspar/natural-interaction}
	\caption{Grab in Portal\cite{_grab_portal} vs Alyx\cite{_grab_alyx}}
\end{figure}

\subsection{Was ist Couchparty?\label{_party_games}}

Couchparty-Spiele, oft auch nur Party-Spiele oder Couch-Games, sind Computerspiele, die eine Gruppe (die sich meist untereinander kennt) gemeinsam spielt. Anders als bei klassischen Online-Multiplayer-Spielen werden diese zusammen in einem Raum gespielt. Sie sind jedoch nicht zu verwechseln mit local-multiplayer-games. LAN-Klassiker wie \emph{Counter Strike}\footnote{Counter Strike ist eine Reihe von taktischen Multiplayer-Ego-Shootern} oder \emph{Starcraft}\footnote{Starcraft ist ein Echtzeit-Strategiespiel} werden meist nur von leidenschaftlichen Gamern gespielt, da sie Vorbereitung, Know-How und viel Ausrüstung benötigen. Das Alleinstellungsmerkmal von Party-Spielen ist, dass diese auf nur einem Bildschirm von einer Couch aus, daher der Name, gespielt werden können und nur minimale Ausrüstung, meistens mehrere Controller, wobei bei manchen auch schon das Handy als Controller fungieren kann, benötigt wird.

In Couchparty-Spielen müssen Spielende schnell verstehen, worum es geht, daher stützen sich viele Spiele auf "`Zufall"'. Dadurch können Anfangende schnell mit den anderen mitkommen, auch wenn sie das Spiel noch nie zuvor gespielt haben. Eine weitere Möglichkeit, das Spiel auch für schlechtere Spieler zugänglich machen wäre die Verwendung von "`Virtual Skills"'\cite[S. 165]{_art_of_gamedesign}.

Es gibt grundsätzlich zwei verschiedene Arten wie Couchparty-Spiele den Bildschirm-Platz nutzen können.

\begin{itemize}
	
\item \subsubsection{Shared-Screen}

In diesem Fall sehen alle Spielende das gleiche und müssen sich daher auch in der Spielwelt im gleichen Raum aufhalten. Der Vorteil eines solchen Systems ist, dass keine Informationen mehrfach gezeigt werden müssen und der Platz des Bildschirms vor allem bei kleinen Laptop-Bildschirmen oder sogar Handys besser genutzt wird. Außerdem hat es einen Performance-Vorteil, da nicht mehrere Kameras gerendert werden müssen. Leider eignet sich diese Art nicht für jedes Spiel, da sich die Spielenden nicht zu weit voneinander entfernen können, vor allem 3D-Spiele mit einer freien Kamera sind hier sehr eingeschränkt. Besonders eignet sich Shared-Screen-Play für Arena-Games wie z.B. \emph{Super Smash Bros} oder \emph{Overcooked}.

\item \subsubsection{Split-Screen}

Ein Split-Screen System bietet hier einen großen Vorteil: Freiheit. Alle Spielenden können sich selbständig bewegen und umschauen, ohne das Erlebnis der anderen zu beeinflussen. Allerdings wird hierfür ein größerer Bildschirm benötigt, da jedem Spielenden nur ein halber oder nur ein Viertel des Bildschirms zugewiesen wird.

\end{itemize}

\noindent Es gibt auch die Möglichkeit nahtlos zwischen den beiden Systemen zu wechseln, je nachdem welches System sich gerade besser eignet, um das gewünschte Spielerlebnis zu erzeugen. Laut einer Reddit Umfrage präferieren Spielende Shared-Screen-Play\cite{_shared_or_splitscreen_preference}.

\subsection{Was sind coop-games?}

Coop-games (= cooperative-games/kooperative-Spiele) sind eine besondere Art von team-basierten-Spielen die besonders auf Gemeinschaft der Spielenden setzt und diese auch durch Game-Design voraussetzt. 

Ein Koalitions- oder Strategiespiel ist kooperativ, wenn die Spielenden verbindliche Vereinbarungen über die Verteilung der Auszahlungen oder die Wahl der Strategien treffen können, auch wenn diese Vereinbarungen nicht durch die Spielregeln spezifiziert oder impliziert sind\cite{_theory_of_cooperative_games}.

Bei klassischen Team-basierten-Spielen ist Kommunikation zwar wichtig jedoch nicht essenziell. Aufgaben der Spielenden werden in coop-games so verteilt, dass ein Ziel nicht ohne die Zusammenarbeit der anderen erreicht werden kann. Es entsteht hierbei, da die Spielenden komplett voneinander abhängig sind, eine interessante Dynamik, die dazu führen kann, dass sich Spannungen oder Konflikte innerhalb des Teams bilden, da eine Person nicht kooperiert, oder Aufgaben auf einem anderen Weg lösen will. Kooperation kann auch, beziehungsweise soll in den meisten Fällen, den Spaßfaktor eines Spieles fördern, mehr dazu in Kapitel \ref{_cooperative_play}.

Gute Beispiele hierfür sind einige Spiele der Lego-Reihe, \emph{Overcooked} und \emph{It Takes Two} welches in Kapitel \ref{_industrie} genauer beschreiben wird.

\subsubsection{Wie erzeugt man Kooperation?}
Kooperation, auch jene in Videospielen, kann zu tiefen Bindungen zwischen den Spielenden führen.
Mit der "`Lens of cooperation"'\cite[S. 311]{_gamemechanics_for_cooperative_games} kann ein Game-Designer besser verstehen, wie Kooperation entsteht und wie man sie fördert. Die wichtigsten Punkte, die Jesse Schell hier anspricht, sind die Notwendigkeit von Kommunikation oder die Tatsache, dass eine Aufgabe nur durch Zusammenarbeit der Spielenden gelöst werden kann. Ein weiterer interessanter Punkt ist Synergie und Antergie, die Veränderung der Stärke einer Gruppe, wenn ein Team zusammenkommt. Synergie beschreibt, dass die Gruppe Stärker als die Summe der einzelnen ist, Antergie ist das Gegenteil davon. Synergie ist damit in kooperativem Umfeld erstrebenswert.

\subsection{Was bedeutet Asymmetrie?}

Asymmetrie beschreibt im Allgemeinen zwei Seiten oder Teile, die nicht die gleiche Größe oder Form haben\cite{_oxford_dict}.

In Spielen bezieht sich Asymmetrie auf das (Board-) Design. Die Spielenden haben in einem solchen Spiel ein hoch unterschiedliches action-set, das bedeutet sie haben einen anderen Einfluss auf die Spielwelt/andere Mitspieler. In Spielen mit asymmetrischen Wahlmöglichkeiten beginnt jeder Spielende typischerweise mit einer Reihe von Aktionen, die sich stark von denen der anderen unterscheiden, was das "`Balancing"' dieser Spiele besonders schwierig macht. Die Herausforderung besteht vor allem darin, den relativen Einfluss einer Aktion auf die Gewinnwahrscheinlichkeit zu bestimmen\cite[S. 18]{_balancing_asymmetric_video_games}.
