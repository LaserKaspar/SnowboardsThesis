\chapter{Das Vorhaben}
\label{cha:sa_Einleitung}

Bei "`Tricks 'n' Treats"' handelt es sich um ein asymmetrisches Couch-Party VR-Spiel. [Hier fehlt eine kurze Beschreibung von unserem Spiel.]

In diesem Kapitel werden die in "`Tricks 'n' Treats"' verwendeten Technologien erklärt und die grundlegenden Konzepte des Game-Designs die verwendet werden um die  Erlebnisse der und die Interaktionen zwischen den Spielenden zu beeinflussen.

\section{VR trifft Couchparty}

\subsection{Was ist Virtual Reality?}

Virtual Reality bezeichnet Bilder und Töne, die von einem Computer erzeugt werden und dem Benutzer, der mit Hilfe von Sensoren mit ihnen interagieren kann, fast real erscheinen. [Brauchen Definitionen Quellen? (oxford dict)]. Sie ist trotz anderer Anwendungsbereiche, dank der Immersion und Interaktionsmöglichkeiten die sie bietet, besonders in der Gaming-Industrie relevant geworden. [Quelle angeben]. Spielenden können neue Erlebnisse geboten werden, welche wiederum eigene  Game-Design-Fragen aufbringen die es zu beantworten gilt.

[Gegenüberstellung von Interaktionen VR - Nicht VR]

\subsection{Was ist Couchparty?}

Couchparty-Spiele, oft auch nur Party-Spiele oder Couch-Multiplayer-Spiele, sind Computerspiele, die eine Gruppe (die sich meist untereinander kennt) gemeinsam spielt. Anders als bei klassischen Online-Multiplayer-Spielen werden diese zusammen in einem Raum gespielt und sind damit eine Unterkategorie der local-multiplayer-games. Das Alleinstellungsmerkmal dieser Art von Spielen ist, dass diese auf nur einem Bildschirm vor einer Couch, daher der Name, aus gespielt werden und nur minimale Ausrüstung benötigt wird.

\subsection{Coop games?}

Kooperation in Team-basierten-Spielen ist sehr wichtig. Coop-Games (= Cooperative-Games/Kooperative-Spiele) verteilen die Aufgaben der Spielenden so, dass eine Aufgabe nicht ohne die Hilfe der anderen gelöst werden kann.

\subsection{Was ist Asymmetrie?}

\section{Warum machen Spiele Spaß?}

\section{Game Design um Interaktionen zwischen den Spielenden zu fördern}

\subsection{MDA-Framework}

\subsection{Player Action Feedback}

\subsection{Target Experience}

