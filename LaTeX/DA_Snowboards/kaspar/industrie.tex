\chapter{Stand der Industrie\label{_industrie}}

\section{Asymmetrical Coop}

\subsection{Keep Talking and Nobody Explodes}
Keep Talking and Nobody Explodes ist ein Spiel in dem eine Person eine Bombe vor sich sieht, die es zu entschärfen gilt, und die andere eine Anleitung zur Entschärfung der Bombe. Das Spiel ist zwar kooperativ jedoch sehr einseitig, da der Bombenentschärfer oft auf auf den anderen Warten muss und es außerhalb der Beschreibung von dem was man sieht nicht viel Kommunikation gibt. Auch kein euphorisches Anfeuern, obwohl alles unter enormem Zeitdruck ausgeführt wird.

\subsection{It Takes Two}
Anders ist es bei "`It Takes Two"'. Es überzeugt besonders durch sein abwechslungsreiches Gameplay und dem sehr guten Pacing, es erfordert von beiden Seiten nicht sehr viel können, was es für alle Zielgruppen offen hält und es setzt Kommunikation voraus. Gemeinsames lösungsorientiertes Denken ist von Vorteil und das Spiel bietet eine entspannende Atmosphäre in der man sich schnell verliert.

\section{Couchparty}

\subsection{Mario Party}
Mario Party ist das klassische Party spiel. Wie für Party-Spiele üblich bietet es eine enorme Vielfalt an Spielen und einen gewisses Maß an Zufall um auch die Schlechteren einer Gruppe im Spiel zu inkludieren.

\subsection{Jackbox}
Bei den Jackbox-Spielen handelt es sich um eine Reihe an verschiedenen Spielen die einen oft an klassische Papier-Spiele erinnern, die man früher und auch heute noch ohne viel Aufwand gespielt hat (z.B. Wörter Raten). Es ist besonders gut darin lustige Geschichten oder andere Kreationen aus den Spielenden herauszuholen. Es bietet aber auch einige komplexere Spiele, die für eine erfahrenere Gruppe noch mehr Abwechslung bieten. Die Experience entsteht hier aber haupsächlich durch die Imagination der Spielenden.

\section{Couchparty VR-Games}

Couchparty VR-Games sind per Definition asymmetrisch. In dieser Kombination wird es schon schwieriger gute Spiele zu finden, da der Markt noch relativ klein ist und sich die großen Studios (Nintendo, EA, etc.) noch nicht in den Markt investieren, da es finanziell noch keinen Sinn macht.

\subsection{Takelings House Party}

Takelings bringt in das Chaos von Couchparty-Spielen noch einen VR-Spieler.

\subsection{Acron: Attack of the Squirrels!}
[Muss ich noch Spielen]