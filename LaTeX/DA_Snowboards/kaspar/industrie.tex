\chapter{Stand der Industrie\label{_industrie}}

\section{Asymmetrical Coop}

\subsection{Keep Talking and Nobody Explodes}
In Keep Talking and Nobody Explodes übernimmt eine Person das Entschärfen einer Bombe, während die andere eine Anleitung zur Entschärfung navigieren muss. Das Spiel ist zwar kooperativ, jedoch sehr einseitig, da der Bombenentschärfende oft auf den Instruktor warten muss und es, abgesehen von Beschreibungen der Bombe oder der Anleitung, nicht viel Kommunikation gibt. Auch kein euphorisches Anfeuern, obwohl alles unter enormem Zeitdruck geschieht. Man muss allerdings anmerken, dass die Experience des Bombenentschärfens damit weiter unterstützt wird.

\subsection{It Takes Two}
Anders ist es bei "`It Takes Two"'. Es überzeugt besonders durch sein abwechslungsreiches Gameplay und dem sehr guten Pacing. Es erfordert von beiden Seiten nicht sehr viel können, dadurch ist es für sehr viele Leute zugänglich und es lässt sich mit fast jeder Person spielen. Kommunikation ist sehr wichtig, da viele Rätsel lösungsorientiertes Denken beider Spielenden benötigen. Außerdem bietet das Spiel eine entspannende Atmosphäre, in der man sich schnell verliert.

\section{Couchparty}

\subsection{Mario Party}
Mario Party ist das klassische Party-Spiel. Wie für das Genre üblich bietet es eine enorme Vielfalt an Mini-Spielen und ein hohes Maß an Zufall, um auch die Schlechteren einer Gruppe im Spiel zu inkludieren.

\subsection{Jackbox}
Bei den Jackbox-Spielen handelt es sich um eine Reihe an verschiedenen Mini-Spielen, die einen oft an klassische Papier-Spiele erinnern, die man früher, und auch heute noch, ohne viel Aufwand gespielt hat (z.B. Wörter Raten). Es ist besonders gut darin lustige Geschichten oder andere Kreationen aus den Spielenden herauszuholen. Es bietet aber auch einige komplexere Spiele, die für eine erfahrenere Gruppe noch mehr Abwechslung bieten. Die Experience entsteht hier aber hauptsächlich durch die Imagination der Spielenden.

\section{Couchparty VR-Games}

Couchparty VR-Games sind per Definition asymmetrisch. In dieser Kombination wird es schon schwieriger gute Spiele zu finden, da der Markt noch klein ist und  die großen Studios (Nintendo, EA, etc.) noch nicht in den Markt investieren, da es finanziell noch keinen Sinn macht.

\subsection{Acron: Attack of the Squirrels!}
Bei diesem Spiel spielt eine Gruppe aus 4 gegen den VR-Spielenden. Ziel ist es dem VR-Spielenden eine Eichel zu stehlen, um das zu verhindern, können VR-Spielende die Gruppe mit "`Steinen"' bewerfen. Hier fehlen leider die Interaktion zwischen den Spielenden in der Gruppe untereinander. Trotzdem kommt es bei der Community gut an, jedoch biet es laut Reviews zu wenig Inhalt und Abwechslung für den verlangten Preis\cite{_acron_reviews}.

\subsection{Takelings House Party}
Takelings bringt in das Chaos von Couchparty-Spielen noch einen VR-Spieler. Es hat verschiedene Spielmodi, in denen das Ziel des VR-Spielenden es ist alle "`Takelings"' einzufangen. Leider ist auch hier wieder wenig Interaktion zwischen den PC-Spielenden.