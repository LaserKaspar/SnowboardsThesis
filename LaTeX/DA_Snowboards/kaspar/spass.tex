\chapter{Spaß in Videospielen}

\section{Warum machen Spiele Spaß?}

[Was ist spaß?]

\subsection{Woher kommt Spaß?}

Gamedesigner Marc LeBlanc hat acht verschiedene Typen von situationsabhängigem Spaß definiert. Einige davon lassen sich gut auf Situationen unseres Spieles anwenden um genauer zu verstehen, woher der Spaß in dem Spiel kommen soll: "`drama, obstacle, social framework"'. 

Das Buch "`A Theory of Fun"'\cite{_theory_of_fun} definiert Spaß als das mentale meistern eines Problems\cite[S. 71]{_theory_of_fun}. Außerdem meint Raph Koster, dass Spaß aus "`richly interpretable situations"'\cite[S. 40]{_theory_of_fun}, also Situationen, die den Spielenden mehrere Möglichkeiten gegeben ein Problem zu lösen und Kreativität fördern, entsteht. 

Spaß kann also nur existieren, wenn Spielende immer neue Probleme lösen können oder bereits gelöste Probleme auf andere Wege ("`richly interpretable"') lösbar sind. Das Meistern eines Spieles, laut seiner Definition, kann also nur Spaß machen, wenn sich die Situationen je nach skill-level ändern oder sie auf andere Wege lösbar sind. Das verbessern der Reaktionsfähigkeit oder das ständige ausführen der selben Aktion macht keinen Spaß. Raph Koster hat außerdem einige wichtige Grundsteine für Spaß in Spielen gelegt, diese werden im praktischen Teil noch genauer behandelt.

\subsection{Spaß bei mehrspieler Spielen}

Ein weiterer Ursprung von Spaß bzw. Vergnügen lässt sich in sozialen Interaktionen finden\cite[S. 72]{_theory_of_fun}.
[competetive vs cooperative fun]

Pary-Spiele sind eine Mischung aus "`Theater"' und "`Arena"' Spielen\cite[S. 65]{_art_of_gamedesign}. 

Fun is more Fun when others are involved

\section{Wie kann man Spaß "`kreieren"'?}

Wie vorher definiert gibt es verschiedene Arten von Spaß. Das Erlebnis der Spielenden lässt sich hauptsächlich durch die Aktionen beeinflussen, welche den Spielenden zu Verfügung gestellt werden und dessen Auswirkungen auf die Spielwelt oder andere Spielende.

\subsection{MDA-Framework}

\subsubsection{Mechanics (Regeln)}

\subsubsection{Dynamics (System)}

\subsubsection{Aesthetics (Spaß)}

\subsection{Player Action Feedback}

\subsection{Target Experience}

\subsection{Target Dynamics}

\subsection{Flow}