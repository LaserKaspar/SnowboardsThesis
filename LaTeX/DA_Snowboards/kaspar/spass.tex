\chapter{Spaß in Videospielen}

\section{Warum machen Spiele Spaß?}

McKee definiert Spaß sehr simpel: "`Vergnügen ohne Ziel"'. Nach dieser Definition kann alles Spaß sein, wenn man es für Vergnügen tut.

\subsection{Woher kommt Spaß?}

Gamedesigner Marc LeBlanc hat acht verschiedene Typen von situationsabhängigem Spaß definiert. Einige davon lassen sich gut auf Situationen unseres Spieles anwenden um genauer zu verstehen, woher der Spaß in dem Spiel kommen soll: "`drama, obstacle, social framework"'. 

Das Buch "`A Theory of Fun"'\cite{_theory_of_fun} definiert Spaß als das mentale meistern eines Problems\cite[S. 71]{_theory_of_fun}. Außerdem meint Raph Koster, dass Spaß aus "`richly interpretable situations"'\cite[S. 40]{_theory_of_fun}, also Situationen, die den Spielenden mehrere Möglichkeiten gegeben ein Problem zu lösen und Kreativität fördern, entsteht. 

Spaß kann also nur existieren, wenn Spielende immer neue Probleme lösen können oder bereits gelöste Probleme auf andere Wege ("`richly interpretable"') lösbar sind. Das Meistern eines Spieles, laut seiner Definition, kann also nur Spaß machen, wenn sich die Situationen je nach skill-level ändern oder sie auf andere Wege lösbar sind. Das verbessern der Reaktionsfähigkeit oder das ständige ausführen der selben Aktion macht keinen Spaß. Raph Koster hat außerdem einige wichtige Grundsteine für Spaß in Spielen gelegt, diese werden im praktischen Teil noch genauer behandelt.

\subsection{Spaß bei mehrspieler Spielen}

Ein weiterer Ursprung von Spaß bzw. Vergnügen lässt sich in sozialen Interaktionen finden\cite[S. 72]{_theory_of_fun}. Hier unterscheidet man zwischen kooperativen und kompetitiven Interaktionen, wobei beide Spaß erzeugen können. 
[competetive vs cooperative fun]

\subsubsection{Arena}
Party-Spiele finden in der "`Arena"' statt\cite[S. 65]{_art_of_gamedesign}. Im Vergleich zu anderen "`Arena-Spielen"' befinden sich die Spielenden bei Party-Spielen im selben Raum und können sich so auf noch mehr andere Weisen beeinflussen, die bei klassischen "`Arena-Spielen"', z.B. Shooter, nicht möglich ist. Es entsteht ein anderer, persönlicherer Umgang miteinander in der Spielwelt.\newline

\noindent Außerdem hat eine Studie herausgefunden, dass Spaß der mit anderen geteilt wird stärker empfunden wird als einsamer Spaß, das hat vermutlich einen evolutionären Ursprung\cite{_fun_is_more_fun}.

\section{Wie kann man Spaß "`kreieren"'?}

Wie vorher definiert gibt es verschiedene Arten von Spaß. Das Erlebnis der Spielenden lässt sich hauptsächlich durch die Aktionen beeinflussen, welche den Spielenden zu Verfügung gestellt werden und dessen Auswirkungen auf die Spielwelt oder andere Spielende.

\subsection{MDA-Framework}

Das MDA-Framework hilft bei der Analyse eines Spieles und dem iterativen Designprozess. Es wurde von dem GameDesigner Marc LeBlanc mitentwickelt und trennt den "`Konsum"' von Spielen in verschiedene Komponenten\cite{_mda}. Es ist unterteilt in Mechanics, Dynamics und Aesthetics. Die unterste Ebene beeinflusst immer das was drüber ist. Es beginnt mit dem Mechanics, diese beeinflussen alle ebenen darüber.

[MDA-Bild]

\subsubsection{Mechanics (Regeln)}

Regeln und Feedback-Loops die den Spielenden in seinen Aktionen limitieren. Sie beschreiben das Ziel des Spieles und wie Spielende es erreichen bzw. nicht erreichen können\cite[S.96]{_art_of_gamedesign}. Sie sind das was das Spiel im Zentrum ausmacht, selbst wenn man alles andere weg lässt. \cite[S.231]{_art_of_gamedesign}
 
\subsubsection{Dynamics (System)}

Das entstehende Verhalten der Spielenden welches aus den Mechanics hervorgeht\cite{_mda}. Sie beschreiben, wie sich das Spiel bzw. wie sich Systeme und Mechaniken im  im Laufe der Zeit verändern. Sie bieten dem Spieler ein Gefühl des Fortschritts und der Veränderung, hier spielt auch Feedback eine große Rolle, dazu kommen wir später. Hier kommen die Regeln, die Spielwelt und die Spielenden zusammen, Spielende werden mit der Zeit besser und das Spiel wird Schwerer und bietet so eine fesselnde Erfahrung.

\subsubsection{Aesthetics (Spaß)}

Die emotionalen Reaktionen der Spielenden. Sobald Spielende zum ersten mal mit dem Spiel interagieren, löst es Gefühle in ihnen aus. Ziel ist es eben diese Gefühle zu steuern und das Interesse der Spielenden so aufrecht zu erhalten.\newline

\noindent Der Game-Designer muss man das Framework im Überblock behalten. Designer verändern die Mechanics des Spieles, sehen aber gleichzeitig deren Einfluss auf die Aesthetics und müssen diese genauso berücksichtigen um das Spiel in die richtige Richtung zu lenken. 
Es ist auch möglich ein Spiel durch Mechaniken zu designen, allerdings, tappt man dann im Dunkeln, und man kann nie wissen, was dabei entsteht. \cite[S.56]{_art_of_gamedesign}

\subsection{Essential Experience}

Als Game-Designer möchte man oft eine Experience, ein Erlebnis, das man vorher definiert hat einfangen. Die "`Essential Experience"' versucht, wenn auch auf einem anderen Weg, die reale Experience einzufangen und durch Mechanics zu erzeugen. Das designen eines Spieles mithilfe der "`Essential Experience"' ermöglicht einem, das Spiel von der Experience getrennt zu sehen, und so entscheiden zu können, was an dem Spiel noch verändert werden muss bzw. welche Teile des Spieles nicht verändert werden dürfen und den Grundgedanken zu erhalten. \cite[S.55]{_art_of_gamedesign}

\subsection{Core Mechanics}

Die Core Mechanics sind die Grundbausteine/der Anfang des Spieles. Sie sind jene Mechaniken, die in dem Spieler die erwünschten Gefühle auslösen und ohne die das Spiel nicht funktionieren könnte. Sie beinhalten grundlegende Dinge wie Interaktionsmöglichkeiten, Fortbewegung und Fähigkeiten.

\subsection{Pacing, Flow}

Die Schwierigkeit von Spielen muss sich in einem gewissen Bereich aufhalten, der von dem Psychologen Cziksentmihalyi als "`Flow Channel"' bezeichnet wird, um die Aufmerksamkeit eines Spielers zu behalten.\cite[S.205]{_art_of_gamedesign}

[Bild: Flowchannel]

Als "`Pacing"' beschreibt man den Balanceakt der Herausforderung des Spieles und der Fähigkeiten der Spieler. Game-Designer haben dabei zum Ziel die Spielenden in einen Flowstate zu bringen.

Flow ist "`ein Gefühl der vollständigen und energiegeladenen Konzentration auf eine Tätigkeit, mit einem hohen Maß an Freude und Erfüllung"'\cite[S.204]{_art_of_gamedesign}.
Cziksentmihalyi beschreibt Flow etwas einfacher als einen Zustand der auftritt, wenn man sich freiwillig an seinem Limit befindet\cite{_flow}.

[Wie bringt man Flow in ein Spiel?]\cite{_theory_of_fun}

\subsection{Player Action Feedback}

Feedback ist auch in Spielen wichtig. Ohne Feedback wissen Spielende nicht ob das was sie tun richtig ist oder das Spiel das was sie tun wollten bereits aufgefasst hat. Visuelle und Auditives Feedback kann den Spieler über Fortschritt und Fehler informieren. Außerdem hilft es bei der Immersion, der Fähigkeit der Spielenden sich in das Spiel hineinzuversetzen.
