\chapter{Spaß in Videospielen}

\section{Warum machen Spiele Spaß?}

McKee definiert Spaß sehr simpel: "`Vergnügen ohne Ziel"'. Nach dieser Definition kann jede Tätigkeit, die für Vergnügen ausgeübt wird, Spaß machen\cite{_fun}.

\subsection{Woher kommt Spaß?}

Gamedesigner Marc LeBlanc hat acht verschiedene Typen von situationsabhängigem Spaß definiert. Einige davon lassen sich gut auf Situationen des in dieser Arbeit entwickelten Spieles anwenden, um genauer zu verstehen, woher der Spaß in dem Spiel kommen soll: "`drama, obstacle, social framework"'. 

Das Buch "`A Theory of Fun"'\cite{_theory_of_fun} definiert Spaß als das mentale meistern eines Problems\cite[S. 71]{_theory_of_fun}. Außerdem meint Raph Koster, dass Spaß aus "`richly interpretable situations"'\cite[S. 40]{_theory_of_fun}, also Situationen, die den Spielenden mehrere Möglichkeiten geben ein Problem zu lösen und Kreativität fördern, entsteht. 

Spaß kann also nur existieren, wenn Spielende immer neue Probleme lösen können oder bereits gelöste Probleme auf andere Wege ("`richly interpretable"') lösbar sind. Das Meistern eines Spieles, laut seiner Definition, kann also nur Spaß machen, wenn sich die Situationen je nach skill-level ändern oder sie auf andere Wege lösbar sind, mehr dazu in Kapitel \ref{_flow}. Das ständige Ausführen derselben Aktion macht somit keinen Spaß. Raph Koster hat außerdem einige wichtige Grundsteine für Spaß in Spielen gelegt, diese werden im praktischen Teil noch genauer behandelt.

\subsection{Spaß bei mehrspieler Spielen}

Ein weiterer Ursprung von Spaß bzw. Vergnügen lässt sich in sozialen Interaktionen finden\cite[S. 72]{_theory_of_fun}. Hier unterscheidet man zwischen kooperativen und kompetitiven Interaktionen, beide sind tief in uns verborgene Instinkte\cite[S. 310]{_art_of_gamedesign}. Einen Unterschied im Spaßfaktor gibt es hier vermutlich nicht\cite{_competition_vs_cooperation}, und Spiele haben auch die Möglichkeit beide Verhalten in verschiedenen Situationen zu fördern.

\begin{itemize}
\item\subsubsection{Competitive Play}
Laut Jesse Schell ist ein wichtiger Aspekt von kompetitiven Spaß Schadenfreude, Freude, die man im Unglück anderer verspürt\cite[S. 196]{_art_of_gamedesign}.
Um den Wettbewerb zwischen Spielenden, oder Teams zu erzeugen, müssen einige Voraussetzungen erfüllt sein. Der wichtigste Aspekt ist, dass sich Spielende untereinander messen. Es muss für erfahrene Spielende möglich sein, mit einer relativen Sicherheit sagen zu können, dass sie nicht von einem Anfangenden geschlagen werden\Cite[S. 311]{_art_of_gamedesign}.

\item\subsubsection{Cooperative Play\label{_cooperative_play}}
Spiele die Kooperation fördern, können in Spielenden positive Gefühle wie soziale Zugehörigkeit, Möglichkeiten zur sozialen Vernetzung und die Förderung der sozialer Integration bieten\cite{_putting_the_fun_factor_into_gaming}. Die Analyse deutet auch darauf hin, dass Flow (mehr dazu in Kapitel \ref{_flow}) in sozialen Spielen auftritt, insbesondere im kooperativen Gameplay\cite{_putting_the_fun_factor_into_gaming}. Soziale Kontexte können so die emotionalen Erfahrungen des Spielens verbessern. Wie man Kooperation in Spielen erzeugt wird, wird in Kapitel \ref{_teamwork_erzeugen} erforscht.
\end{itemize}

Außerdem belegt Quelle \cite{_fun_is_more_fun}, dass Spaß, der mit anderen geteilt wird, stärker empfunden wird, als einsamer Spaß, das hat vermutlich einen evolutionären Ursprung.

\subsubsection{Arena}
Party-Spiele finden in der "`Arena"' statt\cite[S. 65]{_art_of_gamedesign}. Im Vergleich zu anderen "`Arena-Spielen"' befinden sich die Spielenden bei Party-Spielen im selben Raum und können sich so auf noch mehr Weisen beeinflussen, die bei klassischen "`Arena-Spielen"', z.B. Shooter, nicht möglich sind. Es entsteht ein anderer, persönlicherer Umgang miteinander sowohl innerhalb als auch außerhalb der Spielwelt.

\section{Wie kann man Spaß "`kreieren"'?}

Wie vorher definiert gibt es verschiedene Arten von Spaß. Das Erlebnis der Spielenden lässt sich hauptsächlich durch die Aktionen beeinflussen, welche den Spielenden zu Verfügung gestellt werden und deren Auswirkungen auf die Spielwelt oder andere Spielende.

\subsection{MDA-Framework}

Das MDA-Framework hilft bei der Analyse eines Spieles und dem iterativen Designprozess. Es wurde von dem Game-Designer Marc LeBlanc mitentwickelt und trennt den "`Konsum"' von Spielen in verschiedene Komponenten\cite{_mda}. Es ist unterteilt in Mechanics, Dynamics und Aesthetics. Eine Ebene beeinflusst immer die darüber liegenden. Es beginnt mit den Mechanics, diese beeinflussen alle Ebenen darüber.

\begin{figure}[H]
	\centering
	\includegraphics[width=10cm]{images/kaspar/mda}
	\caption{src: ResearchGate}
\end{figure}

\begin{itemize}
\item\subsubsection{Mechanics (Regeln)}

Regeln und Feedback-Loops die die Spielenden in ihren Aktionen limitieren. Die Mechanics beschreiben das Ziel des Spieles und wie Spielende es erreichen bzw. nicht erreichen können\cite[S.96]{_art_of_gamedesign}. Sie beschreiben den Kern des Spieles, selbst wenn man alles andere entfernt, bleiben die Mechanics erhalten\cite[S.231]{_art_of_gamedesign}.
 
\item\subsubsection{Dynamics (System)}

Das entstehende Verhalten der Spielenden welches aus den Mechanics hervorgeht\cite{_mda}. Sie beschreiben, wie sich das Spiel bzw. wie sich Systeme und Mechaniken im Laufe der Zeit verändern. Sie bieten dem Spieler ein Gefühl des Fortschritts und der Veränderung, hier spielt auch Feedback eine große Rolle, dazu mehr in Kapitel \ref{_feedback}. Hier kommen die Regeln, die Spielwelt und die Spielenden zusammen, Spielende werden mit der Zeit besser und das Spiel wird schwerer und bietet so eine fesselnde Erfahrung.

\item\subsubsection{Aesthetics (Spaß)}

Die emotionalen Reaktionen der Spielenden. Sobald Spielende zum ersten Mal mit dem Spiel interagieren, löst es Gefühle in ihnen aus. Ziel ist es eben diese Gefühle zu steuern und das Interesse der Spielenden so aufrecht zu erhalten.\newline
\end{itemize}

\noindent Als Game-Designer muss man das Framework im Überblick behalten. Designer verändern die Mechanics des Spieles, sehen aber gleichzeitig deren Einfluss auf die Aesthetics und müssen diese berücksichtigen, um das Spiel in die richtige Richtung zu lenken. Es ist möglich den Anfang eines Spieles zu designen, indem man eine interessante Mechanic erstellt und sieht welche Experience daraus entsteht, oder indem man Aesthetics festlegt, und die Mechaniken des Spieles so anpasst, dass dabei die gewünschte Experience entsteht. "`Wenn man nicht weiß, was man will, ist es einem vielleicht egal, was man bekommt. Wenn man aber weiß, was man will (...), dann muss man sich überlegen, wie man diese Essential Experience vermittelt."'\cite[S.55]{_art_of_gamedesign}.

\subsection{Essential Experience}

Oft möchte der Designer eine Experience, ein Erlebnis, das man vorher definiert hat, einfangen. Die "`Essential Experience"' versucht, wenn auch auf einem anderen Weg, die reale Experience einzufangen und durch Mechanics zu erzeugen. Das designen eines Spieles mithilfe der "`Essential Experience"' ermöglicht einem, das Spiel von der Experience getrennt zu sehen, und so entscheiden zu können, was an dem Spiel noch verändert werden muss bzw. welche Teile des Spieles nicht verändert werden dürfen um den Grundgedanken zu erhalten\cite[S.55]{_art_of_gamedesign}.

\subsection{Core Mechanics}

Die Core Mechanics sind die Grundbausteine/der Anfang des Designprozesses des Spieles. Sie sind jene Mechaniken, die in dem Spieler die erwünschten Gefühle auslösen und ohne welche das Spiel nicht funktionieren könnte. Sie beinhalten die grundlegendsten Dinge wie Interaktionsmöglichkeiten, Fortbewegung und Fähigkeiten.

\subsection{Pacing, Flow\label{_flow}}

Der Schwierigkeitsgrad von Spielen muss sich in einem gewissen Bereich aufhalten, der von dem Psychologen Cziksentmihalyi als "`Flow Channel"' bezeichnet wird, um die Aufmerksamkeit eines Spielendens zu behalten\cite[S.205]{_art_of_gamedesign}.

\begin{figure}[H]
	\centering
	\includegraphics[width=10cm]{images/kaspar/flowchannel}
	\caption{Flowchannel - The Art of Game Design\cite{_art_of_gamedesign}}
\end{figure}

Als "`Pacing"' beschreibt man die durch die Spielenden gefühlte Intensität und den Rhythmus des Gameplays\cite{_the_level_design_book}. Game-Designer haben dabei zum Ziel die Spielenden in einen Flowstate zu bringen.

Flow ist "`ein Gefühl der vollständigen und energiegeladenen Konzentration auf eine Tätigkeit, mit einem hohen Maß an Freude und Erfüllung"'\cite[S.204]{_art_of_gamedesign}.
Cziksentmihalyi beschreibt Flow etwas einfacher als einen Zustand der auftritt, wenn man sich freiwillig an seinem Limit befindet\cite{_flow}.
Laut Jesse Schell müssen es um Flow zu erzeugen fünf Voraussetzungen gegeben sein\cite[S.211]{_art_of_gamedesign}. Die zwei wichtigsten sind klare Ziele, die die Spielenden auch verfolgen und von denen sie nicht abschweifen, und die passende Schwierigkeit und Intensität für die Spielenden. Es ist allerdings zu beachten, dass Flow nicht gleichzusetzen mit Spaß ist. Spaß setzt keinen Flow voraus und es gibt auch Flow der nicht unbedingt Spaß macht, beispielsweise das Meistern einer Tätigkeit\cite[S.75]{_theory_of_fun}.

\subsection{Player Action Feedback\label{_feedback}}

Feedback ist ein sehr wichtiger Teil von Spielen. Ohne Feedback wissen Spielende nicht, ob ihre Aktionen richtig sind oder das Spiel sie verarbeitet hat. Visuelles und Auditives Feedback kann die Spielenden über Fortschritt und Fehler informieren. Außerdem hilft es den Spielenden gedanklich vollkommen in die Spielwelt einzutauchen.