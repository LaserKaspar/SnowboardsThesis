\documentclass[diplom,german]{hgbthesis}
% Warning because: https://tex.stackexchange.com/questions/177137/how-to-add-a-document-class-to-texstudio

% Zulässige Class Options: 
%   Typ der Arbeit: diplom, master (default), bachelor, praktikum 
%   Hauptsprache: german (default), english
%%------------------------------------------------------------

% Zusatzpakete (bei Bedarf einkommentieren):
% \usepackage{enumitem}    % erlaubt Listen mit einem <key>=<value> - Format
% \usepackage{tikz}        % ermöglicht das Zeichnen von Grafik-Elementen (zB Linien, Kreise etc.) - Bsp.: https://www.overleaf.com/learn/latex/TikZ_package
% \usetikzlibrary{arrows,shapes,positioning} % Zusatzbibliothek für tikz (zB Pfeil etc.)
\usepackage{pdfpages}      % ermöglicht das includieren (mit \include) von pdf-Dateien

\usepackage{listings}

\graphicspath{{images/}}   % Verzeichnis mit den Bildern (zB für \figure)
%\logofile{spengerlogo_cmyk_small}   % Name des .pdf mit dem Logo

\addbibresource{buketits/b-literatur.bib}  % Datei(en) mit der Bibliothek (.bib) für das Quellenverzeichnis 
\addbibresource{kaspar/k-literatur.bib}  % Datei(en) mit der Bibliothek (.bib) für das Quellenverzeichnis 
\addbibresource{wehinger/w-literatur.bib}  % Datei(en) mit der Bibliothek (.bib) für das Quellenverzeichnis 
\addbibresource{memic/m-literatur.bib}  % Datei(en) mit der Bibliothek (.bib) für das Quellenverzeichnis 
% Beispiel für mehrere .bib files: \bibliography{schueler1/literatur,schueler2/literatur}

%%%----------------------------------------------------------
\begin{document}
%%%----------------------------------------------------------

% offizielle Formulare inkludieren
\includepdf[page=-]{formulare/HTL_RDP_Titelseite_DA_A4}
\includepdf[page=-]{formulare/HTL_RDP_Dokumentation_DA_DE_A4}
\includepdf[page=-]{formulare/HTL_RDP_Dokumentation_DA_EN_A4}

%\begin{minipage}{\textwidth}
%	\includepdf[scale=0.85,pages=1]{formulare/HTL_RDP_Titelseite_DA_A4}
%\end{minipage}

% Daten zur Arbeit: _________--------------------------------
\title{"'Tricks 'n' Treats"': Entwicklung eines asymmetrischen VR- und Couch-Partyspiels}
\author{Ian Memic, Simon Buketits, Emma Wehinger und Felix Kaspar}
\studiengang{Mediendesign - Gamedesign}
\studienort{Wien}
\abgabedatum{2023}{03}{31}	% {YYYY}{MM}{DD}

%%%----------------------------------------------------------
\frontmatter
\maketitle
\tableofcontents
%%%----------------------------------------------------------

%\include{vorwort}		% ggfs. weglassen

\chapter{Kurzfassung}

Diese Diplomarbeit befasst sich mit der Entwicklung des asymmetrischen lokalen Multiplayer-Spieles Tricks ‘n’ Treats. 
Der erste Teil der Arbeit befasst sich mit der Entwicklung von asymmetrischen Virtual Reality Spielen, wobei ein besonderer Fokus auf Performance-Optimierung, den aufgetretenen Problemen und technischen Limitationen von asymmetrischen Virtual Reality Spielen. 
Im zweiten Teil wird genauer bearbeitet, wie sich die Dynamiken zwischen PC und VR bilden, wie man diese beeinflusst und im Falle des Spieles eine gute Interaktion zwischen den beiden konstruieren kann. 
Der dritte Teil der Arbeit befasst sich mit der Welt und Atmosphäre des Spiels. Besonders widmet er sich wie Farb- und Formgebung die Gefühle und Aktionen der SpielerInnen beeinflussen. 
Der vierte Teil beschäftigt sich mit dem 3D-Modeling und geht spezifisch auf die Performance Optimierung der 3D-Assets ein. Dabei liegt ein großer Fokus auf der Recherche und Verwendung verschiedener Optimierungs-Methoden und deren Vor- und Nachteile.
   % vom Team gemeinsam zu verfassen
\include{abstract}		% vom Team gemeinsam in englisch zu verfassen	

%%%----------------------------------------------------------
\mainmatter         % Hauptteil (ab hier arab. Seitenzahlen)
%%%----------------------------------------------------------

% Trennseite erzeugt ein Titelblatt für eine neue Arbeit (zB wenn die Arbeit A endet und die Arbeit B beginnt)
% Parameter: {}Titel der Arbeit}{Autor}{Betreuer}{Abteilung}
% Beispiele: 
%	\trennseite{Gestaltpsychologie und stilisiertes Game Audio}{Noah Diem}{Dipl.Ing. Michael Schreiber}{Mediendesign - Gamedesign}
%	\trennseite{Collection and analysis of gameplay metrics}{Ian Hornik}{Mag. Andreas Metz}{Media design - Game design}

\trennseite{Technische Implementierung und Optimierung von asymmetrischen VR-Spielen}{Simon Buketits}{Kristian Ljubek, BSc MSc}{Mediendesign - Gamedesign}
\chapter{Einleitung}
\label{cha:sa_Einleitung}

\section{Virtuelle Realität}
Die Virtuelle Realität (Virtual Reality, VR) ist eine computergenerierte virtuelle Umgebung, welche in Echtzeit berechnet wird. In dieser ist es möglich sich umzuschauen und je nach Anwendung kann man sich auch Bewegen und beliebig mit Objekten interagieren. Es gibt zwei verschiedene Möglichkeiten Virtual Reality zu erleben. Einerseits gibt es speziell angefertigte Räume, in welchen Großbildleinwände angebracht sind, andererseits gibt es auch Head-Mounted-Displays (VR Brillen, die man aufsetzt), welche hier im Fokus stehen werden. In der Virtuellen Realität gibt es eine breit gefächerte Reichweite an Spielen, in welchen man verschiedenste Situationen erleben kann. Anfangs waren Spiele für VR nur simple Testräume, in welchen man mit Objekten interagieren konnte. Mittlerweile ist es jedoch möglich riesige Spiele mit einer eigenen Geschichte zu erleben. Die Interaktionen, welche damals ein ganzes Spiel ausmachten, sind seit einiger Zeit nur noch die Basis, worauf neue Spiele aufbauen.

\subsection{Asymmetrische VR Spiele}
Die meisten VR-Spiele kapseln einen von der Außenwelt ab. Sobald man die VR-Brille aufsetzt, befindet man sich in einer neuen, digitalen Welt. Eine Zeit lang war die zwischenmenschliche Interaktion in der virtuellen Realität gar nicht möglich. Seit neuerem gibt es auch Spiele, welche unterstützen, dass VR-Spieler über das Internet miteinander spielen können (Online-Multiplayer). Somit braucht man jedoch zwei HMDs (VR-Brillen), zwei PCs und man kann in der Regel nicht nebeneinander spielen. Asymmetrische VR-Spiele versuchen, reguläre PC-Spiele mit VR-Spielen zu kombinieren. So können z.B. zwei Freunde mit einem PC und einer VR-Brille miteinander spielen. Das funktioniert z.B., indem die zwei Spieler sich in derselben Spielwelt befinden, jedoch unterschiedliche Charaktere steuern. Durch die verschiedenen Steuerungsmöglichkeiten (VR-Controller, Tastatur und Maus, Gamepad) ergibt sich oft automatisch eine bestimme Rollenverteilung. So könnte der PC-Spieler z.B. einen Menschen spielen, welcher durch ein Level manövrieren muss und zeitgleich steuert der VR-Spieler einen Riesen, welcher den Menschen behindern muss, indem er ihm z.B. Steine in den Weg legt.

\section{Zielsetzung}
Das Ziel dieser Arbeit ist, die technische Implementierung und Optimierung von asymmetrischen VR Spielen genauer zu Untersuchen und gefundene technische Probleme genauer zu erläutern.


\section{Warum Asymmetrisches VR}


\chapter{Überblick über die Materie}
\label{cha:sa_Einleitung}


\section{Farbpsychologie}

\subsection{Was ist Farbe?}
Wissenschaftlich betrachtet, sind Farben Eindrücke, die unsere Sinne durch die Augen und das Gehirn vermittelt bekommen. Farben entstehen durch die Absorption und Reflexion, die das Licht auf Oberflächen wirft. Das menschliche Auge ist für die Sinnesempfindung von Farben verantwortlich, die dann wiederum im Gehirn verarbeitet werden. Man kann Farbe auch als Eigenschaft des Lichts betiteln. Prinzipiell nimmt jeder Mensch Farben und Farbtöne anders wahr. Welche Farbe wahrgenommen wird, hängt von dem Gegenstand ab, der Lichtquelle und den Augen des Betrachters. Licht hat außerdem auch bestimmte Wellenlängen, je nach Länge der Welle nimmt das Auge eine andere Farbe war. Schwarz ist keine Farbe, denn sie kann auch in Abwesenheit des Lichts und völliger Dunkelheit existieren.
Weißes Licht kann mithilfe eines Prismas in Spektralfarben gebrochen werden. Spektralfarben sind diejenigen Farben, die für das menschliche Auge einen sichtbaren Farbeindruck und somit einen Farbton hinterlassen. Insgesamt gibt es sechs verschiedene Spektralfarben. Spektralfarben werden umgangssprachlich auch „Regenbogenfarben“ genannt. Zu ihnen zählen Rot, Orange, Gelb, Grün, Blau, Indigo und Violett. Die durch die Brechung von Licht entstandenen Farben können nicht in weitere Farbtöne zerlegt werden. Das Licht kann manchmal auch infrarotes Licht oder ultraviolettes Licht enthalten, welches für das bloße Auge nicht sichtbar ist.
Wenn man jedoch gelbes Licht oder jede andere Spektralfarbe durch ein Prisma fallen lässt, bekommt man dieselbe Farbe raus, die man im Prisma brechen wollte. Man kann also als Schlussfolgerung daraus ziehen, das weißes Licht das einzige ist, was durch ein Prisma in Spektralfarben gebrochen werden kann.

\begin{figure}[h]
	\centering
	\includegraphics[width=10cm]{Farbprisma}
	\caption{\cite{_steam_hardware}}
\end{figure}

In der Kunst ist Farbe das beste Mittel, um visuell Emotionen auszudrücken oder Emotionen bei Menschen zu erzeugen. Farben sind das beste Instrument, wenn das darum geht, bei einem Menschen Gefühle und Emotionen auszulösen. Aber nicht nur, dass es wird dem Betrachter dadurch auch eine bestimmte Stimmung vermittelt. Künstler können dadurch gezielt bei Menschen bestimmte Gefühle auslösen und das alleine durch das Einsetzen von bestimmten Farben und Farbtönen.
Dabei muss beachtet werden, dass Farben in verschiedenen Kulturen verschiedene Bedeutungen haben und die Symbolik hinter manchen Farbtönen in bestimmten Ländern und Kulturkreisen eine andere ist. Doch grundsätzlich kann man jeder Farbe eine bestimmte Bedeutung und Emotion zuschreiben, die vielleicht in manchen Kulturen ein wenig abweicht, aber im Grunde genommen immer sehr ähnlich ist.
Man kann aber klar die kulturelle Bedeutung von Farben mit der psychologischen Bedeutung von Farben trennen. Die wichtigsten Farben für Kunst und Design sind die Farben Blau, Gelb, Grün, Orange, Rosa, Rot, Violett, Grau und Braun. Jede dieser Farben hat eine andere Wirkung auf die Gefühle und Emotionen der Betrachter. Der psychologische Effekt hinter der Wirkung einer Farbe hat rein gar nichts mit der kulturellen Interpretation dieser zu tun. 
Es gibt eine endliche Anzahl an Variationen, die eine Farbe für das menschliche Auge annehmen kann. Grundsätzlich gibt es eine unendliche Anzahl an Schattierungen, die eine Farbe haben kann, wobei die meisten für das Sinnesorgan eines Menschen nicht sichtbar sind. Die Schattierung einer Farbe ändert prinzipiell den psychologischen Effekt nicht.


\subsection{Der Farbkreis und Farbschemas}
Der Farbkreis besteht normalerweise aus 12 Farben, er kann aber auch aus 24 und manchmal aus bis zu 48 Farben bestehen, das hängt aber von dem Ersteller des Farbkreises ab. Das grundlegende Farbrad besteht aber jedoch nur aus 12 Farben. Um den Farbkreis besser zu verstehen, muss man sich zuerst die additive und die subtraktive Farbmischung ansehen.  

Farben können in Primär -, Sekundär -und Tertiärfarben aufgeteilt werden. Dabei unterscheidet man zwischen der additiven Farbmischung und der subtraktiven Farbmischung. 

Bei der subtraktiven Farbmischung, auch CMYK-Modell genannt, sind die Primärfarben Gelb, Magenta und Cyan, sie können nicht aus anderen Farben gemischt werden. Jedoch kann aus ihnen jede existierende Farbe gemischt werden. Die Sekundärfarben in diesem Modell sind Rot, Grün und Blau und die Tertiärfarbe ist Schwarz. Mischt man nun die Primärfarben gelb, Magenta und Cyan, erhält man infolge dessen die Sekundärfarben Rot, Grün und Blau. Die gesamte Mischung ergibt dann Schwarz.

\centerfirst
\includegraphics[width=8cm]{images/Flyeralarm_CMYK_de}


Die additive Farbmischung, auch RGB-Modell genannt, besteht aus den Lichtfarben Rot, Grün und Blau. Rot, Grün und Blau sind in diesem Modell die Primärfarben, also Grundfarben. Die Sekundärfarben sind Gelb, Magenta und Cyan. Die Tertiärfarbe ist Weiß. Diese entsteht durch die Mischung der Grundfarben Rot, Grün und Blau.

\includegraphics[width=8cm]{images/Flyeralarm_RGB_de}



\subsection{Farbtemperatur}

\subsection{Farbrelativität}

\subsection{Farbsättigung}

\subsection{Erstellung von Farbpaletten}

\subsection{Farbe, Licht und Schatten}

\subsection{Wie wirkt sich Farbe auf die Stimmung aus?}



\section{Formpsychologie}

\subsection{Gestalt und Form}

\subsection{Linien und Linien- Stile}

\subsection{Kompositionslinien}

\subsection{Wie Formen und Linien Emotionen erzeugen}



\section{Environment Design}

\subsection{Character und Environment Shapes}

\subsection{Character Centric Environment Design }

\subsection{Bildkomposition}







\chapter{Möglichkeiten zur Implementierunge}
\label{cha:sa_Einleitung}

\section{Nutzung von Farben in Spielen}
Die Nutzung von Farben in Videospielen war mit ihrer Entstehung sehr mit dem damaligen Stand der Technik verbunden. 1972 wurde die Farbüberlagerung erfunden und hat es ermöglicht, dass Videospiele in Farbe dargestellt werden können. Davor konnten Spiele nur in Schwarz-Weiß oder Monochrom dargestellt werden. Ab Ende der 1980er sind fast alle Videospiele in Farbe. 1985 veröffentlichte Sega das Sega Master System, welches Spiele mit 32 verschiedenen Farben darstellen konnte. Schon 9 Jahre später veröffentlicht Sony die Playstation, welche es ermöglicht, Spiele mit rund 16.7 Millionen Farben darzustellen. Heutzutage können Spiele mit über 16.7 Milliarden verschiedenen Farben dargestellt werden. 

Durch die Nutzung von Farben kann man dem Spieler Gefühle und Emotionen auf eine sehr einfache und natürliche Weise vermitteln. Prinzipiell ist das Farbempfinden von Kulturkreisen immer etwas unterschiedlich, dennoch gibt es auf die meisten Farben eine universelle Reaktion. Wie Menschen Farben wahrnehmen, hängt nicht nur von ihrer Kultur ab, sondern auch von ihrem Alter, ihrer Herkunft, ihrem Geschlecht und ihrem ethnischen Hintergrund. Wenn ein Spieler in einem Spiel zum Beispiel Eierfarbe Rot sieht, wird ihm signalisiert, dass er in Gefahr ist. Rot steht prinzipiell für Energie, Liebe, Selbstbewusstsein und Leidenschaft, wird aber auch als Warnfarbe gesehen. Rot wird mit Gefahren in Verbindung gesetzt. Explosionen, Feuer und Blut etc. wären dafür ein gutes Beispiel. In vielem Actionspielen wird unter anderem ein an den Ecken und Kanten roter Bildschirm eingesetzt, wenn der Spieler verletzt ist. Das signalisiert dem Spieler, dass er ab jetzt wachsamer und vorsichtiger sein muss. 

In vielen MMORPGs oder in Singleplayer-Spielen, werden viele sammelbaren Objekte, Items oder Schatztruhen in bestimmten Farben angezeigt. Meist werden die Farben Weiß, Grün, Blau, Lila und Gold verwendet, wobei Weiß weniger gut ist, als Violett und Gold. Violett wird mit dem Adel verbunden, denn früher konnten sich nur sehr wohlhabende Menschen das violette Farbpigment leisten. Die Farbe Violett steht für Eleganz, Würde und Raffinesse. Die Farbe hat aber auch etwas Magisches und Mystisches an sich. Aus diesem Grund sind sehr seltene und außergewöhnliche Objekte und Items meist Violett, da sie dem Spieler signalisieren sollen, dass er etwas Besonderes und Wertvolles gefunden hat. 

\subsection{Beispiel – Mario Kart 8}

\subsection{Beispiel – Journey}
Journey ist ein Adventure-Spiel, welches ohne jegliche Worte auskommt. Es wurde erstmals 2012 für die Playstation 3 veröffentlicht. Der Entwickler des Spiels ist „thatgamecompany“ und der Publisher „Sony Interactive Entertainment“ und „Annapurna Interactive“. Das Spiel startet in einer Wüste, in der Ferne liegt ein großer Berg, an dessen Gipfel ein Lichtstrahl in den Himmel ragt. Der Spieler spielt eine Figur in einer roten Robe. Auf dem Weg zum Berg kann der Spieler mit bestehender Internetverbindung auf einen anderen Spieler treffen. Die Spielenden können nicht miteinander reden, aber sich dennoch gegenseitig helfen. Das Zeil von Journey ist es, den in der Ferne liegenden Berg zu erreichen. 


\section{Nutzung von Formen in Spielen}

\subsection{Beispiel – Mario Kart 8}

\subsection{Beispiel – Journey}




\chapter{PC und VR Welt}
\section{VR Spieler}
Dieses Dokument ist als vorwiegend technische Starthilfe für das
Erstellen einer Masterarbeit (oder Bachelorarbeit) mit \latex
gedacht und ist die Weiterentwicklung einer früheren
Vorlage\footnote{Nicht mehr verfügbar.} für das Arbeiten mit
Microsoft \emph{Word}. Während ursprünglich daran gedacht war, die
bestehende Vorlage einfach in \latex zu übernehmen, wurde rasch
klar, dass allein aufgrund der großen Unterschiede zum Arbeiten
mit \emph{Word} ein gänzlich anderer Ansatz notwendig wurde. Dazu
kamen zahlreiche Erfahrungen mit Diplomarbeiten in den
nachfolgenden Jahren, die zu einigen zusätzlichen Hinweisen Anlass gaben.


\subsection{Spell System}
Zitat \cite{bobsch 123 bobsch}

\subsection{VR Umgebung}
Zitat \cite{bobsch 123 bobsch}


\section{PC Spieler}
Dieses Dokument ist als vorwiegend technische Starthilfe für das
Erstellen einer Masterarbeit (oder Bachelorarbeit) mit \latex
gedacht und ist die Weiterentwicklung einer früheren
Vorlage\footnote{Nicht mehr verfügbar.} für das Arbeiten mit
Microsoft \emph{Word}. Während ursprünglich daran gedacht war, die
bestehende Vorlage einfach in \latex zu übernehmen, wurde rasch
klar, dass allein aufgrund der großen Unterschiede zum Arbeiten
mit \emph{Word} ein gänzlich anderer Ansatz notwendig wurde. Dazu
kamen zahlreiche Erfahrungen mit Diplomarbeiten in den
nachfolgenden Jahren, die zu einigen zusätzlichen Hinweisen Anlass gaben.

\subsection{Spieler Größe}
Zitat \cite{bobsch 123 bobsch}

\subsection{Fähigkeiten}
Zitat \cite{bobsch 123 bobsch}

\section{Erwartete Probleme}
Dieses Dokument ist als vorwiegend technische Starthilfe für das
Erstellen einer Masterarbeit (oder Bachelorarbeit) mit \latex
gedacht und ist die Weiterentwicklung einer früheren
Vorlage\footnote{Nicht mehr verfügbar.} für das Arbeiten mit
Microsoft \emph{Word}. Während ursprünglich daran gedacht war, die
bestehende Vorlage einfach in \latex zu übernehmen, wurde rasch
klar, dass allein aufgrund der großen Unterschiede zum Arbeiten
mit \emph{Word} ein gänzlich anderer Ansatz notwendig wurde. Dazu
kamen zahlreiche Erfahrungen mit Diplomarbeiten in den
nachfolgenden Jahren, die zu einigen zusätzlichen Hinweisen Anlass gaben.

\subsection{Wenige Bilder pro Sekunde}
Zitat \cite{bobsch 123 bobsch}

\subsection{Bewegungskrankheit}
Zitat \cite{bobsch 123 bobsch}

\subsection{Platz und Bewegung}
Zitat \cite{bobsch 123 bobsch}

\subsection{Interaktionen}
Zitat \cite{bobsch 123 bobsch}
\chapter{Performance Optimierung}
\section{VR Performance}
Dieses Dokument ist als vorwiegend technische Starthilfe für das
Erstellen einer Masterarbeit (oder Bachelorarbeit) mit \latex
gedacht und ist die Weiterentwicklung einer früheren
Vorlage\footnote{Nicht mehr verfügbar.} für das Arbeiten mit
Microsoft \emph{Word}. Während ursprünglich daran gedacht war, die
bestehende Vorlage einfach in \latex zu übernehmen, wurde rasch
klar, dass allein aufgrund der großen Unterschiede zum Arbeiten
mit \emph{Word} ein gänzlich anderer Ansatz notwendig wurde. Dazu
kamen zahlreiche Erfahrungen mit Diplomarbeiten in den
nachfolgenden Jahren, die zu einigen zusätzlichen Hinweisen Anlass gaben.


\subsection{Layer Culling}
Zitat

\section{PC Performance}
Dieses Dokument ist als vorwiegend technische Starthilfe für das
Erstellen einer Masterarbeit (oder Bachelorarbeit) mit \latex
gedacht und ist die Weiterentwicklung einer früheren
Vorlage\footnote{Nicht mehr verfügbar.} für das Arbeiten mit
Microsoft \emph{Word}. Während ursprünglich daran gedacht war, die
bestehende Vorlage einfach in \latex zu übernehmen, wurde rasch
klar, dass allein aufgrund der großen Unterschiede zum Arbeiten
mit \emph{Word} ein gänzlich anderer Ansatz notwendig wurde. Dazu
kamen zahlreiche Erfahrungen mit Diplomarbeiten in den
nachfolgenden Jahren, die zu einigen zusätzlichen Hinweisen Anlass gaben.

\subsection{Level of Detail}
Zitat 

\section{Mesh Optimierungen}
Dieses Dokument ist als vorwiegend technische Starthilfe für das
Erstellen einer Masterarbeit (oder Bachelorarbeit) mit \latex
gedacht und ist die Weiterentwicklung einer früheren
Vorlage\footnote{Nicht mehr verfügbar.} für das Arbeiten mit
Microsoft \emph{Word}. Während ursprünglich daran gedacht war, die
bestehende Vorlage einfach in \latex zu übernehmen, wurde rasch
klar, dass allein aufgrund der großen Unterschiede zum Arbeiten
mit \emph{Word} ein gänzlich anderer Ansatz notwendig wurde. Dazu
kamen zahlreiche Erfahrungen mit Diplomarbeiten in den
nachfolgenden Jahren, die zu einigen zusätzlichen Hinweisen Anlass gaben.



\trennseite{Design eines asymmetrischen Local-Multiplayer-Party-Games unter Verwendung des MDA-Frameworks}{Felix Kaspar}{René Ksuz, BSc MA}{Mediendesign - Gamedesign}
\chapter{Begriffsdefinitionen}

Bei "`Tricks 'n' Treats"' handelt es sich um ein asymmetrisches Couch-Party VR-Spiel. [Hier fehlt, denke ich, eine kurze Beschreibung von unserem Spiel.]

In diesem Kapitel werden die in "`Tricks 'n' Treats"' verwendeten Technologien erklärt und die grundlegenden Konzepte des Game-Designs die verwendet werden um die  Erlebnisse der und die Interaktionen zwischen den Spielenden zu beeinflussen.

\section{VR trifft Couchparty}

\subsection{Was ist Virtual Reality?}

Virtual Reality bezeichnet Bilder und Töne, die von einem Computer erzeugt werden und dem Benutzer, der mit Hilfe von Sensoren mit ihnen interagieren kann, fast real erscheinen.\cite{_oxford_dict} Sie ist trotz anderer Anwendungsbereiche, dank der Immersion und Interaktionsmöglichkeiten die sie bietet, besonders in der Gaming-Industrie relevant geworden. [bitkom research]. Spielenden können neue Erlebnisse geboten werden, welche wiederum eigene  Game-Design-Fragen aufbringen die es zu beantworten gilt.

VR bringt neben der Interaktivität noch einen weiteren gewaltigen Vorteil: Intuition. Greifen, Umschauen, Bewegen, alles funktioniert wie man es erwartet. Gaming-Neulinge haben häufige Probleme die richtigen Tasten auf der Tastatur zu drücken und auch klassische Gamepads sind nicht optimal, in VR gibt es zum einen weniger Knöpfe die gedrückt werden können und zum anderen wird viel unterbewusster gearbeitet. Spielende verstehen in VR schneller wie er das Spiel gesteuert wird und können sich mehr Gedanken darüber machen was das Ziel des Spiels ist. Insbesondere für Party-Spiele (Zu denen wir gleich kommen werden.) ist das sehr wichtig. [VR ist ein gutes Medium für Party-Spiele -> Es ist "`einfach"' man versteht schneller was man tun muss]

[
Bild: 
	Gegenüberstellung von Interaktionen VR - Nicht VR (Greifen, Knöpfe)
	Bewegen? Drehen, Umschauen, Firstperson
]
[Greifen Tastatur-E vs VR-Grab (hl vs alyx)]
[field: interaction design]

\subsection{Was ist Couchparty?}

Couchparty-Spiele, oft auch nur Party-Spiele oder local-multiplayer-games, sind Computerspiele, die eine Gruppe (die sich meist untereinander kennt) gemeinsam spielt. Anders als bei klassischen Online-Multiplayer-Spielen werden diese zusammen in einem Raum gespielt. Sie sind jedoch nicht zu verwechseln mit LAN klassikern wie Counter Strike\footnote{Counter Strike ist eine Reihe von taktischen Multiplayer-Ego-Shootern [verbesserungswürdig]} oder Unreal Tournament\footnote{[Sehr ähnich wie cs:go? anderes Beispiel]} die meist nur bei leidenschaftlichen Gamern gespielt werden, da sie Vorbereitung, Know-How und teure Ausrüstung benötigen. Das Alleinstellungsmerkmal dieser Art von Spielen ist, dass diese auf nur einem Bildschirm vor einer Couch, daher der Name, aus gespielt werden und nur minimale Ausrüstung, wie zum beispielsweise mehrere Controller, wobei oft auch ein einfaches Handy reicht, benötigt wird.

In Couchparty-Spielen müssen Spielende schnell verstehen worum es geht, daher stützen sich viele Spiele auf "`Zufall"'. Dadurch können Anfangende schnell mit den anderen mitkommen, auch wenn sie das Spiel noch nie zuvor gespielt haben.

\subsection{Was sind coop-games?}

Kooperation in Team-basierten-Spielen ist sehr wichtig. Coop-games (= cooperative-games/Kooperative-Spiele) sind eine besondere Art von Team-basierten-Spielen die besonders auf Gemeinschaft der Spielenden setzt und diese auch durch Game-Design voraussetzt. Bei klassischen Team-basierten-Spielen ist Kommunikation zwar wichtig jedoch nicht essentiell. Aufgaben der Spielenden werden in coop-games so verteilt, dass ein Ziel nicht ohne die Zusammenarbeit der anderen erreicht werden kann. Es entsteht hierbei eine interessante Dynamik in der die Spielenden komplett voneinander abhängig sind. [symmetrische coop-games: lego games, battle block theater / ist overcooked asymmetrisch?]

\subsection{Was ist bedeutet Asymmetrie?}

Asymmetrie beschreibt in Videospielen das (board-) design. Die Spielenden haben in einem solchen Spiel ein hoch unterschiedliches action-set, das bedeutet sie haben einen anderen Einfluss auf die Spielwelt und andere Spieler.\cite{_balancing_asymmetric_video_games}
\chapter{Spaß in Videospielen}

\section{Warum machen Spiele Spaß?}

McKee definiert Spaß sehr simpel: "`Vergnügen ohne Ziel"'. Nach dieser Definition kann alles Spaß sein, wenn man es für Vergnügen tut.

\subsection{Woher kommt Spaß?}

Gamedesigner Marc LeBlanc hat acht verschiedene Typen von situationsabhängigem Spaß definiert. Einige davon lassen sich gut auf Situationen unseres Spieles anwenden um genauer zu verstehen, woher der Spaß in dem Spiel kommen soll: "`drama, obstacle, social framework"'. 

Das Buch "`A Theory of Fun"'\cite{_theory_of_fun} definiert Spaß als das mentale meistern eines Problems\cite[S. 71]{_theory_of_fun}. Außerdem meint Raph Koster, dass Spaß aus "`richly interpretable situations"'\cite[S. 40]{_theory_of_fun}, also Situationen, die den Spielenden mehrere Möglichkeiten gegeben ein Problem zu lösen und Kreativität fördern, entsteht. 

Spaß kann also nur existieren, wenn Spielende immer neue Probleme lösen können oder bereits gelöste Probleme auf andere Wege ("`richly interpretable"') lösbar sind. Das Meistern eines Spieles, laut seiner Definition, kann also nur Spaß machen, wenn sich die Situationen je nach skill-level ändern oder sie auf andere Wege lösbar sind. Das verbessern der Reaktionsfähigkeit oder das ständige ausführen der selben Aktion macht keinen Spaß. Raph Koster hat außerdem einige wichtige Grundsteine für Spaß in Spielen gelegt, diese werden im praktischen Teil noch genauer behandelt.

\subsection{Spaß bei mehrspieler Spielen}

Ein weiterer Ursprung von Spaß bzw. Vergnügen lässt sich in sozialen Interaktionen finden\cite[S. 72]{_theory_of_fun}. Hier unterscheidet man zwischen kooperativen und kompetitiven Interaktionen, wobei beide Spaß erzeugen können. 
[competetive vs cooperative fun]

\subsubsection{Arena}
Party-Spiele finden in der "`Arena"' statt\cite[S. 65]{_art_of_gamedesign}. Im Vergleich zu anderen "`Arena-Spielen"' befinden sich die Spielenden bei Party-Spielen im selben Raum und können sich so auf noch mehr andere Weisen beeinflussen, die bei klassischen "`Arena-Spielen"', z.B. Shooter, nicht möglich ist. Es entsteht ein anderer, persönlicherer Umgang miteinander in der Spielwelt.\newline

\noindent Außerdem hat eine Studie herausgefunden, dass Spaß der mit anderen geteilt wird stärker empfunden wird als einsamer Spaß, das hat vermutlich einen evolutionären Ursprung\cite{_fun_is_more_fun}.

\section{Wie kann man Spaß "`kreieren"'?}

Wie vorher definiert gibt es verschiedene Arten von Spaß. Das Erlebnis der Spielenden lässt sich hauptsächlich durch die Aktionen beeinflussen, welche den Spielenden zu Verfügung gestellt werden und dessen Auswirkungen auf die Spielwelt oder andere Spielende.

\subsection{MDA-Framework}

Das MDA-Framework hilft bei der Analyse eines Spieles und dem iterativen Designprozess. Es wurde von dem GameDesigner Marc LeBlanc mitentwickelt und trennt den "`Konsum"' von Spielen in verschiedene Komponenten\cite{_mda}. Es ist unterteilt in Mechanics, Dynamics und Aesthetics. Die unterste Ebene beeinflusst immer das was drüber ist. Es beginnt mit dem Mechanics, diese beeinflussen alle ebenen darüber.

[MDA-Bild]

\subsubsection{Mechanics (Regeln)}

Regeln und Feedback-Loops die den Spielenden in seinen Aktionen limitieren. Sie beschreiben das Ziel des Spieles und wie Spielende es erreichen bzw. nicht erreichen können\cite[S.96]{_art_of_gamedesign}. Sie sind das was das Spiel im Zentrum ausmacht, selbst wenn man alles andere weg lässt. \cite[S.231]{_art_of_gamedesign}
 
\subsubsection{Dynamics (System)}

Das entstehende Verhalten der Spielenden welches aus den Mechanics hervorgeht\cite{_mda}. Sie beschreiben, wie sich das Spiel bzw. wie sich Systeme und Mechaniken im  im Laufe der Zeit verändern. Sie bieten dem Spieler ein Gefühl des Fortschritts und der Veränderung, hier spielt auch Feedback eine große Rolle, dazu kommen wir später. Hier kommen die Regeln, die Spielwelt und die Spielenden zusammen, Spielende werden mit der Zeit besser und das Spiel wird Schwerer und bietet so eine fesselnde Erfahrung.

\subsubsection{Aesthetics (Spaß)}

Die emotionalen Reaktionen der Spielenden. Sobald Spielende zum ersten mal mit dem Spiel interagieren, löst es Gefühle in ihnen aus. Ziel ist es eben diese Gefühle zu steuern und das Interesse der Spielenden so aufrecht zu erhalten.\newline

\noindent Der Game-Designer muss man das Framework im Überblock behalten. Designer verändern die Mechanics des Spieles, sehen aber gleichzeitig deren Einfluss auf die Aesthetics und müssen diese genauso berücksichtigen um das Spiel in die richtige Richtung zu lenken. 
Es ist auch möglich ein Spiel durch Mechaniken zu designen, allerdings, tappt man dann im Dunkeln, und man kann nie wissen, was dabei entsteht. \cite[S.56]{_art_of_gamedesign}

\subsection{Essential Experience}

Als Game-Designer möchte man oft eine Experience, ein Erlebnis, das man vorher definiert hat einfangen. Die "`Essential Experience"' versucht, wenn auch auf einem anderen Weg, die reale Experience einzufangen und durch Mechanics zu erzeugen. Das designen eines Spieles mithilfe der "`Essential Experience"' ermöglicht einem, das Spiel von der Experience getrennt zu sehen, und so entscheiden zu können, was an dem Spiel noch verändert werden muss bzw. welche Teile des Spieles nicht verändert werden dürfen und den Grundgedanken zu erhalten. \cite[S.55]{_art_of_gamedesign}

\subsection{Core Mechanics}

Die Core Mechanics sind die Grundbausteine/der Anfang des Spieles. Sie sind jene Mechaniken, die in dem Spieler die erwünschten Gefühle auslösen und ohne die das Spiel nicht funktionieren könnte. Sie beinhalten grundlegende Dinge wie Interaktionsmöglichkeiten, Fortbewegung und Fähigkeiten.

\subsection{Pacing, Flow}

Die Schwierigkeit von Spielen muss sich in einem gewissen Bereich aufhalten, der von dem Psychologen Cziksentmihalyi als "`Flow Channel"' bezeichnet wird, um die Aufmerksamkeit eines Spielers zu behalten.\cite[S.205]{_art_of_gamedesign}

[Bild: Flowchannel]

Als "`Pacing"' beschreibt man den Balanceakt der Herausforderung des Spieles und der Fähigkeiten der Spieler. Game-Designer haben dabei zum Ziel die Spielenden in einen Flowstate zu bringen.

Flow ist "`ein Gefühl der vollständigen und energiegeladenen Konzentration auf eine Tätigkeit, mit einem hohen Maß an Freude und Erfüllung"'\cite[S.204]{_art_of_gamedesign}.
Cziksentmihalyi beschreibt Flow etwas einfacher als einen Zustand der auftritt, wenn man sich freiwillig an seinem Limit befindet\cite{_flow}.

[Wie bringt man Flow in ein Spiel?]\cite{_theory_of_fun}

\subsection{Player Action Feedback}

Feedback ist auch in Spielen wichtig. Ohne Feedback wissen Spielende nicht ob das was sie tun richtig ist oder das Spiel das was sie tun wollten bereits aufgefasst hat. Visuelle und Auditives Feedback kann den Spieler über Fortschritt und Fehler informieren. Außerdem hilft es bei der Immersion, der Fähigkeit der Spielenden sich in das Spiel hineinzuversetzen.

\chapter{Stand der Industrie\label{_industrie}}

\section{Asymmetrical Coop}

\subsection{Keep Talking and Nobody Explodes}
Keep Talking and Nobody Explodes ist ein Spiel in dem eine Person eine Bombe vor sich sieht, die es zu entschärfen gilt, und die andere eine Anleitung zur Entschärfung der Bombe. Das Spiel ist zwar kooperativ jedoch sehr einseitig, da der Bombenentschärfer oft auf auf den anderen Warten muss und es außerhalb der Beschreibung von dem was man Sieht nicht viel Kommunikation gibt. Keine euphorisches Anfeuern, obwohl alles unter enormem Zeitdruck ausgeführt wird.

\subsection{It Takes Two}
Anders ist es bei "`It Takes Two"'. Es überzeugt besonders durch sein abwechslungsreiches Gameplay und dem sehr guten Pacing, es erfordert von beiden Seiten nicht sehr viel können, was es für alle Zielgruppen offen hält und es setzt Kommunikation voraus. Gemeinsames lösungsorientiertes Denken ist von Vorteil und das Spiel bietet eine entspannende Atmosphäre in der man sich schnell verliert.

\section{Couchparty}

\subsection{Mario Party}
Mario Party ist das klassische Party spiel. Wie für Party-Spiele üblich bietet es eine enorme Vielfalt an Spielen und einen gewisses maß an Zufall um auch die schlechteren einer Gruppe im Spiel zu inkludieren.

\subsection{Jackbox}
Bei den Jackbox-Spielen handelt es sich um eine Reihe an verschiedenen Spielen die einen oft an klassische Papier-Spiele erinnern, die man früher und auch heute noch ohne viel Aufwand gespielt hat (z.B. Wörter Raten). Es ist besonders gut darin lustige Geschichten oder andere Kreationen aus den Spielenden herauszuholen. Es bietet aber auch einige komplexere Spiele, die für eine erfahrenere Gruppe noch mehr Abwechslung bieten. Die Experience entsteht hier aber haupsächlich durch die Imagination der Spielenden.

\section{Couchparty VR-Games}

Couchparty VR-Games sind per Definition asymmetrisch. In dieser Kombination wird es schon schwieriger gute Spiele zu finden, da der Markt noch relativ klein ist und sich die großen Studios (Nintendo, EA, etc.) noch nicht in den Markt trauen.

\subsection{Takelings House Party}

Takelings bringt in das Chaos von Couchparty-Spielen noch einen VR-Spieler.

\subsection{Acron: Attack of the Squirrels!}
[Muss ich noch Spielen]
\chapter{Implementation}

\section{Was muss bei VR-Game-Design beachtet werden?}

VR-Spiele bringen einige Limitationen mit sich, die es zu überwinden gilt. Bei der ersten Limitation handelt es sich um die Größe des Play-Spaces, also jenem physischem Raum, in dem sich VR-Spielende aufhalten. Das zweite Problem ist die Reisekrankheit (Motion-Sickness, eine körperliche Reaktion auf ungewöhnliche/unzusammenhängende Bewegung\cite[S. 533]{_art_of_gamedesign}), vor allem bei Party-Spielen muss ihr Einfluss möglichst gering sein, da das Spiel von einer breiten Masse gespielt werden können soll. Die Reisekrankheit limitiert die Bewegungsmöglichkeiten der Spielenden und diese wiederum die Größe und Gestaltung der virtuellen Umgebung.

Außerdem muss die Größe des Spielenden (um sicherzustellen, dass alles, womit Spielende interagieren sollen, erreichbar ist) als auch Ängste wie zum Beispiel "`Klaustrophobie"', beachtet werden. Im Falle von einem Party-Spiel spielt auch das technische Know-How eine Rolle, da man sehr schnell in das Spiel eintauchen können soll.

\section{Stimmung des Spieles}
Das im Zuge dieser Arbeit entwickelte Spiel soll eine bestimmte Stimmung erzeugen. Am Anfang des Design-Prozesses muss diese Stimmung festgelegt werden.

\subsection{Core Pillars}

\subsubsection{Fast but Strategic}
Das Ziel des Spiels soll es nicht sein, möglichst schnell Spells zu aktivieren. Es muss etwas Strategie dahinter sein, damit Spielende gefordert werden und eine wertvolle Entscheidung treffen können. Da es sich um ein Party-Spiel und kein Strategie-Spiel handelt, sollte es möglich sein, schnelle Entscheidungen zu treffen. Um das zu ermöglichen, muss das Spell-System vergebend sein und muss sich auf die aktuelle Spielsituation anpassen können (mehr dazu in Kapitel \ref{_rubberbanding}). 

\subsubsection{Teamwork \& Kommunikation}
Außerdem muss Strategie und Absprache zwischen den PC-Spielenden angeregt werden, um weiter das Teamwork der Gruppe zu fördern. Wie am Anfang der Arbeit bereits definiert, braucht Teamwork ein gemeinsames Ziel. Das Ende der Strecke kommt hier recht schnell in den Sinn, allerdings ist das allein nicht genug, um Kommunikation zwischen den Spielenden zu erzeugen, da es zu weit in der Zukunft liegt und sich nicht verändert. Eine Lösung und mögliche Lösungsansätze für dieses Problem werden in Kapitel \ref{_teamwork_erzeugen} besprochen.

\subsubsection{God-Like}
Vor allem der VR-Spieler soll sich mächtig fühlen. Erreicht werden soll dieses Gefühl durch den deutlichen Größenunterschied zwischen Magier und Snowboarder und Combos, also der Möglichkeit, mehrere Snowboarder gleichzeitig von der Strecke zu schmeißen. Es muss allerdings beachtet werden, dass das alleinige "`auf die Strecke schmeißen"' keinen Spaß machen wird, wenn davor nicht etwas dafür getan werden musste. Hier kommt die "`Lens of Challenge"'\cite{_art_of_gamedesign} ins Spiel, man sollte besonders darauf achten, dass die Schwierigkeit, das Ziel zu erreichen, mit dem Skill des Spielenden übereinstimmt, außerdem ist es wichtig, die Interaktion nicht zu einseitig zu gestalten. Das Gefühl der Macht kann nur erzeugt werden, wenn Spielende denken, es war ihre alleinige Leistung.

\subsection{Target Experiences}

Da es sich bei dem Projekt um ein asymmetrisches Spiel handelt, wurden für die PC- und die VR-Spielenden jeweils andere Target Experiences festgelegt.

\subsubsection{Snowboarder (PC)}
Für die Snowboarder sind folgende zwei Experiences am wichtigsten: die (Snowboarding-) Action und das Teamwork der Gruppe. Mithilfe der "`Lens of Essential Experience"'\cite[S. 55]{_art_of_gamedesign} kann versucht werden, diese Experiences zu erzeugen.

Um das Gefühl "`Snowboarden"' akkurat wiederzugeben, wurden drei wichtige Komponenten gefunden: Tricks, Stürzen und Beharrlichkeit. Man könnte auch sagen, dass die Kälte ein Teil des Snowboardens darstellt, jedoch wurde dagegen entschieden, da es dem Core Pillar "`God-Like"' widerspricht, wenn sich die Snowboarder über die Körpertemperatur ihrer Charaktere Gedanken machen müssen.

Eine weitere wichtige Experience der Snowboarder ist das Teamwork sogar so wichtig, dass es, wie oben erwähnt, zu einem der Core-Pillars gemacht wurde. Wie Teamwork erzeugt werden soll, wird in Kapitel \ref{_teamwork_erzeugen} behandelt.

\subsubsection{Magier (VR)}
Für den Magier sind die folgenden zwei Experiences am wichtigsten: Macht und Multitasking. 

Diese göttliche/magische Kraft soll vor allem durch das Zaubern, also den Spells (siehe \ref{_spell_design}]) und die Aneinanderreihung und taktische ("`Fast but Strategic"') Platzierung dieser erzeugt werden. Sie wird auch über den visuellen Größenunterschied der Snowboarder und dem Magier dargestellt.

Multitasking ist vor allem im Bereich der Interaktionen wichtig, der Magier sollte immer schon bevor er einen Spell platziert hat, darüber nachdenken, was der nächste Zug sein wird. Um das zu erreichen, sollten Spells nicht immer sofort verfügbar sein. Ziel ist es, dadurch mehr Strategie in das Spiel zu bringen.

\section{Zusammentreffen beider Welten (PC + VR)}

\subsection{Streckendesign}
Der erste Punkt und auch Grundstein für das restliche Design ist das Konzept der Strecke. Es hat sowohl Einfluss auf die PC-Spielenden als auch auf den VR-Spielenden, da es den Ort darstellt, in dem beide miteinander Interagieren.

\subsubsection{Limitationen}
Der VR-Spielende muss die Möglichkeit haben, zu jedem gegebenen Zeitpunkt mit einem Großteil der Strecke zu interagieren (Fallen platzieren etc.). Außerdem ist es wichtig, dass sich Spells immer in Reichweite befinden. Die Strecke muss jedoch auch (aufgrund anderer technischer Limitationen) zu jedem Zeitpunkt komplett geladen und sichtbar sein.

Das Streckendesign muss es zulassen, die Strecke so anzupassen, dass eine Fahrt von oben nach unten ca. drei Minuten dauert. Des Weiteren muss die Breite der Strecke genug Platz, um Hindernissen auszuweichen und den Spielenden eine gewisse Toleranz für Fehler beim Lenken der Snowboarder lassen.

Zur Auswahl stehen drei verschiedene Streckendesigns, deren vor und Nachteile behandelt werden und begründet wird, warum schlussendlich für Streckendesign Nr. 3 entschieden wurde. Alle drei Designs versuchen die Bewegung des VR-Spielenden zu vermeiden, um das Gefühl der Reisekrankheit zu mindern.

\subsubsection{Designoption 1 - Spirale}
Die Snowboarder fahren in diesem Design in einer Spirale um den VR-Spielenden herum. Es ermöglicht eine freie Strecke für die Snowboarder, jedoch ist wenig Platz für VR-Spielende. Dieses Design ist auch sehr unrealistisch, da es sich hier nicht um eine klassische Ski-Strecke handeln kann, da Teile "`Schweben". Dieser Effekt wird verstärkt dadurch, dass es sich für die Spielenden so anfühlt, als wären sie in einer Wasserrutsche. Das widerspricht der Target-Experience des Snowboardens.

\subsubsection{Designoption 2 - Kristallkugel}
Dieses Design bringt auf der Seite der Snowboarder und von dem VR-Spielenden deutliche Verbesserungen. Da VR-Spielende durch eine Kristallkugel auf die Snowboarder herabschaut, kann die Strecke unglaublich frei sein, es kann harte Kurven und Richtungswechsel geben. Der VR-Spielende hat die Möglichkeit, einen eigenen Raum zu bekommen, indem der Zugang zu den Spells einfach ist. Der größte Nachteil ist hier die Technik, die im Hintergrund gebraucht wird. Technisch ist es sehr schwer umzusetzen, da die Welten des VR-Spielenden und der Snowboarder voneinander getrennt und über die Kristallkugel wieder zusammengebracht werden müssen.

\subsubsection{Designoption 3 - Berg}
Die Idee des Berges wurde am Anfang etwas vernachlässigt, da sie sehr eingeschränkt erschien. Jedoch bringt der Berg neben der technisch um einiges leichteren Umsetzung wieder etwas Realismus und eine schöne Atmosphäre ins Spiel, die die Experience des Snowboardens verstärken kann. Auf diesem, aus der Sicht des VR-Spielenden, kleinen Berg, in der Mitte des VR-Play-Spaces, können die Snowboarder spiralförmig hinunterfahren. Wie bei der Spirale hat man hier aber auch das Problem, dass die Snowboarder permanent in die Kurve lenken müssen, dieses Problem wird in Kapitel \ref{_playercontroller} weiter behandelt. Der Berg kann automatisch mit den Snowboardern mitgedreht werden, damit sie immer im Sichtfeld des Magiers sind. Der Magier soll aber auch eine Möglichkeit haben, diese Drehung zu überschreiben, damit er auch Teile der Strecke die außerhalb des interagierbaren Bereiches liegen beeinflussen kann.

\subsection{Interaktionsdesign}
Da der Wettbewerb zwischen PC und VR asymmetrisch ist, müssen beide Parteien die jeweils andere Seite beeinflussen können, um es interessant zu machen. Der Magier muss durch Spells und andere Tools die Snowboarder behindern können und die Snowboarder müssen Einfluss auf die Welt des VR-Spieleden haben.

\subsubsection{Spell Design\label{_spell_design}}
Um dem VR-Spielenden das Gefühl von Zaubern zu übermitteln, braucht es neben der Möglichkeit zu Zaubern auch gute Interaktionen. Das Spell-System muss dem Magier die Möglichkeit geben zu Multitasken und Kombos zu machen, damit das Gefühlt der Macht aufkommt. Da der VR-Spielende zwei Hände hat, ist es naheliegend, dass beide Hände an unterschiedlichen Spells arbeiten können. Um eine gewisse Vielfalt in die möglichen Spells zu bringen, wurden drei Spell-Kategorien eingeführt.

\paragraph{Instant-Spell}
Sollen einen sofortigen Effekt auf die PC-Spielenden haben. Instant-Spells sollen sich besonders gut für die Kombination mit anderen Spell-Typen eignen. Sie dürfen auch in die Masse der Snowboarder platziert werden, da sie niemals sofort töten dürfen. Ein Beispiel hierfür wäre eine Eisfläche, die die Snowboarder zum Rutschen bringt.

\paragraph{Short-Term-Spell}
Dieser Spell-Typ braucht eine gewisse Zeit, um sich zu aktivieren. Somit müssen Spielende sich vorher etwas genauer überlegen, wo sie am besten zu platzieren sind, und die Snowboarder haben etwas Zeit um auszuweichen. Short-Term-Spells dürfen auch tödlich enden, ein Beispiel hierfür wäre eine Bombe, die die Snowboarder in die Luft schleudert. Alle Short-Term-Spells müssen nach einer gewissen Zeit wieder verschwinden.

\paragraph{Long-Term-Spell}
Das Ziel von Long-Term-Spells ist es den Spielenden die Möglichkeit zu geben, etwas strategischer zu Spielen. Sie können nur begrenzt, und an vordefinierten orten aktiviert werden. Ziel ist es Spielende dazu zu bewegen, zu Beginn einer Runde abzuwägen einen Long-Term-Spell einzusetzen, und dafür weniger Instant- und Short-Term-Spells, oder darauf zu verzichten, um flexibler zu sein. Beispiele hierfür wären das Absperren von gewissen Streckenteilen, oder das zum Einsturz bringen von Bäumen oder Geröll auf der Strecke.

\paragraph{Balancing-Cooldown}
Die Anzahl der Spells, welche der Magier verwenden kann, muss limitiert sein. Der einfachste Weg, um Spells zu limitieren, ist ein Cooldown, welcher verhindert, dass zwei Spells direkt nacheinander platziert werden können. Es wird jedoch bei diesem System schnell dazu kommen, dass VR-Spielende immer den "`Stärksten"' Spell nehmen, und es dadurch schwer zu balancen ist.

\paragraph{Balancing-Mana\label{_mana}}
Bevor ein Spell platziert werden kann, muss er in einen Mana-Kessel getaucht werden, um ihn zu aktivieren. Dieser Mana-Kessel kann nur eine begrenzte Menge an Mana halten und befüllt sich über den Verlauf des Spieles von selbst. Ein Mana-Kessel bietet gleich mehrere Vorteile: Balancing und Physische-Interaktion. Mana kann dabei helfen Spells untereinander zu balancen, indem ihre Kosten erhöht oder verringert werden. Außerdem fügt der Kessel einen Zwischenschritt hinzu, das macht die Interaktion komplexer und interessanter. Sobald ein Spell einmal aufgeladen ist, ist man dazu gezwungen ihn auch zu platzieren, da er sonst seine Ladung wieder verliert.

Die Anzahl der Spielenden, die noch im Rennen sind, könnte sich auf die Mana-Regeneration auswirken, so entsteht ein negativer Feedbackloop (siehe \ref{_rubberbanding}), es wird für den Magier schwieriger zu gewinnen, wenn er kurz davor ist den letzten Snowboarder zu Fall zu bringen. Das erzeugt einen weiteren Pacing Spike, wie bei einem Bosskampf, am Ende einer Runde und hat einen automatischen Balancing\cite[S. 296]{_game_design_workshop}-Effekt.

\paragraph{Balancing-Interaction\label{_balancing_interaction}}
Ein weiterer Weg, wie Spells gebalanced werden könnten, wäre durch die Dauer / Komplexität der Interaktion. Ein großer Vorteil hierbei wäre, dass es den Spielenden etwas natürlicher vorkommt als ein künstlicher Cooldown, dadurch sollte es auch weniger frustrierend sein, wenn gerade kein Spell bereit steht, da der Magier selbst dafür verantwortlich war sie herzustellen. Den Spielenden soll dadurch nahegelegt werden, dass sie es beim nächsten Mal, durch verbessern ihrer Skills, besser machen können, das nennt man auch interne Attribution\cite{_internal_attribution}.

\subsection{Playercontroller \& Obstacles\label{_playercontroller}}
\subsubsection{Tricks}
Tricks können zum einen dazu genutzt werden, ein Risiko einzugehen, um die Respawn-Zeit zu verkürzen oder als eine Movement-Mechanik, die dabei hilft Obstacles auszuweichen. In das Spiel wird vermutlich eine Mischung beider Mechanics implementiert.

\subsubsection{Respawns}
Das Wiedereintreten von ausgeschiedenen Spielenden ist eine wichtige Mechanik, um das Gefühl des Snowboardens einzufangen und um den Party-Aspekt des Spieles zu fördern. Es gibt verschiedene Möglichkeiten zu regeln, wann Spielende wieder in die Runde eintreten.

\paragraph{Timer}
Eine sehr simple Möglichkeit wäre, Ausgeschiedene nach einer gewissen Zeit automatisch wieder dem Rennen hinzuzufügen. Das hätte den Vorteil, dass es zum einen sehr leicht umsetzbar ist und zum anderen den Spielenden eine kleine Pause verschafft und der Timer sich durch andere Faktoren beeinflussen ließe, was wiederum beim Balancing helfen kann.

\paragraph{Sekundäres Ziel für ausgeschiedene Spielende}
Ein weiterer Weg, der die ausgeschiedenen Spielenden besser ins Spiel inkludiert, wäre ein Sekundäres Ziel. Sie müssten, um wieder ins Rennen zu kommen, eine kleine Aufgabe lösen, bestimmte Knöpfe schnell hintereinander drücken (verbunden mit viel Game-Juice, um die durch die Spielenden wahrgenommene Intensität des Gameplays zu steigern), oder sie könnten den VR-Spielenden behindern, mehr dazu in Kapitel \ref{_rubberbanding}. Dieser Möglichkeiten könnte wieder dabei helfen den Spielenden ein Gefühl von Kontrolle zu geben. Ähnlich wie bei dem Magier (\ref{_balancing_interaction}) ist es auch auf der Seite der Snowboarder hilfreich, wenn ein ausgeschiedener Spielender die Schuld für die schlechte Team-Performance in sich selbst sucht (interne Attribution) und sich deshalb versucht zu verbessern.

\subsubsection{Automatische Lenkhilfe}
Da die Spielenden bei dem Berg- oder Spiralen-Streckendesign permanent im Kreis lenken müssen, wäre es wichtig, dem entgegenzuwirken, um das Gefühl des "`im Kreis"'-fahrens zumindest etwas zu mindern. Hier wäre eine Automatische Lenkhilfe, die den Snowboarder automatisch nach unten fahren lässt, die einfachste Methode. Verbunden mit einer Lenkung, welche Relativ zur Kamera funktioniert, haben Spielende eine angemessene Lenk-Freiheit in beide Richtungen.

\subsubsection{Obstacles}
Obstacles können sowohl von der Strecke vorgegeben sein als auch durch den Magier platziert werden. Sie zwingen die Snowboarder Entscheidungen zu treffen, insbesondere im Falle von \emph{Positive Obstacles}. Normale Obstacles sollten von den Snowboardern vermieden werden.

\paragraph{Positive Obstacles\label{_positive_obstacles}}
\emph{Positive Obstacles} sind ähnlich wie Pickups in z.B. \emph{Mario Cart}, nur sind sie in die Welt integriert. Sie haben einen positiven Effekt auf die Gruppe, jedoch werden sie von VR-Spielenden eher angegriffen/beschützt.

\subsubsection{Spieler führen}
Wenn man Spielende für riskantes Fahren belohnt, entsteht eine Meaningful Decision. Es entsteht eine Risk-Reward Situation, in der sich schlechte Teammitglieder von den Guten abspalten und andere Wege nehmen. Mithilfe von Rampen und Slalom-Bögen, die die Respawn-Zeit verkürzen, kann man riskantes Fahren fördern.

\subsection{Teamwork erzeugen - Krone \& andere Lösungsansätze\label{_teamwork_erzeugen}}
Um die Snowboard-Gruppe dazu zu bewegen zu kooperieren, müssen sie ein gemeinsames Ziel haben. Dieses Ziel ist schlussendlich das Ende der Strecke jedoch braucht es auch ein kurzzeitiges Ziel wie zum Beispiel einem ausgeschiedenen Spielenden dabei zu helfen, wieder ins Rennen zu kommen. Mögliche Aufgaben dafür wären, das Einsammeln von bestimmten Punkten auf der Strecke, das Halten der Krone, oder eine gemeinsame Aktion (z.B. einen Trick machen).

\subsubsection{Coop-Points}
Bei Coop-Points müssen verschiede Spielende zur selben Zeit an zwei verschiedenen Orten sein. Dadurch muss es zwingend zu einer Absprache zwischen den Spielenden kommen, um auszumachen wer wohin fährt.

\subsubsection{Krone\label{_krone}}
Der erste Spielende hält immer die Krone. Wenn die Krone länger als eine gewisse Zeit vom selben Spielenden gehalten wird, kann ein ausgeschiedener Spielender wieder auf die Piste. Sollte die Krone von einem anderen Spielenden übernommen worden sein, bevor ein Snowboarder respawnt ist, wird der Timer wieder länger. Nach einem Respawn, muss die Krone an einen anderen Spielenden übergeben werden, um Teamwork und Kommunikation innerhalb des Teams zu fördern.

\subsubsection{Gemeinsame Aktion}
Eine gemeinsame Aktion wäre beispielsweise, dass alle Snowboarder zur selben Zeit Springen oder einen Trick ausführen. Das würde wieder ein großes Risiko mit sich bringen, da alle gleichzeitig sterben könnten.

\subsubsection{Negativer Feedbackloop\label{_rubberbanding}}
Ein negativer Feedbackloop versucht Änderungen entgegen zu steuern um so ein System stabil zu halten\cite[S.133]{_game_design_workshop}. Snowboarder, die weiter hinten sind, sollten schneller werden, damit sie mit der Gruppe mithalten können. Wie in \ref{_mana} angesprochen, ist es auch wichtig PC und VR untereinander zu balancen. Neben dem bereits angesprochenen, gibt es noch folgende weitere Möglichkeiten: Ausgeschiedene Spielende könnten den VR-Spieler daran hindern weitere Spells zu aktivieren oder gute Sicht auf die Strecke durch z.B. Nebel verhindern. Spells könnten auch dynamisch gestärkt oder geschwächt werden, indem man deren Range oder Dauer verändert.
\chapter{Post-Mortem}

Im Designprozess ist nicht immer alles so wie man es erwartet, oft kommt es zu unerwarteten Nebeneffekten, die man im ursprünglichen Design nicht beachten konnte. Deswegen ist die Flexibilität aller Komponenten, um späteres Balancing zu ermöglichen, und besonders Playtesting in verschiedenen Phasen des Designprozesses notwendig.

\section{Playtests}

Durch Playtests wurden einige Mängel im Design gefunden, die dann auch rechtzeitig gehandhabt werden konnten. Leider sind Playtests erst spät in der Entwicklung sinnvoll gewesen. Außerdem wurden neue Bereiche aufgedeckt, die vorher vernachlässigt wurden. Sollte das Spiel weiterentwickelt werden, sind Playtests ein wichtiger Punkt um die Qualität des Spieles zu sichern und weiter zu polishen, da die Grundsysteme mittlerweile funktionieren.

\subsubsection{Feedback}

Einige Dinge, auf die wir durch Playtests aufmerksam wurden, wurden bereits implementiert. Darunter ist auch mehr Feedback für tote Spielende und wann sie wieder Respawnen können. Neben dem initialen Feedback für Tot und Respawn, ist es auch wichtig den Spielenden über eine gewisse Zeit hinweg zu sagen, dass sie eine bestimmte Aktion ausführen können, sollten sie den ersten "`Tip"' übersehen haben. Hier haben wir uns für eine leichte Vibration des Controllers entschieden (Diese Idee kam nur dank des Feedbacks unserer Playtester). Es ist manchmal aber auch wichtig den Spielenden Zeit zu geben, und sie selbstständig entscheiden zu lassen wann eine Aktion ausgeführt wird, anstatt ihnen einfach zu sagen "`Jetzt!"'. Neben Feedback ist also auch Interaktion wichtig.

Eine weitere Sache, auf die wir bei Playtests aufmerksam wurden, war wie die Spielenden mit Spells interagieren. Lieder wurde darüber anfangs weniger nachgedacht. Es wurde jedoch schnell klar, dass viele Spielende nicht intrinsisch herausfinden konnten, wie Spells zu aktivieren sind. Es wurden verschiedene Möglichkeiten ausprobiert, um es klarer zu gestalten. Das ursprüngliche Aktivieren per Knopfdruck wurde durch ein Collider basiertes aktivieren ausgetauscht. Schlussendlich wurde sogar der Collider der gehaltenen Spells verkleinert, um den Spielenden die nötige Genauigkeit bei der Platzierung der Spells zu geben.

\subsubsection{Balancing}

Außerdem halfen Playtests enorm beim Balancing des Spieles. Die ersten Settings, die bei beispielsweise der Geschwindigkeit der Snowboarder, oder der Menge an Mana eingestellt wurden waren immer falsch, und mussten immer leicht angepasst werden, um das Spiel möglichst fair zu halten. Auch das ist nur dank Playtests möglich.

\subsubsection{Positives}

Neben den Dingen, die nicht gut funktioniert haben, hat das Konzept der Krone, wie erwartet, sehr gut für Teamwork innerhalb der Gruppe gesorgt. Auch die Tricks und \emph{Positive Obstacles}\ref{_positive_obstacles} haben den gewünschten Effekt, riskantes fahren zu fördern erzieht. Die verschiedenen Spell-Typen wurden von den VR-Spielenden, wie erwartet, kombiniert, da sie gemeinsam einen größeren Effekt auf die Snowboarder hatten.

\section{Kommunikation}
[Hier fehlt noch etwas]

\section{Zukunft des Projektes}

Sollte das Projekt weiterentwickelt werden, ist vor allem mehr auf das Design der Interaktionen des VR-Spielenden zu achten. Derzeit sind alle Interaktionen sehr statisch und es sollten, um das Balancing der Spells für die Spielenden verständlicher zu machen [auch hier wieder internal attribution], komplexere Interaktionen (siehe \ref{_balancing_interaction}) implementiert werden. Leider waren die dafür benötigten Systeme zu komplex, um sie in der kurzen verbleibenden Zeit noch umzusetzen, und es wurde mehr Zeit in das Polishing von dem was bereits da war investiert.

Grundsätzlich war der Grundgedanke, so wenig wie möglich zu Implementieren, dafür mehr darauf zu achten das Implementierte möglichst zu perfektionieren, eine der besten Entscheidungen im Designprozess.


\trennseite{Farb- und Formpsychologie in asymmetrischen VR-Spielen}{Emma Wehinger}{Alexander Hager}{Mediendesign - Gamedesign}
\chapter{Überblick über die Materie}
\label{cha:sa_Einleitung}


\section{Farbpsychologie}

\subsection{Was ist Farbe?}
Wissenschaftlich betrachtet, sind Farben Eindrücke, die unsere Sinne durch die Augen und das Gehirn vermittelt bekommen. Farben entstehen durch die Absorption und Reflexion, die das Licht auf Oberflächen wirft. Das menschliche Auge ist für die Sinnesempfindung von Farben verantwortlich, die dann wiederum im Gehirn verarbeitet werden. Man kann Farbe auch als Eigenschaft des Lichts betiteln. Prinzipiell nimmt jeder Mensch Farben und Farbtöne anders wahr. Welche Farbe wahrgenommen wird, hängt von dem Gegenstand ab, der Lichtquelle und den Augen des Betrachters. Licht hat außerdem auch bestimmte Wellenlängen, je nach Länge der Welle nimmt das Auge eine andere Farbe war. Schwarz ist keine Farbe, denn sie kann auch in Abwesenheit des Lichts und völliger Dunkelheit existieren.
Weißes Licht kann mithilfe eines Prismas in Spektralfarben gebrochen werden. Spektralfarben sind diejenigen Farben, die für das menschliche Auge einen sichtbaren Farbeindruck und somit einen Farbton hinterlassen. Insgesamt gibt es sechs verschiedene Spektralfarben. Spektralfarben werden umgangssprachlich auch „Regenbogenfarben“ genannt. Zu ihnen zählen Rot, Orange, Gelb, Grün, Blau, Indigo und Violett. Die durch die Brechung von Licht entstandenen Farben können nicht in weitere Farbtöne zerlegt werden. Das Licht kann manchmal auch infrarotes Licht oder ultraviolettes Licht enthalten, welches für das bloße Auge nicht sichtbar ist.
Wenn man jedoch gelbes Licht oder jede andere Spektralfarbe durch ein Prisma fallen lässt, bekommt man dieselbe Farbe raus, die man im Prisma brechen wollte. Man kann also als Schlussfolgerung daraus ziehen, das weißes Licht das einzige ist, was durch ein Prisma in Spektralfarben gebrochen werden kann.

\begin{figure}[h]
	\centering
	\includegraphics[width=10cm]{Farbprisma}
	\caption{\cite{_steam_hardware}}
\end{figure}

In der Kunst ist Farbe das beste Mittel, um visuell Emotionen auszudrücken oder Emotionen bei Menschen zu erzeugen. Farben sind das beste Instrument, wenn das darum geht, bei einem Menschen Gefühle und Emotionen auszulösen. Aber nicht nur, dass es wird dem Betrachter dadurch auch eine bestimmte Stimmung vermittelt. Künstler können dadurch gezielt bei Menschen bestimmte Gefühle auslösen und das alleine durch das Einsetzen von bestimmten Farben und Farbtönen.
Dabei muss beachtet werden, dass Farben in verschiedenen Kulturen verschiedene Bedeutungen haben und die Symbolik hinter manchen Farbtönen in bestimmten Ländern und Kulturkreisen eine andere ist. Doch grundsätzlich kann man jeder Farbe eine bestimmte Bedeutung und Emotion zuschreiben, die vielleicht in manchen Kulturen ein wenig abweicht, aber im Grunde genommen immer sehr ähnlich ist.
Man kann aber klar die kulturelle Bedeutung von Farben mit der psychologischen Bedeutung von Farben trennen. Die wichtigsten Farben für Kunst und Design sind die Farben Blau, Gelb, Grün, Orange, Rosa, Rot, Violett, Grau und Braun. Jede dieser Farben hat eine andere Wirkung auf die Gefühle und Emotionen der Betrachter. Der psychologische Effekt hinter der Wirkung einer Farbe hat rein gar nichts mit der kulturellen Interpretation dieser zu tun. 
Es gibt eine endliche Anzahl an Variationen, die eine Farbe für das menschliche Auge annehmen kann. Grundsätzlich gibt es eine unendliche Anzahl an Schattierungen, die eine Farbe haben kann, wobei die meisten für das Sinnesorgan eines Menschen nicht sichtbar sind. Die Schattierung einer Farbe ändert prinzipiell den psychologischen Effekt nicht.


\subsection{Der Farbkreis und Farbschemas}
Der Farbkreis besteht normalerweise aus 12 Farben, er kann aber auch aus 24 und manchmal aus bis zu 48 Farben bestehen, das hängt aber von dem Ersteller des Farbkreises ab. Das grundlegende Farbrad besteht aber jedoch nur aus 12 Farben. Um den Farbkreis besser zu verstehen, muss man sich zuerst die additive und die subtraktive Farbmischung ansehen.  

Farben können in Primär -, Sekundär -und Tertiärfarben aufgeteilt werden. Dabei unterscheidet man zwischen der additiven Farbmischung und der subtraktiven Farbmischung. 

Bei der subtraktiven Farbmischung, auch CMYK-Modell genannt, sind die Primärfarben Gelb, Magenta und Cyan, sie können nicht aus anderen Farben gemischt werden. Jedoch kann aus ihnen jede existierende Farbe gemischt werden. Die Sekundärfarben in diesem Modell sind Rot, Grün und Blau und die Tertiärfarbe ist Schwarz. Mischt man nun die Primärfarben gelb, Magenta und Cyan, erhält man infolge dessen die Sekundärfarben Rot, Grün und Blau. Die gesamte Mischung ergibt dann Schwarz.

\centerfirst
\includegraphics[width=8cm]{images/Flyeralarm_CMYK_de}


Die additive Farbmischung, auch RGB-Modell genannt, besteht aus den Lichtfarben Rot, Grün und Blau. Rot, Grün und Blau sind in diesem Modell die Primärfarben, also Grundfarben. Die Sekundärfarben sind Gelb, Magenta und Cyan. Die Tertiärfarbe ist Weiß. Diese entsteht durch die Mischung der Grundfarben Rot, Grün und Blau.

\includegraphics[width=8cm]{images/Flyeralarm_RGB_de}



\subsection{Farbtemperatur}

\subsection{Farbrelativität}

\subsection{Farbsättigung}

\subsection{Erstellung von Farbpaletten}

\subsection{Farbe, Licht und Schatten}

\subsection{Wie wirkt sich Farbe auf die Stimmung aus?}



\section{Formpsychologie}

\subsection{Gestalt und Form}

\subsection{Linien und Linien- Stile}

\subsection{Kompositionslinien}

\subsection{Wie Formen und Linien Emotionen erzeugen}



\section{Environment Design}

\subsection{Character und Environment Shapes}

\subsection{Character Centric Environment Design }

\subsection{Bildkomposition}







\chapter{Möglichkeiten zur Implementierunge}
\label{cha:sa_Einleitung}

\section{Nutzung von Farben in Spielen}
Die Nutzung von Farben in Videospielen war mit ihrer Entstehung sehr mit dem damaligen Stand der Technik verbunden. 1972 wurde die Farbüberlagerung erfunden und hat es ermöglicht, dass Videospiele in Farbe dargestellt werden können. Davor konnten Spiele nur in Schwarz-Weiß oder Monochrom dargestellt werden. Ab Ende der 1980er sind fast alle Videospiele in Farbe. 1985 veröffentlichte Sega das Sega Master System, welches Spiele mit 32 verschiedenen Farben darstellen konnte. Schon 9 Jahre später veröffentlicht Sony die Playstation, welche es ermöglicht, Spiele mit rund 16.7 Millionen Farben darzustellen. Heutzutage können Spiele mit über 16.7 Milliarden verschiedenen Farben dargestellt werden. 

Durch die Nutzung von Farben kann man dem Spieler Gefühle und Emotionen auf eine sehr einfache und natürliche Weise vermitteln. Prinzipiell ist das Farbempfinden von Kulturkreisen immer etwas unterschiedlich, dennoch gibt es auf die meisten Farben eine universelle Reaktion. Wie Menschen Farben wahrnehmen, hängt nicht nur von ihrer Kultur ab, sondern auch von ihrem Alter, ihrer Herkunft, ihrem Geschlecht und ihrem ethnischen Hintergrund. Wenn ein Spieler in einem Spiel zum Beispiel Eierfarbe Rot sieht, wird ihm signalisiert, dass er in Gefahr ist. Rot steht prinzipiell für Energie, Liebe, Selbstbewusstsein und Leidenschaft, wird aber auch als Warnfarbe gesehen. Rot wird mit Gefahren in Verbindung gesetzt. Explosionen, Feuer und Blut etc. wären dafür ein gutes Beispiel. In vielem Actionspielen wird unter anderem ein an den Ecken und Kanten roter Bildschirm eingesetzt, wenn der Spieler verletzt ist. Das signalisiert dem Spieler, dass er ab jetzt wachsamer und vorsichtiger sein muss. 

In vielen MMORPGs oder in Singleplayer-Spielen, werden viele sammelbaren Objekte, Items oder Schatztruhen in bestimmten Farben angezeigt. Meist werden die Farben Weiß, Grün, Blau, Lila und Gold verwendet, wobei Weiß weniger gut ist, als Violett und Gold. Violett wird mit dem Adel verbunden, denn früher konnten sich nur sehr wohlhabende Menschen das violette Farbpigment leisten. Die Farbe Violett steht für Eleganz, Würde und Raffinesse. Die Farbe hat aber auch etwas Magisches und Mystisches an sich. Aus diesem Grund sind sehr seltene und außergewöhnliche Objekte und Items meist Violett, da sie dem Spieler signalisieren sollen, dass er etwas Besonderes und Wertvolles gefunden hat. 

\subsection{Beispiel – Mario Kart 8}

\subsection{Beispiel – Journey}
Journey ist ein Adventure-Spiel, welches ohne jegliche Worte auskommt. Es wurde erstmals 2012 für die Playstation 3 veröffentlicht. Der Entwickler des Spiels ist „thatgamecompany“ und der Publisher „Sony Interactive Entertainment“ und „Annapurna Interactive“. Das Spiel startet in einer Wüste, in der Ferne liegt ein großer Berg, an dessen Gipfel ein Lichtstrahl in den Himmel ragt. Der Spieler spielt eine Figur in einer roten Robe. Auf dem Weg zum Berg kann der Spieler mit bestehender Internetverbindung auf einen anderen Spieler treffen. Die Spielenden können nicht miteinander reden, aber sich dennoch gegenseitig helfen. Das Zeil von Journey ist es, den in der Ferne liegenden Berg zu erreichen. 


\section{Nutzung von Formen in Spielen}

\subsection{Beispiel – Mario Kart 8}

\subsection{Beispiel – Journey}




\chapter{PC und VR Welt}
\section{VR Spieler}
Dieses Dokument ist als vorwiegend technische Starthilfe für das
Erstellen einer Masterarbeit (oder Bachelorarbeit) mit \latex
gedacht und ist die Weiterentwicklung einer früheren
Vorlage\footnote{Nicht mehr verfügbar.} für das Arbeiten mit
Microsoft \emph{Word}. Während ursprünglich daran gedacht war, die
bestehende Vorlage einfach in \latex zu übernehmen, wurde rasch
klar, dass allein aufgrund der großen Unterschiede zum Arbeiten
mit \emph{Word} ein gänzlich anderer Ansatz notwendig wurde. Dazu
kamen zahlreiche Erfahrungen mit Diplomarbeiten in den
nachfolgenden Jahren, die zu einigen zusätzlichen Hinweisen Anlass gaben.


\subsection{Spell System}
Zitat \cite{bobsch 123 bobsch}

\subsection{VR Umgebung}
Zitat \cite{bobsch 123 bobsch}


\section{PC Spieler}
Dieses Dokument ist als vorwiegend technische Starthilfe für das
Erstellen einer Masterarbeit (oder Bachelorarbeit) mit \latex
gedacht und ist die Weiterentwicklung einer früheren
Vorlage\footnote{Nicht mehr verfügbar.} für das Arbeiten mit
Microsoft \emph{Word}. Während ursprünglich daran gedacht war, die
bestehende Vorlage einfach in \latex zu übernehmen, wurde rasch
klar, dass allein aufgrund der großen Unterschiede zum Arbeiten
mit \emph{Word} ein gänzlich anderer Ansatz notwendig wurde. Dazu
kamen zahlreiche Erfahrungen mit Diplomarbeiten in den
nachfolgenden Jahren, die zu einigen zusätzlichen Hinweisen Anlass gaben.

\subsection{Spieler Größe}
Zitat \cite{bobsch 123 bobsch}

\subsection{Fähigkeiten}
Zitat \cite{bobsch 123 bobsch}

\section{Erwartete Probleme}
Dieses Dokument ist als vorwiegend technische Starthilfe für das
Erstellen einer Masterarbeit (oder Bachelorarbeit) mit \latex
gedacht und ist die Weiterentwicklung einer früheren
Vorlage\footnote{Nicht mehr verfügbar.} für das Arbeiten mit
Microsoft \emph{Word}. Während ursprünglich daran gedacht war, die
bestehende Vorlage einfach in \latex zu übernehmen, wurde rasch
klar, dass allein aufgrund der großen Unterschiede zum Arbeiten
mit \emph{Word} ein gänzlich anderer Ansatz notwendig wurde. Dazu
kamen zahlreiche Erfahrungen mit Diplomarbeiten in den
nachfolgenden Jahren, die zu einigen zusätzlichen Hinweisen Anlass gaben.

\subsection{Wenige Bilder pro Sekunde}
Zitat \cite{bobsch 123 bobsch}

\subsection{Bewegungskrankheit}
Zitat \cite{bobsch 123 bobsch}

\subsection{Platz und Bewegung}
Zitat \cite{bobsch 123 bobsch}

\subsection{Interaktionen}
Zitat \cite{bobsch 123 bobsch}

\trennseite{Performance-Optimierung von 3D Assets in VR-Spielen}{Ian Memic}{Alexander Hager}{Mediendesign - Gamedesign}
\chapter{Begriffsdefinitionen}

Bei "`Tricks 'n' Treats"' handelt es sich um ein asymmetrisches Couch-Party VR-Spiel. [Hier fehlt, denke ich, eine kurze Beschreibung von unserem Spiel.]

In diesem Kapitel werden die in "`Tricks 'n' Treats"' verwendeten Technologien erklärt und die grundlegenden Konzepte des Game-Designs die verwendet werden um die  Erlebnisse der und die Interaktionen zwischen den Spielenden zu beeinflussen.

\section{VR trifft Couchparty}

\subsection{Was ist Virtual Reality?}

Virtual Reality bezeichnet Bilder und Töne, die von einem Computer erzeugt werden und dem Benutzer, der mit Hilfe von Sensoren mit ihnen interagieren kann, fast real erscheinen.\cite{_oxford_dict} Sie ist trotz anderer Anwendungsbereiche, dank der Immersion und Interaktionsmöglichkeiten die sie bietet, besonders in der Gaming-Industrie relevant geworden. [bitkom research]. Spielenden können neue Erlebnisse geboten werden, welche wiederum eigene  Game-Design-Fragen aufbringen die es zu beantworten gilt.

VR bringt neben der Interaktivität noch einen weiteren gewaltigen Vorteil: Intuition. Greifen, Umschauen, Bewegen, alles funktioniert wie man es erwartet. Gaming-Neulinge haben häufige Probleme die richtigen Tasten auf der Tastatur zu drücken und auch klassische Gamepads sind nicht optimal, in VR gibt es zum einen weniger Knöpfe die gedrückt werden können und zum anderen wird viel unterbewusster gearbeitet. Spielende verstehen in VR schneller wie er das Spiel gesteuert wird und können sich mehr Gedanken darüber machen was das Ziel des Spiels ist. Insbesondere für Party-Spiele (Zu denen wir gleich kommen werden.) ist das sehr wichtig. [VR ist ein gutes Medium für Party-Spiele -> Es ist "`einfach"' man versteht schneller was man tun muss]

[
Bild: 
	Gegenüberstellung von Interaktionen VR - Nicht VR (Greifen, Knöpfe)
	Bewegen? Drehen, Umschauen, Firstperson
]
[Greifen Tastatur-E vs VR-Grab (hl vs alyx)]
[field: interaction design]

\subsection{Was ist Couchparty?}

Couchparty-Spiele, oft auch nur Party-Spiele oder local-multiplayer-games, sind Computerspiele, die eine Gruppe (die sich meist untereinander kennt) gemeinsam spielt. Anders als bei klassischen Online-Multiplayer-Spielen werden diese zusammen in einem Raum gespielt. Sie sind jedoch nicht zu verwechseln mit LAN klassikern wie Counter Strike\footnote{Counter Strike ist eine Reihe von taktischen Multiplayer-Ego-Shootern [verbesserungswürdig]} oder Unreal Tournament\footnote{[Sehr ähnich wie cs:go? anderes Beispiel]} die meist nur bei leidenschaftlichen Gamern gespielt werden, da sie Vorbereitung, Know-How und teure Ausrüstung benötigen. Das Alleinstellungsmerkmal dieser Art von Spielen ist, dass diese auf nur einem Bildschirm vor einer Couch, daher der Name, aus gespielt werden und nur minimale Ausrüstung, wie zum beispielsweise mehrere Controller, wobei oft auch ein einfaches Handy reicht, benötigt wird.

In Couchparty-Spielen müssen Spielende schnell verstehen worum es geht, daher stützen sich viele Spiele auf "`Zufall"'. Dadurch können Anfangende schnell mit den anderen mitkommen, auch wenn sie das Spiel noch nie zuvor gespielt haben.

\subsection{Was sind coop-games?}

Kooperation in Team-basierten-Spielen ist sehr wichtig. Coop-games (= cooperative-games/Kooperative-Spiele) sind eine besondere Art von Team-basierten-Spielen die besonders auf Gemeinschaft der Spielenden setzt und diese auch durch Game-Design voraussetzt. Bei klassischen Team-basierten-Spielen ist Kommunikation zwar wichtig jedoch nicht essentiell. Aufgaben der Spielenden werden in coop-games so verteilt, dass ein Ziel nicht ohne die Zusammenarbeit der anderen erreicht werden kann. Es entsteht hierbei eine interessante Dynamik in der die Spielenden komplett voneinander abhängig sind. [symmetrische coop-games: lego games, battle block theater / ist overcooked asymmetrisch?]

\subsection{Was ist bedeutet Asymmetrie?}

Asymmetrie beschreibt in Videospielen das (board-) design. Die Spielenden haben in einem solchen Spiel ein hoch unterschiedliches action-set, das bedeutet sie haben einen anderen Einfluss auf die Spielwelt und andere Spieler.\cite{_balancing_asymmetric_video_games}

% falls es englische Arbeiten gibt, können englische Bereiche der Arbeit mit \begin{english} und \end{english} umschlossen werden.
%\begin{english}
%\trennseite{Collection and analysis of gameplay metrics}{Ian Hornik}{Mag. Andreas Metz}{Media design - Game design}
%\include{hornik/hornik}
%\end{english}

%\trennseite{Titel der Arbeit B}{SchuelerIn B}{Titel + Name Betreuer von Arbeit B}{Mediendesign - Gamedesign}
%\chapter{Einleitung}
\label{cha:sa_Einleitung}

\section{Virtuelle Realität}
Die Virtuelle Realität (Virtual Reality, VR) ist eine computergenerierte virtuelle Umgebung, welche in Echtzeit berechnet wird. In dieser ist es möglich sich umzuschauen und je nach Anwendung kann man sich auch Bewegen und beliebig mit Objekten interagieren. Es gibt zwei verschiedene Möglichkeiten Virtual Reality zu erleben. Einerseits gibt es speziell angefertigte Räume, in welchen Großbildleinwände angebracht sind, andererseits gibt es auch Head-Mounted-Displays (VR Brillen, die man aufsetzt), welche hier im Fokus stehen werden. In der Virtuellen Realität gibt es eine breit gefächerte Reichweite an Spielen, in welchen man verschiedenste Situationen erleben kann. Anfangs waren Spiele für VR nur simple Testräume, in welchen man mit Objekten interagieren konnte. Mittlerweile ist es jedoch möglich riesige Spiele mit einer eigenen Geschichte zu erleben. Die Interaktionen, welche damals ein ganzes Spiel ausmachten, sind seit einiger Zeit nur noch die Basis, worauf neue Spiele aufbauen.

\subsection{Asymmetrische VR Spiele}
Die meisten VR-Spiele kapseln einen von der Außenwelt ab. Sobald man die VR-Brille aufsetzt, befindet man sich in einer neuen, digitalen Welt. Eine Zeit lang war die zwischenmenschliche Interaktion in der virtuellen Realität gar nicht möglich. Seit neuerem gibt es auch Spiele, welche unterstützen, dass VR-Spieler über das Internet miteinander spielen können (Online-Multiplayer). Somit braucht man jedoch zwei HMDs (VR-Brillen), zwei PCs und man kann in der Regel nicht nebeneinander spielen. Asymmetrische VR-Spiele versuchen, reguläre PC-Spiele mit VR-Spielen zu kombinieren. So können z.B. zwei Freunde mit einem PC und einer VR-Brille miteinander spielen. Das funktioniert z.B., indem die zwei Spieler sich in derselben Spielwelt befinden, jedoch unterschiedliche Charaktere steuern. Durch die verschiedenen Steuerungsmöglichkeiten (VR-Controller, Tastatur und Maus, Gamepad) ergibt sich oft automatisch eine bestimme Rollenverteilung. So könnte der PC-Spieler z.B. einen Menschen spielen, welcher durch ein Level manövrieren muss und zeitgleich steuert der VR-Spieler einen Riesen, welcher den Menschen behindern muss, indem er ihm z.B. Steine in den Weg legt.

\section{Zielsetzung}
Das Ziel dieser Arbeit ist, die technische Implementierung und Optimierung von asymmetrischen VR Spielen genauer zu Untersuchen und gefundene technische Probleme genauer zu erläutern.


\section{Warum Asymmetrisches VR}


%\include {schueler_b/abbildungen}
%\include {schueler_b/prozessoren}

%%%----------------------------------------------------------
%%%Anhang
\appendix
\chapter{Projektdokumentation}

\section{Meilensteine}

\subsection{Prototype (23.09)}
Die Hauptmechaniken wurden Implementiert, und erste Systeme Getestet. Der Art-Style von unserem Spiel wurde ausgearbeitet, und dazu wurden erste Style-Guides, Charaktere, Farbpaletten und Target-Experience festgelegt.

\subsection{Iteration Interim 1 (20.10)}
Es wurden die Spell-Types und des Raum-Design definiert und implementiert. Der Gameloop des Spieles funktionierte erstmals komplett. Es wurden außerdem erste Playtests in einem Test-Level durchgeführt, um mögliche Probleme mit dem Design herauszufinden.

\subsection{Iteration Interim 2 (22.12)}
Ein Großteil der Grafiken und 3D-Modelle wurden importiert. Außerdem wurde die Mana-Menge und andere Variablen anhand von Playtests, die durchgeführt wurden, um früher auf Fehler im Design zu stoßen und sie zu beheben, angepasst. Die Technik des Spieles ist auch größtenteils fertiggestellt.

\subsection{Vertical Slice (30.03)}
Ein Vertical Slice des Projekts wurde finalisiert. Der Gameloop ist nun ohne Errors durchspielbar, und die gewünchte Experience, Art/Design und Performance wurden gut umgesetzt. Es gibt außerdem ein vollständiges Presskit inklusive Trailer.

\section{Playtest Protokolle \& Ergebnisse}
Wir haben bereits mehrere Playtests durchgeführt. Die ersten mussten aufgrund unerwarteter Fehler frühzeitig abgebrochen werden. Die Playtests, die Problemlos abgelaufen sind, haben allerdings die gewünschte Experience erzeugt. Nach den Playtests, haben wir die Teilnehmenden anschließend darum gebeten einige Fragen zu beantworten.

\subsection{Welcher Spell ist der Lustigste?}
Der Großteil der PC- als auch der VR-Spielenden gaben an, dass die Bombe der beste Spell sei. Auf Platz zwei war die Mikado-Barrikade, da damit gute Kombinationen erzielt werden konnten.

\subsection{Welcher Spell ist der Unlustigste?}
Die wolke wurde von vielen Spielenden als "`sinnlos"' oder "`Unfair"' beschrieben. Das ist wahrscheinlich darauf zurückzuführen, dass sie erst relativ spät Einfluss auf die Snowboarder hat und nicht sehr genau platziert werden kann, also nicht wirklich ein Zusammenhang von Wolken-kill zu VR-Skill besteht.

\subsection{In welcher Rolle hattest du mehr Spaß?}
Ca. 65\% der Spielenden gaben an, dass sie als Snowboarder mehr Spaß hatten. Manche hatten leider auch Probleme mit Motion-Sickness.

\subsection{Gab es Momente wo du stark mitgefiebert hast?}
Die Spannung wenn nur nur noch ein Snowboarder im Spiel ist, ist sehr intensiv. Vor allem gegen Ende einer Runde.

\subsection{Wie würdest du die Gesamterfahrung bewerten?}
Der Durchschnittliche Score der Gesamterfahrung beträgt ~7.5/10 Punkten. Die Erfahrung der Snowboarder war laut Umfrage etwas besser. "`Ist für ein paar runden ganz witzig."'

\subsection{Gab es Momente in denen dir langweilig war?}
Manche Snowboarder haben sich überfordert gefühlt, wenn sie bereits mit einer erfahrenen Gruppe gespielt haben. Viele VR-Spielende gaben an, dass die Wartezeiten zwischen den Runden verkürzt werden sollte.

%\includepdf[page=-]{formulare/PlaytestProtokolle}

\section{Begleitprotokolle}
%\includepdf[page=-]{formulare/Zeitprotokolle}
	    % Projektdokumentation
\chapter{Kooperationsvereinbarung}

%\includepdf[page=-]{formulare/Kooperationsvereinbarung}	% Kooperationsvereinbarung
\chapter{Inhalt des Datenträgers}

\section{Projektdateien}
\begin{FileList}{/Game}
	\fitem{Projekt} Unity + Wwise Projekt
	\fitem{Build} Windows Build des Spieles
\end{FileList}

\section{Arbeit}
\begin{FileList}{/Diplomarbeit}
	\fitem{_DaBa.pdf} Diplomarbeit (Gesamtdokument) 
	\fitem{LaTex} LaTex Projekt
\end{FileList}

\section{Game Design}
\begin{FileList}{/GameDesign}
	\fitem{Concepts} Konzept Dateien
	\fitem{Playtests} Ergebnisse der Playtests
\end{FileList}
 
\section{Presskit}
\begin{FileList}{/Presskit}
	\fitem{Trailer} Trailer des Spieles
	\fitem{Overview} Ein PDF mit Infos zum Spiel
	\fitem{Screenshots} Screenshots des Spieles
\end{FileList}
	            % Inhalt des Datenträgers

%%%----------------------------------------------------------
\printbibliography
\addcontentsline{toc}{chapter}{Abbildungsverzeichnis}
\listoffigures
\addcontentsline{toc}{chapter}{Tabellenverzeichnis}
\listoftables


%%%----------------------------------------------------------

%%%Messbox zur Druckkontrolle
\include{messbox}

\end{document}
