\documentclass[diplom,german]{hgbthesis}
% Zulässige Class Options: 
%   Typ der Arbeit: diplom, master (default), bachelor, praktikum 
%   Hauptsprache: german (default), english
%%------------------------------------------------------------


% Zusatzpakete (bei Bedarf einkommentieren):
% \usepackage{enumitem}    % erlaubt Listen mit einem <key>=<value> - Format
% \usepackage{tikz}        % ermöglicht das Zeichnen von Grafik-Elementen (zB Linien, Kreise etc.) - Bsp.: https://www.overleaf.com/learn/latex/TikZ_package
% \usetikzlibrary{arrows,shapes,positioning} % Zusatzbibliothek für tikz (zB Pfeil etc.)
\usepackage{pdfpages}      % ermöglicht das includieren (mit \include) von pdf-Dateien

\graphicspath{{images/}}   % Verzeichnis mit den Bildern (zB für \figure)
%\logofile{spengerlogo_cmyk_small}   % Name des .pdf mit dem Logo

\bibliography{literatur}   % Datei(en) mit der Bibliothek (.bib) für das Quellenverzeichnis 
% Beispiel für mehrere .bib files: \bibliography{schueler1/literatur,schueler2/literatur}

%%%----------------------------------------------------------
\begin{document}
%%%----------------------------------------------------------

% offizielle Formulare inkludieren
%\includepdf[page=-]{formulare/HTL_RDP_Titelseite_DA_A4}
%\includepdf[page=-]{formulare/HTL_RDP_Dokumentation_DA_DE_A4}
%\includepdf[page=-]{formulare/HTL_RDP_Dokumentation_DA_EN_A4}

%\begin{minipage}{\textwidth}
%	\includepdf[scale=0.85,pages=1]{formulare/HTL_RDP_Titelseite_DA_A4}
%\end{minipage}

% Daten zur Arbeit: _________--------------------------------
\title{"'Tricks 'n' Treats"': Entwicklung eines asymmetrischen VR- und Couch-Partyspiels}
\author{Ian Memic, Simon Buketits, Emma Wehinger und Felix Kaspar}
\studiengang{Mediendesign - Gamedesign}
\studienort{Wien}
\abgabedatum{2021}{04}{05}	% {YYYY}{MM}{DD}

%%%----------------------------------------------------------
\frontmatter
\maketitle
\tableofcontents
%%%----------------------------------------------------------

%\include{vorwort}		% ggfs. weglassen

\chapter{Kurzfassung}

Diese Diplomarbeit befasst sich mit der Entwicklung des asymmetrischen lokalen Multiplayer-Spieles Tricks ‘n’ Treats. 
Der erste Teil der Arbeit befasst sich mit der Entwicklung von asymmetrischen Virtual Reality Spielen, wobei ein besonderer Fokus auf Performance-Optimierung, den aufgetretenen Problemen und technischen Limitationen von asymmetrischen Virtual Reality Spielen. 
Im zweiten Teil wird genauer bearbeitet, wie sich die Dynamiken zwischen PC und VR bilden, wie man diese beeinflusst und im Falle des Spieles eine gute Interaktion zwischen den beiden konstruieren kann. 
Der dritte Teil der Arbeit befasst sich mit der Welt und Atmosphäre des Spiels. Besonders widmet er sich wie Farb- und Formgebung die Gefühle und Aktionen der SpielerInnen beeinflussen. 
Der vierte Teil beschäftigt sich mit dem 3D-Modeling und geht spezifisch auf die Performance Optimierung der 3D-Assets ein. Dabei liegt ein großer Fokus auf der Recherche und Verwendung verschiedener Optimierungs-Methoden und deren Vor- und Nachteile.
   % vom Team gemeinsam zu verfassen
\include{abstract}		% vom Team gemeinsam in englisch zu verfassen	

%%%----------------------------------------------------------
\mainmatter         % Hauptteil (ab hier arab. Seitenzahlen)
%%%----------------------------------------------------------

% Trennseite erzeugt ein Titelblatt für eine neue Arbeit (zB wenn die Arbeit A endet und die Arbeit B beginnt)
% Parameter: {}Titel der Arbeit}{Autor}{Betreuer}{Abteilung}
% Beispiele: 
%	\trennseite{Gestaltpsychologie und stilisiertes Game Audio}{Noah Diem}{Dipl.Ing. Michael Schreiber}{Mediendesign - Gamedesign}
%	\trennseite{Collection and analysis of gameplay metrics}{Ian Hornik}{Mag. Andreas Metz}{Media design - Game design}

\trennseite{Technische Implementierung und Optimierung von asymmetrischen VR Spielen.
}{Simon Buketits}{Titel + Kristian Ljubek}{Mediendesign - Gamedesign}
\chapter{Einleitung}
\label{cha:sa_Einleitung}

\section{Virtuelle Realität}
Die Virtuelle Realität (Virtual Reality, VR) ist eine computergenerierte virtuelle Umgebung, welche in Echtzeit berechnet wird. In dieser ist es möglich sich umzuschauen und je nach Anwendung kann man sich auch Bewegen und beliebig mit Objekten interagieren. Es gibt zwei verschiedene Möglichkeiten Virtual Reality zu erleben. Einerseits gibt es speziell angefertigte Räume, in welchen Großbildleinwände angebracht sind, andererseits gibt es auch Head-Mounted-Displays (VR Brillen, die man aufsetzt), welche hier im Fokus stehen werden. In der Virtuellen Realität gibt es eine breit gefächerte Reichweite an Spielen, in welchen man verschiedenste Situationen erleben kann. Anfangs waren Spiele für VR nur simple Testräume, in welchen man mit Objekten interagieren konnte. Mittlerweile ist es jedoch möglich riesige Spiele mit einer eigenen Geschichte zu erleben. Die Interaktionen, welche damals ein ganzes Spiel ausmachten, sind seit einiger Zeit nur noch die Basis, worauf neue Spiele aufbauen.

\subsection{Asymmetrische VR Spiele}
Die meisten VR-Spiele kapseln einen von der Außenwelt ab. Sobald man die VR-Brille aufsetzt, befindet man sich in einer neuen, digitalen Welt. Eine Zeit lang war die zwischenmenschliche Interaktion in der virtuellen Realität gar nicht möglich. Seit neuerem gibt es auch Spiele, welche unterstützen, dass VR-Spieler über das Internet miteinander spielen können (Online-Multiplayer). Somit braucht man jedoch zwei HMDs (VR-Brillen), zwei PCs und man kann in der Regel nicht nebeneinander spielen. Asymmetrische VR-Spiele versuchen, reguläre PC-Spiele mit VR-Spielen zu kombinieren. So können z.B. zwei Freunde mit einem PC und einer VR-Brille miteinander spielen. Das funktioniert z.B., indem die zwei Spieler sich in derselben Spielwelt befinden, jedoch unterschiedliche Charaktere steuern. Durch die verschiedenen Steuerungsmöglichkeiten (VR-Controller, Tastatur und Maus, Gamepad) ergibt sich oft automatisch eine bestimme Rollenverteilung. So könnte der PC-Spieler z.B. einen Menschen spielen, welcher durch ein Level manövrieren muss und zeitgleich steuert der VR-Spieler einen Riesen, welcher den Menschen behindern muss, indem er ihm z.B. Steine in den Weg legt.

\section{Zielsetzung}
Das Ziel dieser Arbeit ist, die technische Implementierung und Optimierung von asymmetrischen VR Spielen genauer zu Untersuchen und gefundene technische Probleme genauer zu erläutern.


\section{Warum Asymmetrisches VR}


\chapter{Überblick über die Materie}
\label{cha:sa_Einleitung}


\section{Farbpsychologie}

\subsection{Was ist Farbe?}
Wissenschaftlich betrachtet, sind Farben Eindrücke, die unsere Sinne durch die Augen und das Gehirn vermittelt bekommen. Farben entstehen durch die Absorption und Reflexion, die das Licht auf Oberflächen wirft. Das menschliche Auge ist für die Sinnesempfindung von Farben verantwortlich, die dann wiederum im Gehirn verarbeitet werden. Man kann Farbe auch als Eigenschaft des Lichts betiteln. Prinzipiell nimmt jeder Mensch Farben und Farbtöne anders wahr. Welche Farbe wahrgenommen wird, hängt von dem Gegenstand ab, der Lichtquelle und den Augen des Betrachters. Licht hat außerdem auch bestimmte Wellenlängen, je nach Länge der Welle nimmt das Auge eine andere Farbe war. Schwarz ist keine Farbe, denn sie kann auch in Abwesenheit des Lichts und völliger Dunkelheit existieren.
Weißes Licht kann mithilfe eines Prismas in Spektralfarben gebrochen werden. Spektralfarben sind diejenigen Farben, die für das menschliche Auge einen sichtbaren Farbeindruck und somit einen Farbton hinterlassen. Insgesamt gibt es sechs verschiedene Spektralfarben. Spektralfarben werden umgangssprachlich auch „Regenbogenfarben“ genannt. Zu ihnen zählen Rot, Orange, Gelb, Grün, Blau, Indigo und Violett. Die durch die Brechung von Licht entstandenen Farben können nicht in weitere Farbtöne zerlegt werden. Das Licht kann manchmal auch infrarotes Licht oder ultraviolettes Licht enthalten, welches für das bloße Auge nicht sichtbar ist.
Wenn man jedoch gelbes Licht oder jede andere Spektralfarbe durch ein Prisma fallen lässt, bekommt man dieselbe Farbe raus, die man im Prisma brechen wollte. Man kann also als Schlussfolgerung daraus ziehen, das weißes Licht das einzige ist, was durch ein Prisma in Spektralfarben gebrochen werden kann.

\begin{figure}[h]
	\centering
	\includegraphics[width=10cm]{Farbprisma}
	\caption{\cite{_steam_hardware}}
\end{figure}

In der Kunst ist Farbe das beste Mittel, um visuell Emotionen auszudrücken oder Emotionen bei Menschen zu erzeugen. Farben sind das beste Instrument, wenn das darum geht, bei einem Menschen Gefühle und Emotionen auszulösen. Aber nicht nur, dass es wird dem Betrachter dadurch auch eine bestimmte Stimmung vermittelt. Künstler können dadurch gezielt bei Menschen bestimmte Gefühle auslösen und das alleine durch das Einsetzen von bestimmten Farben und Farbtönen.
Dabei muss beachtet werden, dass Farben in verschiedenen Kulturen verschiedene Bedeutungen haben und die Symbolik hinter manchen Farbtönen in bestimmten Ländern und Kulturkreisen eine andere ist. Doch grundsätzlich kann man jeder Farbe eine bestimmte Bedeutung und Emotion zuschreiben, die vielleicht in manchen Kulturen ein wenig abweicht, aber im Grunde genommen immer sehr ähnlich ist.
Man kann aber klar die kulturelle Bedeutung von Farben mit der psychologischen Bedeutung von Farben trennen. Die wichtigsten Farben für Kunst und Design sind die Farben Blau, Gelb, Grün, Orange, Rosa, Rot, Violett, Grau und Braun. Jede dieser Farben hat eine andere Wirkung auf die Gefühle und Emotionen der Betrachter. Der psychologische Effekt hinter der Wirkung einer Farbe hat rein gar nichts mit der kulturellen Interpretation dieser zu tun. 
Es gibt eine endliche Anzahl an Variationen, die eine Farbe für das menschliche Auge annehmen kann. Grundsätzlich gibt es eine unendliche Anzahl an Schattierungen, die eine Farbe haben kann, wobei die meisten für das Sinnesorgan eines Menschen nicht sichtbar sind. Die Schattierung einer Farbe ändert prinzipiell den psychologischen Effekt nicht.


\subsection{Der Farbkreis und Farbschemas}
Der Farbkreis besteht normalerweise aus 12 Farben, er kann aber auch aus 24 und manchmal aus bis zu 48 Farben bestehen, das hängt aber von dem Ersteller des Farbkreises ab. Das grundlegende Farbrad besteht aber jedoch nur aus 12 Farben. Um den Farbkreis besser zu verstehen, muss man sich zuerst die additive und die subtraktive Farbmischung ansehen.  

Farben können in Primär -, Sekundär -und Tertiärfarben aufgeteilt werden. Dabei unterscheidet man zwischen der additiven Farbmischung und der subtraktiven Farbmischung. 

Bei der subtraktiven Farbmischung, auch CMYK-Modell genannt, sind die Primärfarben Gelb, Magenta und Cyan, sie können nicht aus anderen Farben gemischt werden. Jedoch kann aus ihnen jede existierende Farbe gemischt werden. Die Sekundärfarben in diesem Modell sind Rot, Grün und Blau und die Tertiärfarbe ist Schwarz. Mischt man nun die Primärfarben gelb, Magenta und Cyan, erhält man infolge dessen die Sekundärfarben Rot, Grün und Blau. Die gesamte Mischung ergibt dann Schwarz.

\centerfirst
\includegraphics[width=8cm]{images/Flyeralarm_CMYK_de}


Die additive Farbmischung, auch RGB-Modell genannt, besteht aus den Lichtfarben Rot, Grün und Blau. Rot, Grün und Blau sind in diesem Modell die Primärfarben, also Grundfarben. Die Sekundärfarben sind Gelb, Magenta und Cyan. Die Tertiärfarbe ist Weiß. Diese entsteht durch die Mischung der Grundfarben Rot, Grün und Blau.

\includegraphics[width=8cm]{images/Flyeralarm_RGB_de}



\subsection{Farbtemperatur}

\subsection{Farbrelativität}

\subsection{Farbsättigung}

\subsection{Erstellung von Farbpaletten}

\subsection{Farbe, Licht und Schatten}

\subsection{Wie wirkt sich Farbe auf die Stimmung aus?}



\section{Formpsychologie}

\subsection{Gestalt und Form}

\subsection{Linien und Linien- Stile}

\subsection{Kompositionslinien}

\subsection{Wie Formen und Linien Emotionen erzeugen}



\section{Environment Design}

\subsection{Character und Environment Shapes}

\subsection{Character Centric Environment Design }

\subsection{Bildkomposition}







\chapter{Möglichkeiten zur Implementierunge}
\label{cha:sa_Einleitung}

\section{Nutzung von Farben in Spielen}
Die Nutzung von Farben in Videospielen war mit ihrer Entstehung sehr mit dem damaligen Stand der Technik verbunden. 1972 wurde die Farbüberlagerung erfunden und hat es ermöglicht, dass Videospiele in Farbe dargestellt werden können. Davor konnten Spiele nur in Schwarz-Weiß oder Monochrom dargestellt werden. Ab Ende der 1980er sind fast alle Videospiele in Farbe. 1985 veröffentlichte Sega das Sega Master System, welches Spiele mit 32 verschiedenen Farben darstellen konnte. Schon 9 Jahre später veröffentlicht Sony die Playstation, welche es ermöglicht, Spiele mit rund 16.7 Millionen Farben darzustellen. Heutzutage können Spiele mit über 16.7 Milliarden verschiedenen Farben dargestellt werden. 

Durch die Nutzung von Farben kann man dem Spieler Gefühle und Emotionen auf eine sehr einfache und natürliche Weise vermitteln. Prinzipiell ist das Farbempfinden von Kulturkreisen immer etwas unterschiedlich, dennoch gibt es auf die meisten Farben eine universelle Reaktion. Wie Menschen Farben wahrnehmen, hängt nicht nur von ihrer Kultur ab, sondern auch von ihrem Alter, ihrer Herkunft, ihrem Geschlecht und ihrem ethnischen Hintergrund. Wenn ein Spieler in einem Spiel zum Beispiel Eierfarbe Rot sieht, wird ihm signalisiert, dass er in Gefahr ist. Rot steht prinzipiell für Energie, Liebe, Selbstbewusstsein und Leidenschaft, wird aber auch als Warnfarbe gesehen. Rot wird mit Gefahren in Verbindung gesetzt. Explosionen, Feuer und Blut etc. wären dafür ein gutes Beispiel. In vielem Actionspielen wird unter anderem ein an den Ecken und Kanten roter Bildschirm eingesetzt, wenn der Spieler verletzt ist. Das signalisiert dem Spieler, dass er ab jetzt wachsamer und vorsichtiger sein muss. 

In vielen MMORPGs oder in Singleplayer-Spielen, werden viele sammelbaren Objekte, Items oder Schatztruhen in bestimmten Farben angezeigt. Meist werden die Farben Weiß, Grün, Blau, Lila und Gold verwendet, wobei Weiß weniger gut ist, als Violett und Gold. Violett wird mit dem Adel verbunden, denn früher konnten sich nur sehr wohlhabende Menschen das violette Farbpigment leisten. Die Farbe Violett steht für Eleganz, Würde und Raffinesse. Die Farbe hat aber auch etwas Magisches und Mystisches an sich. Aus diesem Grund sind sehr seltene und außergewöhnliche Objekte und Items meist Violett, da sie dem Spieler signalisieren sollen, dass er etwas Besonderes und Wertvolles gefunden hat. 

\subsection{Beispiel – Mario Kart 8}

\subsection{Beispiel – Journey}
Journey ist ein Adventure-Spiel, welches ohne jegliche Worte auskommt. Es wurde erstmals 2012 für die Playstation 3 veröffentlicht. Der Entwickler des Spiels ist „thatgamecompany“ und der Publisher „Sony Interactive Entertainment“ und „Annapurna Interactive“. Das Spiel startet in einer Wüste, in der Ferne liegt ein großer Berg, an dessen Gipfel ein Lichtstrahl in den Himmel ragt. Der Spieler spielt eine Figur in einer roten Robe. Auf dem Weg zum Berg kann der Spieler mit bestehender Internetverbindung auf einen anderen Spieler treffen. Die Spielenden können nicht miteinander reden, aber sich dennoch gegenseitig helfen. Das Zeil von Journey ist es, den in der Ferne liegenden Berg zu erreichen. 


\section{Nutzung von Formen in Spielen}

\subsection{Beispiel – Mario Kart 8}

\subsection{Beispiel – Journey}




\chapter{PC und VR Welt}
\section{VR Spieler}
Dieses Dokument ist als vorwiegend technische Starthilfe für das
Erstellen einer Masterarbeit (oder Bachelorarbeit) mit \latex
gedacht und ist die Weiterentwicklung einer früheren
Vorlage\footnote{Nicht mehr verfügbar.} für das Arbeiten mit
Microsoft \emph{Word}. Während ursprünglich daran gedacht war, die
bestehende Vorlage einfach in \latex zu übernehmen, wurde rasch
klar, dass allein aufgrund der großen Unterschiede zum Arbeiten
mit \emph{Word} ein gänzlich anderer Ansatz notwendig wurde. Dazu
kamen zahlreiche Erfahrungen mit Diplomarbeiten in den
nachfolgenden Jahren, die zu einigen zusätzlichen Hinweisen Anlass gaben.


\subsection{Spell System}
Zitat \cite{bobsch 123 bobsch}

\subsection{VR Umgebung}
Zitat \cite{bobsch 123 bobsch}


\section{PC Spieler}
Dieses Dokument ist als vorwiegend technische Starthilfe für das
Erstellen einer Masterarbeit (oder Bachelorarbeit) mit \latex
gedacht und ist die Weiterentwicklung einer früheren
Vorlage\footnote{Nicht mehr verfügbar.} für das Arbeiten mit
Microsoft \emph{Word}. Während ursprünglich daran gedacht war, die
bestehende Vorlage einfach in \latex zu übernehmen, wurde rasch
klar, dass allein aufgrund der großen Unterschiede zum Arbeiten
mit \emph{Word} ein gänzlich anderer Ansatz notwendig wurde. Dazu
kamen zahlreiche Erfahrungen mit Diplomarbeiten in den
nachfolgenden Jahren, die zu einigen zusätzlichen Hinweisen Anlass gaben.

\subsection{Spieler Größe}
Zitat \cite{bobsch 123 bobsch}

\subsection{Fähigkeiten}
Zitat \cite{bobsch 123 bobsch}

\section{Erwartete Probleme}
Dieses Dokument ist als vorwiegend technische Starthilfe für das
Erstellen einer Masterarbeit (oder Bachelorarbeit) mit \latex
gedacht und ist die Weiterentwicklung einer früheren
Vorlage\footnote{Nicht mehr verfügbar.} für das Arbeiten mit
Microsoft \emph{Word}. Während ursprünglich daran gedacht war, die
bestehende Vorlage einfach in \latex zu übernehmen, wurde rasch
klar, dass allein aufgrund der großen Unterschiede zum Arbeiten
mit \emph{Word} ein gänzlich anderer Ansatz notwendig wurde. Dazu
kamen zahlreiche Erfahrungen mit Diplomarbeiten in den
nachfolgenden Jahren, die zu einigen zusätzlichen Hinweisen Anlass gaben.

\subsection{Wenige Bilder pro Sekunde}
Zitat \cite{bobsch 123 bobsch}

\subsection{Bewegungskrankheit}
Zitat \cite{bobsch 123 bobsch}

\subsection{Platz und Bewegung}
Zitat \cite{bobsch 123 bobsch}

\subsection{Interaktionen}
Zitat \cite{bobsch 123 bobsch}
\chapter{Performance Optimierung}
\section{VR Performance}
Dieses Dokument ist als vorwiegend technische Starthilfe für das
Erstellen einer Masterarbeit (oder Bachelorarbeit) mit \latex
gedacht und ist die Weiterentwicklung einer früheren
Vorlage\footnote{Nicht mehr verfügbar.} für das Arbeiten mit
Microsoft \emph{Word}. Während ursprünglich daran gedacht war, die
bestehende Vorlage einfach in \latex zu übernehmen, wurde rasch
klar, dass allein aufgrund der großen Unterschiede zum Arbeiten
mit \emph{Word} ein gänzlich anderer Ansatz notwendig wurde. Dazu
kamen zahlreiche Erfahrungen mit Diplomarbeiten in den
nachfolgenden Jahren, die zu einigen zusätzlichen Hinweisen Anlass gaben.


\subsection{Layer Culling}
Zitat

\section{PC Performance}
Dieses Dokument ist als vorwiegend technische Starthilfe für das
Erstellen einer Masterarbeit (oder Bachelorarbeit) mit \latex
gedacht und ist die Weiterentwicklung einer früheren
Vorlage\footnote{Nicht mehr verfügbar.} für das Arbeiten mit
Microsoft \emph{Word}. Während ursprünglich daran gedacht war, die
bestehende Vorlage einfach in \latex zu übernehmen, wurde rasch
klar, dass allein aufgrund der großen Unterschiede zum Arbeiten
mit \emph{Word} ein gänzlich anderer Ansatz notwendig wurde. Dazu
kamen zahlreiche Erfahrungen mit Diplomarbeiten in den
nachfolgenden Jahren, die zu einigen zusätzlichen Hinweisen Anlass gaben.

\subsection{Level of Detail}
Zitat 

\section{Mesh Optimierungen}
Dieses Dokument ist als vorwiegend technische Starthilfe für das
Erstellen einer Masterarbeit (oder Bachelorarbeit) mit \latex
gedacht und ist die Weiterentwicklung einer früheren
Vorlage\footnote{Nicht mehr verfügbar.} für das Arbeiten mit
Microsoft \emph{Word}. Während ursprünglich daran gedacht war, die
bestehende Vorlage einfach in \latex zu übernehmen, wurde rasch
klar, dass allein aufgrund der großen Unterschiede zum Arbeiten
mit \emph{Word} ein gänzlich anderer Ansatz notwendig wurde. Dazu
kamen zahlreiche Erfahrungen mit Diplomarbeiten in den
nachfolgenden Jahren, die zu einigen zusätzlichen Hinweisen Anlass gaben.

\include {buketits/kapitel5}


\trennseite{Design eines asymmetrischen local-multiplayer partygames unter Verwendung des MDA-Frameworks.}{Felix Kaspar}{Titel + René Ksuz}{Mediendesign - Gamedesign}
\chapter{Das Vorhaben}
\label{cha:sa_Einleitung}

Bei "`Tricks 'n' Treats"' handelt es sich um ein asymmetrisches Couch-Party VR-Spiel. [Hier fehlt eine kurze Beschreibung von unserem Spiel.]

In diesem Kapitel werden die in "`Tricks 'n' Treats"' verwendeten Technologien erklärt und die grundlegenden Konzepte des Game-Designs die verwendet werden um die  Erlebnisse der und die Interaktionen zwischen den Spielenden zu beeinflussen.

\section{VR trifft Couchparty}

\subsection{Was ist Virtual Reality?}

Virtual Reality bezeichnet Bilder und Töne, die von einem Computer erzeugt werden und dem Benutzer, der mit Hilfe von Sensoren mit ihnen interagieren kann, fast real erscheinen. [Brauchen Definitionen Quellen? (oxford dict)]. Sie ist trotz anderer Anwendungsbereiche, dank der Immersion und Interaktionsmöglichkeiten die sie bietet, besonders in der Gaming-Industrie relevant geworden. [Quelle angeben]. Spielenden können neue Erlebnisse geboten werden, welche wiederum eigene  Game-Design-Fragen aufbringen die es zu beantworten gilt.

[Gegenüberstellung von Interaktionen VR - Nicht VR]

\subsection{Was ist Couchparty?}

Couchparty-Spiele, oft auch nur Party-Spiele oder Couch-Multiplayer-Spiele, sind Computerspiele, die eine Gruppe (die sich meist untereinander kennt) gemeinsam spielt. Anders als bei klassischen Online-Multiplayer-Spielen werden diese zusammen in einem Raum gespielt und sind damit eine Unterkategorie der local-multiplayer-games. Das Alleinstellungsmerkmal dieser Art von Spielen ist, dass diese auf nur einem Bildschirm vor einer Couch, daher der Name, aus gespielt werden und nur minimale Ausrüstung benötigt wird.

\subsection{Coop games?}

Kooperation in Team-basierten-Spielen ist sehr wichtig. Coop-Games (= Cooperative-Games/Kooperative-Spiele) verteilen die Aufgaben der Spielenden so, dass eine Aufgabe nicht ohne die Hilfe der anderen gelöst werden kann.

\subsection{Was ist Asymmetrie?}

\section{Warum machen Spiele Spaß?}

\section{Game Design um Interaktionen zwischen den Spielenden zu fördern}

\subsection{MDA-Framework}

\subsection{Player Action Feedback}

\subsection{Target Experience}




\trennseite{}{}{Titel + René Ksuz}{Mediendesign - Gamedesign}
\chapter{Das Vorhaben}
\label{cha:sa_Einleitung}

Bei "`Tricks 'n' Treats"' handelt es sich um ein asymmetrisches Couch-Party VR-Spiel. [Hier fehlt eine kurze Beschreibung von unserem Spiel.]

In diesem Kapitel werden die in "`Tricks 'n' Treats"' verwendeten Technologien erklärt und die grundlegenden Konzepte des Game-Designs die verwendet werden um die  Erlebnisse der und die Interaktionen zwischen den Spielenden zu beeinflussen.

\section{VR trifft Couchparty}

\subsection{Was ist Virtual Reality?}

Virtual Reality bezeichnet Bilder und Töne, die von einem Computer erzeugt werden und dem Benutzer, der mit Hilfe von Sensoren mit ihnen interagieren kann, fast real erscheinen. [Brauchen Definitionen Quellen? (oxford dict)]. Sie ist trotz anderer Anwendungsbereiche, dank der Immersion und Interaktionsmöglichkeiten die sie bietet, besonders in der Gaming-Industrie relevant geworden. [Quelle angeben]. Spielenden können neue Erlebnisse geboten werden, welche wiederum eigene  Game-Design-Fragen aufbringen die es zu beantworten gilt.

[Gegenüberstellung von Interaktionen VR - Nicht VR]

\subsection{Was ist Couchparty?}

Couchparty-Spiele, oft auch nur Party-Spiele oder Couch-Multiplayer-Spiele, sind Computerspiele, die eine Gruppe (die sich meist untereinander kennt) gemeinsam spielt. Anders als bei klassischen Online-Multiplayer-Spielen werden diese zusammen in einem Raum gespielt und sind damit eine Unterkategorie der local-multiplayer-games. Das Alleinstellungsmerkmal dieser Art von Spielen ist, dass diese auf nur einem Bildschirm vor einer Couch, daher der Name, aus gespielt werden und nur minimale Ausrüstung benötigt wird.

\subsection{Coop games?}

Kooperation in Team-basierten-Spielen ist sehr wichtig. Coop-Games (= Cooperative-Games/Kooperative-Spiele) verteilen die Aufgaben der Spielenden so, dass eine Aufgabe nicht ohne die Hilfe der anderen gelöst werden kann.

\subsection{Was ist Asymmetrie?}

\section{Warum machen Spiele Spaß?}

\section{Game Design um Interaktionen zwischen den Spielenden zu fördern}

\subsection{MDA-Framework}

\subsection{Player Action Feedback}

\subsection{Target Experience}



% falls es englische Arbeiten gibt, können englische Bereiche der Arbeit mit \begin{english} und \end{english} umschlossen werden.
%\begin{english}
%\trennseite{Collection and analysis of gameplay metrics}{Ian Hornik}{Mag. Andreas Metz}{Media design - Game design}
%\include{hornik/hornik}
%\end{english}

%\trennseite{Titel der Arbeit B}{SchuelerIn B}{Titel + Name Betreuer von Arbeit B}{Mediendesign - Gamedesign}
%\chapter{Einleitung}
\label{cha:sa_Einleitung}

\section{Virtuelle Realität}
Die Virtuelle Realität (Virtual Reality, VR) ist eine computergenerierte virtuelle Umgebung, welche in Echtzeit berechnet wird. In dieser ist es möglich sich umzuschauen und je nach Anwendung kann man sich auch Bewegen und beliebig mit Objekten interagieren. Es gibt zwei verschiedene Möglichkeiten Virtual Reality zu erleben. Einerseits gibt es speziell angefertigte Räume, in welchen Großbildleinwände angebracht sind, andererseits gibt es auch Head-Mounted-Displays (VR Brillen, die man aufsetzt), welche hier im Fokus stehen werden. In der Virtuellen Realität gibt es eine breit gefächerte Reichweite an Spielen, in welchen man verschiedenste Situationen erleben kann. Anfangs waren Spiele für VR nur simple Testräume, in welchen man mit Objekten interagieren konnte. Mittlerweile ist es jedoch möglich riesige Spiele mit einer eigenen Geschichte zu erleben. Die Interaktionen, welche damals ein ganzes Spiel ausmachten, sind seit einiger Zeit nur noch die Basis, worauf neue Spiele aufbauen.

\subsection{Asymmetrische VR Spiele}
Die meisten VR-Spiele kapseln einen von der Außenwelt ab. Sobald man die VR-Brille aufsetzt, befindet man sich in einer neuen, digitalen Welt. Eine Zeit lang war die zwischenmenschliche Interaktion in der virtuellen Realität gar nicht möglich. Seit neuerem gibt es auch Spiele, welche unterstützen, dass VR-Spieler über das Internet miteinander spielen können (Online-Multiplayer). Somit braucht man jedoch zwei HMDs (VR-Brillen), zwei PCs und man kann in der Regel nicht nebeneinander spielen. Asymmetrische VR-Spiele versuchen, reguläre PC-Spiele mit VR-Spielen zu kombinieren. So können z.B. zwei Freunde mit einem PC und einer VR-Brille miteinander spielen. Das funktioniert z.B., indem die zwei Spieler sich in derselben Spielwelt befinden, jedoch unterschiedliche Charaktere steuern. Durch die verschiedenen Steuerungsmöglichkeiten (VR-Controller, Tastatur und Maus, Gamepad) ergibt sich oft automatisch eine bestimme Rollenverteilung. So könnte der PC-Spieler z.B. einen Menschen spielen, welcher durch ein Level manövrieren muss und zeitgleich steuert der VR-Spieler einen Riesen, welcher den Menschen behindern muss, indem er ihm z.B. Steine in den Weg legt.

\section{Zielsetzung}
Das Ziel dieser Arbeit ist, die technische Implementierung und Optimierung von asymmetrischen VR Spielen genauer zu Untersuchen und gefundene technische Probleme genauer zu erläutern.


\section{Warum Asymmetrisches VR}


%\include {schueler_b/abbildungen}
%\include {schueler_b/prozessoren}

%%%----------------------------------------------------------
%%%Anhang
\appendix
\chapter{Projektdokumentation}

\section{Meilensteine}

\subsection{Prototype (23.09)}
Die Hauptmechaniken wurden Implementiert, und erste Systeme Getestet. Der Art-Style von unserem Spiel wurde ausgearbeitet, und dazu wurden erste Style-Guides, Charaktere, Farbpaletten und Target-Experience festgelegt.

\subsection{Iteration Interim 1 (20.10)}
Es wurden die Spell-Types und des Raum-Design definiert und implementiert. Der Gameloop des Spieles funktionierte erstmals komplett. Es wurden außerdem erste Playtests in einem Test-Level durchgeführt, um mögliche Probleme mit dem Design herauszufinden.

\subsection{Iteration Interim 2 (22.12)}
Ein Großteil der Grafiken und 3D-Modelle wurden importiert. Außerdem wurde die Mana-Menge und andere Variablen anhand von Playtests, die durchgeführt wurden, um früher auf Fehler im Design zu stoßen und sie zu beheben, angepasst. Die Technik des Spieles ist auch größtenteils fertiggestellt.

\subsection{Vertical Slice (30.03)}
Ein Vertical Slice des Projekts wurde finalisiert. Der Gameloop ist nun ohne Errors durchspielbar, und die gewünchte Experience, Art/Design und Performance wurden gut umgesetzt. Es gibt außerdem ein vollständiges Presskit inklusive Trailer.

\section{Playtest Protokolle \& Ergebnisse}
Wir haben bereits mehrere Playtests durchgeführt. Die ersten mussten aufgrund unerwarteter Fehler frühzeitig abgebrochen werden. Die Playtests, die Problemlos abgelaufen sind, haben allerdings die gewünschte Experience erzeugt. Nach den Playtests, haben wir die Teilnehmenden anschließend darum gebeten einige Fragen zu beantworten.

\subsection{Welcher Spell ist der Lustigste?}
Der Großteil der PC- als auch der VR-Spielenden gaben an, dass die Bombe der beste Spell sei. Auf Platz zwei war die Mikado-Barrikade, da damit gute Kombinationen erzielt werden konnten.

\subsection{Welcher Spell ist der Unlustigste?}
Die wolke wurde von vielen Spielenden als "`sinnlos"' oder "`Unfair"' beschrieben. Das ist wahrscheinlich darauf zurückzuführen, dass sie erst relativ spät Einfluss auf die Snowboarder hat und nicht sehr genau platziert werden kann, also nicht wirklich ein Zusammenhang von Wolken-kill zu VR-Skill besteht.

\subsection{In welcher Rolle hattest du mehr Spaß?}
Ca. 65\% der Spielenden gaben an, dass sie als Snowboarder mehr Spaß hatten. Manche hatten leider auch Probleme mit Motion-Sickness.

\subsection{Gab es Momente wo du stark mitgefiebert hast?}
Die Spannung wenn nur nur noch ein Snowboarder im Spiel ist, ist sehr intensiv. Vor allem gegen Ende einer Runde.

\subsection{Wie würdest du die Gesamterfahrung bewerten?}
Der Durchschnittliche Score der Gesamterfahrung beträgt ~7.5/10 Punkten. Die Erfahrung der Snowboarder war laut Umfrage etwas besser. "`Ist für ein paar runden ganz witzig."'

\subsection{Gab es Momente in denen dir langweilig war?}
Manche Snowboarder haben sich überfordert gefühlt, wenn sie bereits mit einer erfahrenen Gruppe gespielt haben. Viele VR-Spielende gaben an, dass die Wartezeiten zwischen den Runden verkürzt werden sollte.

%\includepdf[page=-]{formulare/PlaytestProtokolle}

\section{Begleitprotokolle}
%\includepdf[page=-]{formulare/Zeitprotokolle}
	    % Projektdokumentation
\chapter{Kooperationsvereinbarung}

%\includepdf[page=-]{formulare/Kooperationsvereinbarung}	% Kooperationsvereinbarung
\chapter{Inhalt des Datenträgers}

\section{Projektdateien}
\begin{FileList}{/Game}
	\fitem{Projekt} Unity + Wwise Projekt
	\fitem{Build} Windows Build des Spieles
\end{FileList}

\section{Arbeit}
\begin{FileList}{/Diplomarbeit}
	\fitem{_DaBa.pdf} Diplomarbeit (Gesamtdokument) 
	\fitem{LaTex} LaTex Projekt
\end{FileList}

\section{Game Design}
\begin{FileList}{/GameDesign}
	\fitem{Concepts} Konzept Dateien
	\fitem{Playtests} Ergebnisse der Playtests
\end{FileList}
 
\section{Presskit}
\begin{FileList}{/Presskit}
	\fitem{Trailer} Trailer des Spieles
	\fitem{Overview} Ein PDF mit Infos zum Spiel
	\fitem{Screenshots} Screenshots des Spieles
\end{FileList}
	            % Inhalt des Datenträgers

%%%----------------------------------------------------------
\MakeBibliography
\addcontentsline{toc}{chapter}{Abbildungsverzeichnis}
\listoffigures
\addcontentsline{toc}{chapter}{Tabellenverzeichnis}
\listoftables


%%%----------------------------------------------------------

%%%Messbox zur Druckkontrolle
\include{messbox}

\end{document}
