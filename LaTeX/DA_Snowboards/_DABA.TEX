\documentclass[diplom,german]{hgbthesis}
% Zulässige Class Options: 
%   Typ der Arbeit: diplom, master (default), bachelor, praktikum 
%   Hauptsprache: german (default), english
%%------------------------------------------------------------


% Zusatzpakete (bei Bedarf einkommentieren):
% \usepackage{enumitem}    % erlaubt Listen mit einem <key>=<value> - Format
% \usepackage{tikz}        % ermöglicht das Zeichnen von Grafik-Elementen (zB Linien, Kreise etc.) - Bsp.: https://www.overleaf.com/learn/latex/TikZ_package
% \usetikzlibrary{arrows,shapes,positioning} % Zusatzbibliothek für tikz (zB Pfeil etc.)
\usepackage{pdfpages}      % ermöglicht das includieren (mit \include) von pdf-Dateien

\usepackage{listings}

\graphicspath{{images/}}   % Verzeichnis mit den Bildern (zB für \figure)
%\logofile{spengerlogo_cmyk_small}   % Name des .pdf mit dem Logo

\bibliography{buketits/b_literatur}  % Datei(en) mit der Bibliothek (.bib) für das Quellenverzeichnis 
\bibliography{kaspar/k_literatur}  % Datei(en) mit der Bibliothek (.bib) für das Quellenverzeichnis 
\bibliography{wehinger/w_literatur}  % Datei(en) mit der Bibliothek (.bib) für das Quellenverzeichnis 
\bibliography{memic/m_literatur}  % Datei(en) mit der Bibliothek (.bib) für das Quellenverzeichnis 
% Beispiel für mehrere .bib files: \bibliography{schueler1/literatur,schueler2/literatur}

%%%----------------------------------------------------------
\begin{document}
%%%----------------------------------------------------------

% offizielle Formulare inkludieren
%\includepdf[page=-]{formulare/HTL_RDP_Titelseite_DA_A4}
%\includepdf[page=-]{formulare/HTL_RDP_Dokumentation_DA_DE_A4}
%\includepdf[page=-]{formulare/HTL_RDP_Dokumentation_DA_EN_A4}

%\begin{minipage}{\textwidth}
%	\includepdf[scale=0.85,pages=1]{formulare/HTL_RDP_Titelseite_DA_A4}
%\end{minipage}

% Daten zur Arbeit: _________--------------------------------
\title{"'Tricks 'n' Treats"': Entwicklung eines asymmetrischen VR- und Couch-Partyspiels}
\author{Ian Memic, Simon Buketits, Emma Wehinger und Felix Kaspar}
\studiengang{Mediendesign - Gamedesign}
\studienort{Wien}
\abgabedatum{2023}{03}{31}	% {YYYY}{MM}{DD}

%%%----------------------------------------------------------
\frontmatter
\maketitle
\tableofcontents
%%%----------------------------------------------------------

%\chapter{Vorwort} 	% engl. Preface



Dies ist \textbf{Version \hgbthesisDate} der \latex-Dokumentenvorlage für 
verschiedene Abschlussarbeiten an der FH Hagenberg, die mittlerweile auch 
an anderen Hochschulen im In- und Ausland gerne verwendet wird.

Das Dokument entstand ursprünglich auf Anfragen von Studierenden,
nachdem im Studienjahr 2000/01 erstmals ein offizieller
\latex-Grundkurs im Studiengang Medientechnik und -design an der
FH Hagenberg angeboten wurde. Eigentlich war die Idee, die bereits
bestehende \emph{Word}-Vorlage für Diplomarbeiten "`einfach"' in
\latex\ zu übersetzen und dazu eventuell einige spezielle
Ergänzungen einzubauen. Das erwies sich rasch als wenig
zielführend, da \latex, \va was den Umgang mit Literatur und
Graphiken anbelangt, doch eine wesentlich andere Arbeitsweise
verlangt. Das Ergebnis ist -- von Grund auf neu geschrieben und
wesentlich umfangreicher als das vorherige Dokument --
letztendlich eine Anleitung für das Schreiben mit \latex, ergänzt
mit einigen speziellen (mittlerweile entfernten) Hinweisen für \emph{Word}-Benutzer.
Technische Details zur aktuellen Version finden sich in Anhang \ref{ch:TechnischeInfos}.

Während dieses Dokument anfangs ausschließlich für die Erstellung
von Diplomarbeiten gedacht war, sind nunmehr auch  
\emph{Masterarbeiten}, \emph{Bachelor\-arbeiten} und \emph{Praktikumsberichte} 
abgedeckt, wobei die Unterschiede bewusst gering gehalten wurden.

Bei der Zusammenstellung dieser Vorlage wurde versucht, mit der
Basisfunktionalität von \latex das Auslangen zu finden und -- soweit möglich --
auf zusätzliche Pakete zu verzichten. Das ist nur zum Teil gelungen;
tat\-säch\-lich ist eine Reihe von ergänzenden "`Paketen"' notwendig, wobei jedoch
nur auf gängige Erweiterungen zurückgegriffen wurde.
Selbstverständlich gibt es darüber hinaus eine Vielzahl weiterer Pakete,
die für weitere Verbesserungen und Finessen nützlich sein können. Damit kann
sich aber jeder selbst beschäftigen, sobald das notwendige Selbstvertrauen und
genügend Zeit zum Experimentieren vorhanden sind.
Eine Vielzahl von Details und Tricks sind zwar in diesem Dokument nicht explizit
angeführt, können aber im zugehörigen Quelltext jederzeit ausgeforscht
werden.

Zahlreiche KollegInnen haben durch sorgfältiges Korrekturlesen und
konstruktive Verbesserungsvorschläge wertvolle Unterstützung
geliefert. Speziell bedanken möchte ich mich bei Heinz Dobler für
die konsequente Verbesserung meines "`Computer Slangs"', bei
Elisabeth Mitterbauer für das bewährte orthographische Auge und
bei Wolfgang Hochleitner für die Tests unter Mac~OS.

Die Verwendung dieser Vorlage ist jedermann freigestellt und an
keinerlei Erwähnung gebunden. Allerdings -- wer sie als Grundlage
seiner eigenen Arbeit verwenden möchte, sollte nicht einfach
("`ung'schaut"') darauf los werken, sondern zumindest die
wichtigsten Teile des Dokuments \emph{lesen} und nach Möglichkeit
auch beherzigen. Die Erfahrung zeigt, dass dies die Qualität der
Ergebnisse deutlich zu steigern vermag.

Der Quelltext zu diesem Dokument sowie das zugehörige
\latex-Paket sind in der jeweils aktuellen Version online
verfügbar unter
%
\begin{quote}
\url{www.fh-hagenberg.at/staff/burger/diplomarbeit/}
\end{quote}
%
oder auch unter
%
\begin{quote}
\url{http://elearning.fh-hagenberg.at/} \newline
im Kurs "`Anleitung/Vorlage für Master-/Bachelor-/Diplomarbeiten"'.
\end{quote}
%
Trotz großer Mühe enthält dieses Dokument zweifellos Fehler und Unzulänglichkeiten
-- Kommentare, Verbesserungsvorschläge und passende Ergänzungen
sind daher stets willkommen, am einfachsten per E-Mail direkt an mich:
\begin{center}%
\begin{tabular}{l}
\nolinkurl{wilhelm.burger@fh-hagenberg.at} \\
Dr.\ Wilhelm Burger \\
FH Hagenberg -- Digitale Medien\\
Austria
\end{tabular}
\end{center}

\noindent
Übrigens, hier im Vorwort (das bei Diplomarbeiten üblich, bei Bachelorarbeiten 
aber entbehrlich ist) kann man kurz auf die Entstehung  des Dokuments eingehen.
Hier ist auch der Platz für allfällige Danksagungen (\zB an den Betreuer, 
den Begutachter, die Familie, den Hund, ...), Widmungen und philosophische 
Anmerkungen. Das sollte man allerdings auch nicht übertreiben und sich auf 
einen Umfang von maximal zwei Seiten beschränken.




		% ggfs. weglassen

\chapter{Kurzfassung}

Diese Diplomarbeit befasst sich mit der Entwicklung des asymmetrischen lokalen Multiplayer-Spieles Tricks ‘n’ Treats. 
Der erste Teil der Arbeit befasst sich mit der Entwicklung von asymmetrischen Virtual Reality Spielen, wobei ein besonderer Fokus auf Performance-Optimierung, den aufgetretenen Problemen und technischen Limitationen von asymmetrischen Virtual Reality Spielen. 
Im zweiten Teil wird genauer bearbeitet, wie sich die Dynamiken zwischen PC und VR bilden, wie man diese beeinflusst und im Falle des Spieles eine gute Interaktion zwischen den beiden konstruieren kann. 
Der dritte Teil der Arbeit befasst sich mit der Welt und Atmosphäre des Spiels. Besonders widmet er sich wie Farb- und Formgebung die Gefühle und Aktionen der SpielerInnen beeinflussen. 
Der vierte Teil beschäftigt sich mit dem 3D-Modeling und geht spezifisch auf die Performance Optimierung der 3D-Assets ein. Dabei liegt ein großer Fokus auf der Recherche und Verwendung verschiedener Optimierungs-Methoden und deren Vor- und Nachteile.
   % vom Team gemeinsam zu verfassen
\chapter{Abstract}

\begin{english} 
This thesis is about the development of Tricks 'n' Treats, an asymmetric local-multiplayer game.
The first part of the thesis deals with the technical development of asymmetric virtual reality games, with a special focus on performance optimization, the problems that were encountered and technical limitations of asymmetric virtual reality games.
The second part discusses how the dynamics between PC and VR are formed, how to influence them and, in the case of the game, how to construct a good interaction between the two worlds. 
The third part of the work explores the world and atmosphere of the game. In particular, it is devoted to how color and shape design influences the feelings and actions of the players. 
The last part deals with 3D modeling and specifically addresses the performance optimization of 3D-game-assets. Thereby a big focus is on the research and use of different optimization methods and their advantages and disadvantages.
\end{english}
		% vom Team gemeinsam in englisch zu verfassen	

%%%----------------------------------------------------------
\mainmatter         % Hauptteil (ab hier arab. Seitenzahlen)
%%%----------------------------------------------------------

% Trennseite erzeugt ein Titelblatt für eine neue Arbeit (zB wenn die Arbeit A endet und die Arbeit B beginnt)
% Parameter: {}Titel der Arbeit}{Autor}{Betreuer}{Abteilung}
% Beispiele: 
%	\trennseite{Gestaltpsychologie und stilisiertes Game Audio}{Noah Diem}{Dipl.Ing. Michael Schreiber}{Mediendesign - Gamedesign}
%	\trennseite{Collection and analysis of gameplay metrics}{Ian Hornik}{Mag. Andreas Metz}{Media design - Game design}

\trennseite{Technische Implementierung und Optimierung von asymmetrischen VR-Spielen}{Simon Buketits}{Kristian Ljubek, BSc MSc}{Mediendesign - Gamedesign}
\chapter{Einleitung}
\label{cha:sa_Einleitung}

\section{Zielsetzung}
Dieses Dokument ist als vorwiegend technische Starthilfe für das
Erstellen einer Masterarbeit (oder Bachelorarbeit) mit \latex
gedacht und ist die Weiterentwicklung einer früheren
Vorlage\footnote{Nicht mehr verfügbar.} für das Arbeiten mit
Microsoft \emph{Word}. Während ursprünglich daran gedacht war, die
bestehende Vorlage einfach in \latex zu übernehmen, wurde rasch
klar, dass allein aufgrund der großen Unterschiede zum Arbeiten
mit \emph{Word} ein gänzlich anderer Ansatz notwendig wurde. Dazu
kamen zahlreiche Erfahrungen mit Diplomarbeiten in den
nachfolgenden Jahren, die zu einigen zusätzlichen Hinweisen Anlass gaben.


\section{Warum {\latex}?}

Diplomarbeiten, Dissertationen und Bücher im
technisch-natur\-wissen\-schaft\-lichen Bereich werden
traditionell mithilfe des Textverarbeitungssystems \latex
\cite{Lamport94,Lamport95} gesetzt. Das hat gute Gründe, denn
\latex ist bzgl.\ der Qualität des Druckbilds, des Umgangs mit
mathematischen Elementen, Literaturverzeichnissen etc.\
unübertroffen und ist noch dazu frei verfügbar. Wer mit \latex
bereits vertraut ist, sollte es auch für die Diplomarbeit
unbedingt in Betracht ziehen, aber auch für den Anfänger sollte
sich die zusätzliche Mühe am Ende durchaus lohnen.

Für den professionellen elektronischen Buchsatz wurde früher
häufig \emph{Adobe Framemaker} verwendet, allerdings ist diese
Software teuer und komplex. Eine modernere Alternative dazu ist
\emph{Adobe InDesign}, wobei allerdings die Erstellung
mathematischer Elemente und die Verwaltung von Literaturverweisen
zur Zeit nur rudimentär unterstützt werden.%
\footnote{Angeblich werden aber für den (sehr sauberen) Schriftsatz 
in \emph{InDesign} ähnliche Algorithmen wie in \latex\ verwendet.}

Microsoft \emph{Word} gilt im Unterschied zu \latex, 
\emph{Framemaker} und \emph{InDesign} übrigens nicht als professionelle
Textverarbeitungssoftware, obwohl es immer häufiger auch von
großen Verlagen verwendet wird.%
\footnote{Siehe auch \url{http://latex.tugraz.at/mythen.php}.}
Das Schriftbild in \emph{Word}
lässt -- zumindest für das geschulte Auge -- einiges zu wünschen
übrig und das Erstellen von Büchern und ähnlich großen Dokumenten
wird nur unzureichend unterstützt. Allerdings ist \emph{Word} sehr
verbreitet, flexibel und vielen Benutzern zumindest oberflächlich
vertraut, sodass das Erlernen eines speziellen Werkzeugs wie
\latex\ ausschließlich für das Erstellen einer Diplomarbeit
manchen verständlicherweise zu mühevoll ist. Man sollte es daher
niemandem übel nehmen, wenn er/sie sich auch bei der Diplomarbeit
auf \emph{Word} verlässt. Im Endeffekt lässt sich mit etwas
Sorgfalt (und ein paar Tricks) auch damit ein durchaus akzeptables
Ergebnis erzielen. 
Für alle, die so denken, finden sich in
%Kap.~\ref{chap:Word} einige spezielle Hinweise zum Arbeiten mit
%\emph{Word}. 
Ansonsten sollten auch für \emph{Word}-Benutzer 
einige Teile dieses Dokuments von Interesse sein, insbesondere die
Abschnitte über Abbildungen und Tabellen und mathematische Elemente.

Übrigens, genau hier am Ende des Einleitungskapitels (und nicht
etwa in der Kurzfassung) ist der richtige Platz, um die
inhaltliche Gliederung der nachfolgenden Arbeit zu beschreiben.
Hier soll dargestellt werden, welche Teile (Kapitel) der Arbeit
welche Funktion haben und wie sie inhaltlich zusammenhängen. Auch
die Inhalte des \emph{Anhangs} -- sofern vorgesehen -- sollten hier
kurz beschrieben werden.


\chapter{Überblick über die Materie}
\section{Virtuelle Realität}
Die Virtuelle Realität (Virtual Reality, VR) ist eine computergenerierte virtuelle Umgebung welche in Echtzeit berechnet wird. In dieser ist es möglich sich um zuschauen und je nach Anwendung kann man sich auch Bewegen und beliebig mit Objekten interagieren. Es gibt 2 verschiedene Möglichkeiten Virtual Reality zu erleben. Einerseits gibt es speziell angefertigte Räume, in welchen Großbildleinwände angebracht sind, andererseits gibt es auch Head-Mounted-Displays (VR Brillen,  die man aufsetzt), welche hier im Fokus stehen werden.


\subsection{Einführung in die Virtuelle Realität}
In der Virtuellen Realität gibt es eine breit gefächerte Reichweite an Spielen, in welchen man verschiedenste Situationen erleben kann. Anfangs waren Spiele für VR nur simple Testräume, in welchen man mit Objekten interagieren konnte. Mittlerweile ist es jedoch möglich riesige Spiele mit einer eigenen Geschichte zu erleben. Die Interaktionen, welche damals ein ganzes Spiel ausmachten, sind seit einiger Zeit nur noch die Basis, worauf neue Spiele aufbauen.

\subsection{Asymmetrische VR Spiele}
Die meisten VR Spiele kapseln einen von der Außenwelt ab. Sobald man die VR Brille aufsetzt befindet man sich in einer neuen, digitalen Welt. Eine Zeit lang war die zwischenmenschliche Interaktion in der Virtuellen Realität nicht möglich. Manche Spiele, welche etwas neuer sind, unterstützen dass VR Spieler über das Internet miteinander spielen können(Online Multiplayer). Somit braucht man jedoch 2 HMDs(VR Brillen), 2 PCs und man kann in der Regel nicht nebeneinander spielen. Asymmetrische VR Spiele versuchen, reguläre PC Spiele mit VR Spielen zu kombinieren. So können z.B. zwei Freunde mit einem PC und einer VR Brille miteinander spielen. Das funktioniert z.B., indem die zwei Spieler sich in der selben Spielwelt befinden, jedoch unterschiedliche Charaktere steuern. Durch die verschiedenen Steuerungsmöglichkeiten (VR Controller, Tastatur und Maus, Gamepad) ergibt sich oft automatisch eine bestimme Rollenverteilung. So könnte der PC Spieler z.B. einen Menschen spielen, welcher durch ein Level manövrieren muss und zeitgleich steuert der VR Spieler einen Riesen, welcher den Menschen behindern muss, indem er ihm z.B. Steine in den Weg legt.  

\subsection{Verfügbare Brillen}
Es gibt mittlerweile viele verschiedene Anbieter für VR Brillen, welche verschieden gute HMDs in verschiedenen Preisreichweiten anbieten.
Das Steam Hardware Survey zeigt, dass momentane Marktführer Facebooks Meta (ehemals Oculus) mit der Oculus Quest und der Oculus Quest 2 im Low-Budget Bereich und Valve mit der Valve Index im High-Budget Bereich sind.

\vspace{0.75cm}
\includegraphics{images/Steam_HardwareSurvey}

\section{Existierende Spiele}
Es gibt viele verschiedene VR Spiele und eine Breite Reichweite an Genres. Besonders relevant sind jedoch die asymmetrischen Spiele. Einige wichtige Beispiele sind hier aufgelistet.

\subsection{Keep Talking and Nobody Explodes}
YOU’RE ALONE
IN A ROOM WITH A BOMB.
Your friends have the info you need to defuse it.

But there’s a catch. They can’t see the bomb. So everyone will need to talk it out–fast!

\subsection{Davigo}
DAVIGO is a VR vs. PC "cross-reality" battle game. The VR player embodies a giant and faces off against one or more PC warriors in fast-paced, explosive combat!

\subsection{Carly and the Reaperman}
Step into the recently-dead shoes of Carly as you run and jump your way through incredible platforming challenges — or take up the mantle of the Reaperman and change the environment around you in real-time to give Carly new places to explore, avoid dangerous obstacles, and help her to achieve the impossible.

\section{Erwartete Probleme}
Dieses Dokument ist als vorwiegend technische Starthilfe für das
Erstellen einer Masterarbeit (oder Bachelorarbeit) mit \latex
gedacht und ist die Weiterentwicklung einer früheren
Vorlage\footnote{Nicht mehr verfügbar.} für das Arbeiten mit
Microsoft \emph{Word}. Während ursprünglich daran gedacht war, die
bestehende Vorlage einfach in \latex zu übernehmen, wurde rasch
klar, dass allein aufgrund der großen Unterschiede zum Arbeiten
mit \emph{Word} ein gänzlich anderer Ansatz notwendig wurde. Dazu
kamen zahlreiche Erfahrungen mit Diplomarbeiten in den
nachfolgenden Jahren, die zu einigen zusätzlichen Hinweisen Anlass gaben.

\subsection{Wenige Bilder pro Sekunde}
Zitat \cite{bobsch 123 bobsch}

\subsection{Bewegungskrankheit}
Zitat \cite{bobsch 123 bobsch}

\subsection{Platz und Bewegung}
Zitat \cite{bobsch 123 bobsch}

\subsection{Interaktionen}
Zitat \cite{bobsch 123 bobsch}
\chapter{Möglichkeiten zur Implementierung}

\section{Nutzung von Farben in Spielen}
Die Nutzung von Farben in Videospielen war mit ihrer Entstehung sehr mit dem damaligen Stand der Technik verbunden. 1972 wurde die Farbüberlagerung erfunden. Diese hat es ermöglicht, dass Videospiele in Farbe dargestellt werden können. Davor konnten Spiele nur in Schwarz-Weiß oder Monochrom dargestellt werden. Ab Ende der 1980er sind fast alle Videospiele in Farbe. 1985 veröffentlichte „Sega“ das Sega Master System, das Spiele mit 32 verschiedenen Farben darstellen konnte. Schon 9 Jahre später veröffentlicht „Sony“ die Playstation, die es ermöglicht, Spiele mit rund 16.7 Millionen Farben darzustellen. Heutzutage können Spiele mit über 16.7 Milliarden verschiedenen Farben dargestellt werden.  
\cite{_the_video_game_explosion}
\cite{_farbkontraste_im_gaming}

\subsection{Beispiel – Outlast}
„Outlast“ ist ein Horror-Spiel, das 2013 für die Playstation 4, die Xbox One und den PC erschienen ist. Der Entwickler und Publisher ist „Red Barrels“. Der Journalist Miles Upshur erhält durch eine anonyme Quelle die Information, dass im Mount Massive Asylum unmenschliche Experimente an Patienten durchgeführt werden. Er geht dieser Spur nach und begibt sich auf den Weg zur Nervenklinik. Dort angekommen läuft alles ganz anders als erwartet. Es gibt keinen Weg nach draußen mehr und alle Insassen sind ausgebrochen. Ziel ist es, einen Weg aus der Nervenklinik zu finden und währenddessen alles mit einem Camcorder zu filmen. 
\cite{_outlast}
\cite{_farbkontraste_im_gaming}

\begin{figure}[H]
	\centering
	\includegraphics[width=10cm]{Outlast}
	\caption{"'Outlast"' In-Game Screenshot\cite{_drawing_basics_and_video_game_art}}
\end{figure}

In „Outlast“ wird die Atmosphäre und Stimmung des Spiels durch den Qualitätskontrast der Farben erzeugt. Der Qualitätskontrast wird durch die Farbqualität der Farbe erzeugt. Durch das Mischen mit Schwarz, Weiß oder Grau wird eine Farbe unrein. Die Farbe verliert als Folge an Leuchtkraft und Qualität. Farben, die mit Grau gemischt werden, wirken trüb. Wird eine Farbe mit Schwarz gemischt, so wirkt sie bedrohlicher. Wenn man eine Farbe mit Weiß mischt, so wird sie kälter. Wendet man einen Qualitätskontrast neben reinen Farben an, fallen sie stärker auf. 
\cite{_farbkontraste}
\cite{_outlast}
\cite{_farbkontraste_im_gaming}

In „Outlast“ wird dem Spielenden durch den Qualitätskontrast eine stark bedrohliche Umgebung vermittelt. Alles ist sehr düster und dunkel, erst wenn der Spielende den Camcorder benutzt, wird die Umgebung in ein blasses Grün getaucht. Die Farben wirken größtenteils sehr leblos, aber gleichzeitig bedrohlich. Die einzig gesättigte, reine Farbe ist Rot in Form von Blut überall in der Anstalt. 
\cite{_outlast}
\cite{_farbkontraste}
\cite{_farbkontraste_im_gaming}

\subsection{Beispiel – Journey}
„Journey“ ist ein Adventure-Spiel, das ohne jegliche Worte auskommt. Es wurde erstmals 2012 für die Playstation 3 veröffentlicht. Der Entwickler des Spiels ist „thatgamecompany“ und der Publisher „Sony Interactive Entertainment“ und „Annapurna Interactive“. Das Spiel startet in einer Wüste. In der Ferne liegt ein großer Berg, an dessen Gipfel ein Lichtstrahl in den Himmel ragt. Der Spielende spielt eine Figur in einer roten Robe. Auf dem Weg zum Berg kann der Spielende mit bestehender Internetverbindung auf einen anderen Spielenden treffen. Die Spielenden können nicht miteinander reden, aber sich dennoch gegenseitig helfen. Das Zeil von „Journey“ ist es, den in der Ferne liegenden Berg zu erreichen.
\cite{_drawing_basics_and_video_game_art}
\cite{_farbkontraste_im_gaming}

„Journey“ nutzt die Vorteile von Farbkontrasten, um bei den Spielenden Emotionen zu erzeugen. Das Spiel bedient sich am Warm-Kalt-Kontrast. Beim Warm-Kalt-Kontrast werden warme und kalte Farben einander gegenübergestellt. Warme und kalte Farben stehen sich im Farbkreis gegenüber und erzeugen deshalb Dissonanz. Die kalten Farben des Farbkreises sind Gelbgrün, Grün, Blaugrün, Blau, Blauviolett und Violett. Zu den warmen Farben des Farbkreises zählen Gelb, gelborange, Orange, Rotorange, Rot und Rotviolett.
\cite{_farbkontraste_im_gaming}
\cite{_drawing_basics_and_video_game_art}
\cite{_farbkontraste}

\begin{figure}[H]
	\centering
	\includegraphics[width=10cm]{Journey2}
	\caption{"'Journey"' In-Game Screenshot\cite{_drawing_basics_and_video_game_art}}
\end{figure}

In „Journey“ wechseln sich die Warm-Kalt-Kontraste immer wieder ab. Am Anfand des Spiels ist der Spielende in einer Wüste, nach der Wüste führt ihn seine Reise in eine kühle Höhle. Nach den warmen, sandigen Ruinen begibt sich der Spielende zum eiskalten Berg in der Ferne. Die Figur selbst trägt eine rote Robe die in der kühlen Höhle und am kalten Berg den Warm-Kalt-Kontrast nochmal verdeutlicht. 
\cite{_drawing_basics_and_video_game_art}
\cite{_farbkontraste_im_gaming}
\cite{_farbkontraste}

\section{Nutzung von Formen in Spielen}
Das Konzept und die Nutzung von Formen und Linien in Videospielen stammt aus der Natur. Objekte und Dinge die aus abgerundeten Formen bestehen empfinden Menschen als sicher. Kantige Objekte und Formen gelten als gefährlich. Der Kreis steht für Freundlichkeit und Positivität. Das Quadrat soll Vertrauen, Stabilität und Sicherheit kommunizieren. Das Dreieck steht für Gefahr und Aggressivität. Man kann diese Formen in einer Skala der Emotionen einordnen. Links ist die positivste Form, der Kreis und ganz rechts das Dreieck, die negativste Form. 
\cite{_drawing_basics_and_video_game_art}

In Videospielen bestehen die „guten“ Charaktere beziehungsweise die Helden aus vielen Kreisen und abgerundeten Linien. Oft enthalten ihre Designs auch Quadrate, um Sicherheit und Stabilität zu signalisieren. Die Antagonisten in Spielen haben meist sehr dominante Silhouetten, die aus Dreiecken oder sehr kantigen Linien bestehen. Weiters, ist es äußerst wichtig, Silhouetten aus großen Formen zu gestalten, um die Figur des Charakters herausstechen zu lassen.
\cite{_drawing_basics_and_video_game_art}

\begin{figure}[H]
	\centering
	\includegraphics[width=10cm]{CharacterShapes}
	\caption{Video-Game Character-Shapes\cite{_drawing_basics_and_video_game_art}}
\end{figure}

In Abbildung 12.3 kann man erkennen, dass in der linken Spalte die Protagonisten aus sehr abgerundeten Linien und Kreisen bestehen. Sie wirken auf den Betrachtenden sehr positiv. Auf der rechten Seite sind die Antagonisten, die aus Dreiecken und vielen kantigen Linien bestehen. Sie wirken auf den Betrachtenden sehr aggressiv und bedrohlich. Generell sind alle abgebildeten Charaktere aus markanten Formen gestaltet, damit sie möglichst gut herausstechen. 
\cite{_drawing_basics_and_video_game_art}

\subsection{Beispiel – Journey}
In „Journey“ wird auch eine positive, harmonische Stimmung durch die Nutzung von Dreiecken und kantigen Linien erzeugt. Der Charakter besteht aus einem Dreieck und sein Umfeld ebenso. Die Umgebung ist geprägt durch kantige Objekte und Formen und dreieckige Figuren. Durch die Nutzung von Dreiecken für Charakter und Umfeld wird Harmonie erzeugt. 
\cite{_drawing_basics_and_video_game_art}
\cite{_farbkontraste_im_gaming}

\begin{figure}[H]
	\centering
	\includegraphics[width=10cm]{images/wehinger/Journey1}
	\caption{"'Journey"' In-Game Screenshot \cite{_journey}}
\end{figure}

\subsection{Beispiel – Super Mario Galaxy}
„Super Mario Galaxy“ ist ein 3D-Plattformer, der 2007 für die Wii veröffentlicht wurde. Der Entwickler und Publisher des Spiels ist „Nintendo“. Der Protagonist Mario reist von Planet zu Planet, kämpft gegen Gegner, löst Rätsel und sammelt Items. Ziel des Spiels ist es, das Pilzkönigreich und Prinzessin Peach zu retten. 
\cite{_super_mario_galaxy}
\cite{_drawing_basics_and_video_game_art}

Mit Formen und Linien kann man bei den Spielenden Gefühle und Emotionen erzeugen. Je nachdem, welche Emotion beim Spielenden erzeugt werden soll, werden andere Charakter -und Environment-Shapes verwendet. Um Harmonie in einer Umgebung zu erzeugen, besteht der Charakter und sein Umfeld aus Kreisen und runden Linien. Harmonie kann aber auch durch Dreiecke und kantige Linien entstehen. Um Dissonanz zu erzeugen, besteht der Charakter aus Kreisen und gebogenen Linien oder Dreiecken und kantigen Linien und das Umfeld steht im Gegensatz dazu.
\cite{_drawing_basics_and_video_game_art}

\begin{figure}[H]
	\centering
	\includegraphics[width=10cm]{SuperMarioGalaxy}
	\caption{"'Super Mario Galaxy"' In-Game Screenshot\cite{_drawing_basics_and_video_game_art}}
\end{figure}

In „Super Mario Galaxy“ besteht auf den meisten Planeten Harmonie zwischen dem Charakter und seiner Umgebung. Mario besteht größtenteils aus Kreisen und runden, gebogenen Linien. Seine Umgebung besteht im Großen und Ganzen auch aus Kreisen und abgerundeten Linien. Diese Beziehung zwischen Charakter und Umgebung erzeugt Harmonie und löst bei den Spielenden positive Gefühle aus. 
\cite{_drawing_basics_and_video_game_art}




\chapter{Umsetzung der PC- und VR-Welt}
\section{Level Aufbau}
Jedes Level in \emph{Tricks ´n´ Treats} ist ähnlich aufgebaut. 
In der Mitte befindet sich der Berg, auf dem die Snowboardenden fahren, und rund herum befindet sich der Raum des VR-Spielenden. Die PC-Spielenden starten oben auf dem Berggipfel und fahren den Hang bis zum Ende der Piste hinab. Der VR-Spielende kann mit einem Joystick auf seinem Controller den Berg rotieren, um sich besser auf der Strecke zu orientieren und die Spielenden in Sicht zu behalten. Währenddessen kann er hinter sich bestimmte Zauberkugeln aufheben und auf den PC-Spielenden anwenden.


\begin{figure}[H]
	\centering
	\includegraphics[width=13cm]{images/buketits_room}
	\caption{Editor View eines unfertigen Levels, Die rote Kapsel stellt den VR-Charakter dar}
\end{figure}

\section{VR-Charakter} \label{simon_vrspieler}
Der VR-Charakter besteht aus mehreren Systemen, die zusammenarbeiten. In \emph{Tricks ´n´ Treats} sind die Kamera und Hände mit Hilfe der SteamVR Integration für Unity gemacht. Der VR-Charakter verwendet die physikalischen Hände und die "`Player"' Komponente, die beide von SteamVR bereitgestellt werden. Ein großes Problem unseres Spiels ist, dass der VR-Charakter, der physikalische Hände besitzt nicht richtig skaliert werden kann. Sobald dieser skaliert wird, funktionieren die Hand-Collider nicht mehr planmäßig. Die Lösung dieses Problems ist in \ref{simon_problems} erklärt.

\subsection{Zauber und Interaktionssystem}
Für die Basis des Interaktionssystems in \emph{Tricks ´n´ Treats}, dient das System, das SteamVR für Unity bereitstellt. Auf jedem Objekt, mit dem der VR-Spielende interagieren kann, befindet sich das "`Interactable"' Script, dass von SteamVR bereitgestellt ist. Unser eigenes System für die Zauberkugeln hakt sich in dieses Script ein und kann somit bestimmte Eigenschaften leichter implementieren. Somit wurde eine Hervorhebung für Objekte geschaffen mit denen interagiert werden kann und gleichzeitig wird auf die Eingabe des VR-Controllers reagiert. Wenn eine Zauberkugel mit einem Objekt kollidiert (während man sie in der Hand hält), wird dann mit dem, von Steam vorgegebenen, "`Velocity Estimator"' die Aufprallgeschwindigkeit ausgerechnet und je nach Geschwindigkeit ein Zauberspruch gewirkt.

\subsection{VR-Umgebung}
Die Form der VR-Umgebung basiert auf einem Kreis, bei dem sich der Berg in der Mitte befindet und der Spielbereich des VR-Spielenden auf einem Ring um den Berg.

\section{PC-Charakter}
\subsection{Charakter Größe}
Die PC-Charaktere sind sehr klein. Das kommt daher, dass der VR-Charakter nicht skaliert werden kann und daher die PC-Charaktere runter skaliert werden müssen. Das Iterieren der Werte für den Character Controller ist durch die geringe Größe etwas komplexer und ungewohnt. Dadurch, dass die PC-Charaktere so viel kleiner sind, muss auch die Schwerkraft anders eingestellt sein. Diese haben eine viel geringere Schwerkraft als die physikalischen Objekte, die sich im VR-Bereich befinden.

\subsection{Fähigkeiten}
Der PC-Character Controller kann lenken, springen und Tricks machen. Außerdem muss dieser auf bestimmte Zauber des VR-Charakters reagieren. Somit kann der PC-Charakter beispielsweise künstlich in eine Richtung geschleudert werden, wobei er auch eine Rotation bekommt. Die PC-Spielenden können dann, in der Luft, den jeweiligen Charakter austarieren und bei richtiger Ausführung wieder sicher landen. Sollte der Charakter auf dem Kopf oder in einem falschen Winkel landen, stürzt dieser und wird zu einem Ragdoll. (Physikalischer Körper, bei der jedes Körperteil eigene Kollisionen hat und somit einen realistischen Fall simuliert).
Der Charakter kann zusätzlich, um besser ausweichen zu können, vorwärts und rückwärts Saltos, mit denen er schnell seine Position ändern kann.


\section{Lösung der erwarteten Probleme}\label{simon_problems}

\subsection{Wenige Bilder pro Sekunde}
Um möglichst viele Fps (Bilder pro Sekunde/Frames per second) zu erreichen und das Spiel auch auf älteren Systemen spielbar zu machen, muss es gut optimiert sein. Die Möglichkeiten zur Rechenoptimierung aus \ref{simon_performance}, die verwendet wurden und die, die nicht implementierbar waren sind anbei zu sehen.

\subsection{Level of Detail}
Die Unity-Engine unterstützt Level of Detail per Camera. Das heißt, dass jede Kamera selbst berechnet, welche Meshqualität gerendert werden soll. In \emph{Tricks ´n´ Treats} wird diese Technik nicht verwendet, weil durch andere Optimierungen genug Rechenleistung gespart werden konnte. Sollte sich der Spielumfang in Zukunft ändern, wird es Essenziell sein Level of Detail zu implementieren.

\subsection{Culling}
Dadurch dass die PC und VR-Kamera gleichzeitig Rendern kann Culling nur begrenzt angewendet werden. Occlusion Culling könnte durch eigenen Code, der anhand des View Frustrums die Objekte prüft und dementsprechend cullt implementiert werden. Eine Frage, die vorab gestellt werden muss, ist jedoch: Entsteht dadurch eine nennenswerte Verbesserung? Man kann nur testen, ob der Code nicht mehr Rechenleistung braucht als der Rendering Prozess der zusätzlichen Objekte. Da durch andere Maßnahmen genug optimiert wurde, konnte die Zeit, um in diesem Bereich zu forschen anders verwendet werden.

\subsection{Collider}
 In \emph{Tricks ´n´ Treats} bestehen alle Objekte mit denen kollidiert werden kann aus simplen Collider Formen und komplexere Strukturen werden aus mehreren primitiven gebaut. 
 Für die Effizienz der Collider gilt Sphere>Capsule>Box>Mesh. Das heißt, dass ein Sphere Collider am effizientesten ist und ein Mesh Collider am ineffizientesten.
\begin{figure}[H]
	\centering
	\includegraphics[width=9cm]{images/buketits_correctCollider}
	\caption{Korrekte Collider}
\end{figure}

\subsection{Object Pooling}
In \emph{Tricks ´n´ Treats} gibt es keine Objekte, die gepoolt werden könnten, um Rechenleistung zu sparen. Somit gibt es im gesamten Projekt kein Pooling System.

\subsection{Drawcall Batching}
In \emph{Tricks ´n´ Treats} wird sowohl Static als auch Dynamic Batching angewendet. Um Dynamic zu batchen werden Meshes mit weniger Vertices verwendet, wobei alle dasselbe Material nutzen. Für Textur des Materials wird ein Textur Atlas verwendet, der alle Farben aller Meshes beinhaltet. Dies gilt für alle Meshes der Umgebung. Da die Spielenden wegen einiger Rahmenbedingungen des Dynamic Batching nicht gebatcht werden können, verwenden diese einen anderen Atlas und Material. Static ist alles, was sich aus der Sicht der PC-Spielenden nicht bewegt. Das heißt beispielsweise der Berg und die Objekte auf diesem. Hindernisse, die zerbrechen oder sich bewegen können jedoch nicht Static sein.

\subsection{Bewegungskrankheit und Rotationsproblem}
Um die Bewegungskrankheit (Motion Sickness) zu vermeiden wurden die in \ref{simon_motionsickness} angebrachten Ursachen dafür beachtet und zugehörige Systeme passend dazu entworfen und implementiert.
Wie in \ref{simon_vrspieler} angeführt wurde, kann man den VR-Spielenden nicht skalieren und sondern ihn nur drehen. Die Strecke wiederum kann skaliert aber nicht gedreht werden. Das heißt, dass also der VR-Spielenden in einem Ring um den Berg gedreht werden muss, um den Spielenden folgen zu können. Einige Menschen werden bei einer derartigen Drehung jedoch seekrank. Damit dem VR-Spielenden nicht übel oder schwindelig wird, muss diesem das Gefühl gegeben werden, dass er den Berg rotiert und nicht er sich um den Berg. Um das zu umgehen, wird der gesamte Raum des VR-Spielenden und die sich darin befindenden Lichter zusammen mit dem VR-Spielenden gedreht. Somit wirkt es für den VR-Spielenden so, als würde sich eigentlich nur der Berg drehen, da für ihn alles andere statisch wirkt.


\begin{figure}[h]
	\centering
	\includegraphics[width=9cm]{images/buketits_layout}
	\caption{Top Down Layout des VR-Raumes mit dem Berg in der Mitte, blaue Zone stellt rotierbaren Bereich dar}
\end{figure}



\subsection{Platz und Bewegung}
Um möglichst vielen Menschen das Spielen von \emph{Tricks ´n´ Treats} zu ermöglichen ist der Spielbereich so klein wie möglich, ohne ihn einengend zu gestalten. Das heißt, dass alles Wichtige in Griffreichweite ist und stationär erreicht werden kann. Personen, die mehr Platz zur Verfügung haben, können diesen jedoch auch nutzen. Abgesehen davon muss der VR-Spielende sich nicht im Ring um den Berg bewegen, da man den Berg augenscheinlich drehen kann.

\subsection{Interaktionen}
Die Interaktionen des VR-Spielenden müssen sich real anfühlen, um die Immersion nicht zu brechen. Deswegen werden in \emph{Tricks ´n´ Treats} physikalische Hände verwendet, die nicht durch Objekte durch transformieren können. Außerdem vibriert die jeweilige Hand etwas, sobald man versucht durch ein Objekt zu greifen. Damit wird das Gefühl der Berührung ersetzt, dass mit den meisten VR-Controllern nicht umsetzbar ist.

\subsubsection{Interaktion mit Knöpfen und Items}
Der VR-Spielende kann Zauberkugeln aufheben. Wenn diese aufgehoben werden, nimmt die virtuelle Hand eine greifende Pose ein. Um die Zauberkugel nun gegen die PC-Spielenden zu en, kann der VR-Spielende diese Kugel auf dem Berg platzieren, indem er diese auf die Oberfläche schlagt. Man kann die Interaktion damit vergleichen, ein Ei auf einen Tisch zu schlagen. Um mit Knöpfen zu interagieren, gibt es zwei Methoden. Die erste Methode ist, den Knopf so zu machen, dass man nur mit der Hand in die Nähe einer unsichtbaren Zone kommen muss, um den Knopf hineinzudrücken. Die zweite und auch immersivere Methode ist, einen physikalischen Knopf zu implementieren, der vom Gefühl, näher an einen realen Knopfdruck heranrankommt.
\chapter{Performance Optimierung}
\section{VR Performance}
Dieses Dokument ist als vorwiegend technische Starthilfe für das
Erstellen einer Masterarbeit (oder Bachelorarbeit) mit \latex
gedacht und ist die Weiterentwicklung einer früheren
Vorlage\footnote{Nicht mehr verfügbar.} für das Arbeiten mit
Microsoft \emph{Word}. Während ursprünglich daran gedacht war, die
bestehende Vorlage einfach in \latex zu übernehmen, wurde rasch
klar, dass allein aufgrund der großen Unterschiede zum Arbeiten
mit \emph{Word} ein gänzlich anderer Ansatz notwendig wurde. Dazu
kamen zahlreiche Erfahrungen mit Diplomarbeiten in den
nachfolgenden Jahren, die zu einigen zusätzlichen Hinweisen Anlass gaben.


\subsection{Layer Culling}
Zitat \cite{bobsch 123 bobsch}

\section{PC Performance}
Dieses Dokument ist als vorwiegend technische Starthilfe für das
Erstellen einer Masterarbeit (oder Bachelorarbeit) mit \latex
gedacht und ist die Weiterentwicklung einer früheren
Vorlage\footnote{Nicht mehr verfügbar.} für das Arbeiten mit
Microsoft \emph{Word}. Während ursprünglich daran gedacht war, die
bestehende Vorlage einfach in \latex zu übernehmen, wurde rasch
klar, dass allein aufgrund der großen Unterschiede zum Arbeiten
mit \emph{Word} ein gänzlich anderer Ansatz notwendig wurde. Dazu
kamen zahlreiche Erfahrungen mit Diplomarbeiten in den
nachfolgenden Jahren, die zu einigen zusätzlichen Hinweisen Anlass gaben.

\subsection{Level of Detail}
Zitat \cite{bobsch 123 bobsch}

\section{Mesh Optimierungen}
Dieses Dokument ist als vorwiegend technische Starthilfe für das
Erstellen einer Masterarbeit (oder Bachelorarbeit) mit \latex
gedacht und ist die Weiterentwicklung einer früheren
Vorlage\footnote{Nicht mehr verfügbar.} für das Arbeiten mit
Microsoft \emph{Word}. Während ursprünglich daran gedacht war, die
bestehende Vorlage einfach in \latex zu übernehmen, wurde rasch
klar, dass allein aufgrund der großen Unterschiede zum Arbeiten
mit \emph{Word} ein gänzlich anderer Ansatz notwendig wurde. Dazu
kamen zahlreiche Erfahrungen mit Diplomarbeiten in den
nachfolgenden Jahren, die zu einigen zusätzlichen Hinweisen Anlass gaben.



\trennseite{Design eines asymmetrischen Local-Multiplayer-Party-Games unter Verwendung des MDA-Frameworks}{Felix Kaspar}{René Ksuz, BSc MA}{Mediendesign - Gamedesign}
\chapter{Begriffsdefinitionen}

Bei "`Tricks 'n' Treats"' handelt es sich um ein asymmetrisches Couch-Party VR-Spiel. [Hier fehlt, denke ich, eine kurze Beschreibung von unserem Spiel.]

In diesem Kapitel werden die in "`Tricks 'n' Treats"' verwendeten Technologien erklärt und die grundlegenden Konzepte des Game-Designs die verwendet werden, um die  Erlebnisse der und die Interaktionen zwischen den Spielenden zu beeinflussen.

\section{VR trifft Couchparty-coop}

\subsection{Was ist Virtual Reality?}

Virtual Reality bezeichnet Bilder und Töne, die von einem Computer erzeugt werden und dem Benutzer, der mit Hilfe von Sensoren mit ihnen interagieren kann, fast real erscheinen\cite{_oxford_dict}. Sie ist trotz anderer Anwendungsbereiche, dank der Immersion und Interaktionsmöglichkeiten die sie bietet, besonders in der Gaming-Industrie relevant geworden\cite{_bitkom_vr}. Spielenden können neue Erlebnisse geboten werden, welche wiederum eigene Game-Design-Fragen aufwerfen die es zu beantworten gilt.

VR bringt neben der Interaktivität noch einen weiteren gewaltigen Vorteil: Intuition. Greifen, Umschauen, Bewegen, alles funktioniert wie man es erwartet. Gaming-Neulinge haben häufig Probleme die richtigen Tasten auf der Tastatur zu drücken und auch klassische Gamepads sind nicht optimal. In VR gibt es zum einen weniger Knöpfe an den Controllern, die gedrückt werden können und zum anderen können alltägliche Interaktionen, wie Umschauen / Orientieren und das in die Hand nehmen von Objekten, in VR unterbewusster durchgeführt werden und sind daher für Anfangende weniger frustrierend. Spielende verstehen in VR schneller wie das Spiel gesteuert wird und können sich mehr Gedanken darüber machen was das Ziel des Spiels ist\cite{_natural_interaction_in__augmented_reality_context}. Insbesondere für Party-Spiele (Siehe \ref{_party_games}) ist das schelle Verstehen von der Steuerung und der intuitiven Ausführung von grundlegenden Aktionen in Stresssituationen sehr wichtig.

[Bild: Greifen Tastatur-E vs VR-Grab (hl vs alyx)]

\subsection{Was ist Couchparty?\label{_party_games}}

Couchparty-Spiele, oft auch nur Party-Spiele oder Couch-Games, sind Computerspiele, die eine Gruppe (die sich meist untereinander kennt) gemeinsam spielt. Anders als bei klassischen Online-Multiplayer-Spielen werden diese zusammen in einem Raum gespielt. Sie sind jedoch nicht zu verwechseln mit local-multiplayer-games. LAN-Klassiker wie Counter Strike\footnote{Counter Strike ist eine Reihe von taktischen Multiplayer-Ego-Shootern [verbesserungswürdig]} oder Starcraft\footnote{[Hier fehlt noch eine Beschreibung von Starcraft]} werden meist nur von leidenschaftlichen Gamern gespielt, da sie Vorbereitung, Know-How und viel Ausrüstung benötigen. Das Alleinstellungsmerkmal von Party-Spielen ist, dass diese auf nur einem Bildschirm von einer Couch aus, daher der Name, gespielt werden können und nur minimale Ausrüstung, meistens mehrere Controller, wobei bei manchen auch schon das Handy als Controller fungieren kann, benötigt wird.

In Couchparty-Spielen müssen Spielende schnell verstehen worum es geht, daher stützen sich viele Spiele auf "`Zufall"'. Dadurch können Anfangende schnell mit den anderen mitkommen, auch wenn sie das Spiel noch nie zuvor gespielt haben. Eine weitere Möglichkeit, das Spiel auch für schlechtere Spieler zugänglich machen wäre die Verwendung von "`Virtual Skills"'\cite[S. 165]{_art_of_gamedesign}.

Es gibt grundsätzlich zwei verschiedene Arten wie Couchparty-Spiele den Bildschirm-Platz nutzen können.

\begin{itemize}
	
\item \subsubsection{Shared-Screen}

In diesem Fall sehen alle Spielende das gleiche und müssen sich daher auch im gleichen Raum aufhalten. Der Vorteil eines solchen Systems ist, dass keine Informationen mehrfach gezeigt werden müssen und der Platz des Bildschirms vor allem bei kleinen Laptop-Bildschirmen oder sogar Handys besser genutzt wird. Außerdem hat es einen Performance-Vorteil, da nicht mehrere Kameras geändert werden müssen. Leider, eignet sich diese Art nicht für jedes Spiel, da sich die Spielenden nicht zu weit voneinander entfernen können, vor allem 3D-Spiele mit einer freien Kamera sind hier sehr eingeschränkt. Besonders eignet sich Shared-Screen-Play für Arena-Games wie z.B. "`Super Smash Bros"' oder "`Overcooked"'.

\item \subsubsection{Split-Screen}

Ein Split-Screen System bietet hier einen großen Vorteil: Freiheit. Alle Spielenden können sich selbständig bewegen und umschauen, ohne das Erlebnis der anderen zu beeinflussen. Allerdings wird hierfür ein größerer Bildschirm benötigt, da jedem Spielenden nur ein halber oder nur ein viertel des Bildschirms zugewiesen wird.

\end{itemize}

\noindent Es gibt auch die Möglichkeit nahtlos zwischen den beiden Systemen zu wechseln je nachdem, welches sich gerade besser eignet, um das gewünschte Spielerlebnis zu erzeugen. Laut einer Reddit Umfrage präferieren Spielende Shared-Screen-Play. [Darf ich eine Reddit-Umfrage als Quelle]

\subsection{Was sind coop-games?}

Coop-games (= cooperative-games/Kooperative-Spiele) sind eine besondere Art von team-basierten-Spielen die besonders auf Gemeinschaft der Spielenden setzt und diese auch durch Game-Design voraussetzt. 

Ein Koalitions- oder Strategiespiel ist kooperativ, wenn die Spielenden verbindliche Vereinbarungen über die Verteilung der Auszahlungen oder die Wahl der Strategien treffen können, auch wenn diese Vereinbarungen nicht durch die Spielregeln spezifiziert oder impliziert sind\cite{_introduction_to_the_theory_of_cooperative_games}.

Bei klassischen Team-basierten-Spielen ist Kommunikation zwar wichtig jedoch nicht essentiell. Aufgaben der Spielenden werden in coop-games so verteilt, dass ein Ziel nicht ohne die Zusammenarbeit der anderen erreicht werden kann. Es entsteht hierbei eine interessante Dynamik [hier könntest du noch kurz beleuchten welche Auswirkungen diese Dynamik hat und vielleicht sogar, was dann passiert, wenn die Stakes größer werden (zB Schuldzuweisung)] in der die Spielenden komplett voneinander abhängig sind. Kooperation kann auch den Spaßfaktor eines Spieles fördern, mehr dazu in Kapitel \ref{_cooperative_play}.

Gute Beispiele hierfür sind einige Spiele der Lego-Reihe, Overcooked und It Takes Two welches ich in Kapitel \ref{_industrie} genauer beschreiben werde.

\subsubsection{Wie erzeugt man Kooperation}
Kooperation in, auch jene in Videospielen, kann zu tiefen Bindungen zwischen den Spielenden führen.
Mit der "`Lens of cooperation"'\cite[S. 311]{_gamemechanics_for_cooperative_games} kann ein Gamedesigner besser verstehen wie Kooperation entsteht und wie man sie fördert. Die wichtigsten Punkte, die Jesse Schell hier anspricht, sind die Notwendigkeit von Kommunikation oder die Tatsache, dass ein Task nur durch Zusammenarbeit der Spielenden gelöst werden kann. Ein weiterer interessanter Punkt ist Synergie und Antergie, die Veränderung der Stärke einer Gruppe, wenn ein team zusammenkommt, Synergie beschreibt, dass die Gruppe Stärker als die Summe der einzelnen ist, Antergie ist das Gegenteil davon. Synergie ist damit in kooperativem Umfeld bestrebenswert.

\subsection{Was bedeutet Asymmetrie?}

Asymmetrie beschreibt im Allgemeinen zwei Seiten oder Teile, die nicht die gleiche Größe oder Form haben\cite{_oxford_dict}.

In Spielen bezieht sich Asymmetrie auf das (Board-) Design. Die Spielenden haben in einem solchen Spiel ein hoch unterschiedliches action-set, das bedeutet sie haben einen anderen Einfluss auf die Spielwelt/andere Mitspieler. In Spielen mit asymmetrischen Wahlmöglichkeiten beginnt jeder Spielende jedoch typischerweise mit einer Reihe von Aktionen, die sich stark von denen der anderen unterscheiden, was das "`Balancing"' dieser Spiele besonders schwierig macht. Die Herausforderung besteht vor allem darin, den relativen Einfluss einer Aktion auf die Gewinnwahrscheinlichkeit zu bestimmen\cite[S. 18]{_balancing_asymmetric_video_games}.

\chapter{Spaß in Videospielen}

\section{Warum machen Spiele Spaß?}

McKee definiert Spaß sehr simpel: "`Vergnügen ohne Ziel"'. Nach dieser Definition kann alles Spaß sein, wenn man es für Vergnügen tut\cite{_fun}.

\subsection{Woher kommt Spaß?}

Gamedesigner Marc LeBlanc hat acht verschiedene Typen von situationsabhängigem Spaß definiert. Einige davon lassen sich gut auf Situationen unseres Spieles anwenden um genauer zu verstehen, woher der Spaß in dem Spiel kommen soll: "`drama, obstacle, social framework"'. 

Das Buch "`A Theory of Fun"'\cite{_theory_of_fun} definiert Spaß als das mentale meistern eines Problems\cite[S. 71]{_theory_of_fun}. Außerdem meint Raph Koster, dass Spaß aus "`richly interpretable situations"'\cite[S. 40]{_theory_of_fun}, also Situationen, die den Spielenden mehrere Möglichkeiten geben ein Problem zu lösen und Kreativität fördern, entsteht. 

Spaß kann also nur existieren, wenn Spielende immer neue Probleme lösen können oder bereits gelöste Probleme auf andere Wege ("`richly interpretable"') lösbar sind. Das Meistern eines Spieles, laut seiner Definition, kann also nur Spaß machen, wenn sich die Situationen je nach skill-level ändern oder sie auf andere Wege lösbar sind, mehr dazu in Sektion \ref{_flow}. Das ständige Ausführen der selben Aktion macht also keinen Spaß. Raph Koster hat außerdem einige wichtige Grundsteine für Spaß in Spielen gelegt, diese werden im praktischen Teil noch genauer behandelt.

\subsection{Spaß bei mehrspieler Spielen}

Ein weiterer Ursprung von Spaß bzw. Vergnügen lässt sich in sozialen Interaktionen finden\cite[S. 72]{_theory_of_fun}. Hier unterscheidet man zwischen kooperativen und kompetitiven Interaktionen, wobei beide Spaß erzeugen können. 

\subsubsection{Competitive Play}
[Hier fehlt noch was]
[Lens of Competition]

\subsubsection{Cooperative Play\label{_cooperative_play}}
Spiele die Kooperation fördern, können in Spielenden positive Gefühle wie soziale Zugehörigkeit, Möglichkeiten zur sozialen Vernetzung und die Förderung der sozialen Integration bieten\cite{_putting_the_fun_factor_into_gaming}. Die Analyse deutet auch darauf hin, dass Flow (dazu komme ich in Kaptiel \ref{_flow}) in sozialen Spielen auftritt, insbesondere im kooperativen Gameplay\cite{_putting_the_fun_factor_into_gaming}. Soziale-Kontexte können so die emotionalen Erfahrungen des Spielens verbessern.

\subsubsection{Arena}
Party-Spiele finden in der "`Arena"' statt\cite[S. 65]{_art_of_gamedesign}. Im Vergleich zu anderen "`Arena-Spielen"' befinden sich die Spielenden bei Party-Spielen im selben Raum und können sich so auf noch mehr Weisen beeinflussen, die bei klassischen "`Arena-Spielen"', z.B. Shooter, nicht möglich sind. Es entsteht ein anderer, persönlicherer Umgang miteinander in und außerhalb der Spielwelt.\newline

\noindent Außerdem hat eine Studie herausgefunden, dass Spaß der mit anderen geteilt wird stärker empfunden wird als einsamer Spaß, das hat vermutlich einen evolutionären Ursprung\cite{_fun_is_more_fun}.

\section{Wie kann man Spaß "`kreieren"'?}

Wie vorher definiert gibt es verschiedene Arten von Spaß. Das Erlebnis der Spielenden lässt sich hauptsächlich durch die Aktionen beeinflussen, welche den Spielenden zu Verfügung gestellt werden und deren Auswirkungen auf die Spielwelt oder andere Spielende.

\subsection{MDA-Framework}

Das MDA-Framework hilft bei der Analyse eines Spieles und dem iterativen Designprozess. Es wurde von dem Game-Designer Marc LeBlanc mitentwickelt und trennt den "`Konsum"' von Spielen in verschiedene Komponenten\cite{_mda}. Es ist unterteilt in Mechanics, Dynamics und Aesthetics. Die unterste Ebene beeinflusst immer das was drüber ist. Es beginnt mit den Mechanics, diese beeinflussen alle Ebenen darüber.

[MDA-Bild]


\begin{itemize}
\item\subsubsection{Mechanics (Regeln)}

Regeln und Feedback-Loops die die Spielenden in ihren Aktionen limitieren. Die Mechanics beschreiben das Ziel des Spieles und wie Spielende es erreichen bzw. nicht erreichen können\cite[S.96]{_art_of_gamedesign}. Sie sind das was das Spiel im Zentrum ausmacht, selbst wenn man alles andere weg lässt \cite[S.231]{_art_of_gamedesign}.
 
\item\subsubsection{Dynamics (System)}

Das entstehende Verhalten der Spielenden welches aus den Mechanics hervorgeht\cite{_mda}. Sie beschreiben, wie sich das Spiel bzw. wie sich Systeme und Mechaniken im  im Laufe der Zeit verändern. Sie bieten dem Spieler ein Gefühl des Fortschritts und der Veränderung, hier spielt auch Feedback eine große Rolle, dazu kommen wir später. Hier kommen die Regeln, die Spielwelt und die Spielenden zusammen, Spielende werden mit der Zeit besser und das Spiel wird Schwerer und bietet so eine fesselnde Erfahrung.

\item\subsubsection{Aesthetics (Spaß)}

Die emotionalen Reaktionen der Spielenden. Sobald Spielende zum ersten mal mit dem Spiel interagieren, löst es Gefühle in ihnen aus. Ziel ist es eben diese Gefühle zu steuern und das Interesse der Spielenden so aufrecht zu erhalten.\newline

\end{itemize}

\noindent Als Game-Designer muss man das Framework im Überblick behalten. Designer verändern die Mechanics des Spieles, sehen aber gleichzeitig deren Einfluss auf die Aesthetics und müssen diese genauso berücksichtigen um das Spiel in die richtige Richtung zu lenken. 
Es ist auch möglich ein Spiel durch Mechaniken zu designen, allerdings, tappt man dann im Dunkeln, und man kann nie wissen, was dabei entsteht \cite[S.56]{_art_of_gamedesign}.

\subsection{Essential Experience}

Oft möchte der Designer eine Experience, ein Erlebnis, das man vorher definiert hat, einfangen. Die "`Essential Experience"' versucht, wenn auch auf einem anderen Weg, die reale Experience einzufangen und durch Mechanics zu erzeugen. Das designen eines Spieles mithilfe der "`Essential Experience"' ermöglicht einem, das Spiel von der Experience getrennt zu sehen, und so entscheiden zu können, was an dem Spiel noch verändert werden muss bzw. welche Teile des Spieles nicht verändert werden dürfen um den Grundgedanken zu erhalten\cite[S.55]{_art_of_gamedesign}.

\subsection{Core Mechanics}

Die Core Mechanics sind die Grundbausteine/der Anfang des Designprozesses des Spieles. Sie sind jene Mechaniken, die in dem Spieler die erwünschten Gefühle auslösen und ohne welche das Spiel nicht funktionieren könnte. Sie beinhalten grundlegende Dinge wie Interaktionsmöglichkeiten, Fortbewegung und Fähigkeiten.

\subsection{Pacing, Flow\label{_flow}}

Der Schwierigkeitsgrad von Spielen muss sich in einem gewissen Bereich aufhalten, der von dem Psychologen Cziksentmihalyi als "`Flow Channel"' bezeichnet wird, um die Aufmerksamkeit eines Spielers zu behalten\cite[S.205]{_art_of_gamedesign}.

[Bild: Flowchannel]

Als "`Pacing"' beschreibt man den Balanceakt der Herausforderung des Spieles und der Fähigkeiten der Spieler. [hast du dafür ne Quelle? Das ist ne sehr seltsame Definition, um nicht zu sagen falsch :D] Game-Designer haben dabei zum Ziel die Spielenden in einen Flowstate zu bringen.

Flow ist "`ein Gefühl der vollständigen und energiegeladenen Konzentration auf eine Tätigkeit, mit einem hohen Maß an Freude und Erfüllung"'\cite[S.204]{_art_of_gamedesign}.
Cziksentmihalyi beschreibt Flow etwas einfacher als einen Zustand der auftritt, wenn man sich freiwillig an seinem Limit befindet\cite{_flow}.

[Wie bringt man Flow in ein Spiel? +art of gamedesign]\cite{_theory_of_fun}

\subsection{Player Action Feedback}

Feedback ist auch in Spielen wichtig. Ohne Feedback wissen Spielende nicht ob das was sie tun richtig ist oder das Spiel deren Aktionen richtig verarbeitet hat. Visuelles und Auditives Feedback kann den Spielenden über Fortschritt und Fehler informieren. Außerdem hilft es den Spielenden sich mit ihren ganzen Gedanken in die Spielwelt einzutauchen.
\chapter{Stand der Industrie\label{_industrie}}

\section{Asymmetrical Coop}

\subsection{Keep Talking and Nobody Explodes}
In Keep Talking and Nobody Explodes übernimmt eine Person das Entschärfen einer Bombe, während die andere eine Anleitung zur Entschärfung navigieren muss. Das Spiel ist zwar kooperativ, jedoch sehr einseitig, da der Bombenentschärfende oft auf auf den Instruktor warten muss und es außerhalb der Beschreibung von dem was man sieht nicht viel Kommunikation gibt. Auch kein euphorisches Anfeuern, obwohl alles unter enormem Zeitdruck ausgeführt wird. Man muss allerdings anmerken, dass die Experience des Bombenentschärfens damit weiter unterstützt wird.

\subsection{It Takes Two}
Anders ist es bei "`It Takes Two"'. Es überzeugt besonders durch sein abwechslungsreiches Gameplay und dem sehr guten Pacing. Es erfordert von beiden Seiten nicht sehr viel können, dadurch ist es für sehr viele Leute zugänglich und es lässt sich so gut wie mit jeder anderen Person spielen. Kommunikation ist sehr wichtig, da viele Rätsel lösungsorientiertes Denken beider Spielenden benötigen. Außerdem bietet das Spiel eine entspannende Atmosphäre in der man sich schnell verliert.

\section{Couchparty}

\subsection{Mario Party}
Mario Party ist das klassische Party-Spiel. Wie für das Genre üblich bietet es eine enorme Vielfalt an Mini-Spielen und einen hohes Maß an Zufall um auch die Schlechteren einer Gruppe im Spiel zu inkludieren.

\subsection{Jackbox}
Bei den Jackbox-Spielen handelt es sich um eine Reihe an verschiedenen Mini-Spielen die einen oft an klassische Papier-Spiele erinnern, die man früher und auch heute noch ohne viel Aufwand gespielt hat (z.B. Wörter Raten). Es ist besonders gut darin lustige Geschichten oder andere Kreationen aus den Spielenden herauszuholen. Es bietet aber auch einige komplexere Spiele, die für eine erfahrenere Gruppe noch mehr Abwechslung bieten. Die Experience entsteht hier aber haupsächlich durch die Imagination der Spielenden.

\section{Couchparty VR-Games}

Couchparty VR-Games sind per Definition asymmetrisch. In dieser Kombination wird es schon schwieriger gute Spiele zu finden, da der Markt noch relativ klein ist und sich die großen Studios (Nintendo, EA, etc.) noch nicht in den Markt investieren, da es finanziell noch keinen Sinn macht.

\subsection{Takelings House Party}

Takelings bringt in das Chaos von Couchparty-Spielen noch einen VR-Spieler. [Hier fehlt noch was]

\subsection{Acron: Attack of the Squirrels!}
[Muss ich noch Spielen]
\chapter{Implementation}

\section{Konzept}

[Konzept vom Spiel, kommt an den Anfang der gesamten Arbeit]

\section{Was muss bei VR-Game-Design beachtet werden?}

VR-Spiele bringen einige Limitationen mit sich, die es zu überwinden gilt. Bei der ersten Limitation handelt es sich um die Größe des Play-Spaces, also jenem Physischem Raum, in dem sich VR-Spielende aufhalten. Das zweite Problem ist die Reisekrankheit (Motion-Sickness, eine körperliche Reaktion auf ungewöhnliche/unzusammenhängende Bewegung\cite[S. 533]{_art_of_gamedesign}), vor allem bei Party-Spielen muss ihr Einfluss möglichst gering sein, da das Spiel von einer breiten Masse gespielt werden können soll. Die Reisekrankheit limitiert die Bewegungsmöglichkeiten der Spielenden und diese wiederum die Größe und Gestaltung der virtuellen Umgebung.

Außerdem muss die Größe des Spielenden (um sicherzustellen, dass alles womit Spielende interagieren sollen erreichbar ist), als auch Ängste, wie zum Beispiel "`Klaustrophobie"', beachtet werden. Im Falle von einem Party Spiel spielt auch das technische Know-How eine Rolle, da man sehr schnell in das Spiel eintauchen können soll.

\section{Stimmung des Spieles}
Das im Zuge dieser Arbeit entwickelte Spiel soll eine bestimmte Stimmung erzeugen. Am Anfang des Design-Prozesses muss diese Stimmung festgelegt werden.

\subsection{Core Pillars}

\subsubsection{Fast but Strategic}
Das Ziel des Spiels soll es nicht sein, möglichst schnell Spells zu aktivieren. Es muss etwas Strategie dahinter sein, damit Spielende gefordert werden und eine wertvolle Entscheidung treffen können. Da es sich um ein Party-Spiel und kein Strategie-Spiel handelt, sollte es möglich sein, schnelle Entscheidungen zu treffen. Um das zu ermöglichen, muss das Spell-System vergebend sein, und muss sich auf die aktuelle Spielsituation anpassen können\ref{_rubberbanding}. 

\subsubsection{Teamwork \& Kommunikation}
Außerdem muss Strategie und Absprache zwischen den PC-Spielenden angeregt werden, um weiter das Teamwork der Gruppe zu fördern. Wie am Anfang der Arbeit bereits definiert, braucht Teamwork ein gemeinsames Ziel. Das Ende der Strecke kommt hier recht schnell in den Sinn, allerdings ist das allein nicht genug, um Kommunikation zwischen den Spielenden zu erzeugen, da es zu weit in der Zukunft liegt, und sich nicht verändert. Eine Lösung, und mögliche Lösungsansätze, für dieses Problem werden in Kapitel \ref{_teamwork_erzeugen} besprochen.

\subsubsection{God-Like}
Vor allem der VR-Spieler soll sich mächtig fühlen. Erreicht werden soll dieses Gefühl durch den deutlichen Größenunterschied zwischen Magier und Snowboarder und Combos, also der Möglichkeit mehrere Snowboarder gleichzeitig von der Strecke zu schmeißen. Es muss allerdings beachtet werden, dass das alleinige "`auf die Strecke schmeißen"' keinen Spaß machen wird, wenn davor nicht etwas dafür getan werden musste. Hier kommt die "`Lens of Challenge"'\cite{_art_of_gamedesign} ins Spiel, man sollte besonders darauf achten, dass die Schwierigkeit das Ziel zu erreichen mit dem Skill des Spielenden übereinstimmt, außerdem ist es wichtig die Interaktion nicht zu einseitig zu gestalten. Das Gefühl der Macht kann nur erzeugt werden, wenn Spielende denken, es war ihre alleinige Leistung.

\subsection{Target Experiences}

Da es sich bei dem Projekt um ein asymmetrisches Spiel handelt, wurden für die PC- und die VR-Spielenden jeweils andere Target Experiences festgelegt.

\subsubsection{Snowboarder (PC)}
Für die Snowboarder sind folgende zwei Experiences am wichtigsten: die (Snowboarding-) Action und das Teamwork der Gruppe. Mithilfe der "`Lens of Essential Experience"'\cite[S. 55]{_art_of_gamedesign} kann versucht werden, diese Experiences zu erzeugen.

Um das Gefühl "`Snowboarden"' akkurat wiederzugeben wurden 3 wichtige Komponenten gefunden: Tricks, Stürzen und Beharrlichkeit. Man könnte auch sagen, dass die Kälte ein Teil des Snowboardens darstellt, jedoch wurde dagegen entschieden, da es dem Core Pillar "`God-Like"' widerspricht, wenn sich die Snowboarder über die Körpertemperatur ihrer Charaktere Gedanken machen müssen.

Eine weitere wichtige Experience der Snowboarder ist das Teamwork, sogar so wichtig, dass es, wie oben erwähnt, zu einem der Core Pillars gemacht wurde. Wie Teamwork erzeugt werden soll, wird in Kapitel \ref{_teamwork_erzeugen} behandelt.

\subsubsection{Magier (VR)}
Für den Magier sind die folgenden zwei Experiences am wichtigsten: Macht und Multitasking. 

Diese, im Konkreten Göttliche/Magische Kraft, soll vor allem durch das Zaubern, also den Spells (\ref{_spell_design}]), und die Aneinanderreihung und taktische ("`Fast but Strategic"') Platzierung dieser erzeugt werden. Sie wird auch über den visuellen Größenunterschied der Snowboarder und dem Magier dargestellt.

Multitasking ist vor allem im Bereich der Interaktionen wichtig, der Magier sollte immer schon bevor er einen Spell platziert hat, darüber nachdenken, was der nächste Zug sein wird. Um des zu erreichen, sollten Spells nicht immer sofort verfügbar sein. Ziel ist es dadurch mehr Strategie in das Spiel zu bringen.

\section{Zusammentreffen beider Welten (PC + VR)}

\subsection{Streckendesign}
Der erste Punkt, und auch Grundstein für das restliche Design, ist das Konzept der Stecke. Es hat sowohl Einfluss auf die PC-Spielenden als auch auf den VR-Spielenden, da es den Ort darstellt, in dem beide miteinander Interagieren.

\subsubsection{Limitationen}
Der VR-Spielende muss die Möglichkeit haben, zu jedem gegebenen Zeitpunkt, mit einem Großteil der Strecke zu interagieren (Fallen platzieren, etc.). Außerdem ist es wichtig, dass sich Spells immer in Reichweite befinden. Die Strecke muss jedoch auch (aufgrund anderer technischer Limitationen) zu jedem Zeitpunkt komplett geladen und sichtbar sein.

Das Streckendesign muss es zulassen, die Strecke so anzupassen, dass eine Fahrt von oben nach unten ca. 3 Minuten dauert. Des Weiteren muss die Breite der Strecke genug Platz, um Hindernissen auszuweichen, und den Spielenden eine gewissen Toleranz für Fehler beim Lenken der Snowboarder lassen.

Zur Auswahl stehen drei verschiedene Streckendesigns, deren vor und Nachteile behandelt werden und begründet wird, warum schlussendlich für Streckendesign Nr. 3 entschieden wurde. Alle drei Designs versuchen die Bewegung des VR-Spielenden zu vermeiden, um das Gefühl der Reisekrankheit zu mindern.

\subsubsection{Designoption 1 - Spirale}
Die Snowboarder fahren, in diesem Design, in einer Spirale um den VR-Spielenden herum. Es ermöglicht eine freie Strecke für die Snowboarder, jedoch ist wenig Platz für VR-Spielende. Dieses Design ist auch sehr unrealistisch, da es sich hier nicht um eine klassische Ski-Strecke handeln kann, da Teile "`Schweben". Dieser Effekt wird weiters verstärkt, da es sich für die Spielenden so anfühlt, als wären sie in einer Wasserrutsche. Das widerspricht alles der Target-Experience des Snowboardens.

\subsubsection{Designoption 2 - Kristallkugel}
Dieses Design bringt auf der Seite der Snowboarder und des VR-Spielenden deutliche Verbesserungen. Da VR-Spielende durch eine Kristallkugel auf die Snowboarder herabschaut, kann die Strecke unglaublich frei sein, es kann harte Kurven und Richtungswechsel geben. Der VR-Spielende, hat die Möglichkeit, einen eigenen Raum zu bekommen, in dem der Zugang zu den Spells einfach ist. Der größte Nachteil ist hier die Technik, die im Hintergrund gebraucht wird. Technisch ist es sehr schwer umzusetzen, da die Welten des VR-Spielenden und der Snowboarder voneinander getrennt und über die Kristallkugel wieder zusammengebracht werden müssen.

\subsubsection{Designoption 3 - Berg}
Die Idee des Berges wurde am Anfang etwas vernachlässigt, da es sehr eingeschränkt erschien. Jedoch bringt der Berg neben der technisch um einiges leichteren Umsetzung wieder etwas Realismus und eine schöne Atmosphäre ins Spiel, die die Experience des Snowboardens verstärken kann. Auf diesem, aus der Sicht des VR-Spielenden, kleinen Berg, in der Mitte des VR-Play-Spaces, können die Snowboarder spiralförmig hinunterfahren. Wie bei der Spirale hat man hier aber auch das Problem, dass die Snowboarder permanent in die Kurve lenken müssen, dieses Problem wird in Kapitel \ref{_playercontroller} weiter behandelt. Der Berg kann automatisch mit den Snowboardern mitgedreht werden, damit sie immer im Sichtfeld des Magiers sind. Der Magier soll aber auch eine Möglichkeit haben diese Drehung zu überschreiben, damit auch Teile der Strecke die außerhalb des interagierbaren Bereiches liegen beeinflussen kann.

\subsection{Interaktionsdesign}
Da der Wettbewerb zwischen PC und VR asymmetrisch ist, müssen beide Parteien die jeweils andere Seite beeinflussen können, um es interessant zu machen. Der Magier muss durch Spells und andere Tools die Snowboarder behindern können und die Snowboarder müssen Einfluss auf die Welt des VR-Spieleden haben.

\subsubsection{Spell Design\label{_spell_design}}
Um dem VR-Spielenden das Gefühl von Zaubern zu übermitteln, braucht es neben der Möglichkeit zu Zaubern auch gute Interaktionen. Das Spell-System muss dem Magier die Möglichkeit geben zu Multitasken und Kombos zu machen, damit das Gefühlt der Macht aufkommt. Da der VR-Spielende zwei Hände hat, ist es naheliegend, dass beide Hände an unterschiedlichen Spells arbeiten können. Um eine gewisse Vielfalt in die möglichen Spells zu bringen, wurden drei Spell-Kategorien eingeführt.

\paragraph{Instant-Spell}
Sollen einen sofortigen Effekt auf die PC-Spielenden haben. Instant-Spells sollen sich besonders gut für die Kombination mit anderen Spell-Typen eignen. Sie dürfen auch in die Masse der Snowboarder platziert werden, da sie niemals sofort töten dürfen. Ein Beispiel hierfür wäre eine Eisfläche, die die Snowboarder zum Rutschen bringt.

\paragraph{Short-Term-Spell}
Dieser Spell-Typ braucht eine gewisse Zeit, um sich zu aktivieren. Somit müssen Spielende sich vorher etwas genauer überlegen, wo sie am besten zu platzieren sind, und die Snowboarder haben etwas Zeit um auszuweichen. Short-Term-Spells dürfen auch tödlich enden, ein Beispiel hierfür wäre eine Bombe, die die Snowboarder in die Luft schleudert. Alle Short-Term-Spells müssen nach einer gewissen Zeit wieder verschwinden.

\paragraph{Long-Term-Spell}
Das Ziel von Long-Term-Spells ist es den Spielenden die Möglichkeit zu geben, etwas strategischer zu Spielen. Sie können nur begrenzt, und an vordefinierten orten aktiviert werden. Ziel ist es Spielende dazu zu bewegen, zu Beginn einer Runde abzuwägen einen Long-Term-Spell einzusetzen, und dafür weniger Instant- und Short-Term-Spells, oder darauf zu verzichten, um flexibler zu sein. Beispiele hierfür wären das Absperren von gewissen Streckenteilen, oder das zum Einsturz bringen von Bäumen oder Geröll auf der Strecke.

\paragraph{Balancing-Cooldown}
Die Anzahl der Spells, welche der Magier verwenden kann, muss limitiert sein. Der einfachste Weg, um Spells zu limitieren, ist ein Cooldown, welcher verhindert, dass zwei Spells direkt nacheinander platziert werden können. Es wird jedoch bei diesem System schnell dazu kommen, dass VR-Spielende immer den "`Stärksten"'' Spell nehmen, und es dadurch schwer zu balancen ist.

\paragraph{Balancing-Mana\label{_mana}}
Bevor ein Spell platziert werden kann, muss er in einen Mana-Kessel getaucht werden, um ihn zu aktivieren. Dieser Mana-Kessel kann nur eine begrenzte Menge an Mana halten und befüllt sich über den Verlauf des Spieles von selbst. Ein Mana-Kessel bietet gleich mehrere Vorteile: Balancing und Physische-Interaktion. Mana kann dabei helfen Spells untereinander zu balancen, indem ihre Kosten erhöht oder verringert werden. Außerdem fügt der Kessel einen Zwischenschritt hinzu, das macht die Interaktion komplexer und interessanter. Sobald ein Spell einmal aufgeladen ist, ist man dazu gezwungen ist ihn auch zu platzieren, da er sonst seine Ladung wieder verliert.

Die Anzahl der Spielenden, die noch im Rennen sind, könnte sich auf die Mana-Regeneration auswirken, so entsteht ein negativer Feedbackloop, es wird für den Magier schwieriger zu gewinnen, wenn er kurz davor ist den letzten Snowboarder zu Fall zu bringen. Das erzeugt einen weiteren Pacing Spike, wie bei einem Bosskampf, am Ende einer Runde und hat einen automatischen Balancing\cite[S. 296]{_game_design_workshop}-Effekt.

\paragraph{Balancing-Interaction\label{_balancing_interaction}}
Ein weiterer Weg, wie Spells gebalanced werden könnten, wäre durch die Dauer / Komplexität der Interaktion. Ein großer Vorteil hierbei wäre, dass es den Spielenden etwas natürlicher vorkommt als ein künstlicher Cooldown, dadurch sollte es auch weniger frustrierend sein, wenn gerade kein Spell bereit steht, da der Magier selbst dafür verantwortlich war. [Ich denke, das heißt "`internal attribution"', gibts dafür Quellen? In Ksuz Mail]

\subsection{Playercontroller \& Obstacles\label{_playercontroller}}
\subsubsection{Tricks}
Tricks können zum einen dazu genutzt werden, ein Risiko einzugehen, um die Respawn-Zeit zu verkürzen oder als eine Movement-Mechanik, die dabei hilft Obstacles auszuweichen. In das Spiel wird vermutlich eine Mischung beider Mechanics implementiert.

\subsubsection{Respawns}
Das Wiedereintreten von ausgeschiedenen Spielenden ist eine wichtige Mechanik, um das Gefühl des Snowboardens einzufangen und um den Party-Aspekt des Spieles zu fördern. Es gibt verschiedene Möglichkeiten zu regeln, wann Spielende wieder in die Runde eintreten.

\paragraph{Timer}
Eine sehr simple Möglichkeit wäre, Ausgeschiedene nach einer gewissen Zeit automatisch wieder dem Rennen hinzuzufügen. Das hätte den Vorteil, dass es zum einen sehr leicht umsetzbar ist und zum anderen den Spielenden eine kleine Pause verschafft und der Timer sich durch andere Faktoren beeinflussen ließe, was wiederum beim Balancing helfen kann.

\paragraph{Sekundäres Ziel für ausgeschiedene Spielende}
Ein weiterer Weg, die die ausgeschiedenen Spielenden besser ins Spiel inkludiert, wäre ein Sekundäres Ziel. Sie müssten, um wieder ins Rennen zu kommen, eine kleine Aufgabe lösen, bestimmte Knöpfe schnell hintereinander drücken (verbunden mit viel Game-Juice, um die durch die Spielenden wahrgenommene Intensität des Gameplays zu steigern), oder sie könnten den VR-Spielenden behindern mehr dazu in \ref{_rubberbanding}. [hier könntest wieder die internal Attribution aufgreifen, weil es ja in der Hand der Spielenden liegen würde, schnell wieder was beizutragen.]

\subsubsection{Automatische Lenkhilfe}
Da die Spielenden bei dem Berg- oder Spiralen-Streckendesign permanent im Kreis lenken müssen, wäre es wichtig, dem entgegenzuwirken, um das Gefühl des "`im Kreis"'-fahrens zumindest etwas zu mindern. Hier wäre eine Automatische Lenkhilfe, die den Snowboarder automatisch nach unten fahren lässt, die einfachste Methode. Verbunden mit einer Lenkung, welche Relativ zur Kamera funktioniert, haben Spielende eine angemessene Lenk-Freiheit in beide Richtungen.

\subsubsection{Obstacles}
Obstacles können sowohl von der Strecke vorgegeben sein als auch durch den Magier platziert werden. Sie zwingen die Snowboarder Entscheidungen zu treffen, insbesondere im Falle von \emph{Positive Obstacles}. Normale Obstacles sollten von den Snowboardern vermieden werden.

\paragraph{Positive Obstacles\label{_positive_obstacles}}
\emph{Positive Obstacles} sind ähnlich wie Pickups in z.B. Mario Cart, nur sind sie in die Welt integriert. Sie haben einen positiven Effekt auf die Gruppe, jedoch werden sie von VR-Spielenden eher angegriffen/beschützt.

\subsubsection{Spieler führen}
Wenn man Spielende für riskantes Fahren belohnt, entsteht eine Meaningful Decision. Es entsteht eine Risk-Reward Situation, in der sich schlechte Teammitglieder von den Guten abspalten und andere Wege nehmen. Mithilfe von Rampen und Slalom-Bögen, die die Respawn-Zeit verkürzen, kann man riskantes Fahren fördern.

\subsection{Teamwork erzeugen - Krone \& andere Lösungsansätze\label{_teamwork_erzeugen}}
Um die Snowboard-Gruppe dazu zu bewegen zu kooperieren, müssen sie ein gemeinsames Ziel haben. Dieses Ziel ist schlussendlich das Ende der Strecke jedoch braucht es auch ein kurzzeitiges Ziel wie zum Beispiel einem ausgeschiedenen Spielenden dabei zu helfen, wieder ins Rennen zu kommen. Mögliche Aufgaben dafür wären, das Einsammeln von bestimmten Punkten auf der Strecke, das Halten der Krone, oder eine gemeinsame Aktion (z.B. einen Trick machen).

\subsubsection{Coop-Points}
Bei Coop-Points müssen verschiede Spielende zur selben Zeit an zwei verschiedenen Orten sein. Dadurch muss es zwingend zu einer Absprache zwischen den Spielenden kommen, um auszumachen wer wohin fährt.

\subsubsection{Krone\label{_krone}}
Der erste Spielende hält immer die Krone. Wenn die Krone länger als eine gewisse Zeit vom selben Spielenden gehalten wird, kann ein ausgeschiedener Spielender wieder auf die Piste. Sollte die Krone von einem anderen Spielenden übernommen worden sein, bevor ein Snowboarder respawnt ist, wird der Timer wieder länger. Nach einem Respawn, muss die Krone an einen anderen Spielenden übergeben werden, um Teamwork und Kommunikation innerhalb des Teams zu fördern.

\subsubsection{Gemeinsame Aktion}
Eine gemeinsame Aktion wäre beispielsweise, dass alle Snowboarder zur selben Zeit Springen oder einen Trick ausführen. Das würde wieder ein großes Risiko mit sich bringen, da alle gleichzeitig sterben könnten.

\subsubsection{Rubberbanding [Ich finde nichts literarisches zu diesem Wort soll ich es durch dynamic balancing austauschen? Ksuz Mail]\label{_rubberbanding}}
Snowboarder, die weiter hinten sind, sollten schneller werden, damit sie mit der Gruppe mithalten können. Wie in \ref{_mana} angesprochen, ist es auch wichtig PC und VR untereinander zu balancen. Neben dem bereits angesprochenen, gibt es noch folgende weitere Möglichkeiten. Ausgeschiedene Spielende könnten den VR-Spieler daran hindern weitere Spells zu aktivieren oder gute Sicht auf die Strecke durch z.B. Nebel verhindern. Spells könnten auch dynamisch gestärkt oder geschwächt werden, indem man deren Range oder Dauer verändert.
\chapter{Post-Mortem}

Im Designprozess ist nicht immer alles so wie man es erwartet, oft kommt es zu unerwarteten Nebeneffekten, die man im ursprünglichen Design nicht beachten konnte. Deswegen ist die Flexibilität aller Komponenten, um späteres Balancing zu ermöglichen, und besonders Playtesting in verschiedenen Phasen des Designprozesses notwendig.

\section{Playtests}

Durch Playtests wurden einige Mängel im Design gefunden, die dann auch rechtzeitig gehandhabt werden konnten. Leider sind Playtests erst spät in der Entwicklung sinnvoll gewesen. Außerdem wurden neue Bereiche aufgedeckt, die vorher vernachlässigt wurden. Sollte das Spiel weiterentwickelt werden, sind Playtests ein wichtiger Punkt um die Qualität des Spieles zu sichern und weiter zu polishen, da die Grundsysteme mittlerweile funktionieren.

\subsubsection{Feedback}

Einige Dinge, auf die wir durch Playtests aufmerksam wurden, wurden bereits implementiert. Darunter ist auch mehr Feedback für tote Spielende und wann sie wieder Respawnen können. Neben dem initialen Feedback für Tot und Respawn, ist es auch wichtig den Spielenden über eine gewisse Zeit hinweg zu sagen, dass sie eine bestimmte Aktion ausführen können, sollten sie den ersten "`Tip"' übersehen haben. Hier haben wir uns für eine leichte Vibration des Controllers entschieden (Diese Idee kam nur dank des Feedbacks unserer Playtester). Es ist manchmal aber auch wichtig den Spielenden Zeit zu geben, und sie selbstständig entscheiden zu lassen wann eine Aktion ausgeführt wird, anstatt ihnen einfach zu sagen "`Jetzt!"'. Neben Feedback ist also auch Interaktion wichtig.

Eine weitere Sache, auf die wir bei Playtests aufmerksam wurden, war wie die Spielenden mit Spells interagieren. Lieder wurde darüber anfangs weniger nachgedacht. Es wurde jedoch schnell klar, dass viele Spielende nicht intrinsisch herausfinden konnten, wie Spells zu aktivieren sind. Es wurden verschiedene Möglichkeiten ausprobiert, um es klarer zu gestalten. Das ursprüngliche Aktivieren per Knopfdruck wurde durch ein Collider basiertes aktivieren ausgetauscht. Schlussendlich wurde sogar der Collider der gehaltenen Spells verkleinert, um den Spielenden die nötige Genauigkeit bei der Platzierung der Spells zu geben.

\subsubsection{Balancing}

Außerdem halfen Playtests enorm beim Balancing des Spieles. Die ersten Settings, die bei beispielsweise der Geschwindigkeit der Snowboarder, oder der Menge an Mana eingestellt wurden waren immer falsch, und mussten immer leicht angepasst werden, um das Spiel möglichst fair zu halten. Auch das ist nur dank Playtests möglich.

\subsubsection{Positives}

Neben den Dingen, die nicht gut funktioniert haben, hat das Konzept der Krone, wie erwartet, sehr gut für Teamwork innerhalb der Gruppe gesorgt. Auch die Tricks und \emph{Positive Obstacles}\ref{_positive_obstacles} haben den gewünschten Effekt, riskantes fahren zu fördern erzieht. Die verschiedenen Spell-Typen wurden von den VR-Spielenden, wie erwartet, kombiniert, da sie gemeinsam einen größeren Effekt auf die Snowboarder hatten.

\section{Kommunikation}
[Hier fehlt noch etwas]

\section{Zukunft des Projektes}

Sollte das Projekt weiterentwickelt werden, ist vor allem mehr auf das Design der Interaktionen des VR-Spielenden zu achten. Derzeit sind alle Interaktionen sehr statisch und es sollten, um das Balancing der Spells für die Spielenden verständlicher zu machen [auch hier wieder internal attribution], komplexere Interaktionen (siehe \ref{_balancing_interaction}) implementiert werden. Leider waren die dafür benötigten Systeme zu komplex, um sie in der kurzen verbleibenden Zeit noch umzusetzen, und es wurde mehr Zeit in das Polishing von dem was bereits da war investiert.

Grundsätzlich war der Grundgedanke, so wenig wie möglich zu Implementieren, dafür mehr darauf zu achten das Implementierte möglichst zu perfektionieren, eine der besten Entscheidungen im Designprozess.


\trennseite{Farb- und Formpsychologie in asymmetrischen VR-Spielen}{Emma Wehinger}{Alexander Hager}{Mediendesign - Gamedesign}
\chapter{Überblick über die Materie}
\section{Virtuelle Realität}
Die Virtuelle Realität (Virtual Reality, VR) ist eine computergenerierte virtuelle Umgebung welche in Echtzeit berechnet wird. In dieser ist es möglich sich um zuschauen und je nach Anwendung kann man sich auch Bewegen und beliebig mit Objekten interagieren. Es gibt 2 verschiedene Möglichkeiten Virtual Reality zu erleben. Einerseits gibt es speziell angefertigte Räume, in welchen Großbildleinwände angebracht sind, andererseits gibt es auch Head-Mounted-Displays (VR Brillen,  die man aufsetzt), welche hier im Fokus stehen werden.


\subsection{Einführung in die Virtuelle Realität}
In der Virtuellen Realität gibt es eine breit gefächerte Reichweite an Spielen, in welchen man verschiedenste Situationen erleben kann. Anfangs waren Spiele für VR nur simple Testräume, in welchen man mit Objekten interagieren konnte. Mittlerweile ist es jedoch möglich riesige Spiele mit einer eigenen Geschichte zu erleben. Die Interaktionen, welche damals ein ganzes Spiel ausmachten, sind seit einiger Zeit nur noch die Basis, worauf neue Spiele aufbauen.

\subsection{Asymmetrische VR Spiele}
Die meisten VR Spiele kapseln einen von der Außenwelt ab. Sobald man die VR Brille aufsetzt befindet man sich in einer neuen, digitalen Welt. Eine Zeit lang war die zwischenmenschliche Interaktion in der Virtuellen Realität nicht möglich. Manche Spiele, welche etwas neuer sind, unterstützen dass VR Spieler über das Internet miteinander spielen können(Online Multiplayer). Somit braucht man jedoch 2 HMDs(VR Brillen), 2 PCs und man kann in der Regel nicht nebeneinander spielen. Asymmetrische VR Spiele versuchen, reguläre PC Spiele mit VR Spielen zu kombinieren. So können z.B. zwei Freunde mit einem PC und einer VR Brille miteinander spielen. Das funktioniert z.B., indem die zwei Spieler sich in der selben Spielwelt befinden, jedoch unterschiedliche Charaktere steuern. Durch die verschiedenen Steuerungsmöglichkeiten (VR Controller, Tastatur und Maus, Gamepad) ergibt sich oft automatisch eine bestimme Rollenverteilung. So könnte der PC Spieler z.B. einen Menschen spielen, welcher durch ein Level manövrieren muss und zeitgleich steuert der VR Spieler einen Riesen, welcher den Menschen behindern muss, indem er ihm z.B. Steine in den Weg legt.  

\subsection{Verfügbare Brillen}
Es gibt mittlerweile viele verschiedene Anbieter für VR Brillen, welche verschieden gute HMDs in verschiedenen Preisreichweiten anbieten.
Das Steam Hardware Survey zeigt, dass momentane Marktführer Facebooks Meta (ehemals Oculus) mit der Oculus Quest und der Oculus Quest 2 im Low-Budget Bereich und Valve mit der Valve Index im High-Budget Bereich sind.

\vspace{0.75cm}
\includegraphics{images/Steam_HardwareSurvey}

\section{Existierende Spiele}
Es gibt viele verschiedene VR Spiele und eine Breite Reichweite an Genres. Besonders relevant sind jedoch die asymmetrischen Spiele. Einige wichtige Beispiele sind hier aufgelistet.

\subsection{Keep Talking and Nobody Explodes}
YOU’RE ALONE
IN A ROOM WITH A BOMB.
Your friends have the info you need to defuse it.

But there’s a catch. They can’t see the bomb. So everyone will need to talk it out–fast!

\subsection{Davigo}
DAVIGO is a VR vs. PC "cross-reality" battle game. The VR player embodies a giant and faces off against one or more PC warriors in fast-paced, explosive combat!

\subsection{Carly and the Reaperman}
Step into the recently-dead shoes of Carly as you run and jump your way through incredible platforming challenges — or take up the mantle of the Reaperman and change the environment around you in real-time to give Carly new places to explore, avoid dangerous obstacles, and help her to achieve the impossible.

\section{Erwartete Probleme}
Dieses Dokument ist als vorwiegend technische Starthilfe für das
Erstellen einer Masterarbeit (oder Bachelorarbeit) mit \latex
gedacht und ist die Weiterentwicklung einer früheren
Vorlage\footnote{Nicht mehr verfügbar.} für das Arbeiten mit
Microsoft \emph{Word}. Während ursprünglich daran gedacht war, die
bestehende Vorlage einfach in \latex zu übernehmen, wurde rasch
klar, dass allein aufgrund der großen Unterschiede zum Arbeiten
mit \emph{Word} ein gänzlich anderer Ansatz notwendig wurde. Dazu
kamen zahlreiche Erfahrungen mit Diplomarbeiten in den
nachfolgenden Jahren, die zu einigen zusätzlichen Hinweisen Anlass gaben.

\subsection{Wenige Bilder pro Sekunde}
Zitat \cite{bobsch 123 bobsch}

\subsection{Bewegungskrankheit}
Zitat \cite{bobsch 123 bobsch}

\subsection{Platz und Bewegung}
Zitat \cite{bobsch 123 bobsch}

\subsection{Interaktionen}
Zitat \cite{bobsch 123 bobsch}
\chapter{Möglichkeiten zur Implementierung}

\section{Nutzung von Farben in Spielen}
Die Nutzung von Farben in Videospielen war mit ihrer Entstehung sehr mit dem damaligen Stand der Technik verbunden. 1972 wurde die Farbüberlagerung erfunden. Diese hat es ermöglicht, dass Videospiele in Farbe dargestellt werden können. Davor konnten Spiele nur in Schwarz-Weiß oder Monochrom dargestellt werden. Ab Ende der 1980er sind fast alle Videospiele in Farbe. 1985 veröffentlichte „Sega“ das Sega Master System, das Spiele mit 32 verschiedenen Farben darstellen konnte. Schon 9 Jahre später veröffentlicht „Sony“ die Playstation, die es ermöglicht, Spiele mit rund 16.7 Millionen Farben darzustellen. Heutzutage können Spiele mit über 16.7 Milliarden verschiedenen Farben dargestellt werden.  
\cite{_the_video_game_explosion}
\cite{_farbkontraste_im_gaming}

\subsection{Beispiel – Outlast}
„Outlast“ ist ein Horror-Spiel, das 2013 für die Playstation 4, die Xbox One und den PC erschienen ist. Der Entwickler und Publisher ist „Red Barrels“. Der Journalist Miles Upshur erhält durch eine anonyme Quelle die Information, dass im Mount Massive Asylum unmenschliche Experimente an Patienten durchgeführt werden. Er geht dieser Spur nach und begibt sich auf den Weg zur Nervenklinik. Dort angekommen läuft alles ganz anders als erwartet. Es gibt keinen Weg nach draußen mehr und alle Insassen sind ausgebrochen. Ziel ist es, einen Weg aus der Nervenklinik zu finden und währenddessen alles mit einem Camcorder zu filmen. 
\cite{_outlast}
\cite{_farbkontraste_im_gaming}

\begin{figure}[H]
	\centering
	\includegraphics[width=10cm]{Outlast}
	\caption{"'Outlast"' In-Game Screenshot\cite{_drawing_basics_and_video_game_art}}
\end{figure}

In „Outlast“ wird die Atmosphäre und Stimmung des Spiels durch den Qualitätskontrast der Farben erzeugt. Der Qualitätskontrast wird durch die Farbqualität der Farbe erzeugt. Durch das Mischen mit Schwarz, Weiß oder Grau wird eine Farbe unrein. Die Farbe verliert als Folge an Leuchtkraft und Qualität. Farben, die mit Grau gemischt werden, wirken trüb. Wird eine Farbe mit Schwarz gemischt, so wirkt sie bedrohlicher. Wenn man eine Farbe mit Weiß mischt, so wird sie kälter. Wendet man einen Qualitätskontrast neben reinen Farben an, fallen sie stärker auf. 
\cite{_farbkontraste}
\cite{_outlast}
\cite{_farbkontraste_im_gaming}

In „Outlast“ wird dem Spielenden durch den Qualitätskontrast eine stark bedrohliche Umgebung vermittelt. Alles ist sehr düster und dunkel, erst wenn der Spielende den Camcorder benutzt, wird die Umgebung in ein blasses Grün getaucht. Die Farben wirken größtenteils sehr leblos, aber gleichzeitig bedrohlich. Die einzig gesättigte, reine Farbe ist Rot in Form von Blut überall in der Anstalt. 
\cite{_outlast}
\cite{_farbkontraste}
\cite{_farbkontraste_im_gaming}

\subsection{Beispiel – Journey}
„Journey“ ist ein Adventure-Spiel, das ohne jegliche Worte auskommt. Es wurde erstmals 2012 für die Playstation 3 veröffentlicht. Der Entwickler des Spiels ist „thatgamecompany“ und der Publisher „Sony Interactive Entertainment“ und „Annapurna Interactive“. Das Spiel startet in einer Wüste. In der Ferne liegt ein großer Berg, an dessen Gipfel ein Lichtstrahl in den Himmel ragt. Der Spielende spielt eine Figur in einer roten Robe. Auf dem Weg zum Berg kann der Spielende mit bestehender Internetverbindung auf einen anderen Spielenden treffen. Die Spielenden können nicht miteinander reden, aber sich dennoch gegenseitig helfen. Das Zeil von „Journey“ ist es, den in der Ferne liegenden Berg zu erreichen.
\cite{_drawing_basics_and_video_game_art}
\cite{_farbkontraste_im_gaming}

„Journey“ nutzt die Vorteile von Farbkontrasten, um bei den Spielenden Emotionen zu erzeugen. Das Spiel bedient sich am Warm-Kalt-Kontrast. Beim Warm-Kalt-Kontrast werden warme und kalte Farben einander gegenübergestellt. Warme und kalte Farben stehen sich im Farbkreis gegenüber und erzeugen deshalb Dissonanz. Die kalten Farben des Farbkreises sind Gelbgrün, Grün, Blaugrün, Blau, Blauviolett und Violett. Zu den warmen Farben des Farbkreises zählen Gelb, gelborange, Orange, Rotorange, Rot und Rotviolett.
\cite{_farbkontraste_im_gaming}
\cite{_drawing_basics_and_video_game_art}
\cite{_farbkontraste}

\begin{figure}[H]
	\centering
	\includegraphics[width=10cm]{Journey2}
	\caption{"'Journey"' In-Game Screenshot\cite{_drawing_basics_and_video_game_art}}
\end{figure}

In „Journey“ wechseln sich die Warm-Kalt-Kontraste immer wieder ab. Am Anfand des Spiels ist der Spielende in einer Wüste, nach der Wüste führt ihn seine Reise in eine kühle Höhle. Nach den warmen, sandigen Ruinen begibt sich der Spielende zum eiskalten Berg in der Ferne. Die Figur selbst trägt eine rote Robe die in der kühlen Höhle und am kalten Berg den Warm-Kalt-Kontrast nochmal verdeutlicht. 
\cite{_drawing_basics_and_video_game_art}
\cite{_farbkontraste_im_gaming}
\cite{_farbkontraste}

\section{Nutzung von Formen in Spielen}
Das Konzept und die Nutzung von Formen und Linien in Videospielen stammt aus der Natur. Objekte und Dinge die aus abgerundeten Formen bestehen empfinden Menschen als sicher. Kantige Objekte und Formen gelten als gefährlich. Der Kreis steht für Freundlichkeit und Positivität. Das Quadrat soll Vertrauen, Stabilität und Sicherheit kommunizieren. Das Dreieck steht für Gefahr und Aggressivität. Man kann diese Formen in einer Skala der Emotionen einordnen. Links ist die positivste Form, der Kreis und ganz rechts das Dreieck, die negativste Form. 
\cite{_drawing_basics_and_video_game_art}

In Videospielen bestehen die „guten“ Charaktere beziehungsweise die Helden aus vielen Kreisen und abgerundeten Linien. Oft enthalten ihre Designs auch Quadrate, um Sicherheit und Stabilität zu signalisieren. Die Antagonisten in Spielen haben meist sehr dominante Silhouetten, die aus Dreiecken oder sehr kantigen Linien bestehen. Weiters, ist es äußerst wichtig, Silhouetten aus großen Formen zu gestalten, um die Figur des Charakters herausstechen zu lassen.
\cite{_drawing_basics_and_video_game_art}

\begin{figure}[H]
	\centering
	\includegraphics[width=10cm]{CharacterShapes}
	\caption{Video-Game Character-Shapes\cite{_drawing_basics_and_video_game_art}}
\end{figure}

In Abbildung 12.3 kann man erkennen, dass in der linken Spalte die Protagonisten aus sehr abgerundeten Linien und Kreisen bestehen. Sie wirken auf den Betrachtenden sehr positiv. Auf der rechten Seite sind die Antagonisten, die aus Dreiecken und vielen kantigen Linien bestehen. Sie wirken auf den Betrachtenden sehr aggressiv und bedrohlich. Generell sind alle abgebildeten Charaktere aus markanten Formen gestaltet, damit sie möglichst gut herausstechen. 
\cite{_drawing_basics_and_video_game_art}

\subsection{Beispiel – Journey}
In „Journey“ wird auch eine positive, harmonische Stimmung durch die Nutzung von Dreiecken und kantigen Linien erzeugt. Der Charakter besteht aus einem Dreieck und sein Umfeld ebenso. Die Umgebung ist geprägt durch kantige Objekte und Formen und dreieckige Figuren. Durch die Nutzung von Dreiecken für Charakter und Umfeld wird Harmonie erzeugt. 
\cite{_drawing_basics_and_video_game_art}
\cite{_farbkontraste_im_gaming}

\begin{figure}[H]
	\centering
	\includegraphics[width=10cm]{images/wehinger/Journey1}
	\caption{"'Journey"' In-Game Screenshot \cite{_journey}}
\end{figure}

\subsection{Beispiel – Super Mario Galaxy}
„Super Mario Galaxy“ ist ein 3D-Plattformer, der 2007 für die Wii veröffentlicht wurde. Der Entwickler und Publisher des Spiels ist „Nintendo“. Der Protagonist Mario reist von Planet zu Planet, kämpft gegen Gegner, löst Rätsel und sammelt Items. Ziel des Spiels ist es, das Pilzkönigreich und Prinzessin Peach zu retten. 
\cite{_super_mario_galaxy}
\cite{_drawing_basics_and_video_game_art}

Mit Formen und Linien kann man bei den Spielenden Gefühle und Emotionen erzeugen. Je nachdem, welche Emotion beim Spielenden erzeugt werden soll, werden andere Charakter -und Environment-Shapes verwendet. Um Harmonie in einer Umgebung zu erzeugen, besteht der Charakter und sein Umfeld aus Kreisen und runden Linien. Harmonie kann aber auch durch Dreiecke und kantige Linien entstehen. Um Dissonanz zu erzeugen, besteht der Charakter aus Kreisen und gebogenen Linien oder Dreiecken und kantigen Linien und das Umfeld steht im Gegensatz dazu.
\cite{_drawing_basics_and_video_game_art}

\begin{figure}[H]
	\centering
	\includegraphics[width=10cm]{SuperMarioGalaxy}
	\caption{"'Super Mario Galaxy"' In-Game Screenshot\cite{_drawing_basics_and_video_game_art}}
\end{figure}

In „Super Mario Galaxy“ besteht auf den meisten Planeten Harmonie zwischen dem Charakter und seiner Umgebung. Mario besteht größtenteils aus Kreisen und runden, gebogenen Linien. Seine Umgebung besteht im Großen und Ganzen auch aus Kreisen und abgerundeten Linien. Diese Beziehung zwischen Charakter und Umgebung erzeugt Harmonie und löst bei den Spielenden positive Gefühle aus. 
\cite{_drawing_basics_and_video_game_art}




\chapter{Umsetzung der PC- und VR-Welt}
\section{Level Aufbau}
Jedes Level in \emph{Tricks ´n´ Treats} ist ähnlich aufgebaut. 
In der Mitte befindet sich der Berg, auf dem die Snowboardenden fahren, und rund herum befindet sich der Raum des VR-Spielenden. Die PC-Spielenden starten oben auf dem Berggipfel und fahren den Hang bis zum Ende der Piste hinab. Der VR-Spielende kann mit einem Joystick auf seinem Controller den Berg rotieren, um sich besser auf der Strecke zu orientieren und die Spielenden in Sicht zu behalten. Währenddessen kann er hinter sich bestimmte Zauberkugeln aufheben und auf den PC-Spielenden anwenden.


\begin{figure}[H]
	\centering
	\includegraphics[width=13cm]{images/buketits_room}
	\caption{Editor View eines unfertigen Levels, Die rote Kapsel stellt den VR-Charakter dar}
\end{figure}

\section{VR-Charakter} \label{simon_vrspieler}
Der VR-Charakter besteht aus mehreren Systemen, die zusammenarbeiten. In \emph{Tricks ´n´ Treats} sind die Kamera und Hände mit Hilfe der SteamVR Integration für Unity gemacht. Der VR-Charakter verwendet die physikalischen Hände und die "`Player"' Komponente, die beide von SteamVR bereitgestellt werden. Ein großes Problem unseres Spiels ist, dass der VR-Charakter, der physikalische Hände besitzt nicht richtig skaliert werden kann. Sobald dieser skaliert wird, funktionieren die Hand-Collider nicht mehr planmäßig. Die Lösung dieses Problems ist in \ref{simon_problems} erklärt.

\subsection{Zauber und Interaktionssystem}
Für die Basis des Interaktionssystems in \emph{Tricks ´n´ Treats}, dient das System, das SteamVR für Unity bereitstellt. Auf jedem Objekt, mit dem der VR-Spielende interagieren kann, befindet sich das "`Interactable"' Script, dass von SteamVR bereitgestellt ist. Unser eigenes System für die Zauberkugeln hakt sich in dieses Script ein und kann somit bestimmte Eigenschaften leichter implementieren. Somit wurde eine Hervorhebung für Objekte geschaffen mit denen interagiert werden kann und gleichzeitig wird auf die Eingabe des VR-Controllers reagiert. Wenn eine Zauberkugel mit einem Objekt kollidiert (während man sie in der Hand hält), wird dann mit dem, von Steam vorgegebenen, "`Velocity Estimator"' die Aufprallgeschwindigkeit ausgerechnet und je nach Geschwindigkeit ein Zauberspruch gewirkt.

\subsection{VR-Umgebung}
Die Form der VR-Umgebung basiert auf einem Kreis, bei dem sich der Berg in der Mitte befindet und der Spielbereich des VR-Spielenden auf einem Ring um den Berg.

\section{PC-Charakter}
\subsection{Charakter Größe}
Die PC-Charaktere sind sehr klein. Das kommt daher, dass der VR-Charakter nicht skaliert werden kann und daher die PC-Charaktere runter skaliert werden müssen. Das Iterieren der Werte für den Character Controller ist durch die geringe Größe etwas komplexer und ungewohnt. Dadurch, dass die PC-Charaktere so viel kleiner sind, muss auch die Schwerkraft anders eingestellt sein. Diese haben eine viel geringere Schwerkraft als die physikalischen Objekte, die sich im VR-Bereich befinden.

\subsection{Fähigkeiten}
Der PC-Character Controller kann lenken, springen und Tricks machen. Außerdem muss dieser auf bestimmte Zauber des VR-Charakters reagieren. Somit kann der PC-Charakter beispielsweise künstlich in eine Richtung geschleudert werden, wobei er auch eine Rotation bekommt. Die PC-Spielenden können dann, in der Luft, den jeweiligen Charakter austarieren und bei richtiger Ausführung wieder sicher landen. Sollte der Charakter auf dem Kopf oder in einem falschen Winkel landen, stürzt dieser und wird zu einem Ragdoll. (Physikalischer Körper, bei der jedes Körperteil eigene Kollisionen hat und somit einen realistischen Fall simuliert).
Der Charakter kann zusätzlich, um besser ausweichen zu können, vorwärts und rückwärts Saltos, mit denen er schnell seine Position ändern kann.


\section{Lösung der erwarteten Probleme}\label{simon_problems}

\subsection{Wenige Bilder pro Sekunde}
Um möglichst viele Fps (Bilder pro Sekunde/Frames per second) zu erreichen und das Spiel auch auf älteren Systemen spielbar zu machen, muss es gut optimiert sein. Die Möglichkeiten zur Rechenoptimierung aus \ref{simon_performance}, die verwendet wurden und die, die nicht implementierbar waren sind anbei zu sehen.

\subsection{Level of Detail}
Die Unity-Engine unterstützt Level of Detail per Camera. Das heißt, dass jede Kamera selbst berechnet, welche Meshqualität gerendert werden soll. In \emph{Tricks ´n´ Treats} wird diese Technik nicht verwendet, weil durch andere Optimierungen genug Rechenleistung gespart werden konnte. Sollte sich der Spielumfang in Zukunft ändern, wird es Essenziell sein Level of Detail zu implementieren.

\subsection{Culling}
Dadurch dass die PC und VR-Kamera gleichzeitig Rendern kann Culling nur begrenzt angewendet werden. Occlusion Culling könnte durch eigenen Code, der anhand des View Frustrums die Objekte prüft und dementsprechend cullt implementiert werden. Eine Frage, die vorab gestellt werden muss, ist jedoch: Entsteht dadurch eine nennenswerte Verbesserung? Man kann nur testen, ob der Code nicht mehr Rechenleistung braucht als der Rendering Prozess der zusätzlichen Objekte. Da durch andere Maßnahmen genug optimiert wurde, konnte die Zeit, um in diesem Bereich zu forschen anders verwendet werden.

\subsection{Collider}
 In \emph{Tricks ´n´ Treats} bestehen alle Objekte mit denen kollidiert werden kann aus simplen Collider Formen und komplexere Strukturen werden aus mehreren primitiven gebaut. 
 Für die Effizienz der Collider gilt Sphere>Capsule>Box>Mesh. Das heißt, dass ein Sphere Collider am effizientesten ist und ein Mesh Collider am ineffizientesten.
\begin{figure}[H]
	\centering
	\includegraphics[width=9cm]{images/buketits_correctCollider}
	\caption{Korrekte Collider}
\end{figure}

\subsection{Object Pooling}
In \emph{Tricks ´n´ Treats} gibt es keine Objekte, die gepoolt werden könnten, um Rechenleistung zu sparen. Somit gibt es im gesamten Projekt kein Pooling System.

\subsection{Drawcall Batching}
In \emph{Tricks ´n´ Treats} wird sowohl Static als auch Dynamic Batching angewendet. Um Dynamic zu batchen werden Meshes mit weniger Vertices verwendet, wobei alle dasselbe Material nutzen. Für Textur des Materials wird ein Textur Atlas verwendet, der alle Farben aller Meshes beinhaltet. Dies gilt für alle Meshes der Umgebung. Da die Spielenden wegen einiger Rahmenbedingungen des Dynamic Batching nicht gebatcht werden können, verwenden diese einen anderen Atlas und Material. Static ist alles, was sich aus der Sicht der PC-Spielenden nicht bewegt. Das heißt beispielsweise der Berg und die Objekte auf diesem. Hindernisse, die zerbrechen oder sich bewegen können jedoch nicht Static sein.

\subsection{Bewegungskrankheit und Rotationsproblem}
Um die Bewegungskrankheit (Motion Sickness) zu vermeiden wurden die in \ref{simon_motionsickness} angebrachten Ursachen dafür beachtet und zugehörige Systeme passend dazu entworfen und implementiert.
Wie in \ref{simon_vrspieler} angeführt wurde, kann man den VR-Spielenden nicht skalieren und sondern ihn nur drehen. Die Strecke wiederum kann skaliert aber nicht gedreht werden. Das heißt, dass also der VR-Spielenden in einem Ring um den Berg gedreht werden muss, um den Spielenden folgen zu können. Einige Menschen werden bei einer derartigen Drehung jedoch seekrank. Damit dem VR-Spielenden nicht übel oder schwindelig wird, muss diesem das Gefühl gegeben werden, dass er den Berg rotiert und nicht er sich um den Berg. Um das zu umgehen, wird der gesamte Raum des VR-Spielenden und die sich darin befindenden Lichter zusammen mit dem VR-Spielenden gedreht. Somit wirkt es für den VR-Spielenden so, als würde sich eigentlich nur der Berg drehen, da für ihn alles andere statisch wirkt.


\begin{figure}[h]
	\centering
	\includegraphics[width=9cm]{images/buketits_layout}
	\caption{Top Down Layout des VR-Raumes mit dem Berg in der Mitte, blaue Zone stellt rotierbaren Bereich dar}
\end{figure}



\subsection{Platz und Bewegung}
Um möglichst vielen Menschen das Spielen von \emph{Tricks ´n´ Treats} zu ermöglichen ist der Spielbereich so klein wie möglich, ohne ihn einengend zu gestalten. Das heißt, dass alles Wichtige in Griffreichweite ist und stationär erreicht werden kann. Personen, die mehr Platz zur Verfügung haben, können diesen jedoch auch nutzen. Abgesehen davon muss der VR-Spielende sich nicht im Ring um den Berg bewegen, da man den Berg augenscheinlich drehen kann.

\subsection{Interaktionen}
Die Interaktionen des VR-Spielenden müssen sich real anfühlen, um die Immersion nicht zu brechen. Deswegen werden in \emph{Tricks ´n´ Treats} physikalische Hände verwendet, die nicht durch Objekte durch transformieren können. Außerdem vibriert die jeweilige Hand etwas, sobald man versucht durch ein Objekt zu greifen. Damit wird das Gefühl der Berührung ersetzt, dass mit den meisten VR-Controllern nicht umsetzbar ist.

\subsubsection{Interaktion mit Knöpfen und Items}
Der VR-Spielende kann Zauberkugeln aufheben. Wenn diese aufgehoben werden, nimmt die virtuelle Hand eine greifende Pose ein. Um die Zauberkugel nun gegen die PC-Spielenden zu en, kann der VR-Spielende diese Kugel auf dem Berg platzieren, indem er diese auf die Oberfläche schlagt. Man kann die Interaktion damit vergleichen, ein Ei auf einen Tisch zu schlagen. Um mit Knöpfen zu interagieren, gibt es zwei Methoden. Die erste Methode ist, den Knopf so zu machen, dass man nur mit der Hand in die Nähe einer unsichtbaren Zone kommen muss, um den Knopf hineinzudrücken. Die zweite und auch immersivere Methode ist, einen physikalischen Knopf zu implementieren, der vom Gefühl, näher an einen realen Knopfdruck heranrankommt.

\trennseite{Performance-Optimierung von 3D Assets in VR-Spielen}{Ian Memic}{Alexander Hager}{Mediendesign - Gamedesign}
\chapter{Begriffsdefinitionen}

Bei "`Tricks 'n' Treats"' handelt es sich um ein asymmetrisches Couch-Party VR-Spiel. [Hier fehlt, denke ich, eine kurze Beschreibung von unserem Spiel.]

In diesem Kapitel werden die in "`Tricks 'n' Treats"' verwendeten Technologien erklärt und die grundlegenden Konzepte des Game-Designs die verwendet werden, um die  Erlebnisse der und die Interaktionen zwischen den Spielenden zu beeinflussen.

\section{VR trifft Couchparty-coop}

\subsection{Was ist Virtual Reality?}

Virtual Reality bezeichnet Bilder und Töne, die von einem Computer erzeugt werden und dem Benutzer, der mit Hilfe von Sensoren mit ihnen interagieren kann, fast real erscheinen\cite{_oxford_dict}. Sie ist trotz anderer Anwendungsbereiche, dank der Immersion und Interaktionsmöglichkeiten die sie bietet, besonders in der Gaming-Industrie relevant geworden\cite{_bitkom_vr}. Spielenden können neue Erlebnisse geboten werden, welche wiederum eigene Game-Design-Fragen aufwerfen die es zu beantworten gilt.

VR bringt neben der Interaktivität noch einen weiteren gewaltigen Vorteil: Intuition. Greifen, Umschauen, Bewegen, alles funktioniert wie man es erwartet. Gaming-Neulinge haben häufig Probleme die richtigen Tasten auf der Tastatur zu drücken und auch klassische Gamepads sind nicht optimal. In VR gibt es zum einen weniger Knöpfe an den Controllern, die gedrückt werden können und zum anderen können alltägliche Interaktionen, wie Umschauen / Orientieren und das in die Hand nehmen von Objekten, in VR unterbewusster durchgeführt werden und sind daher für Anfangende weniger frustrierend. Spielende verstehen in VR schneller wie das Spiel gesteuert wird und können sich mehr Gedanken darüber machen was das Ziel des Spiels ist\cite{_natural_interaction_in__augmented_reality_context}. Insbesondere für Party-Spiele (Siehe \ref{_party_games}) ist das schelle Verstehen von der Steuerung und der intuitiven Ausführung von grundlegenden Aktionen in Stresssituationen sehr wichtig.

[Bild: Greifen Tastatur-E vs VR-Grab (hl vs alyx)]

\subsection{Was ist Couchparty?\label{_party_games}}

Couchparty-Spiele, oft auch nur Party-Spiele oder Couch-Games, sind Computerspiele, die eine Gruppe (die sich meist untereinander kennt) gemeinsam spielt. Anders als bei klassischen Online-Multiplayer-Spielen werden diese zusammen in einem Raum gespielt. Sie sind jedoch nicht zu verwechseln mit local-multiplayer-games. LAN-Klassiker wie Counter Strike\footnote{Counter Strike ist eine Reihe von taktischen Multiplayer-Ego-Shootern [verbesserungswürdig]} oder Starcraft\footnote{[Hier fehlt noch eine Beschreibung von Starcraft]} werden meist nur von leidenschaftlichen Gamern gespielt, da sie Vorbereitung, Know-How und viel Ausrüstung benötigen. Das Alleinstellungsmerkmal von Party-Spielen ist, dass diese auf nur einem Bildschirm von einer Couch aus, daher der Name, gespielt werden können und nur minimale Ausrüstung, meistens mehrere Controller, wobei bei manchen auch schon das Handy als Controller fungieren kann, benötigt wird.

In Couchparty-Spielen müssen Spielende schnell verstehen worum es geht, daher stützen sich viele Spiele auf "`Zufall"'. Dadurch können Anfangende schnell mit den anderen mitkommen, auch wenn sie das Spiel noch nie zuvor gespielt haben. Eine weitere Möglichkeit, das Spiel auch für schlechtere Spieler zugänglich machen wäre die Verwendung von "`Virtual Skills"'\cite[S. 165]{_art_of_gamedesign}.

Es gibt grundsätzlich zwei verschiedene Arten wie Couchparty-Spiele den Bildschirm-Platz nutzen können.

\begin{itemize}
	
\item \subsubsection{Shared-Screen}

In diesem Fall sehen alle Spielende das gleiche und müssen sich daher auch im gleichen Raum aufhalten. Der Vorteil eines solchen Systems ist, dass keine Informationen mehrfach gezeigt werden müssen und der Platz des Bildschirms vor allem bei kleinen Laptop-Bildschirmen oder sogar Handys besser genutzt wird. Außerdem hat es einen Performance-Vorteil, da nicht mehrere Kameras geändert werden müssen. Leider, eignet sich diese Art nicht für jedes Spiel, da sich die Spielenden nicht zu weit voneinander entfernen können, vor allem 3D-Spiele mit einer freien Kamera sind hier sehr eingeschränkt. Besonders eignet sich Shared-Screen-Play für Arena-Games wie z.B. "`Super Smash Bros"' oder "`Overcooked"'.

\item \subsubsection{Split-Screen}

Ein Split-Screen System bietet hier einen großen Vorteil: Freiheit. Alle Spielenden können sich selbständig bewegen und umschauen, ohne das Erlebnis der anderen zu beeinflussen. Allerdings wird hierfür ein größerer Bildschirm benötigt, da jedem Spielenden nur ein halber oder nur ein viertel des Bildschirms zugewiesen wird.

\end{itemize}

\noindent Es gibt auch die Möglichkeit nahtlos zwischen den beiden Systemen zu wechseln je nachdem, welches sich gerade besser eignet, um das gewünschte Spielerlebnis zu erzeugen. Laut einer Reddit Umfrage präferieren Spielende Shared-Screen-Play. [Darf ich eine Reddit-Umfrage als Quelle]

\subsection{Was sind coop-games?}

Coop-games (= cooperative-games/Kooperative-Spiele) sind eine besondere Art von team-basierten-Spielen die besonders auf Gemeinschaft der Spielenden setzt und diese auch durch Game-Design voraussetzt. 

Ein Koalitions- oder Strategiespiel ist kooperativ, wenn die Spielenden verbindliche Vereinbarungen über die Verteilung der Auszahlungen oder die Wahl der Strategien treffen können, auch wenn diese Vereinbarungen nicht durch die Spielregeln spezifiziert oder impliziert sind\cite{_introduction_to_the_theory_of_cooperative_games}.

Bei klassischen Team-basierten-Spielen ist Kommunikation zwar wichtig jedoch nicht essentiell. Aufgaben der Spielenden werden in coop-games so verteilt, dass ein Ziel nicht ohne die Zusammenarbeit der anderen erreicht werden kann. Es entsteht hierbei eine interessante Dynamik [hier könntest du noch kurz beleuchten welche Auswirkungen diese Dynamik hat und vielleicht sogar, was dann passiert, wenn die Stakes größer werden (zB Schuldzuweisung)] in der die Spielenden komplett voneinander abhängig sind. Kooperation kann auch den Spaßfaktor eines Spieles fördern, mehr dazu in Kapitel \ref{_cooperative_play}.

Gute Beispiele hierfür sind einige Spiele der Lego-Reihe, Overcooked und It Takes Two welches ich in Kapitel \ref{_industrie} genauer beschreiben werde.

\subsubsection{Wie erzeugt man Kooperation}
Kooperation in, auch jene in Videospielen, kann zu tiefen Bindungen zwischen den Spielenden führen.
Mit der "`Lens of cooperation"'\cite[S. 311]{_gamemechanics_for_cooperative_games} kann ein Gamedesigner besser verstehen wie Kooperation entsteht und wie man sie fördert. Die wichtigsten Punkte, die Jesse Schell hier anspricht, sind die Notwendigkeit von Kommunikation oder die Tatsache, dass ein Task nur durch Zusammenarbeit der Spielenden gelöst werden kann. Ein weiterer interessanter Punkt ist Synergie und Antergie, die Veränderung der Stärke einer Gruppe, wenn ein team zusammenkommt, Synergie beschreibt, dass die Gruppe Stärker als die Summe der einzelnen ist, Antergie ist das Gegenteil davon. Synergie ist damit in kooperativem Umfeld bestrebenswert.

\subsection{Was bedeutet Asymmetrie?}

Asymmetrie beschreibt im Allgemeinen zwei Seiten oder Teile, die nicht die gleiche Größe oder Form haben\cite{_oxford_dict}.

In Spielen bezieht sich Asymmetrie auf das (Board-) Design. Die Spielenden haben in einem solchen Spiel ein hoch unterschiedliches action-set, das bedeutet sie haben einen anderen Einfluss auf die Spielwelt/andere Mitspieler. In Spielen mit asymmetrischen Wahlmöglichkeiten beginnt jeder Spielende jedoch typischerweise mit einer Reihe von Aktionen, die sich stark von denen der anderen unterscheiden, was das "`Balancing"' dieser Spiele besonders schwierig macht. Die Herausforderung besteht vor allem darin, den relativen Einfluss einer Aktion auf die Gewinnwahrscheinlichkeit zu bestimmen\cite[S. 18]{_balancing_asymmetric_video_games}.


% falls es englische Arbeiten gibt, können englische Bereiche der Arbeit mit \begin{english} und \end{english} umschlossen werden.
%\begin{english}
%\trennseite{Collection and analysis of gameplay metrics}{Ian Hornik}{Mag. Andreas Metz}{Media design - Game design}
%\include{hornik/hornik}
%\end{english}

%\trennseite{Titel der Arbeit B}{SchuelerIn B}{Titel + Name Betreuer von Arbeit B}{Mediendesign - Gamedesign}
%\chapter{Einleitung}
\label{cha:sa_Einleitung}

\section{Zielsetzung}
Dieses Dokument ist als vorwiegend technische Starthilfe für das
Erstellen einer Masterarbeit (oder Bachelorarbeit) mit \latex
gedacht und ist die Weiterentwicklung einer früheren
Vorlage\footnote{Nicht mehr verfügbar.} für das Arbeiten mit
Microsoft \emph{Word}. Während ursprünglich daran gedacht war, die
bestehende Vorlage einfach in \latex zu übernehmen, wurde rasch
klar, dass allein aufgrund der großen Unterschiede zum Arbeiten
mit \emph{Word} ein gänzlich anderer Ansatz notwendig wurde. Dazu
kamen zahlreiche Erfahrungen mit Diplomarbeiten in den
nachfolgenden Jahren, die zu einigen zusätzlichen Hinweisen Anlass gaben.


\section{Warum {\latex}?}

Diplomarbeiten, Dissertationen und Bücher im
technisch-natur\-wissen\-schaft\-lichen Bereich werden
traditionell mithilfe des Textverarbeitungssystems \latex
\cite{Lamport94,Lamport95} gesetzt. Das hat gute Gründe, denn
\latex ist bzgl.\ der Qualität des Druckbilds, des Umgangs mit
mathematischen Elementen, Literaturverzeichnissen etc.\
unübertroffen und ist noch dazu frei verfügbar. Wer mit \latex
bereits vertraut ist, sollte es auch für die Diplomarbeit
unbedingt in Betracht ziehen, aber auch für den Anfänger sollte
sich die zusätzliche Mühe am Ende durchaus lohnen.

Für den professionellen elektronischen Buchsatz wurde früher
häufig \emph{Adobe Framemaker} verwendet, allerdings ist diese
Software teuer und komplex. Eine modernere Alternative dazu ist
\emph{Adobe InDesign}, wobei allerdings die Erstellung
mathematischer Elemente und die Verwaltung von Literaturverweisen
zur Zeit nur rudimentär unterstützt werden.%
\footnote{Angeblich werden aber für den (sehr sauberen) Schriftsatz 
in \emph{InDesign} ähnliche Algorithmen wie in \latex\ verwendet.}

Microsoft \emph{Word} gilt im Unterschied zu \latex, 
\emph{Framemaker} und \emph{InDesign} übrigens nicht als professionelle
Textverarbeitungssoftware, obwohl es immer häufiger auch von
großen Verlagen verwendet wird.%
\footnote{Siehe auch \url{http://latex.tugraz.at/mythen.php}.}
Das Schriftbild in \emph{Word}
lässt -- zumindest für das geschulte Auge -- einiges zu wünschen
übrig und das Erstellen von Büchern und ähnlich großen Dokumenten
wird nur unzureichend unterstützt. Allerdings ist \emph{Word} sehr
verbreitet, flexibel und vielen Benutzern zumindest oberflächlich
vertraut, sodass das Erlernen eines speziellen Werkzeugs wie
\latex\ ausschließlich für das Erstellen einer Diplomarbeit
manchen verständlicherweise zu mühevoll ist. Man sollte es daher
niemandem übel nehmen, wenn er/sie sich auch bei der Diplomarbeit
auf \emph{Word} verlässt. Im Endeffekt lässt sich mit etwas
Sorgfalt (und ein paar Tricks) auch damit ein durchaus akzeptables
Ergebnis erzielen. 
Für alle, die so denken, finden sich in
%Kap.~\ref{chap:Word} einige spezielle Hinweise zum Arbeiten mit
%\emph{Word}. 
Ansonsten sollten auch für \emph{Word}-Benutzer 
einige Teile dieses Dokuments von Interesse sein, insbesondere die
Abschnitte über Abbildungen und Tabellen und mathematische Elemente.

Übrigens, genau hier am Ende des Einleitungskapitels (und nicht
etwa in der Kurzfassung) ist der richtige Platz, um die
inhaltliche Gliederung der nachfolgenden Arbeit zu beschreiben.
Hier soll dargestellt werden, welche Teile (Kapitel) der Arbeit
welche Funktion haben und wie sie inhaltlich zusammenhängen. Auch
die Inhalte des \emph{Anhangs} -- sofern vorgesehen -- sollten hier
kurz beschrieben werden.


%\chapter{Abbildungen, Tabellen, Quellcode}
\label{chap:Abbildungen}

\section{Allgemeines}

Abbildungen (\emph{figures}) und Tabellen (\emph{tables}) werden üblicherweise
zusammen mit einem nummerierten Titel (\emph{caption}) zentriert
angeordnet (siehe Abb.~\ref{fig:CocaCola}).
Im Text \emph{muss} es zu jeder Abbildung einen Verweis geben und die eigentliche Abbildung
sollte erst \emph{nach} dem ersten Verweis platziert werden.

\begin{figure}
\centering
\includegraphics[width=.95\textwidth]{cola-public-domain-photo-p} %{CS0031}
\caption{Coca-Cola Werbung 1940 \cite{CocaCola1940}.}
\label{fig:CocaCola}
\end{figure}



\section{\emph{Let Them Float!}}

Das Platzieren von Abbildungen und Tabellen gehört zu den
schwierigsten Aufgaben im Schriftsatz, weil diese meist viel Platz
benötigen und häufig nicht auf der aktuellen Seite im laufenden
Text untergebracht werden können. Diese Elemente müssen daher an
eine geeignete Stelle auf nachfolgenden Seiten verschoben werden,
was manuell sehr mühsam (jedoch in \emph{Word} beispielsweise unerlässlich) ist.

In \latex funktioniert das weitgehend automatisch, indem
Abbildungen, Tabellen und ähnliche als "`Floating Bodies"'
behandelt werden. Bei der Positionierung dieser Elemente wird
versucht, einerseits im Textfluss möglichst wenig Leer\-raum
entstehen zu lassen und andererseits die Abbildungen und Tabellen
nicht zu weit von der ursprünglichen Textstelle zu entfernen.

Der Gedanke, dass etwa Abbildungen kaum jemals genau an der
ge\-wünsch\-ten Stelle und möglicherweise nicht einmal auf
derselben Seite Platz finden, ist für viele Anfänger aber offenbar sehr
ungewohnt oder sogar beängstigend. Dennoch sollte man zunächst einmal
getrost \latex\ diese Arbeit überlassen und \emph{nicht} manuell
eingreifen. Erst am Ende, wenn das gesamte Dokument "`steht"' und
man mit der automatischen Platzierung wirklich nicht zurande
kommt, sollte man (durch gezielte Platzierungsanweisungen
\cite[S.~33]{Oetiker01}) \textbf{in Einzelfällen} eingreifen.



\section{Captions}

Bei Abbildungen steht der Titel üblicherweise \emph{unten}, bei
Tabellen hingegen -- je nach Konvention -- \emph{oben} (wie in diesem Dokument) 
oder ebenfalls \emph{unten}. In \latex\ erfolgt
auch die Nummerierung der Abbildungen automatisch, ebenso der
Eintrag in das (optionale)
Abbildungsverzeichnis%
\footnote{Ein eigenes Verzeichnis der Abbildungen am Anfang des Dokuments
ist zwar leicht erstellt, in einer Diplomarbeit aber (und eigentlich
überall sonst auch) überflüssig. Man sollte es daher weglassen.}
am Beginn des Dokuments.

Die Markierung der Captions%
\footnote{Ausnahmsweise wird das Wort "`Caption"' im Folgenden
ohne deutsche Übersetzung verwendet.} erfolgt in \latex mithilfe
der \verb!\label{}! Anweisung, die unmittelbar auf die
\verb!\caption{}! Anweisung folgen muss:
%
\begin{LaTeXCode}
\begin{figure}
\centering
\includegraphics[width=.95\textwidth]{cola-public-domain-photo-p}
\caption{Coca-Cola Werbung 1940 \cite{CocaCola1940}.}
\label{fig:CocaCola}
\end{figure}
\end{LaTeXCode}
%
Der Name des Labels (\texttt{fig:CocaCola}) kann beliebig gewählt werden. 
Die Kennzeichnung \texttt{fig:} ist (wie in Abschn.\ \ref{sec:querverweise} 
erwähnt) nur eine nützliche Hilfe, um beim Schreiben verschiedene Arten 
von Labels besser unterscheiden zu können.

Die Länge der Captions kann dabei sehr unterschiedlich sein. Je
nach Anwendung und Stil ergibt sich manchmal eine sehr kurze
Caption (Abb.~\ref{fig:CocaCola}) oder eine längere
(Abb.~\ref{fig:ibm360}).
Man beachte, wie bei kurzen Captions ein
zentrierter Satz und bei langen Captions ein Blocksatz verwendet
wird (\latex macht das automatisch).
Captions sollten \emph{immer} mit einem Punkt abgeschlossen sein.%
\footnote{Kurioserweise verlangen manche Anleitungen
genau das Gegenteil, angeblich, weil beim klassischen Bleisatz 
die abschließenden Punkte im Druck häufig "`weggebrochen"' sind. 
Das kann man glauben oder nicht, im Digitaldruck 
spielt es jedenfalls keine Rolle.}

\begin{figure}
\centering
\FramePic{\includegraphics[width=.85\textwidth]{ibm-360-color}}  %{CS1065}}
\caption{Beispiel für einen langen Caption-Text. \textsc{Univac}
brachte 1961 mit dem Modell 751 den ersten Hochleistungsrechner
mit Halbleiterspeicher auf den Markt. Von diesem Computer wurden
in den U.S.A.\ bereits im ersten Produktionsjahr über fünfzig
Exemplare verkauft, vorwiegend an militärische Dienststellen,
Versicherungen und Großbanken. Die Ablöse erfolgte zwei Jahre
später durch das zusammen mit \textsc{Sperry} entwickelte Modell 820.
Das klingt vielleicht plausibel, ist aber völliger Unsinn, denn das
Bild zeigt in Wirklichkeit eine System/360 Anlage von IBM. 
Bildquelle~\cite{IBM360}.} 
\label{fig:ibm360}
\end{figure}





\section{Abbildungen}

Für die Einbindung von Grafiken in \latex wird die Verwendung des Stan\-dard-Pakets
\texttt{graphicx} \cite{Carlisle99} empfohlen 
(wird durch das \texttt{hagenberg}-Paket bereits eingebunden). 
Mit dem aktuell verwendeten Workflow (\texttt{pdflatex})
können Bild- bzw.\ Grafikformate ausschließlich 
in folgenden Formaten eingebunden werden:
%
\begin{itemize}
	\item \textbf{PNG}: für Grau-, S/W- und Farb-Rasterbilder (bevorzugt),
	\item \textbf{JPEG}: für Fotos (wenn nicht anders vorhanden),
	\item \textbf{PDF}: für Vektorgrafiken (Illustrationen, Strichzeichnungen etc.).
\end{itemize}
%
Bei Rasterbildern sollte wenn möglich PNG verwendet werden, weil die darin 
enthaltenen Bilder verlustfrei komprimiert sind und daher keine sichtbaren Kompressionsartefakte
aufweisen. Im Gegensatz dazu sollte man JPEG nur dann verwenden, wenn das Originalmaterial
(Foto) bereits in dieser Form vorliegt.


\subsection{Wo liegen die Grafikdateien?} 

Die Bilder werden üblicherweise in einem Unterverzeichnis (oder in mehreren Unterverzeichnissen) abgelegt,
im Fall dieses Dokuments in \nolinkurl{images/}.
Dazu dient die folgende Anweisung
am Beginn des Hauptdokuments \nolinkurl{_DaBa.tex} (\sa\ Anhang \ref{app:latex}):
%
\begin{quote}
\verb!\graphicspath{{images/}}!
\end{quote}
%
Der (zum Hauptdokument relative) Pfad \texttt{graphicspath} kann innerhalb des
Dokuments jederzeit geändert werden, was durchaus nützlich ist, wenn man
\zB\ die Grafiken einzelner Kapitel getrennt in entsprechenden Verzeichnissen
ablegen möchte.
Die Größe der Abbildung im Druck kann durch Vorgabe einer bestimmten
Breite oder Höhe oder eines Skalierungsfaktors gesteuert werden, {\zB}:
%
\begin{quote}
\verb!\includegraphics[width=.85\textwidth]{ibm-360-color}! \\
\verb!\includegraphics[scale=1.5]{ibm-360-color}!
\end{quote}
%
Man beachte, dass dabei die Dateiendung nicht explizit angegeben werden muss. 
Das ist \va\ dann praktisch, wenn man verschiedene Workflows mit jeweils
unterschiedlichen Dateitypen verwendet.


\subsection{Grafiken einrahmen} 

Mit dem Makro \verb!\FramePic{}! (definiert in \texttt{hgb.sty}) kann man optional einen dünnen 
Rahmen rund um die Grafik erzeugen, \zB:
%
\begin{quote}
\verb!\FramePic{\includegraphics[height=50mm]{ibm-360-color}}!
\end{quote}
%
Das wird man üblicherweise nur bei Rasterbildern tun, insbesondere wenn sie zum Rand hin sehr hell sind
und ohne Rahmen nicht vom Hintergrund abgrenzbar wären.

\subsection{Rasterbilder (Pixelgrafiken)}

Generell sollte man Bilder bereits vorher so aufbereiten,
dass sie später beim Druck möglichst wenig an Qualität verlieren.
Es empfiehlt sich daher, die Bildgröße (Auflösung) bereits im Vorhinein
(\zB mit \emph{Photoshop})
richtig einzustellen.
Brauchbare Auflösungen bezogen auf die endgültige Bildgröße sind:
%
\begin{itemize}
  \item \textbf{Farb- und Grauwertbilder:} 150--300 dpi
  \item \textbf{Binärbilder (Schwarz/Weiß):} 300--600 dpi
\end{itemize}
%
Eine wesentlich höhere Auflösung macht aufgrund der beim Laserdruck notwendigen
Rasterung keinen Sinn, auch bei 1200 dpi-Druckern.
Speziell \emph{Screen\-shots} sollte man nicht zu klein darstellen,
da sie sonst schlecht lesbar sind (max.\ 200 dpi, besser 150 dpi).
Dabei ist zu bedenken, dass die Arbeit auch als Kopie in allen
Details noch gut lesbar sein sollte.

\subsubsection{JPEG-Problematik}

In der Regel sollte man Bilder, die für den Einsatz in
Druckdokumenten gedacht sind, nicht mit verlustbehafteten
Kompressionsverfahren abspeichern. Insbesondere sollte man die Verwendung
von JPEG möglichst vermeiden, auch wenn viele Dateien dadurch
wesentlich kleiner werden. 
Eine Ausnahme ist, wenn die Originaldaten nur in JPEG vorliegen und für die 
Einbindung nicht bearbeitet oder verkleinert wurden. Ansonsten sollte man immer
PNG verwenden.

Besonders gerne werden farbige \textbf{Screenshots} einer JPEG-Kompression%
\footnote{Das JPEG-Verfahren ist für natürliche Fotos konzipiert und dafür auch gut geeignet,
seine undifferenzierte Verwendung ist aber zu einer globalen Plage geworden.}
unterzogen, obwohl deren verheerende Folgen auch für jeden Laien sichtbar sein sollten
(Abb.~\ref{fig:jpeg-pfusch}).

\begin{figure}
\centering\small
\begin{tabular}{cc}
\FramePic{\includegraphics[width=0.45\textwidth]{screenshot-dirty}} &		% JPEG file
\FramePic{\includegraphics[width=0.45\textwidth]{screenshot-clean}} \\	% PNG file
(a) & (b) 
\end{tabular}
\caption{Typischer JPEG-Pfusch. Screenshots und ähnliche im Original
verfügbare Rasterbilder sollten für Druckdokumente \emph{keinesfalls} mit
JPEG komprimiert werden. Das Ergebnis~(a) sieht gegenüber dem
unkomprimierten Original~(b) nicht nur schmutzig aus, sondern wird
im Druck auch schnell unleserlich.} 
\label{fig:jpeg-pfusch}
\end{figure}






\subsection{Vektorgrafiken}

Für schematische Abbildungen (\zB Flussdiagramme, Entity-Relationship-Diagramme
oder sonstige strukturelle Darstellungen) sollte man unbedingt
Vektorgrafiken (PDF) verwenden. % (\zB Abb.~\ref{fig:latex-pdf-workflow}).
Gerasterte Grafiken, wie man sie üblicherweise als GIF- oder PNG-Dateien
auf Webseiten findet, haben in einem Druckdokument nichts zu suchen, notfalls
muss man sie mit einem entsprechenden Werkzeug \emph{neu} zeichnen (natürlich
unter Angabe der ursprünglichen Quelle).

In diesem Fall kommt als Datenformat nur PDF %(oder EPS im DVI-PS-Workflow) 
in Frage,
dieses bietet sich aber auch in anderen Umgebungen als universelles
Vektor-Format an.
Zur Erstellung von PDF-Vektorgrafiken benötigt man ein geeignetes
Grafikprogramm, \zB\ \emph{Freehand} %von \emph{Macromedia}
oder \emph{Illustrator} von \emph{Adobe}.
Manche gängige Grafikprogramme 
unterstützen allerdings keinen direkten Export von PDF-Dateien
oder erzeugen unsaubere Dateien. Vor der Entscheidung
für eine bestimmte Zeichensoftware sollte man das im Zweifelsfall
ausprobieren.
PDF kann im Notfall über einen entsprechenden Druckertreiber erzeugt werden.




\subsubsection{Einbettung von Schriften}

Die Wiedergabe von Textelementen ist abhängig von der auf dem
Computer (oder Drucker) installierten Schriften und der Form der
Schrifteinbettung im Quelldokument. Die korrekte Darstellung am
Bildschirm eines Computers bedeutet nicht, dass dasselbe Dokument
auf einem anderen Computer oder Drucker genau so dargestellt wird.
Dieser Umstand ist besonders wichtig, wenn Druckdokumente online
zur Verfügung gestellt werden. Kontrollieren Sie daher genau, ob
die innerhalb Ihrer Grafiken verwendeten Schriften auch exakt wie
beabsichtigt im Ausdruck aufscheinen.


\subsubsection{Strichstärken -- \emph{Hairlines} vermeiden!}

In Grafik-Programmen wie \emph{Freehand} und \emph{Illustrator},
die sich im Wesentlichen an der \emph{PostScript}-Funktionalität
orientieren, ist es möglich, Linien bzgl.\ ihrer Stärke als
"`Hairline"' zu definieren. Im zugehörigen \emph{PostScript}-Code
wird dies als \texttt{linewidth} mit dem Wert \texttt{0} ausgedrückt und
sollte am Ausgabegerät "`möglichst dünne"' Linien ergeben. 
Das Ergebnis ist ausschließlich vom jeweiligen Drucker
abhängig und somit kaum vorhersagbar.
\textbf{Fazit:} Hairlines vermeiden und stattdessen immer konkrete
Strichstärken ($\geq 0.25\,\mathrm{pt}$) einstellen!


\begin{comment}

\subsection{Erzeugung von EPS-Dateien (nur im DVI-PS-Work\-flow relevant)}

\subsubsection{EPS-Bilder }

Im DVI-PS-Workflow können -- wie oben erwähnt -- nur
EPS-Grafi\-ken eingebunden werden. Da EPS für \emph{Photoshop} ein
Standardformat ist, lässt sich damit besonders einfach arbeiten.
Dabei ist allerdings zu beachten, dass zur Verwendung in \latex
die Bilder \emph{ohne} Voransicht (\emph{Preview}) gespeichert
werden müssen (Abb.~\ref{fig:photoshop-eps-screen}). 
%
\begin{figure}
\centering
\includegraphics[scale=1.5]{photoshop-eps-screen}
\caption{Speichern von Bildern als EPS-Dateien in \emph{Photoshop}.
Dabei ist unbedingt zu beachten, dass ohne Preview
(\emph{Preview-None}) gespeichert wird.
Sowohl ASCII- wie auch Binärkodierung funktionieren mit \latex.
%, die Anzeige der EPS-Grafiken im DVI-Viewer (mit \emph{GhostScript}) funktioniert aber nur bei ASCII-Kodierung!
}
\label{fig:photoshop-eps-screen}
\end{figure}
%
Binärkodierte
EPS-Dateien sind normalerweise kein Problem. Nicht alle
Bildbearbeitungsprogramme bieten allerdings die Möglichkeit,
direkt im EPS-Format zu exportieren.


\subsubsection{EPS-Export aus \emph{Adobe Illustrator} oder \emph{Freehand}}

Auch mit professioneller Software ergeben sich in diesem
Zusammenhang -- zum Teil wegen der Vielzahl an
Einstellmöglichkeiten -- immer wieder Probleme. Speziell daher zum
Export von EPS-Grafiken aus \emph{Freehand} einige Tips %
(\sa\ Abb.~\ref{fig:freehand-export-screen-setup}):
%
\begin{itemize}
\item Exportieren Sie nur die ausgewählten Grafik-Elemente
("`Selected objects only"'). %
\item Exportieren Sie \emph{ohne} Vorschau ("`Preview None"'). %
\item Vermeiden Sie die Farbkonvertierung auf CMYK -- sie kann beim
Druck zu störender Rasterung und ungenügenden Schwarzwerten füh\-ren.
\item Binden Sie die Schriften der Grafik in das EPS-Dokument ein
("`Include Fonts in EPS"' im Setup-Menü). Falls das nicht möglich
ist (bestimmte Schriften sind aus rechtlichen Gründen gesperrt),
sollten alle Texte vor dem Export in Zeichenpfade aufgelöst werden
("`Convert text to paths"').
\end{itemize}
%
Ähnliches gilt auch für den EPS-Export aus \emph{Adobe Illustrator}.

\begin{figure}
\centering
\includegraphics[scale=1.5]{freehand-export-screen-setup}
\caption{Exportieren von Vektor-Grafik als EPS-Datei in {\em
Freehand}. Das kleinere Fenster zeigt das geöffnete Setup-Menü.
Wichtig ist, dass beim Export sicherheitshalber die Schriften in
das EPS-Dokument eingebunden werden (\emph{Include Fonts in
EPS})! Die Ausgabe in RGB-Farben widerspricht zwar der üblichen Konvention, 
ist aber dennoch zu empfehlen, da bei CMYK-Konversion ungenügende Schwarzwerte 
beim S/W-Druck auftreten können.} \label{fig:freehand-export-screen-setup}
\end{figure}

\end{comment}


\subsection{\tex-Schriften auch in Grafiken?}
\label{sec:tex-schriften-in-grafiken}

Während man sich bei Abbildungen, die mit externen
Grafik-Programmen erzeugt werden, meist mit ähnlich aussehenden
Schriften (wie \emph{Times-Roman} oder \emph{Garamond}) abhilft,
besteht bei Puristen oft der verständliche Wunsch, die 
\emph{Computer-Modern} (CM) Schriftfamilie von {\tex}/{\latex} auch
innerhalb von eingebetteten Grafiken einzusetzen.

\subsubsection{\emph{BaKoMa}-Schriften (TrueType)}

Glücklicherweise stehen einige Portierungen von CM als {\em
TrueType}-Schriften zur Verfügung, die man auch in herkömmlichen
DTP-Anwendungen unter \emph{Windows} und \emph{Mac~OS} verwenden
kann. Empfehlenswert ist \zB\ die \emph{BaKoMa Fonts
Collection}\footnote{Von Basil K.\ Malyshev -- die BaKoMa-Fonts
liegen dieser Vorlage bei, ansonsten findet man sie \zB\ unter
\url{www.ctan.org/tex-archive/fonts/cm/ps-type1/bakoma/}.}, die
neben den CM-Standardschriften auch die mathematischen Schriften
der AMS-Familie ent\-hält und zudem kostenfrei ist. Natürlich
müssen die TrueType Schriften vor der Verwendung zunächst auf dem
eigenen PC installiert werden. 
%Ein Beispiel für eine in {\em
%Freehand} erzeugte Grafik mit CM-Schriften findet sich in
%Abb.~\ref{fig:latex-pdf-workflow} (Seite
%\pageref{fig:latex-pdf-workflow}).

\subsubsection{\emph{Latin Modern Roman} Fonts (OpenType)}

Eine Alternative dazu sind die  "`LM-Roman"'%
\footnote{\url{http://www.gust.org.pl/projects/e-foundry/latin-modern}}
 Open-Type Schriften, die speziell für die Verwendung im Umfeld von \latex\ entwickelt wurden.
Sie sind auch Teil der MikTeX-Installation.%
\footnote{\url{C:/Program Files (x86)/MikTeX 2.9/fonts/opentype/public/lm/}}
Diese Schriften enthalten \ua\ Zeichen mit Umlauten und sind daher auch für deutsche Texte recht
bequem zu verwenden.

\begin{comment}

\subsubsection{Für Gourmets: Ersetzung von Symbolen mit dem \texttt{psfrag}-Paket}
\label{sec:psfrag}

Eine interessante Alternative zur Verwendung gesonderter Schriften ist die Verwendung des \texttt{psfrag}-Paket,
mit dem Zeichenketten in EPS-Dateien \emph{nachträglich} durch beliebige \latex-Elemente ersetzt werden können. 
Das ist allerdings \textbf{nur im DVI-PS-Workflow} möglich!
Auf diesem Weg kann das Schriftbild in Abbildungen perfekt dem übrigen Text angepasst werden, was besonders für Illustrationen und Diagramme mit mathematischen Beschriftungen interessant ist. Ein Beispiel dazu ist in 
Abb.~\ref{fig:psfrag-example} gezeigt, Details finden sich im zugehörigen \latex-Code.

\texttt{psfrag} funktioniert allerdings nicht mit allen EPS-Dateien. Weitgehend problemlos sind Dateien, die aus \emph{Macromedia Freehand}, \emph{Adobe Illustrator} oder \emph{Mathematica} exportiert wurden. Schwierig\-keiten gibt es hingegen erfahrungsgemäß mit \emph{Microsoft Visio}, \emph{PowerPoint} und \emph{Excel}, aber auch mit Grafiken aus \emph{Canvas}.

%
%\begin{quote}
%\begin{sloppypar}
%\textbf{Achtung:} Die Ersetzung mit \texttt{psfrag} \textbf{funktioniert nicht} mit \texttt{pdflatex} 
%(Ausgabeprofil \verb!LaTeX => PDF!), da hier keine PostScript-Zwischendatei mehr erzeugt wird. 
%Um \texttt{psfrag} zu verwenden, muss man zumindest beim finalen Durchlauf das Ausgabeprofil auf 
%\verb!LaTeX => PS! oder \verb!LaTeX => PS => PDF! ändern.
%\end{sloppypar}
%\end{quote}

\begin{figure}
\centering\small
\begin{tabular}{cc}
\includegraphics[width=0.45\textwidth]{ellipse-parameters-1}
&
%\begin{psfrags} %ACHTUNG: Ersetzung funktioniert NUR im klassischen DVI-PS-Workflow!
%\small
%\psfrag{xc}[Bl]{$x_c$}	% Bl.. align on baseline (vert.) and left (horiz.) 
%\psfrag{yc}[Bl]{$y_c$}
%\psfrag{p}[Bl]{$\mathbf{p}=(u,v)$}
%\psfrag{p}[Bl]{$\boldsymbol{p}=(u,v)$}
%\psfrag{a}[Bl]{$\alpha$}
%\psfrag{b}[Bl]{$\beta$}
%\psfrag{w}[Bl]{$\varphi$}
%verwendet die PNG-Screenshot im PDF-workflow, EPS-Version (Vektorgrafik) im DVI-PS-Workflow:
\includegraphics[width=0.45\textwidth]{ellipse-parameters-2}
%\end{psfrags}
\\
(a) & (b)
\end{tabular}
\caption{Beispiel für die Textersetzung in EPS-Grafiken mit dem \texttt{psfrag}-Paket. Original-Vektorgrafik, erstellt in Freehand (a); dieselbe Grafik mit Textelementen ersetzt durch mathematische Symbole oder Ausdrücke (b). 
\textbf{Achtung:} Die tatsächliche Ersetzung funktioniert nur bei Verarbeitung im klassischen 
DVI-PS-Workflow -- im PDF-Workflow wird in (b) nur ein Screenshot angezeigt.}
\label{fig:psfrag-example}
\end{figure}

\end{comment}


\begin{comment} % dieses Thema ist unter PDF-LaTeX kaum mehr relevant

\subsection{Probleme mit EPS-Dateien}

Obwohl EPS im Grunde ein einfaches und wohldefiniertes Datenformat
ist, produzieren leider viele Programme unsauberen EPS-Code, der
sich entweder überhaupt nicht einbinden lässt oder später, \zB auf
dem Drucker, zu Fehlern führt. Typische Probleme sind etwa das
Auftreten unbekannter PostScript-Befehle, fehlende
Schriftdefinitionen, falsch angegebene oder fehlende
\emph{Boun\-ding-Box\-es} und binär kodierte \emph{Preview}-Daten.


\subsubsection{EPS-Dateien überprüfen}

Einige dieser Probleme kann man bereits im Vorfeld erkennen, in
dem man die betroffenen EPS-Dateien einfach mit einem Texteditor,
der auch große Dateien darstellen kann (beispielsweise mit \emph{TeXnicCenter}, \emph{Ultra\-Edit}, 
{\em Win\-Edt} oder \emph{Emacs}),%
\footnote{Dazu zählt auch der in der
Unix-Welt beliebte \texttt{vi}-Editor, in dem allerdings auch
abgehärtete Benutzer ein gewisses masochistisches Element zu
erkennen glauben.}
öffnet. Eine reguläre EPS-Datei muss jedenfalls mit der Zeichenfolge 
\begin{center}
\verb|%!PS-Adobe|
\end{center}
beginnen,
wie das Beispiel in Abb.~\ref{fig:eps-source} zeigt.
%
\begin{figure}
\centering
\begin{GenericCode}
%!PS-Adobe-2.0 EPSF-1.2
%%Title: workflow.fh8
%%Creator: FreeHand 8.0
%%CreationDate: Mon Jul 16 20:04:10 2001
%%BoundingBox: 0 0 420 138
...
end
%%Trailer
\end{GenericCode}
%
\caption{Inhalt einer typischen EPS-Datei.
Wichtig für die Einbindung in \latex ist v.a., dass keinerlei
Zeichen \emph{vor} dem "`\texttt{\%!}"' erscheinen und dass
im Kopf ein korrekter Eintrag für die \emph{Bounding Box} enthalten ist.
Die Maßeinheit für die Größe der \emph{Bounding Box} ist übrigens
$\frac{1}{72}$ Zoll.}
\label{fig:eps-source}
\end{figure}
%
Durch das Öffnen mit Adobe \emph{Photoshop} (in den ein
vollständiger \emph{Post\-Script}-Interpreter eingebaut ist) lässt
sich übrigens leicht feststellen, wie eine EPS-Datei in einer
beliebigen Auflösung "`rendert"'. Probleme mit fehlenden Schriften
können dabei allerdings unbemerkt bleiben, \va\ dann, wenn die
ursprüngliche Grafik auf dem selben Computer erstellt wurde.


\subsubsection{EPS-Grafiken über PDF generieren}

Falls man Grafiken aus einer Applikation übernehmen möchte, die
keine Möglichkeit zum EPS-Export anbietet (\zB\ {\em
PowerPoint}) oder unbrauchbare EPS-Dateien erzeugt, kann man
auch den Umweg
über eine PDF-Datei nehmen: 
%
\begin{itemize}
\item Dazu erzeugt man zunächst mit Adobe \emph{Distiller} (als
virtueller Drucker) eine PDF-Datei. %
\item Dann öffnet man die PDF-Datei mit \emph{Acrobat}, schneidet
diese (mit dem \emph{Crop Tool}) auf das richtige Format und %
\item speichert schließlich die Datei (mit \emph{Save As}) im
\texttt{eps}-Format, natürlich wiederum \emph{ohne} Preview
(in \emph{Settings} einstellbar). %
\end{itemize}
Das ist zwar etwas mühsamer als auf dem direkten Weg, aber im
Notfall meist einfacher als ein Grafik ganz neu zu machen. 
Eventuelle Probleme bei \texttt{psfrag}-Ersetzungen (siehe Abschn.\ \ref{sec:tex-schriften-in-grafiken}) sind dabei 
allerdings nicht auszuschließen.

\end{comment}


\subsection{Abbildungen mit mehreren Elementen}

Werden mehrere Bilder oder Grafiken zu einer Abbildung zusammengefasst, 
verwendet man üblicherweise eine gemeinsame Caption, wie in Abb.~\ref{fig:Bearings}
dargestellt. Im Text könnte ein Verweis auf einen einzelnen Teil der Abbildung, etwa das 
einreihige Rollenlager in Abb.~\ref{fig:Bearings}\,(c), so aussehen:
%
\begin{itemize}
\item[] \verb!Abb.~\ref{fig:Bearings}\,(c)! 
\end{itemize}
%
Für kompliziertere Abbildungen sollte man die Verwendung des 
\texttt{subfig}-Pakets \cite{Cochran05} in Betracht ziehen.


\subsection{Quellenangaben in Captions}
\label{sec:QuellenangabenInCaptions}

Wenn Bilder, Grafiken oder Tabellen aus anderen Quellen verwendet werden, dann 
muss ihre Herkunft in jedem Fall klar ersichtlich gemacht werden, und zwar am 
besten direkt in der Caption.
Verwendet man beispielsweise eine Grafik aus einem Buch oder einer sonstigen 
zitierfähigen Publikation, dann sollte man diese in das Literaturverzeichnis 
aufnehmen und wie üblich mit
\verb!\cite{..}! zitieren, wie in Abb.\ \ref{fig:Bearings} demonstriert. 
Weitere Details zu dieser Art von Quellenangaben finden sich in 
Kap.\ \ref{cha:Literatur} (insbes.\ Abschnitt \ref{sec:KategorieOnline}).

\begin{figure}
\centering\small
\setlength{\tabcolsep}{0mm}	% alle Spaltenränder auf 0mm
\begin{tabular}{c@{\hspace{12mm}}c} % mittlerer Abstand = 12mm
  \includegraphics[width=.45\textwidth]{overhang-mounting} &
  \includegraphics[width=.45\textwidth]{straddle-mounting} \\
  (a) & (b)
\\[4pt]	%vertical extra spacing (4 points)
  \includegraphics[width=.45\textwidth]{ball-bearing-1} &
  \includegraphics[width=.45\textwidth]{ball-bearing-2} \\
  (c) & (d)
\end{tabular}
%
\caption{Diverse Maschinenelemente als Beispiel für eine
Abbildung mit mehreren Elementen.
Overhang Mounting (a), Straddle Mounting (b),
einreihiges Rollenlager (c), Schmierung von Rollenlagern (d).
Die Abbildung verwendet im oberen Teile eine $2 \times 2$
Tabelle (\texttt{tabular}), in der die Breite der Spaltenränder 
gesondert spezifiziert ist (Details finden sich im Quelltext).
Bildquelle~\cite{Faires34}.
}
\label{fig:Bearings}
\end{figure}




\section{Tabellen}

Tabellen werden häufig eingesetzt um numerische Zusammenhänge, Testergebnisse
etc.\ in übersichtlicher Form darzustellen.
Ein einfaches Beispiel ist Tab.~\ref{tab:processors}, der \latex-Quelltext dazu
findet sich in Prog.~\ref{prog:processors-source}.


\begin{table}
\caption{Prozessor-Familien im Überblick.}
\label{tab:processors}
\centering
\setlength{\tabcolsep}{5mm}	% separator between columns
\def\arraystretch{1.25}			% vertical stretch factor (Standard = 1.0)
\begin{tabular}{|r||c|c|c|} \hline
& \emph{PowerPC} & \emph{Pentium} & \emph{Athlon} \\
\hline\hline
Manufacturer & Motorola & Intel & AMD \\
\hline
Speed & high & medium & high   \\
\hline
Price & high & high   & medium \\
\hline
\end{tabular}
\end{table}

\begin{program}
% place caption consistently either at the top or bottom:
\caption{\latex\ Quelltext zu Tab.~\ref{tab:processors}.
Die Erzeugung des dargestellten Listings selbst ist in Abschn.\ \ref{sec:programmtexte} beschrieben.}
\label{prog:processors-source}
%
\begin{LaTeXCode}[numbers=none]
\begin{table}
	\caption{Prozessor-Familien im Überblick.}
	\label{tab:processors}
	\centering
	\setlength{\tabcolsep}{5mm}	% separator between columns
	\def\arraystretch{1.25}		% vertical stretch factor
	\begin{tabular}{|r||c|c|c|} 
		\hline
		& \emph{PowerPC} & \emph{Pentium} & \emph{Athlon} \\
		\hline
		\hline
		Manufacturer & Motorola & Intel & AMD \\
		\hline
		Speed & high & medium & high   \\
		\hline
		Price & high & high   & medium \\
		\hline
	\end{tabular}
\end{table}
\end{LaTeXCode}
%
\end{program}

Manchmal ist es notwendig, in Tabellen relativ viel Text in engen Spalten
unter zu bringen, wie in Tab.~\ref{tab:synthesis-techniques}. In diesem Fall
ist es sinnvoll, auf den Blocksatz zu verzichten und gleichzeitig die
strengen Abteilungsregeln zu lockern. Details dazu finden sich im zugehörigen
\latex-Quelltext.


%--------------------------------------------------------------------------------
% Table with narrow columns
%--------------------------------------------------------------------------------
\begin{table}
\caption{Beispiel für eine Tabelle mit mehrzeiligem Text in engen Spalten.
Hier werden die Zeilen für den Blocksatz zu kurz, daher wird linksbündig
gesetzt (im "`Flattersatz"').}
\label{tab:synthesis-techniques}
\centering
\def\rr{\rightskip=0pt plus1em \spaceskip=.3333em \xspaceskip=.5em\relax}
\setlength{\tabcolsep}{1ex}
\def\arraystretch{1.20}
\setlength{\tabcolsep}{1ex}
\small
\begin{english}
\begin{tabular}{|p{0.2\textwidth}|c|p{0.3\textwidth}|p{0.2\textwidth}|}
\hline
   \multicolumn{1}{|c}{\emph{Method}} &
   \multicolumn{1}{|c}{\emph{Implem.}} &
   \multicolumn{1}{|c}{\emph{Features}} &
   \multicolumn{1}{|c|}{\emph{Status}} \\
\hline\hline
   {\rr polygon shading} &
   SW/HW &
   {\rr flat-shaded polygons} &
   \\
\hline
  {\rr flat shading with z-buffer} &
  SW/HW &
  {\rr depth values} &
  \\
\hline
  {\rr goraud shading with z-buffer} &
  SW/HW &
  {\rr smooth shading, simple fog, point light sources} &
  {\rr SGI entry models} \\
\hline
  {\rr phong shading with z-buffer} &
  SW/HW &
  {\rr highlights} &
  \\
\hline
  {\rr texture mapping with z-buffer} &
  SW/HW &
  {\rr surface textures, simple shadows} &
  {\rr SGI high end, flight simulators} \\
\hline
%  {\rr reflection mapping with z-buffer} &
%  SW/HW &
%  {\rr reflections} &
%  {\rr SGI next generation} \\
%\hline
%  {\rr raytracing} &
%  SW &
%  {\rr refraction, real camera model, area light sources with penumbra, realistic material models} &
%  {\rr common ray\-tracers} \\
%\hline
%  {\rr raytracing + global illumination simulation} &
%  SW &
%  {\rr indirect illumination} &
%  \textit{Radiance} \\
%\hline
%  {\rr raytracing + global illumination simulation + dissipating media} &
%  none &
%  {\rr realistic clouds, scattering, ...} &
%  {\rr research} \\
%\hline
\end{tabular}
\end{english}
\end{table}

%--------------------------------------------------------------------------------



\section{Programmtexte}
\label{sec:programmtexte}

Die Einbindung von Programmtexten (source code) ist eine häufige Notwendigkeit,
\va natürlich bei Arbeiten im Bereich der Informatik.

\subsection{Formatierung von Programmcode}
\label{sec:FormatierungVonProgrammcode}

Es gibt für \latex\ spezielle Pakete zur Darstellung von Programmen, die \ua\ auch die automatische Nummerierung der Zeilen vornehmen, insbesondere das \texttt{listings}-Package.%
\footnote{\url{http://www.ctan.org/tex-archive/macros/latex/contrib/listings/}}
Damit sind auch die in Tabelle~\ref{tab:CodeUmgebungen} aufgelisteten Code-Umgebungen 
realisiert.
%
\begin{table}
\caption{In \nolinkurl{hgb.sty} vordefinierte Code-Umgebungen.}
\label{tab:CodeUmgebungen}
\centering
\begin{tabular}{llll}
	\hline
	C (ANSI):   & \verb!\begin{CCode}! & \verb!...! \verb!\end{CCode}! \\
	C++ (ISO):  & \verb!\begin{CppCode}! & \verb!...! \verb!\end{CppCode}! \\
	Java:       & \verb!\begin{JavaCode}! & \verb!...! \verb!\end{JavaCode}! \\
	JavaScript:  			& \verb!\begin{JsCode}! & \verb!...! \verb!\end{JsCode}! \\
	PHP:  			& \verb!\begin{PhpCode}! & \verb!...! \verb!\end{PhpCode}! \\
	HTML:  			& \verb!\begin{HtmlCode}! & \verb!...! \verb!\end{HtmlCode}! \\
	CSS:  			& \verb!\begin{CssCode}! & \verb!...! \verb!\end{CssCode}! \\
	XML:  			& \verb!\begin{XmlCode}! & \verb!...! \verb!\end{XmlCode}! \\
	\latex:     & \verb!\begin{LaTeXCode}! & \verb!...! \verb!\end{LaTeXCode}! \\
	Generisch:  & \verb!\begin{GenericCode}! & \verb!...! \verb!\end{GenericCode}! \\
	\hline
\end{tabular}
\end{table}
%
Die Verwendung ist äußerst einfach, \zB\ für Quellcode in der Programmiersprache C schreibt man
%
\begin{quote}
\begin{verbatim}
\begin{CCode}
    ... 
\end{CCode}
\end{verbatim}
\end{quote}
%
Der Quellcode innerhalb dieser Umgebungen wird in der jeweiligen Programmiersprache interpretiert, wobei Kommentare erhalten bleiben. Diese Umgebungen können sowohl alleinstehend (im Fließtext) oder innerhalb von Float-Umgebungen (insbes.\ \texttt{program}) verwendet werden. Im ersten Fall wird der Quelltext auch über Seitengrenzen umgebrochen. Mit \verb!/+! ... \verb!+/! ist eine Escape-Möglichkeit nach \latex\ vorgesehen, die \va\ zum Setzen von Labels für Verweise auf einzelne Programmzeilen nützlich ist, \zB\ mit
%
\begin{quote}
\verb!/+\label{ExampleCodeLabel}+/!
\end{quote}
%
Ein Beispiel mit Java ist in Prog.~\ref{prog:CodeExample} gezeigt, wobei der oben angeführte Label in Zeile \ref{ExampleCodeLabel} steht.
Man beachte, dass innerhalb der Kommentare auch mathematischer Text (wie etwa in Zeile \ref{MathInCode} von Prog.~\ref{prog:CodeExample}) stehen kann.


\subsubsection{Nummerierung der Code-Zeilen}

Alle in Tabelle~\ref{tab:CodeUmgebungen} angeführten Code-Umgebungen können
mit optionalen Argumenten verwendet werden, die insbesondere zur Steuerung der
Zeilennummerierung hilfreich. 
Im Normalfall (also ohne zusätzliche Angabe) mit
%
\begin{quote}
\verb!\begin{!\texttt{\emph{some}Code}\verb!} ... !
\end{quote}
%
werden alle Code-Zeilen (einschließlich der Leerzeilen) bei 1 beginnend und 
fortlaufend nummeriert.
%
Bei aufeinanderfolgenden Codesegmenten ist es oft hilfreich, die Nummerierung 
aus dem vorherigen Abschnitt kontinuierlich weiter laufen zu lassen,
ermöglicht durch die Angabe des optionalen Arguments 
\texttt{firstnumber={\optbreaknh}last}:
%
\begin{quote}
\verb!\begin{!\texttt{\emph{some}Code}\verb!}[firstnumber=last] ... !
\end{quote}
%
Um die Nummerierung der Codezeilen gänzlich zu unterbinden genügt die Angabe
des optionalen Arguments
\texttt{numbers={\optbreaknh}none}:
%
\begin{quote}
\verb!\begin{!\texttt{\emph{some}Code}\verb!}[numbers=none] ... !
\end{quote}
%
In diesem Fall ist natürlich die Verwendung von Zeilenlabels im Code nicht
sinnvoll.



\subsection{Platzierung von Programmcode}

Da Quelltexte sehr umfangreich werden können, ist diese Aufgabe nicht
immer leicht zu lösen. Abhängig vom Umfang und vom Bezug zum Haupttext
gibt es grundsätzlich drei Möglichkeiten zur Einbindung von Programmtext:
%
\begin{itemize}
\item[a)] im laufenden Text für kurze Programmstücke,
\item[b)] als Float-Element (\texttt{program}) für mittlere Programmtexte bis max.\ eine Seite oder
\item[c)] im Anhang (für lange Programmtexte).
\end{itemize}

\subsubsection{Programmtext im laufenden Text}

Kurze Codesequenzen kann man ohne weiteres im laufenden Text
einbetten, sofern sie an den gegebenen Stellen von unmittelbarer
Bedeutung sind. Die folgende (rudimentäre) Java-Methode {\tt
extractEmail} sucht nach einer E-Mail Adresse in der Zeichenkette
\texttt{line}:
%
\begin{JavaCode}[numbers=none]
static String extractEmail(String line) {
    line = line.trim(); // find the first blank
    int i = line.indexOf(' '); 
    if (i > 0)
        return line.substring(i).trim();
    else
        return null;
}
\end{JavaCode}
\medskip

\noindent
Dieses Codestück wurde mit 
%
\begin{quote}
\begin{verbatim}
\begin{JavaCode}[numbers=none]
static String extractEmail(String line) {
    line = line.trim(); // find the first blank
    ...
}
\end{JavaCode}
\end{verbatim}
\end{quote}
%
erstellt (siehe Abschn.\ \ref{sec:FormatierungVonProgrammcode}). 
In-line Programmstücke sollten maximal einige Zeilen lang sein und 
nach Möglichkeit nicht durch Seitenumbrüche geteilt werden.
%Um auch längere Programmzeilen unterzubringen, empfiehlt es sich, dafür
%eine entsprechend kleine Schriftgröße zu wählen (als Standardgröße ist
%\texttt{footnotesize} eingestellt). 


\subsubsection{Programmtexte als Float-Elemente}
Sind längere Codesequenzen notwendig, die in unmittelbarer Nähe des laufenden Texts
stehen müssen, sollte man diese genauso wie andere Abbildungen als Float-Elemente
behandeln. Diese Programmtexte sollten den Umfang von einer Seite nicht übersteigen.
Im Notfall kann man auch bis zu zwei Seiten in aufeinanderfolgende Abbildungen packen,
jeweils mit eigener Caption. In \texttt{hgb.sty} ist eine neue Float-Umgebung \texttt{program} definiert, die analog zu \texttt{table} verwendet wird:
%
\begin{quote}
\begin{verbatim}
\begin{program}
\caption{Der Titel zu diesem Programmstück.}
\label{prog:xyz}
\begin{JavaCode}
  class IrgendWas {
    ...
  }
\end{JavaCode}
\end{program}
\end{verbatim}
\end{quote}
%
Wenn man möchte, kann man in diesem Fall die Caption auch unten anbringen 
(jedenfalls aber konsistent und nicht gemischt).
Natürlich darf man auch hier  nicht mit einer linearen  Abfolge im fertigen
Druckbild rechnen, daher sind Wendungen wie
"`... im  folgenden Programmstück ..."' zu vermeiden und entsprechende Verweise
einzusetzen. Beispiele dafür sind die Programme \ref{prog:processors-source} und \ref{prog:CodeExample}.

\begin{program}
% place caption consistently either at the top or bottom:
\caption{Beispiel für die Auflistung von Programmcode als Float-Element.}
\label{prog:CodeExample}
\begin{JavaCode}
import ij.ImagePlus;
import ij.plugin.filter.PlugInFilter;
import ij.process.ImageProcessor;

public class My_Inverter implements PlugInFilter {
	int agent_velocity;
  String title = ""; // just to test printing of double quotes

	public int setup (String arg, ImagePlus im) {
		return DOES_8G;	// this plugin accepts 8-bit grayscale images \label{pr:IjSamplePlugin10}
	}

	public void run (ImageProcessor ip) {
		int w = ip.getWidth();	/+\label{ExampleCodeLabel}+/
		int h = ip.getHeight(); 
		
		/* iterate over all image coordinates */
		for (int u = 0; u < w; u++) { 
			for (int v = 0; v < h; v++) {
				int p = ip.getPixel(u, v); 
				ip.putPixel(u, v, 255-p); // invert: $I'(u,v) \leftarrow 255 - I(u,v)$\label{MathInCode}
			}
		}
	}
			
} // end of class {\tt My\_Inverter}
\end{JavaCode}
%
\end{program}


\subsubsection{Programmtext im Anhang}

Für längere Programmtexte, speziell wenn sie vollständige
Implementierungen umfassen und im aktuellen Kontext nicht
unmittelbar relevant sind, muss man zur Ablage in einem getrennten
Anhang am Ende des Dokuments greifen. Für Hinweise auf bestimmte
Details kann man entweder kurze Ausschnitte in den laufenden Text
stellen oder mit entsprechenden Seitenverweisen arbeiten. Ein
solches Beispiel ist der \latex-Quellcode in Anhang
\ref{app:latex} (Seite \pageref{app:latex}).%
\footnote{%
Grundsätzlich ist zu überlegen, ob die gedruckte Einbindung der gesamten
Programmtexte einer Implementierung für den Leser überhaupt sinnvoll ist, oder
ob man diese nicht besser elektronisch (auf CD-ROM) beifügt und nur exemplarisch
beschreibt.}


%\include {schueler_b/prozessoren}

%%%----------------------------------------------------------
%%%Anhang
\appendix
\chapter{Projektdokumentation}

\section{Meilensteine}

\subsection{Meilenstein-Bezeichnung (TT. Monat JJJJ)}
To be done.

\subsection{Meilenstein-Bezeichnung (TT. Monat JJJJ)}
To be done.

\subsection{Meilenstein-Bezeichnung (TT. Monat JJJJ)}
To be done.

\subsection{Meilenstein-Bezeichnung (TT. Monat JJJJ)}
To be done.

\subsection{Goldmaster (TT. Monat JJJJ)}
To be done.

\section{Burndown}
\begin{figure}[H]
	\centering
	\includegraphics[width=0.9\textwidth]{burndown}
	\caption{Burndown chart}
\end{figure}

\section{Playtest Protokolle \& Ergebnisse}
To be done.
%\includepdf[page=-]{formulare/PlaytestProtokolle}

\section{Begleitprotokolle}
To be done.
%\includepdf[page=-]{formulare/Zeitprotokolle}	    % Projektdokumentation
\chapter{Kooperationsvereinbarung}

\includepdf[page=-]{formulare/Kooperationsvereinbarung}	% Kooperationsvereinbarung
\chapter{Inhalt des Datenträgers}

\section{Projektdateien}
\begin{FileList}{/Game}
	\fitem{Projekt} Unity + Wwise Projekt
	\fitem{Build} Windows Build des Spieles
\end{FileList}

\section{Arbeit}
\begin{FileList}{/Diplomarbeit}
	\fitem{_DaBa.pdf} Diplomarbeit (Gesamtdokument) 
	\fitem{LaTex} LaTex Projekt
\end{FileList}

\section{Game Design}
\begin{FileList}{/GameDesign}
	\fitem{Concepts} Konzept Dateien
	\fitem{Playtests} Ergebnisse der Playtests
\end{FileList}
 
\section{Presskit}
\begin{FileList}{/Presskit}
	\fitem{Trailer} Trailer des Spieles
	\fitem{Overview} Ein PDF mit Infos zum Spiel
	\fitem{Screenshots} Screenshots des Spieles
\end{FileList}
	            % Inhalt des Datenträgers

%%%----------------------------------------------------------
\MakeBibliography
\addcontentsline{toc}{chapter}{Abbildungsverzeichnis}
\listoffigures
\addcontentsline{toc}{chapter}{Tabellenverzeichnis}
\listoftables


%%%----------------------------------------------------------

%%%Messbox zur Druckkontrolle
\chapter*{Messbox zur Druckkontrolle}



\begin{center}
{\Large --- Druckgröße kontrollieren! ---}

\bigskip

\Messbox{100}{50} % Angabe der Breite/Hoehe in mm

\bigskip

{\Large --- Diese Seite nach dem Druck entfernen! ---}

\end{center}



\end{document}
