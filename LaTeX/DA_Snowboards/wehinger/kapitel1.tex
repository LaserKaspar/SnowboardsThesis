\chapter{Überblick über die Materie}

\section{Farbpsychologie}

\subsection{Was ist Farbe?}
Wissenschaftlich betrachtet, sind Farben Eindrücke, die unsere Sinne durch die Augen und das Gehirn vermittelt bekommen. Farben entstehen durch die Absorption und Reflexion, die das Licht auf Oberflächen wirft. Das menschliche Auge ist für die Sinnesempfindung von Farben verantwortlich, die dann wiederum im Gehirn verarbeitet werden. Man kann Farbe auch als Eigenschaft des Lichts betiteln. Prinzipiell nimmt jeder Mensch Farben und Farbtöne anders wahr. Welche Farbe wahrgenommen wird, hängt von dem Gegenstand ab, der Lichtquelle und den Augen des Betrachtenden. Licht hat außerdem auch bestimmte Wellenlängen, je nach Länge der Welle nimmt das Auge eine andere Farbe war. Schwarz ist keine Farbe, denn sie kann auch in Abwesenheit des Lichts und völliger Dunkelheit existieren.
Weißes Licht kann mithilfe eines Prismas in Spektralfarben gebrochen werden. Spektralfarben sind diejenigen Farben, die für das menschliche Auge einen sichtbaren Farbeindruck und somit einen Farbton hinterlassen. Insgesamt gibt es sechs verschiedene Spektralfarben. Spektralfarben werden umgangssprachlich auch „Regenbogenfarben“ genannt. Zu ihnen zählen Rot, Orange, Gelb, Grün, Blau, Indigo und Violett. Die durch die Brechung von Licht entstandenen Farben können nicht in weitere Farbtöne zerlegt werden. Das Licht kann manchmal auch infrarotes Licht oder ultraviolettes Licht enthalten, das für das bloße Auge nicht sichtbar ist.
Wenn man jedoch gelbes Licht oder jede andere Spektralfarbe durch ein Prisma fallen lässt, bekommt man dieselbe Farbe raus, die man im Prisma brechen wollte. Man kann also als Schlussfolgerung daraus ziehen, das weißes Licht das einzige ist, was durch ein Prisma in Spektralfarben gebrochen werden kann.
\cite{_special_subjects}

\begin{figure}[H]
	\centering
	\includegraphics[width=10cm]{Farbprisma}
	\caption{Farbprisma\cite{_basicColorTheory}}
\end{figure}

In der Kunst ist Farbe das beste Mittel, um visuell Emotionen auszudrücken oder Emotionen bei Menschen zu erzeugen. Farben sind ein gutes Instrument, wenn das darum geht, bei einem Menschen Gefühle und Emotionen auszulösen. Aber nicht nur dass, es wird dem Betrachtenden dadurch auch eine bestimmte Stimmung vermittelt. Künstler können dadurch gezielt bei Menschen bestimmte Gefühle auslösen und das alleine durch das Einsetzen von bestimmten Farben und Farbtönen.
Dabei muss beachtet werden, dass Farben in verschiedenen Kulturen verschiedene Bedeutungen haben und die Symbolik hinter manchen Farbtönen in bestimmten Ländern und Kulturkreisen eine andere ist. Doch grundsätzlich kann man jeder Farbe eine bestimmte Bedeutung und Emotion zuschreiben, die vielleicht in manchen Kulturen ein wenig abweicht, aber im Grunde genommen immer sehr ähnlich ist.
Man kann aber klar die kulturelle Bedeutung von Farben mit der psychologischen Bedeutung von Farben trennen. Die wichtigsten Farben für Kunst und Design sind die Farben Blau, Gelb, Grün, Orange, Rosa, Rot, Violett, Grau und Braun. Jede dieser Farben hat eine andere Wirkung auf die Gefühle und Emotionen der Betrachtenden. Der psychologische Effekt hinter der Wirkung einer Farbe hat rein gar nichts mit der kulturellen Interpretation dieser zu tun. 
Es gibt eine endliche Anzahl an Variationen, die eine Farbe für das menschliche Auge annehmen kann. Grundsätzlich gibt es eine unendliche Anzahl an Schattierungen, die eine Farbe haben kann, wobei die meisten für das Sinnesorgan eines Menschen nicht sichtbar sind. Die Schattierung einer Farbe ändert prinzipiell den psychologischen Effekt nicht.
\cite{_special_subjects}

\subsection{Der Farbkreis und Farbschemas}
Der Farbkreis besteht normalerweise aus 12 Farben, er kann aber auch aus 24 und manchmal aus bis zu 48 Farben bestehen, das hängt aber von dem Ersteller des Farbkreises ab. Das grundlegende Farbrad besteht aber jedoch nur aus 12 Farben. Um den Farbkreis besser zu verstehen, muss man sich zuerst die additive und die subtraktive Farbmischung ansehen.  
\cite{_special_subjects}

\begin{figure}[H]
	\centering
	\includegraphics[width=10cm]{Farbkreis}
	\caption{Farbrad\cite{_basicColorTheory}}
\end{figure}

Farben können in Primär -, Sekundär -und Tertiärfarben aufgeteilt werden. Dabei unterscheidet man zwischen der additiven Farbmischung und der subtraktiven Farbmischung. 
\cite{_special_subjects}

Bei der subtraktiven Farbmischung, auch CMYK-Modell genannt, sind die Primärfarben Gelb, Magenta und Cyan, sie können nicht aus anderen Farben gemischt werden. Jedoch kann aus ihnen jede existierende Farbe gemischt werden. Die Sekundärfarben in diesem Modell sind Rot, Grün und Blau und die Tertiärfarbe ist Schwarz. Mischt man nun die Primärfarben Gelb, Magenta und Cyan, erhält man infolge dessen die Sekundärfarben Rot, Grün und Blau. Die gesamte Mischung ergibt dann Schwarz.
\cite{_special_subjects}

Die additive Farbmischung, auch RGB-Modell genannt, besteht aus den Lichtfarben Rot, Grün und Blau. Rot, Grün und Blau sind in diesem Modell die Primärfarben, also Grundfarben. Die Sekundärfarben sind Gelb, Magenta und Cyan. Die Tertiärfarbe ist Weiß. Diese entsteht durch die Mischung der Grundfarben Rot, Grün und Blau.
\cite{_special_subjects}

Im grundlegenden Farbkreis gibt es jedoch einen Unterschied bei den Primär -, Sekundär -und Tertiärfarben zum CMYK - und RGB Modell. Im Farbrad mit 12 Farben sind die Primärfarben Rot, Gelb und Blau. Die Sekundärfarben sind die Farben, die durch das Mischen von zwei Primärfarben entstehen, das sind Farben wie Orange, Grün und Violett. Die Tertiärfarben sind die Farben, die beim Mischen von einer Primärfarbe und einer benachbarten Sekundärfarbe entstehen. Die Tertiärfarben sind all jene Farben, die die noch vorhandenen Lücken im Farbkreis füllen.
\cite{_special_subjects}

Die wichtigste Aufgabe des Farbrads ist es, viele Farben auf einmal zu sehen, um sie dadurch besser zu verstehen und zu lernen, Farben zu wählen, die gut zueinander passen. Es zeigt uns visuell das Verhalten der Farbtöne zueinander, dadurch kann man harmonische und ästhetische Kombinationen bilden.
Außerdem gibt es fünf verschiedene Farbschemen, die einem helfen harmonische und dynamische Atmosphäre in seinen Kompositionen zu kreieren.
\cite{_special_subjects}

Beim komplementären Farbschema wählt man die Farben aus, die einander gegenüber liegen. Komplementäre Farben wirken nebeneinander viel intensiver und heller als Farben, die im Farbrad benachbart sind. Ein Beispiel für Komplementärfarben wären zum Beispiel Rot und Grün, sie liegen im Kreis gegenüber.
\cite{_special_subjects}

Das triadische Farbschema besteht aus einem gleichseitigen Dreieck. Das Modell des triadischen Farbschemas fokussiert sich auf eine dominante Farbe und zwei zueinander sehr harmonische Farben. Ein Beispiel dafür wäre die Kombination aus den Farbtönen Violett, Grün und Orange. Diese Farben erzeugen sehr viel Kontrast, wirken aber sehr harmonisch auf das menschliche Auge. 
\cite{_special_subjects}

Beim tetradischen Farbschema besteht die Kombination aus Farben aus denjenigen Farbtönen, die an den Ecken des Rechtecks oder Quadrats liegen. Dieses Schema besteht aus jeweils 2 Paaren von Komplementärfarben, die zwar zueinander viel Kontrast haben, aber im Gesamtbild gibt es keine sehr dominante Farbe. Die Kombination soll ein harmonisches Bild erzeugen.
\cite{_special_subjects}

Beim analogischen Farbschema werden meistens 2 Farben ausgewählt, die einer anderen Farbe benachbart sind. Diese Kombination von Farbtönen soll eine Einheit erzeugen, um das Gesamtbild ausgewogen wirken zu lassen. Ein Beispiel für solch ein Schema wären die Farben Rot, Rot-orange und Orange.
\cite{_special_subjects}

Das letzte Farbschema ist das komplemenänter-geteilte Schema, es besteht aus eine Primär -oder Sekundärfarbe und zwei Tertiärfarben, die der Komplementärfarbe der Hauptfarbe benachbart sind. So eine Farbkombination wäre zum Beispiel Rot, Gelb-Grün und Blau-Grün.
\cite{_special_subjects}

\subsection{Farbtemperatur}
Die Farbtemperatur ist grundsätzlich der Farbton, den eine Farbe annimmt, wenn sie von einer Lichtquelle beleuchtet wird. In der Kunst bezieht sich die Farbtemperatur auf das Gefühl, das der Betrachtende von bestimmten Farbtönen oder Gruppen von Farben bekommt. Dabei unterscheidet man zwischen zwei Gruppierungen, den warmen Farben und den kalten Farben. Warme Farben wären zum Beispiel Rot, Gelb und Orange. Kalte Farben wären unter anderem Blau, Violett und Grün. Während warme Farben auf den Betrachtenden so wirken, als würden sie auf einen zugehen, so wirken kalte Farben so, als würden sie sich vom Betrachtenden entfernen. Wenn man Tiefe erzeugen will oder ein Bild dynamisch wirken lassen möchte, sollte man dafür warme und kalte Farben verwenden. Man kann diese auch klar im Farbkreis trennen, indem man zwischen Gelb und Violett einen Strich durch die Mitte zieht, als Ergebnis hat man auf einer Seite die warmen Farben und auf der anderen Seite die kalten Farben.
\cite{_special_subjects}

\begin{figure}[H]
	\centering
	\includegraphics[width=10cm]{Farbtemperatur}
	\caption{Farbrad in warme und kalte Farben unterteilt\cite{_special_subjects}}
\end{figure}

\subsection{Farbrelativität}
Prinzipiell werden Farben in warme Farben und kalte Farben eingestuft, sie können aber auch in relativ warme oder kalte Farben innerhalb ihres Farbtons eingestuft werden. Die Farbe Gelb gehört zu den warmen Farben, mischt man aber die Farbe Blau hinzu, wird der Farbton eher kühler. Mischt man zu der Farbe Gelb zum Beispiel die Farbe Rot hinzu, wird der Farbton wärmer. Die Eigenschaften einer Farbe sind immer relativ zu ihrer Umgebung. Man kann Farben heller oder dunkler und wärmer oder kälter wirken lassen, indem man sie neben bestimmten Farben platziert.
\cite{_special_subjects}

\begin{figure}[H]
	\centering
	\includegraphics[width=12cm]{Farbrelativität}
	\caption{Farbrelativität zwischen zwei Farben\cite{_special_subjects}}
\end{figure}

\subsection{Farbsättigung}
Die Farbsättigung einer Farbe kann an ihrer Reinheit und Leuchtkraft gemessen werden. Wenn eine Farbe weder in ihrer Reinheit noch in ihrer Leuchtkraft gesteigert werden kann, ist sie gesättigt. Die Farbsättigung beschreibt also das Verhältnis, das die Farbigkeit zur Helligkeit eines Farbtons hat. Die Farbsättigung wird auch „Intensität“ oder „Chroma“ einer Farbe genannt. Schwarz, Weiß und Grau haben eine Sättigung vom Wert null, sie haben somit keine Farbsättigung. 
\cite{_special_subjects}

\subsection{Farbe, Licht und Schatten}
Mit Farbe, Licht und Schatten kann die Stimmung und die Dramatik eines Bildes beeinflusst werden. Um zu verstehen, wie Licht und Schatten Objekte beeinflussen, muss das Objekt in drei Teile geteilt werden. Das Objekt wird in Licht, lokale Farbe und Schatten eingeteilt. Die lokale Farbe bezieht sich auf die tatsächliche Farbe eines Objekts ohne Einfluss von Licht und Schatten. Es gibt verschiedene Lichtquellen, mit denen Objekte beleuchtet werden können. Darunter fallen die Sonne, der Mond und künstliches Licht. Wenn die Sonne ein Objekt beleuchtet, werden die lokalen Farben des Objekts wärmer und die Schatten kühler. Wenn der Mond ein Objekt beleuchtet, werden die lokalen Farben des Objekts kühler und die Schatten wärmer.
\cite{_special_subjects}

\subsection{Wie wirkt sich Farbe auf unsere Stimmung aus?}
Farbpsychologie beschreibt, wie Farben die Stimmung und das Verhalten von Menschen beeinflussen. Wie Menschen Farben wahrnehmen, hängt von ihrem Geschlecht, ihrem Alter, ihrer Kultur und ihrem ethnischen Hintergrund ab. Dennoch gibt es viele universelle Reaktionen auf bestimmte Farben und Farbgruppen. 
Gelb steht für Wärme, Glück, Hoffnung und Positivität. Die Farbe Gelb wird oft auch mit der Sonne oder dem Sonnenschein in Zusammenhang gesetzt.
Rot wird oft mit Feuer und Blut assoziiert und sie steht für Energie, Macht, Liebe und Passion. Es wurde erwiesen, dass die Farbe Rot den Blutdruck und den Puls erhöht. 
Die Farbe Pink oder Rosarot wird mit Liebe und Romantik in Verbindung gesetzt. Die Farbe hat einen beruhigenden Effekt, wobei zu viel für den Menschen anstrengend sein kann.
Blau wird mit dem Himmel und dem Wasser assoziiert. Die Farbe löst beim Betrachtenden Gelassenheit aus und wirkt entspannend. Dunklere Blautöne werden jedoch mit Verzweiflung und Traurigkeit in Zusammenhang gesetzt. 
Violett wird mit dem Adel verbunden, denn früher konnten sich nur sehr wohlhabende Menschen das violette Farbpigment leisten. Die Farbe Violett steht für Eleganz, Würde und Raffinesse. 
Schwarz wird mit dem Tod, der Angst, dem Bösen und Negativität assoziiert. Die Farbe Schwarz wird oft verwendet, damit andere Farben und Farbtöne mehr herausstechen. 
Weiß symbolisiert Reinheit, Unschuld und Göttlichkeit, sie wird auch als Farbe der Perfektion angesehen. 
Die Farbe Grün ist die Farbe der Natur. Sie wird mit Harmonie und Fruchtbarkeit assoziiert. Grün ist für das menschliche Auge die erholsamste Farbe. 
Orange steht für Enthusiasmus, Kreativität und Belebung. Das menschliche Auge nimmt Orange als die wärmste Farbe wahr. 
\cite{_special_subjects}

\section{Formpsychologie}
In der Formpsychologie geht es darum, wie unser Gehirn versucht, einfache und stabile Formen herauszufiltern, die ihm vertraut sind. Das Gehirn versucht Elemente, die zusammengehören zu verbinden. Formen, die sich sehr ähnlich sind, werden in eine Gruppe gegeben. Wenn unser Gehirn die Entfernung nicht verwenden kann, versucht es, ähnliche Formen zu gruppieren. Es wird immer versucht, Formen zu vervollständigen.
\cite{_drawing_basics_and_video_game_art}

\subsection{Gestalt und Form}
Die Kommunikation mit Formen taucht in allen künstlerischen Disziplinen auf und ist erstaunlich universell. Die drei Grundformen sind der Kreis, das Quadrat und das Dreieck. Der Kreis steht für Freundlichkeit und Positivität. Das Quadrat soll Vertrauen, Stabilität und Sicherheit kommunizieren. Das Dreieck steht für Gefahr und Aggressivität. Man kann die Formen von links nach rechts anordnen, in eine „Skala der Emotionen“. Ganz Links ist der Kreis in der Mitte das Quadrat und ganz Rechts das Dreieck. Traditionell stehen der Kreis und gebogene Linien für Feminität und Quadrate und Dreiecke für Maskulinität.
Kreise beziehen sich auf gebogene und geschwungene Linien. Quadrate sind verwandt mit geraden, vertikalen und horizontalen Linien. Dreiecke beziehen sich auf eckige und diagonale Linien. 
Objekte und Dinge, die Menschen als sicher ansehen, bestehen meist aus runden Formen und gebogenen Linien. Gefährliche Objekte oder Dinge bestehen meistens aus scharfen Linien und Dreiecken. In der Kunst reagieren Menschen auf Formen mit Erfahrungen aus dem echten Leben.
\cite{_drawing_basics_and_video_game_art}

\begin{figure}[H]
	\centering
	\includegraphics[width=10cm]{GestaltFormen}
	\caption{Skala der Formen\cite{_drawing_basics_and_video_game_art}}
\end{figure}

\subsection{Linien und Linien- Stile}
Linien sind das einfachste Werkzeug für visuelle Kommunikation. Linien und Formen geben Konzepten, Werken und Bildern erst eine Form. Ist gibt verschiedene Linienstile, mit denen der Erschaffende bestimmte Effekte erzeugen möchte. 
\cite{_drawing_basics_and_video_game_art}
\cite{_line_color_form}

Es ist möglich, in einem Bild Licht zu erschaffen, ohne etwas zu schattieren. Je nachdem wie viel Druck man einer Linie gibt, so kann man dunkle oder helle Konturlinien kreieren, die einer Oberfläche mehr oder weniger Licht geben. 
\cite{_drawing_basics_and_video_game_art}
\cite{_line_color_form}

Es gibt Linien, mit denen man in einem Bild oder in einem Werk Orientierung kommunizieren kann. Je nachdem in welchem Winkel eine Linie konstruiert wurde, kann sie die Orientierung oder Richtung eines Objekts bestimmen. Horizontal und vertikal gezeichnete Linien sind sehr statisch und haben wenig Balance. Linien, die diagonal sind erzeugen eine bestimmte Illusion von Energie.
\cite{_drawing_basics_and_video_game_art}

Man kann auch Tiefe in einem Werk erzeugen, indem man Linien in bestimmten Längen und Abständen aufträgt. 
Mit Linien kann auch Tiefe kreiert werden. Je nachdem, wie man die Stärke einer Linie ändert, erschafft das die Illusion von Atmosphäre und Perspektive. 
\cite{_drawing_basics_and_video_game_art}

Um eine starke Tiefe in Bildern zu erzeugen, werden oft Vorder -und Hintergrund Linien und Objekte übereinander gezeichnet. Das kreiert beim Betrachtenden das Gefühl von Tiefe in einer Komposition.
\cite{_drawing_basics_and_video_game_art}

Um die Oberfläche eines Objekts zu beschreiben, ist auch die Form und Orientierung der Linie äußerst wichtig. 
\cite{_drawing_basics_and_video_game_art}

Linien, die sich in einem Werk entgegenkommen, erzeugen die Illusion von Bewegung und kreieren dadurch einen visuellen Pfad oder Weg für den Betrachtenden. Mit dieser Art von Linien kann man statische Objekte dynamisch wirken lassen.
\cite{_drawing_basics_and_video_game_art}

\subsection{Kompositionslinien}
Kompositionslinien dienen zur Orientierung für den Betrachtenden. Sie setzen bewusst Blickpunkte, um das Auge zu lenken und den Schwerpunkt im Bild zu kommunizieren. Außerdem sollen Kompositionslinien Elemente im Bild miteinander verbinden, um Beziehungen zwischen Objekten und Bildelementen herzustellen. Kompositionslinien müssen für den Betrachtenden nicht immer sichtbar sein. 
\cite{_drawing_basics_and_video_game_art}

\begin{figure}[H]
	\centering
	\includegraphics[width=10cm]{Kompositionslinien}
	\caption{Kompositionslinien in einem Werk\cite{_drawing_basics_and_video_game_art}}
\end{figure}

\subsection{Wie Formen und Linien Emotionen erzeugen}
Generell kann man sagen, dass jeder Form bestimmte Emotionen zugewiesen werden können. Jede Form hat eine andere Wirkung auf den Betrachtenden. Man kann so zwischen Gut und Böse unterscheiden, zwischen dynamischen oder statischen Objekten oder Charakteren und passivem und aggressivem Verhalten. Die drei primären Formen sind das Dreieck, der Kreis und das Quadrat. Der Kreis steht für Freundlichkeit und Positivität. Das Quadrat soll Vertrauen, Stabilität und Sicherheit kommunizieren. Das Dreieck steht für Gefahr und Aggressivität. Für Linien gilt dasselbe. Gebogen Linien wirken sehr harmonisch und ergänzend. Kantige, harte Linien sollen auf den Betrachtenden eher bedrohlich wirken. 
\cite{_drawing_basics_and_video_game_art}

\section{Environment Design}
Auch im Environment-Design sind Formen und Farben sehr wichtig, um beim Spielenden bestimmte Emotionen und Reaktion zu erzeugen. Die Formen und Farben die beim Designen gewählt werden haben einen sehr großen Einfluss auf das Erlebnis des Spielenden. 
\cite{_drawing_basics_and_video_game_art}

\subsection{Character und Environment Shapes}
Welche Formen für Charaktere und Environment verwendet werden, hängt von dem Design des Spiels ab und von den Emotionen die man beim Spielenden erzeugen möchte. Die Beziehung vom Spielenden zum Charakter und dem Environment kann sich auch über die Zeit ändern, um bestimmte Emotionen im Prozess des Spieles zu erzeugen. 
Um zwischen dem Charakter und dem Environment Harmonie zu kreieren, wird der Charakter aus Kreisen und geschwungenen Linien bestehen sowie das Environment. Man kann auch ein harmonisches Verhältnis zwischen dem Charakter und dem Environment erzeugen, indem der Charakter so gestaltet, dass er aus vielen Dreiecken besteht, genauso wie die Umgebung beziehungsweise das Environment.
\cite{_drawing_basics_and_video_game_art}

Um Unruhe und Dissonanz zwischen dem Charakter und der Umgebung zu schaffen, wird oft der Charakter des Spielenden aus vielen Kreisen gestaltet und das Environment aus vielen Dreiecken und kantigen Formen. Dissonanz kann auch andersrum geschaffen werden, der Charakter kann auch aus fast nur Dreiecken bestehen und die Umgebung aus Kreisen und weichen, geschwungenen Formen. 
\cite{_drawing_basics_and_video_game_art}

\begin{figure}[H]
	\centering
	\includegraphics[width=10cm]{CharacterCentricDesign}
	\caption{Character Centric Design\cite{_drawing_basics_and_video_game_art}}
\end{figure}

\subsection{Character Centric Environment Design }
Beim Character-Centric-Environment-Design ist schon ein Charakter und eine Idee für die Umgebung vordefiniert. Es wird um den Charakter herum eine „Welt“, ein Environment kreiert. Je nachdem wie der Charakter gestaltet wurde, welche Formen und Farben verwendet wurden und welche Emotionen man erzeugen möchte, wird die Umgebung um den Charakter gestaltet. Oft beeinflussen die Formen des Charakters die Gestaltung des Environments sehr. Es sollten bei dieser Art von Design immer zuerst die Charaktere und Bewohner der Umgebung gestaltet werden.
\cite{_drawing_basics_and_video_game_art}








