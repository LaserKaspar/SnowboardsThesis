\chapter{Überblick über die Materie}
\label{cha:sa_Einleitung}


\section{Farbpsychologie}

\subsection{Was ist Farbe?}
Wissenschaftlich betrachtet, sind Farben Eindrücke, die unsere Sinne durch die Augen und das Gehirn vermittelt bekommen. Farben entstehen durch die Absorption und Reflexion, die das Licht auf Oberflächen wirft. Das menschliche Auge ist für die Sinnesempfindung von Farben verantwortlich, die dann wiederum im Gehirn verarbeitet werden. Man kann Farbe auch als Eigenschaft des Lichts betiteln. Prinzipiell nimmt jeder Mensch Farben und Farbtöne anders wahr. Welche Farbe wahrgenommen wird, hängt von dem Gegenstand ab, der Lichtquelle und den Augen des Betrachters. Licht hat außerdem auch bestimmte Wellenlängen, je nach Länge der Welle nimmt das Auge eine andere Farbe war. Schwarz ist keine Farbe, denn sie kann auch in Abwesenheit des Lichts und völliger Dunkelheit existieren.
Weißes Licht kann mithilfe eines Prismas in Spektralfarben gebrochen werden. Spektralfarben sind diejenigen Farben, die für das menschliche Auge einen sichtbaren Farbeindruck und somit einen Farbton hinterlassen. Insgesamt gibt es sechs verschiedene Spektralfarben. Spektralfarben werden umgangssprachlich auch „Regenbogenfarben“ genannt. Zu ihnen zählen Rot, Orange, Gelb, Grün, Blau, Indigo und Violett. Die durch die Brechung von Licht entstandenen Farben können nicht in weitere Farbtöne zerlegt werden. Das Licht kann manchmal auch infrarotes Licht oder ultraviolettes Licht enthalten, welches für das bloße Auge nicht sichtbar ist.
Wenn man jedoch gelbes Licht oder jede andere Spektralfarbe durch ein Prisma fallen lässt, bekommt man dieselbe Farbe raus, die man im Prisma brechen wollte. Man kann also als Schlussfolgerung daraus ziehen, das weißes Licht das einzige ist, was durch ein Prisma in Spektralfarben gebrochen werden kann.

\begin{figure}[h]
	\centering
	\includegraphics[width=10cm]{Farbprisma}
	\caption{\cite{_basicColorTheory}}
\end{figure}

In der Kunst ist Farbe das beste Mittel, um visuell Emotionen auszudrücken oder Emotionen bei Menschen zu erzeugen. Farben sind das beste Instrument, wenn das darum geht, bei einem Menschen Gefühle und Emotionen auszulösen. Aber nicht nur, dass es wird dem Betrachter dadurch auch eine bestimmte Stimmung vermittelt. Künstler können dadurch gezielt bei Menschen bestimmte Gefühle auslösen und das alleine durch das Einsetzen von bestimmten Farben und Farbtönen.
Dabei muss beachtet werden, dass Farben in verschiedenen Kulturen verschiedene Bedeutungen haben und die Symbolik hinter manchen Farbtönen in bestimmten Ländern und Kulturkreisen eine andere ist. Doch grundsätzlich kann man jeder Farbe eine bestimmte Bedeutung und Emotion zuschreiben, die vielleicht in manchen Kulturen ein wenig abweicht, aber im Grunde genommen immer sehr ähnlich ist.
Man kann aber klar die kulturelle Bedeutung von Farben mit der psychologischen Bedeutung von Farben trennen. Die wichtigsten Farben für Kunst und Design sind die Farben Blau, Gelb, Grün, Orange, Rosa, Rot, Violett, Grau und Braun. Jede dieser Farben hat eine andere Wirkung auf die Gefühle und Emotionen der Betrachter. Der psychologische Effekt hinter der Wirkung einer Farbe hat rein gar nichts mit der kulturellen Interpretation dieser zu tun. 
Es gibt eine endliche Anzahl an Variationen, die eine Farbe für das menschliche Auge annehmen kann. Grundsätzlich gibt es eine unendliche Anzahl an Schattierungen, die eine Farbe haben kann, wobei die meisten für das Sinnesorgan eines Menschen nicht sichtbar sind. Die Schattierung einer Farbe ändert prinzipiell den psychologischen Effekt nicht.


\subsection{Der Farbkreis und Farbschemas}
Der Farbkreis besteht normalerweise aus 12 Farben, er kann aber auch aus 24 und manchmal aus bis zu 48 Farben bestehen, das hängt aber von dem Ersteller des Farbkreises ab. Das grundlegende Farbrad besteht aber jedoch nur aus 12 Farben. Um den Farbkreis besser zu verstehen, muss man sich zuerst die additive und die subtraktive Farbmischung ansehen.  

\begin{figure}[h]
	\centering
	\includegraphics[width=10cm]{Farbkreis}
	\caption{\cite{_basicColorTheory}}
\end{figure}

Farben können in Primär -, Sekundär -und Tertiärfarben aufgeteilt werden. Dabei unterscheidet man zwischen der additiven Farbmischung und der subtraktiven Farbmischung. 

Bei der subtraktiven Farbmischung, auch CMYK-Modell genannt, sind die Primärfarben Gelb, Magenta und Cyan, sie können nicht aus anderen Farben gemischt werden. Jedoch kann aus ihnen jede existierende Farbe gemischt werden. Die Sekundärfarben in diesem Modell sind Rot, Grün und Blau und die Tertiärfarbe ist Schwarz. Mischt man nun die Primärfarben gelb, Magenta und Cyan, erhält man infolge dessen die Sekundärfarben Rot, Grün und Blau. Die gesamte Mischung ergibt dann Schwarz.

Die additive Farbmischung, auch RGB-Modell genannt, besteht aus den Lichtfarben Rot, Grün und Blau. Rot, Grün und Blau sind in diesem Modell die Primärfarben, also Grundfarben. Die Sekundärfarben sind Gelb, Magenta und Cyan. Die Tertiärfarbe ist Weiß. Diese entsteht durch die Mischung der Grundfarben Rot, Grün und Blau.

Im grundlegenden Farbkreis gibt es jedoch einen Unterschied bei den Primär -, Sekundär -und Tertiärfarben zum CMYK - und RGB Modell. Im Farbrad mit 12 Farben sind die Primärfarben Rot, Gelb und Blau. Die Sekundärfarben sind die Farben, die durch das Mischen von zwei Primärfarben entstehen, das sind Farben wie Orange, Grün und Violett. Die Tertiärfarben sind die Farben, die beim Mischen von einer Primärfarbe und einer benachbarten Sekundärfarbe entstehen. Die Tertiärfarben sind all jene Farben, die die noch vorhandenen Lücken im Farbkreis füllen.


Die wichtigste Aufgabe des Farbrads ist es, viele Farben auf einmal zu sehen, um sie dadurch besser zu verstehen und zu lernen, Farben zu wählen, die gut zueinander passen. Es zeigt uns visuell das Verhalten der Farbtöne zueinander, dadurch kann man harmonische und ästhetische Kombinationen bilden.
Weiters gibt es fünf verschiedene Farbschemen, die einem helfen harmonische und dynamische Atmosphäre in seinen Kompositionen zu kreieren.

Beim komplementären Farbschema wählt man die Farben aus, die einander gegenüber liegen. Komplementäre Farben wirken nebeneinander viel intensiver und heller als Farben, die im Farbrad benachbart sind. Ein Beispiel für Komplementärfarben wären zum Beispiel Rot und Grün, sie liegen im Kreis gegenüber.

Das triadische Farbschema besteht aus einem gleichseitigen Dreieck. Das Modell des triadischen Farbschemas fokussiert sich auf eine dominante Farbe und zwei zueinander sehr harmonische Farben. Ein Beispiel dafür wäre die Kombination aus den Farbtönen Violett, Grün und Orange. Diese Farben erzeugen sehr viel Kontrast, wirken aber sehr harmonisch auf das menschliche Auge. 

Beim tetradischen Farbschema besteht die Kombination aus Farben aus denjenigen Farbtönen, die an den Ecken des Rechtecks oder Quadrats liegen. Dieses Schema besteht aus jeweils 2 Paaren von Komplementärfarben, die zwar zueinander viel Kontrast haben, aber im Gesamtbild gibt es keine sehr dominante Farbe. Die Kombination soll ein harmonisches Bild erzeugen.

Beim analogischen Farbschema werden meistens 2 Farben ausgewählt, die einer anderen Farbe benachbart sind. Diese Kombination von Farbtönen soll eine Einheit erzeugen, um das Gesamtbild ausgewogen wirken zu lassen. Ein Beispiel für solch ein Schema wären die Farben Rot, Rot-orange und Orange.

Das letzte Farbschema ist das komplemenänter-geteilte Schema, es besteht aus eine Primär -oder Sekundärfarbe und zwei Tertiärfarben, die der Komplementärfarbe der Hauptfarbe benachbart sind. So eine Farbkombination wäre zum Beispiel Rot, Gelb-Grün und Blau-Grün.

\subsection{Farbtemperatur}

\subsection{Farbrelativität}

\subsection{Farbsättigung}

\subsection{Erstellung von Farbpaletten}

\subsection{Farbe, Licht und Schatten}

\subsection{Wie wirkt sich Farbe auf die Stimmung aus?}



\section{Formpsychologie}

\subsection{Gestalt und Form}

\subsection{Linien und Linien- Stile}

\subsection{Kompositionslinien}

\subsection{Wie Formen und Linien Emotionen erzeugen}



\section{Environment Design}

\subsection{Character und Environment Shapes}

\subsection{Character Centric Environment Design }

\subsection{Bildkomposition}






