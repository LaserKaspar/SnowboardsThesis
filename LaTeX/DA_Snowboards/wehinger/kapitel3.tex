\chapter{Konkrete Implementierung}
\label{cha:sa_Einleitung}
Das Ziel ist es, eine harmonische Atmosphäre zwischen den Charakteren und der Umgebung zu schaffen. Einerseits gibt es die Umgebung der PC-Spielenden und die Umgebung des VR Spielenden. Die Snowboarder sind niedliche, freundliche Tiere, die dem Betrachtenden ein positives Gefühl vermitteln sollen. Der Magier ist der Antagonist des Spiels und soll somit einen eher gefährlicheren, aber auch mystischen und magischen Eindruck hinterlassen. Die Atmosphäre des Umfelds des VR Spielenden und der PC-Spielenden soll harmonisch mit den Designs der Charaktere wirken. Das Environment der PC-Spielenden soll sehr niedlich, positiv und freundlich auf die Betrachtenden wirken. Es soll das Thema einer winterlichen Schneelandschaft gemischt mit Süßigkeiten gut widerspiegeln. Die Umgebung des VR Spielenden soll eher eine düstere, mysteriöse Stimmung haben. Sie soll aber auch das Überthema „Süßigkeiten“ wieder aufgreifen. Das Hauptmenü der PC-Spielenden soll ebenfalls das süße, niedliche und positive Thema aufgreifen, um ein harmonisches Gesamtbild zu ergeben.

\section{Konzeption der Charaktere}
In Tricks `N` Treats gibt es verschiedene Charaktere, die dem PC-Spielenden zur Verfügung stehen. Der VR-Spielende hat nur einen Charakter zur Verfügung. Es gibt vier verschiedene Tiere, einen Hasen, einen Fuchs, einen Eisbär und einen Wolf, zwischen denen die PC-Spielenden wählen können. Der VR-Spielende spielt einen Magier. Die Charaktere wurden so konzipiert, um in der jeweiligen Umgebung harmonisch zu wirken und um dem Spiel eine atmosphärische Wirkung zu geben. Generell sollen die Tiere sehr niedlich, süß und freundlich auf die Spielenden wirken. Der Magier soll eine mystische, geheimnisvolle, aber auch bedrohliche Stimmung erzeugen. Die verwendeten Farben und Formen sollen die Gefühle die den Spielenden vermittelt werden sollen verstärken.
 
 \begin{figure}[H]
 	\centering
 	\includegraphics[width=13cm]{CharaktereProportionen}
 \end{figure}

Der Eisbär ist einer der vier Charaktere, die den Spielenden zur Auswahl stehen. Die Hauptfarben des Bären sind Weiß, Pink, Orange und ein helles und etwas dunkleres Blau. Obwohl der Eisbär außerhalb des Spiels ein Raubtier ist, wirkt sein Design sehr niedlich und freundlich. Vor allem die Farben Pink und Weiß bekräftigen diese Gefühle stark. Die Farbe Pink wirkt auf den Betrachter äußerst beruhigend. Weiß steht für Reinheit und Unschuld. Außerdem sticht das Pink der Jacke des Bären in der winterlichen Umgebung stark heraus.
Da der Bär ein größeres, sehr stabiles Tier ist, besteht er überwiegend aus Quadraten. Quadrate signalisieren Stabilität und Sicherheit. Der Eisbär soll aber auch sehr niedlich und süß wirken und besteht aufgrund dessen auch aus einigen Kreisen. Kreise signalisieren Freundlichkeit und Positivität. Durch die Verwendung von Quadraten und Kreisen wirkt der Charakter sehr sicher und stabil, aber auch niedlich und süß.


Der Hase ist ebenfalls einer der vier Charaktere, die den Spielenden zur Verfügung stehen. Die Hauptfarben des Tiers sind Hellbraun, Gelb, Rosa und ein dunkles Blau. Der Hase soll wie die anderen Charaktere eine positive Ausstrahlung auf den Spielenden haben. Das signalisiert vor allem die Farbe Gelb, Gelb wird mit der Sonne assoziiert und mit glücklichen Emotionen. Auch Rosa hat auf den Betrachtenden eine freundliche, positive Wirkung. Auch hier heben sich die Farben des Gewandes und des Hasen von der winterlichen Schneelandschaft ab. 
Der Hase wirkt nicht nur durch seine Farben sehr freundlich und süß, sondern auch durch die verwendeten Formen. Er besteht größtenteils aus Kreisen und abgerundeten Linien, dadurch vermittelt er dem Betrachtenden ein sicheres, gutes Gefühl, er wirkt sehr harmlos und ungefährlich. Da die Tiere Snowboarder sind und einen Sport ausüben, soll er auch stabil wirken, das wird durch einen geringen Anteil an Quadraten in seinem Design erreicht.


Der Wolf ist einer der vier spielbaren Charaktere. Die Hauptfarben dieses Tiers sind Orange, Violett und ein wärmeres Grau. Da Orange für Kreativität und Enthusiasmus steht, ist sie auch eine Farbe, die mit positiven Emotionen verbunden wird. Außerdem wird sie vom menschlichen Auge als wärmste Farbe im Farbkreis wahrgenommen. Die graue Fellfarbe des Wolfes ist auch mit einem warmen Farbton gemischt und wirkt deshalb freundlich auf den Betrachtenden. Generell heben sich die drei verschiedenen Hauptfarben des Wolfes gut von der Schneelandschaft und seiner Umgebung ab.
Die Figur des Wolfes besteht aus vielen Kreisen und abgerundeten Linien, das macht ihn zu einem sehr freundlichen, eher harmlosen Charakter. Sein Design beinhaltet aber auch einige Quadrate und Vierecke, um zu symbolisieren, dass er eine sehr stabile Statur hat, denn der Wolf ist einer der beiden größeren spielbaren Charaktere.

Der letzte spielbare Snowboarder ist der Fuchs. Er ist wie der Hase einer der beiden kleineren Figuren. Die Hauptfarben des Fuchses sind ein helles und dunkleres Orange, ein helles Grün und ein heller rosafarbener Farbton. Grün ist für das menschliche Auge die erholsamste Farbe und wird mit der Natur verbunden. Orange steht für Kreativität und Enthusiasmus, weshalb sie eine sehr positive Wirkung auf den Betrachtenden hat. Auch Rosa hat eine beruhigende Wirkung und verursacht beim Spielenden gute Emotionen. Durch diese Kombination von Farben und Farbtönen wirkt der Fuchs sehr niedlich und freundlich auf den Spielenden. Weiters heben sich seine Hauptfarben auch sehr gut von der winterlichen Umgebung ab.
Der Fuchs besteht auch aus sehr vielen abgerundeten Formen und Linien. Sein Design basiert auf dem Kreis und dem Quadrat, welche beim Betrachtenden positive Gefühle erzeugen. Die abgerundeten Formen lassen den Charakter sehr niedlich, süß und ungefährlich wirken. 



Der VR Spielende steuert den Magier, der Magier ist der Antagonist des Spiels. Seine Farben wurden so gewählt, um seine Rolle als Rivale der Snowboarder zu verstärken. Seine Hauptfarben sind ein sehr dunkles Grau, Violett und Pink. Dunkles Grau wird genauso wie Schwarz mit eher schlechten Emotionen verbunden. Schwarz steht für Tod, Angst und Negativität. Die Farbe Violett steht für Magie und Mystik, weshalb sie sehr gut zum Charakter passt.
Der Magier besteht größtenteils aus Dreiecken und härteren Linien. Dreiecke wirken auf den Betrachtenden aggressiv und signalisieren Gefahr. Da der Charakter des VR Spielenden der Antagonist des Spiels ist, wurde er auch formlich so gestaltet. Besonders die dreieckigen, spitzen Augen lassen ihn besonders gefährlich erscheinen.

\begin{figure}[H]
	\centering
	\includegraphics[width=12cm]{ZaubererArtwork}
\end{figure}


\section{UI/UX Design}
Das „User Interface“ beschränkt sich größtenteils auf das Erlebnis der PC-Spielenden. Aus diesem Grund wurde das UI passend zu den Snowboardern und dem Environment der Snowboarder gestaltet, um ein harmonisches Gesamtbild zu kreieren. Die Atmosphäre soll eine sehr positive, süße und niedliche sein, die beim Betrachtenden gute Emotionen und Gefühle erzeugt. Das UI soll diese Thematik widerspiegeln.

Im Hauptmenü wurden die Buttons mit einem pinken Farbton versehen. Die Farbe Pink hat einen beruhigenden Effekt für den Betrachtenden und löst glückliche Emotionen aus. Die Farbe wurde auch passend zu dem poppigen Überthema „Süßigkeiten“ gewählt. Die Buttons sollen auch in einem harmonischen Zusammenhang mit den Hintergrundbildern des Hauptmenüs stehen. Außerdem hebt sich die Farbe sehr gut von den Hintergründen ab und dadurch entsteht ein Kontrast, der sich stark auf das UI fokussiert. 

\begin{figure}[H]
	\centering
	\includegraphics[width=12cm]{MainMenu_Mockup}
\end{figure}

Das Design des Hauptmenüs wird auch in allen anderen Menüs wieder aufgegriffen. Die abgerundeten Ecken und runden Formen sowie die knalligen Farben ziehen sich durch das ganze User-Interface des Spiels. Die UI-Elemente wurden immer mit dunkleren Linien umrandet, um einen Kontrast zwischen dem UI-Element und dem Hintergrund zu schaffen. Dadurch ist das User-Interface für die Spielenden klar erkennbar beziehungsweise gut sichtbar. Das „Selection-Menu“ wurde mit diesen gestalterischen Mitteln kreiert. Es ist ein gutes Beispiel für ein harmonisches Gesamtbild zwischen UI-Elementen, Hintergrund und anderen Bildelementen. Die Farben der UI-Elemente stechen in der Komposition am meisten heraus und sind somit klar für die Spielenden erkennbar.

\begin{figure}[H]
	\centering
	\includegraphics[width=12cm]{SelectionMenuMockup}
\end{figure}

Auch das In-Game UI der PC-Spielenden wurde passend zu der Umgebung und der Charaktere gestaltet. Die Hauptfarben des User-Interface sind Pink und Gelb. Sie spiegeln die fröhliche Stimmung wieder und sehr hervor, damit sie für die Spielenden gut erkennbar sind. Weiters, sind die Icons noch mit weißen Linien umrandet, damit sie deutlich erkennbar sind. Es sind außerdem alle Elemente mit abgerundeten Linien und Kreisen gestaltet worden, um ein harmonisches Gesamtbild zu ergeben.

\section{Environment Design und Komposition}
Die Umgebung der Spielenden wurde farblich und formlich gestalterisch an ihre Charaktere angepasst. Es wurden insgesamt zwei verschiedene Umgebungen gestaltet, die der PC-Spielenden und jene des VR-Spielenden. 

Für das Environment der PC-Spielenden wurden sehr knallige, poppige Farben verwendet, die sich gut von der weiß-blauen, schneebedeckten Strecke abheben. Dadurch können die Spielenden Hindernisse und Objekte auf der Strecke gut voneinander unterscheiden. Durch diese Farbkonzeption ist es klar geregelt wo es für die Spielenden sicher ist zu fahren und wo nicht. Weiters, sind auch alle Assets formlich passend zu den Charakteren gestaltet worden. Sie sind alle sehr rundlich und ergeben so ein harmonisches Gesamtbild.

\begin{figure}[H]
	\centering
	\includegraphics[width=10cm]{EnvironmentDesign}
\end{figure}

Es wurde für den Start-Screen des Spiels ein Hintergrundbild gestaltet, das das Environment und die Charaktere noch mal in einem harmonischen Gesamtbild zeigt. Die Charaktere heben sich klar vom Hintergrund ab, nicht nur durch ihre poppigen Farben, sondern auch durch ihre Silhouetten. Alle Vordergrundelemente wurden mit sehr knalligen Farben versehen, um einen Kontrast zwischen Vorder -und Hintergrund entstehen zu lassen. Durch diese farbliche Trennung entsteht auch Tiefe in der Bildkomposition. Außerdem wurden bei der Gestaltung Kompositionslinien verwendet, die das Auge des Betrachtenden auf die wichtigsten Bildelemente fokussiert. Durch die Verwendung von Kompositionslinien wurde das Bild sehr dynamisch.

\begin{figure}[H]
	\centering
	\includegraphics[width=12cm]{TricksNTreats_MS}
\end{figure}

Die Umgebung des VR-Spielenden wurde passend zu dem Design des Magiers gestaltet. Atmosphärisch ist alles recht düster und dunkel, dennoch wurde versucht, es etwas farbenfroher zu designen. Die Farben des Magiers finden sich in seiner Umgebung immer wieder, um ein harmonisches Gesamtbild zu kreieren. Generell ist der Raum des VR-Spielenden im Gegensatz zu dem Environment der PC-Spielenden sehr kantig und eckig. Es finden sich viele harte Formen und Dreiecke. Von der formlichen Gestaltung stehen so Charakter und Umgebung in einer harmonischen Beziehung zueinander.

\section{Andere 2D-Assets}
Weitere nennenswerte Assets wären der Comic für die Start-Sequenz sowie Texturen für 3D-Assets und jegliche Icons und Hintergründe. Alle anderen Assets, die in den oberen Absätzen nicht genannt wurden, wurden farblich und formlich an den Stil des Spiels angepasst. 
Viel Arbeit wurde in den Comic investiert, der die Vorgeschichte des Spiels erzählt. Von den Farben wurde er passend zum Gesamtbild von „Tricks ‚N‘ Treats“ gestaltet. Er geht auch weiter auf die Persönlichkeiten der einzelnen Charaktere ein.


\section{Fazit}
Im Großen und Ganzen gab es bei der 2D-Asset Produktion keine großen Schwierigkeiten. Die Produktion der Assets verlief wie geplant und die Endergebnisse sind geworden, wie sie angedacht wurden. Von der Konzeption bis hin zur Umsetzung verlief die Ausarbeitung sehr gut. 
Würde an dem Projekt in Zukunft weitergearbeitet werden, würde das Projekt um weitere Strecken erweitert werden. In Folge würde das bedeuten, dass mehr Environment-Konzepte nötig wären und viele weitere 2D-Assets produziert werden müssten. Von User-Interface Elementen bis hin zu Hintergründen und Texturen.
Das Projekt war für die persönliche Weiterentwicklung ein wichtiger Schritt, da das Gefühl für Farben und Formen bzw. Farb -und Formpsychologie stark vertieft wurde. 



