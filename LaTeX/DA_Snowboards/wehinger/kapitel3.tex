\chapter{Konkrete Implementierung}
\label{cha:sa_Einleitung}

\section{Konzeption der Charaktere}
In Tricks `N` Treats gibt es verschiedene Charaktere, die dem PC-Spielenden zur Verfügung stehen. Der VR-Spielende hat nur einen Charakter zur Verfügung. Es gibt vier verschiedene Tiere, einen Hasen, einen Fuchs, einen Eisbär und einen Wolf, zwischen denen die PC-Spielenden wählen können. Der VR-Spielende spielt einen Magier. Die Charaktere wurden so konzipiert, um in der jeweiligen Umgebung harmonisch zu wirken und um dem Spiel eine atmosphärische Wirkung zu geben. Generell sollen die Tiere sehr niedlich, süß und freundlich auf die Spielenden wirken. Der Magier soll eine mystische, geheimnisvolle, aber auch bedrohliche Stimmung erzeugen. Die verwendeten Farben und Formen sollen die Gefühle die den Spielenden vermittelt werden sollen verstärken.
 
 
Der Eisbär ist einer der vier Charaktere, die den Spielenden zur Auswahl stehen. Die Hauptfarben des Bären sind Weiß, Pink, Orange und ein helles und etwas dunkleres Blau. Obwohl der Eisbär außerhalb des Spiels ein Raubtier ist, wirkt sein Design sehr niedlich und freundlich. Vor allem die Farben Pink und Weiß bekräftigen diese Gefühle stark. Die Farbe Pink wirkt auf den Betrachter äußerst beruhigend. Weiß steht für Reinheit und Unschuld. Außerdem sticht das Pink der Jacke des Bären in der winterlichen Umgebung stark heraus.
Da der Bär ein größeres, sehr stabiles Tier ist, besteht er überwiegend aus Quadraten. Quadrate signalisieren Stabilität und Sicherheit. Der Eisbär soll aber auch sehr niedlich und süß wirken und besteht aufgrund dessen auch aus einigen Kreisen. Kreise signalisieren Freundlichkeit und Positivität. Durch die Verwendung von Quadraten und Kreisen wirkt der Charakter sehr sicher und stabil, aber auch niedlich und süß.


Der Hase ist ebenfalls einer der vier Charaktere, die den Spielenden zur Verfügung stehen. Die Hauptfarben des Tiers sind Hellbraun, Gelb, Rosa und ein dunkles Blau. Der Hase soll wie die anderen Charaktere eine positive Ausstrahlung auf den Spielenden haben. Das signalisiert vor allem die Farbe Gelb, Gelb wird mit der Sonne assoziiert und mit glücklichen Emotionen. Auch Rosa hat auf den Betrachtenden eine freundliche, positive Wirkung. Auch hier heben sich die Farben des Gewandes und des Hasen von der winterlichen Schneelandschaft ab. 
Der Hase wirkt nicht nur durch seine Farben sehr freundlich und süß, sondern auch durch die verwendeten Formen. Er besteht größtenteils aus Kreisen und abgerundeten Linien, dadurch vermittelt er dem Betrachtenden ein sicheres, gutes Gefühl, er wirkt sehr harmlos und ungefährlich. Da die Tiere Snowboarder sind und einen Sport ausüben, soll er auch stabil wirken, das wird durch einen geringen Anteil an Quadraten in seinem Design erreicht.

\section{UI/UX Design}

\section{Environment Design und Komposition}

\section{Andere Assets}


