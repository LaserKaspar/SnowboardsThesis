\chapter{Möglichkeiten zur Implementierunge}
\label{cha:sa_Einleitung}

\section{Nutzung von Farben in Spielen}
Die Nutzung von Farben in Videospielen war mit ihrer Entstehung sehr mit dem damaligen Stand der Technik verbunden. 1972 wurde die Farbüberlagerung erfunden und hat es ermöglicht, dass Videospiele in Farbe dargestellt werden können. Davor konnten Spiele nur in Schwarz-Weiß oder Monochrom dargestellt werden. Ab Ende der 1980er sind fast alle Videospiele in Farbe. 1985 veröffentlichte Sega das Sega Master System, welches Spiele mit 32 verschiedenen Farben darstellen konnte. Schon 9 Jahre später veröffentlicht Sony die Playstation, welche es ermöglicht, Spiele mit rund 16.7 Millionen Farben darzustellen. Heutzutage können Spiele mit über 16.7 Milliarden verschiedenen Farben dargestellt werden. 

Durch die Nutzung von Farben kann man dem Spieler Gefühle und Emotionen auf eine sehr einfache und natürliche Weise vermitteln. Prinzipiell ist das Farbempfinden von Kulturkreisen immer etwas unterschiedlich, dennoch gibt es auf die meisten Farben eine universelle Reaktion. Wie Menschen Farben wahrnehmen, hängt nicht nur von ihrer Kultur ab, sondern auch von ihrem Alter, ihrer Herkunft, ihrem Geschlecht und ihrem ethnischen Hintergrund. Wenn ein Spieler in einem Spiel zum Beispiel Eierfarbe Rot sieht, wird ihm signalisiert, dass er in Gefahr ist. Rot steht prinzipiell für Energie, Liebe, Selbstbewusstsein und Leidenschaft, wird aber auch als Warnfarbe gesehen. Rot wird mit Gefahren in Verbindung gesetzt. Explosionen, Feuer und Blut etc. wären dafür ein gutes Beispiel. In vielem Actionspielen wird unter anderem ein an den Ecken und Kanten roter Bildschirm eingesetzt, wenn der Spieler verletzt ist. Das signalisiert dem Spieler, dass er ab jetzt wachsamer und vorsichtiger sein muss. 

In vielen MMORPGs oder in Singleplayer-Spielen, werden viele sammelbaren Objekte, Items oder Schatztruhen in bestimmten Farben angezeigt. Meist werden die Farben Weiß, Grün, Blau, Lila und Gold verwendet, wobei Weiß weniger gut ist, als Violett und Gold. Violett wird mit dem Adel verbunden, denn früher konnten sich nur sehr wohlhabende Menschen das violette Farbpigment leisten. Die Farbe Violett steht für Eleganz, Würde und Raffinesse. Die Farbe hat aber auch etwas Magisches und Mystisches an sich. Aus diesem Grund sind sehr seltene und außergewöhnliche Objekte und Items meist Violett, da sie dem Spieler signalisieren sollen, dass er etwas Besonderes und Wertvolles gefunden hat. 

\subsection{Beispiel – Mario Kart 8}

\subsection{Beispiel – Journey}
Journey ist ein Adventure-Spiel, welches ohne jegliche Worte auskommt. Es wurde erstmals 2012 für die Playstation 3 veröffentlicht. Der Entwickler des Spiels ist „thatgamecompany“ und der Publisher „Sony Interactive Entertainment“ und „Annapurna Interactive“. Das Spiel startet in einer Wüste, in der Ferne liegt ein großer Berg, an dessen Gipfel ein Lichtstrahl in den Himmel ragt. Der Spieler spielt eine Figur in einer roten Robe. Auf dem Weg zum Berg kann der Spieler mit bestehender Internetverbindung auf einen anderen Spieler treffen. Die Spielenden können nicht miteinander reden, aber sich dennoch gegenseitig helfen. Das Zeil von Journey ist es, den in der Ferne liegenden Berg zu erreichen. 


\section{Nutzung von Formen in Spielen}

\subsection{Beispiel – Mario Kart 8}

\subsection{Beispiel – Journey}



