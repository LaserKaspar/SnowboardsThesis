\chapter{Möglichkeiten zur Implementierunge}
\label{cha:sa_Einleitung}

\section{Nutzung von Farben in Spielen}
Die Nutzung von Farben in Videospielen war mit ihrer Entstehung sehr mit dem damaligen Stand der Technik verbunden. 1972 wurde die Farbüberlagerung erfunden und hat es ermöglicht, dass Videospiele in Farbe dargestellt werden können. Davor konnten Spiele nur in Schwarz-Weiß oder Monochrom dargestellt werden. Ab Ende der 1980er sind fast alle Videospiele in Farbe. 1985 veröffentlichte Sega das Sega Master System, welches Spiele mit 32 verschiedenen Farben darstellen konnte. Schon 9 Jahre später veröffentlicht Sony die Playstation, welche es ermöglicht, Spiele mit rund 16.7 Millionen Farben darzustellen. Heutzutage können Spiele mit über 16.7 Milliarden verschiedenen Farben dargestellt werden. 

Durch die Nutzung von Farben kann man dem Spieler Gefühle und Emotionen auf eine sehr einfache und natürliche Weise vermitteln. Prinzipiell ist das Farbempfinden von Kulturkreisen immer etwas unterschiedlich, dennoch gibt es auf die meisten Farben eine universelle Reaktion. Wie Menschen Farben wahrnehmen, hängt nicht nur von ihrer Kultur ab, sondern auch von ihrem Alter, ihrer Herkunft, ihrem Geschlecht und ihrem ethnischen Hintergrund. Wenn ein Spieler in einem Spiel zum Beispiel Eierfarbe Rot sieht, wird ihm signalisiert, dass er in Gefahr ist. Rot steht prinzipiell für Energie, Liebe, Selbstbewusstsein und Leidenschaft, wird aber auch als Warnfarbe gesehen. Rot wird mit Gefahren in Verbindung gesetzt. Explosionen, Feuer und Blut etc. wären dafür ein gutes Beispiel. In vielem Actionspielen wird unter anderem ein an den Ecken und Kanten roter Bildschirm eingesetzt, wenn der Spieler verletzt ist. Das signalisiert dem Spieler, dass er ab jetzt wachsamer und vorsichtiger sein muss. 

In vielen MMORPGs oder in Singleplayer-Spielen, werden viele sammelbaren Objekte, Items oder Schatztruhen in bestimmten Farben angezeigt. Meist werden die Farben Weiß, Grün, Blau, Lila und Gold verwendet, wobei Weiß weniger gut ist, als Violett und Gold. Violett wird mit dem Adel verbunden, denn früher konnten sich nur sehr wohlhabende Menschen das violette Farbpigment leisten. Die Farbe Violett steht für Eleganz, Würde und Raffinesse. Die Farbe hat aber auch etwas Magisches und Mystisches an sich. Aus diesem Grund sind sehr seltene und außergewöhnliche Objekte und Items meist Violett, da sie dem Spieler signalisieren sollen, dass er etwas Besonderes und Wertvolles gefunden hat. 

\subsection{Beispiel – Outlast}
Outlast ist ein Horror-Spiel, welches 2013 für die Playstation 4, die Xbox One und den PC erschienen ist. Der Entwickler und Publisher ist „Red Barrels“. Der Journalist Miles Upshur erhält durch eine anonyme Quelle die Information, dass im Mount Massive Asylum unmenschliche Experimente an Patienten durchgeführt werden. Er geht dieser Spur nach und begibt sich auf den Weg zur Nervenklinik. Dort angekommen läuft alles ganz anders als erwartet. Es gibt keinen Weg nach draußen mehr und alle Insassen sind ausgebrochen. Ziel ist es, einen Weg aus der Nervenklinik zu finden und währenddessen alles mein einem Camcorder zu filmen. 

\begin{figure}[H]
	\centering
	\includegraphics[width=10cm]{Outlast}
	\caption{\cite{solarski2012drawing}}
\end{figure}

In Outlast wird die Atmosphäre und Stimmung des Spiels durch den Qualitätskontrast der Farben erzeugt. Der Qualitätskontrast wird durch die Farbqualität der Farbe erzeugt. Durch das Mischen mit Schwarz, Weiß oder Grau wird eine Farbe unrein. Die Farbe verliert als Folge an Leuchtkraft und Qualität. Farben, die mit Grau gemischt werden, wirken trüb. Wird eine Farbe mit Schwarz gemischt, so wirkt sie bedrohlicher. Wenn man eine Farbe mit Weiß mischt, so wird die Farbe kälter. Wendet man einen Qualitätskontrast neben reinen Farben an, fallen die reinen Farben stark auf.

In Outlast wird dem Spielenden durch den Qualitätskontrast eine stark bedrohliche Umgebung vermittelt. Alles ist sehr düster und dunkel, erst wenn der Spielende den Camcorder benutzt, wird die Umgebung in ein blasses Grün getaucht. Die Farben wirken größtenteils sehr leblos, aber gleichzeitig bedrohlich. Die einzig gesättigte, reine Farbe ist Rot in Form von Blut überall in der Anstalt. 

\subsection{Beispiel – Journey}
Journey ist ein Adventure-Spiel, welches ohne jegliche Worte auskommt. Es wurde erstmals 2012 für die Playstation 3 veröffentlicht. Der Entwickler des Spiels ist „thatgamecompany“ und der Publisher „Sony Interactive Entertainment“ und „Annapurna Interactive“. Das Spiel startet in einer Wüste, in der Ferne liegt ein großer Berg, an dessen Gipfel ein Lichtstrahl in den Himmel ragt. Der Spieler spielt eine Figur in einer roten Robe. Auf dem Weg zum Berg kann der Spieler mit bestehender Internetverbindung auf einen anderen Spieler treffen. Die Spielenden können nicht miteinander reden, aber sich dennoch gegenseitig helfen. Das Zeil von Journey ist es, den in der Ferne liegenden Berg zu erreichen. 
\cite{solarski2012drawing}

Journey nutzt die Vorteile von Farbkontrasten, um bei den Spielenden Emotionen zu erzeugen. Das Spiel bedient sich am Warm-Kalt-Kontrast. Beim Warm-Kalt-Kontrast werden warme und kalte Farben einander gegenübergestellt. Warme beziehungsweise kalte Farben liegen im Farbkreis sehr nah beieinander, sie sind verwandte Farben und kreieren deshalb Harmonie. Warme und kalte Farben stehen sich im Farbkreis gegenüber und erzeugen deshalb Dissonanz. Die kalten Farben des Farbkreises sind Gelbgrün, Grün, Blaugrün, Blau, Blauviolett und Violett. Zu den warmen Farben des Farbkreises zählen Gelb, gelborange, Orange, Rotorange, Rot und Rotviolett.
\cite{solarski2012drawing}

\begin{figure}[H]
	\centering
	\includegraphics[width=10cm]{Journey2}
	\caption{\cite{solarski2012drawing}}
\end{figure}

In Journey wechseln sich die Warm-Kalt-Kontraste immer wieder ab. Am Anfand des Spiels ist der Spielende in einer Wüste, nach der Wüste führt ihn seine Reise in eine kühle Höhle. Nach den warmen, sandigen Ruinen begibt sich der Spielende zum eiskalten Berg in der Ferne. Die Figur selbst trägt eine rote Robe die in der kühlen Höhle und am kalten Berg den Warm-Kalt-Kontrast nochmal verdeutlicht.
\cite{solarski2012drawing}


\section{Nutzung von Formen in Spielen}
Das Konzept und die Nutzung von Formen und Linien in Videospielen stammt aus der Natur. Objekte und Dinge, die aus abgerundeten Formen bestehen, empfinden Menschen als sicher. Kantige Objekte und Formen gelten als gefährlich. Der Kreis steht für Freundlichkeit und Positivität. Das Quadrat soll Vertrauen, Stabilität und Sicherheit kommunizieren. Das Dreieck steht für Gefahr und Aggressivität. Man kann diese Formen in einer Skala der Emotionen einordnen. Links ist die positivste Form, der Kreis und ganz rechts das Dreieck, die negativste Form.
\cite{solarski2012drawing}

In Videospielen bestehen die „guten“ Charaktere beziehungsweise die Helden aus vielen Kreisen und abgerundeten Linien. Oft enthalten ihre Designs auch Quadrate, um Sicherheit und Stabilität zu signalisieren. Die Antagonisten in Spielen haben meist sehr dominante Silhouetten, welche aus Dreiecken oder sehr kantigen Linien bestehen. Weiters, ist es äußerst wichtig, Silhouetten aus großen Formen zu gestalten, um die Figur des Charakters herausstechen zu lassen.
\cite{solarski2012drawing}

\begin{figure}[H]
	\centering
	\includegraphics[width=10cm]{CharacterShapes}
	\caption{\cite{solarski2012drawing}}
\end{figure}

In Abbildung 12.3 kann man erkennen, dass in der linken Spalte die Protagonisten aus sehr abgerundeten Linien und kreisen bestehen. Sie wirken auf den Betrachtenden sehr positiv. Auf der rechten Seite sind die Antagonisten, welche aus Dreiecken und vielen kantigen Linien bestehen. Sie wirken auf den Betrachtenden sehr aggressiv und bedrohlich. Generell sind alle abgebildeten Charaktere aus markanten Formen gestaltet, damit sie möglichst gut herausstechen. 
\cite{solarski2012drawing}

\subsection{Beispiel – Journey}
Journey ist ein Adventure-Spiel, welches ohne jegliche Worte auskommt. Es wurde erstmals 2012 für die Playstation 3 veröffentlicht. Der Entwickler des Spiels ist „thatgamecompany“ und der Publisher „Sony Interactive Entertainment“ und „Annapurna Interactive“. Das Spiel startet in einer Wüste, in der Ferne liegt ein großer Berg, an dessen Gipfel ein Lichtstrahl in den Himmel ragt. Der Spieler spielt eine Figur in einer roten Robe. Auf dem Weg zum Berg kann der Spieler mit bestehender Internetverbindung auf einen anderen Spieler treffen. Die Spielenden können nicht miteinander reden, aber sich dennoch gegenseitig helfen. Das Zeil von Journey ist es, den in der Ferne liegenden Berg zu erreichen.
\cite{solarski2012drawing}

In Journey wird auch eine positive, harmonische Stimmung durch die Nutzung von Dreiecken und kantigen Linien erzeugt. Der Charakter besteht aus einem Dreieck und sein Umfeld ebenso. Die Umgebung ist geprägt durch kantige Objekte und Formen und dreieckige Figuren. Durch die Nutzung von Dreiecken für Charakter und Umfeld wird Harmonie erzeugt. 
\cite{solarski2012drawing}

\begin{figure}[H]
	\centering
	\includegraphics[width=10cm]{Journey1}
	\caption{\cite{solarski2012drawing}}
\end{figure}

\subsection{Beispiel – Super Mario Galaxy}
Super Mario Galaxy ist ein 3D-Plattformer, der 2007 für die Wii veröffentlicht wurde. Der Entwickler und Publisher des Spiels ist Nintendo. Der Protagonist Mario reist von Planet zu Planet, kämpft gegen Gegner, löst Rätsel und sammelt Items. Ziel des Spiels ist es, das Pilzkönigreich und Prinzessin Peach zu retten. 

Mit Formen und Linien kann man bei den Spielenden Gefühle und Emotionen erzeugen. Je nachdem, welche Emotion beim Spielenden erzeugt werden soll, werden andere Charakter und Environment Shapes verwendet. Um Harmonie in einer Umgebung zu erzeugen, besteht der Charakter und sein Umfeld aus Kreisen und runden Linien. Harmonie kann aber auch durch Dreiecke und kantige Linien entstehen. Um Dissonanz zu erzeugen, besteht der Charakter aus Kreisen und gebogenen Linien oder Dreiecken und kantigen Linien und das Umfeld steht im Gegensatz dazu.
\cite{solarski2012drawing}

\begin{figure}[H]
	\centering
	\includegraphics[width=10cm]{SuperMarioGalaxy}
	\caption{\cite{solarski2012drawing}}
\end{figure}

In Super Mario Galaxy besteht auf den meisten Planeten Harmonie zwischen dem Charakter und seiner Umgebung. Mario besteht größtenteils aus Kreisen und runden, gebogenen Linien. Seine Umgebung besteht im Großen und Ganzen auch aus Kreisen und abgerundeten Linien. Diese Beziehung zwischen Charakter und Umgebung erzeugt Harmonie und löst bei den Spielenden positive Gefühle aus. 
\cite{solarski2012drawing}



