
\chapter{Kurzfassung}

Diese Diplomarbeit befasst sich mit der Entwicklung des asymmetrischen lokalen Multiplayer-Spieles Tricks ‘n’ Treats. 
Der erste Teil der Arbeit befasst sich mit der Entwicklung von asymmetrischen Virtual Reality Spielen, wobei ein besonderer Fokus auf Performance-Optimierung, den aufgetretenen Problemen und technischen Limitationen von asymmetrischen Virtual Reality Spielen. 
Im zweiten Teil wird genauer bearbeitet, wie sich die Dynamiken zwischen PC und VR bilden, wie man diese beeinflusst und im Falle des Spieles eine gute Interaktion zwischen den beiden konstruieren kann. 
Der dritte Teil der Arbeit befasst sich mit der Welt und Atmosphäre des Spiels. Besonders widmet er sich wie Farb- und Formgebung die Gefühle und Aktionen der SpielerInnen beeinflussen. 
Der vierte Teil beschäftigt sich mit dem 3D-Modeling und geht spezifisch auf die Performance Optimierung der 3D-Assets ein. Dabei liegt ein großer Fokus auf der Recherche und Verwendung verschiedener Optimierungs-Methoden und deren Vor- und Nachteile.
